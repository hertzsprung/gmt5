%	$Id: GMT_API.tex,v 1.1 2006-03-31 07:20:34 pwessel Exp $
\documentclass[12pt]{article}
\usepackage{amsmath}
\usepackage{epsfig}
\usepackage{makeidx}
\usepackage{float}
\usepackage{times}
\usepackage{mathptm}

\begin{document}
{\center \bf Documentation of the GMT Application Program Interface}\\

This documentation serves two related purposes:
\begin{enumerate}
\item It will document the API library for application developers that wish
to call functions in the API.
\item It will document the underlying API utilities for developers of
the API itself.
\end{enumerate}
Of course, many developers might be interested in learning about both sets of
functions.\\


{\bf 1. Overview of the GMT Application Program Interface}\\

Users who wish to create their own GMT-like applications based on the API
must make sure their program goes through these steps:

\begin{enumerate}
\item Initialize a GMT session by calling \texttt{GMTAPI\_Create\_Session} which
will return a pointer to a GMT API control structure.  This pointer will be used
as the first argument to all subsequent GMT API function calls.
\item Register any input sources and output destinations with the session using
\texttt{GMTAPI\_Register\_Import} and \texttt{GMTAPI\_Register\_Export}, respectively.
These will typically be used to specify memory locations and open file handles;
i/o involving files are already handled by GMT itself.
\item Each registration will generate a unique ID number.  These numbers are to be
used with the relevant GMT function as filenames of the form ``GMTAPI-\#-ID''.  When
GMT  functions encounter such filenames they will extract the ID and make a connection
to the source of destination registered under that ID.
\item When a source or destination is no longer needed, you should free it up by calls
to ?.
\item Repeat steps 2--4 as many times as your application desires.  All functions
return a status code which is 0 if all is well.  For non-zero return values, use
\texttt{GMTAPI\_Report\_Error} to generate an error message on {\it stderr}.
\item To terminate the GMT session, call \texttt{GMTAPI\_Destroy\_Session}.
\end{enumerate}

Normally, only one GMT session will be active at one time, but developers are free
to initiate any number of session, each with their own internal GMT parameter base.
Such session can run concurrently and are thread-safe.

{\bf 2. The GMT Application Program Interface}\\

All calls to the GMT functions themselves have identical syntax that come in two different flavors.
They differ in how the command options are passed to the function which can be done via
\begin{enumerate}
\item a text-based command specification.
\item A pointer to a GMT common option structure and a linked list of objects with individual options
for the current program.
\end{enumerate}

The former interface is used by the stand-alone GMT applications and is expected to be used by
FORTRAN developers; however, there are no restrictions on the use of these functions.  The
latter interface allows developers of higher-level programming languages to pass all command
options via a common structure with the basic GMT domain and projection options as well as a
NULL-terminated, linked list of structures, each containing information about one program option.
The syntax of the two formats are:

\texttt{int GMT\_program\_cmd (struct GMTAPI\_CTRL *API, int n\_args, char *args[])}
\texttt{int GMT\_program (struct GMTAPI\_CTRL *API, struct GMT\_COMMON *C, struct GMT\_OPTIONS *options)}

where ``program'' is replaced by the desired application name (e.g., blockmean, psxy, grdmath).

\end{document}
