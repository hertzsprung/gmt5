%------------------------------------------
%	$Id: GMT_Chapter_6.tex,v 1.18 2008-01-23 03:22:47 guru Exp $
%
%	The GMT Documentation Project
%	Copyright 2000-2008.
%	Paul Wessel and Walter H. F. Smith
%------------------------------------------
%
\chapter{\gmt\ Map Projections}
\label{ch:6}

\GMT\ implements 25 different map projections.  They all project the input coordinates
longitude and latitude to positions on a map.  In general, $x' = f(x,y,z)$ and $y' = g(x,y,z)$, where
$z$ is implicitly given as the radial vector length to the $(x,y)$ point on the chosen ellipsoid.  The functions $f$ and $g$ can be
quite nasty and we will refrain from presenting details in this document.  The interested read is referred to
{\it Snyder} [1987]\footnote{Snyder, J. P., 1987, Map Projections \- A Working Manual, U.S. Geological Survey Prof. Paper 1395.}.
We will mostly be using the \GMTprog{pscoast} command to demonstrate each of the projections.
\GMT\ map projections are grouped into four categories depending on the
nature of the projection.  The groups are

\begin{enumerate}
\item Conic map projections
\item Azimuthal map projections
\item Cylindrical map projections
\item Miscellaneous projections
\end{enumerate}

Because $x$ and $y$ are coupled we can only specify one plot-dimensional scale, typically
a map \emph{scale} (for lower-case map projection code) or a map \emph{width} (for upper-case
map projection code).  However, in some cases it would be more
practical to specify map \emph{height} instead of \emph{width}, while in other situations it would be nice
to set either the \emph{shortest} or \emph{longest} map dimension.  Users may select
these alternatives by appending a character code to their map dimension.  To specify map \emph{height},
append {\bf h} to the given dimension; to select the minimum map dimension, append {\bf -}, whereas you may
append {\bf +} to select the maximum map dimension.  Without the modifier the map width is
selected by default.

%------------------------------------------
%	$Id: GMT_Chapter_6.tex,v 1.19 2008-04-30 03:53:30 remko Exp $
%
%	The GMT Documentation Project
%	Copyright 2000-2008.
%	Paul Wessel and Walter H. F. Smith
%------------------------------------------
%
\chapter{\gmt\ Map Projections}
\label{ch:6}

\GMT\ implements more than 30 different projections.  They all project the input coordinates
longitude and latitude to positions on a map.  In general, $x' = f(x,y,z)$ and $y' = g(x,y,z)$, where
$z$ is implicitly given as the radial vector length to the $(x,y)$ point on the chosen ellipsoid.  The functions $f$ and $g$ can be
quite nasty and we will refrain from presenting details in this document.  The interested read is referred to
{\it Snyder} [1987]\footnote{Snyder, J. P., 1987, Map Projections \- A Working Manual, U.S. Geological Survey Prof. Paper 1395.}.
We will mostly be using the \GMTprog{pscoast} command to demonstrate each of the projections.
\GMT\ map projections are grouped into four categories depending on the
nature of the projection.  The groups are

\begin{enumerate}
\item Conic map projections
\item Azimuthal map projections
\item Cylindrical map projections
\item Miscellaneous projections
\end{enumerate}

Because $x$ and $y$ are coupled we can only specify one plot-dimensional scale, typically
a map \emph{scale} (for lower-case map projection code) or a map \emph{width} (for upper-case
map projection code).  However, in some cases it would be more
practical to specify map \emph{height} instead of \emph{width}, while in other situations it would be nice
to set either the \emph{shortest} or \emph{longest} map dimension.  Users may select
these alternatives by appending a character code to their map dimension.  To specify map \emph{height},
append \textbf{h} to the given dimension; to select the minimum map dimension, append \textbf{-}, whereas you may
append \textbf{+} to select the maximum map dimension.  Without the modifier the map width is
selected by default.

In \GMT\ version 4.3.0 we noticed we ran out of the alphabet for 1-letter (and sometimes 2-letter) projection codes. To allow more flexibility, and to make it easier to remember the codes, we implemented the option to use the abbreviations used by the \progname{Proj4} mapping package. Since some of the \GMT\ projections are not in \progname{Proj4}, we invented some of our own as well. For a full list of both the old 1- and 2-letter codes, as well as the \progname{Proj4}-equivalents see the quick reference cards in Section~\ref{sec:purpose}. For example, \Opt{JM15c} and \Opt{JMerc/15c} have the same meaning.

%------------------------------------------
%	$Id: GMT_Chapter_6.tex,v 1.19 2008-04-30 03:53:30 remko Exp $
%
%	The GMT Documentation Project
%	Copyright 2000-2008.
%	Paul Wessel and Walter H. F. Smith
%------------------------------------------
%
\chapter{\gmt\ Map Projections}
\label{ch:6}

\GMT\ implements more than 30 different projections.  They all project the input coordinates
longitude and latitude to positions on a map.  In general, $x' = f(x,y,z)$ and $y' = g(x,y,z)$, where
$z$ is implicitly given as the radial vector length to the $(x,y)$ point on the chosen ellipsoid.  The functions $f$ and $g$ can be
quite nasty and we will refrain from presenting details in this document.  The interested read is referred to
{\it Snyder} [1987]\footnote{Snyder, J. P., 1987, Map Projections \- A Working Manual, U.S. Geological Survey Prof. Paper 1395.}.
We will mostly be using the \GMTprog{pscoast} command to demonstrate each of the projections.
\GMT\ map projections are grouped into four categories depending on the
nature of the projection.  The groups are

\begin{enumerate}
\item Conic map projections
\item Azimuthal map projections
\item Cylindrical map projections
\item Miscellaneous projections
\end{enumerate}

Because $x$ and $y$ are coupled we can only specify one plot-dimensional scale, typically
a map \emph{scale} (for lower-case map projection code) or a map \emph{width} (for upper-case
map projection code).  However, in some cases it would be more
practical to specify map \emph{height} instead of \emph{width}, while in other situations it would be nice
to set either the \emph{shortest} or \emph{longest} map dimension.  Users may select
these alternatives by appending a character code to their map dimension.  To specify map \emph{height},
append \textbf{h} to the given dimension; to select the minimum map dimension, append \textbf{-}, whereas you may
append \textbf{+} to select the maximum map dimension.  Without the modifier the map width is
selected by default.

In \GMT\ version 4.3.0 we noticed we ran out of the alphabet for 1-letter (and sometimes 2-letter) projection codes. To allow more flexibility, and to make it easier to remember the codes, we implemented the option to use the abbreviations used by the \progname{Proj4} mapping package. Since some of the \GMT\ projections are not in \progname{Proj4}, we invented some of our own as well. For a full list of both the old 1- and 2-letter codes, as well as the \progname{Proj4}-equivalents see the quick reference cards in Section~\ref{sec:purpose}. For example, \Opt{JM15c} and \Opt{JMerc/15c} have the same meaning.

%------------------------------------------
%	$Id: GMT_Chapter_6.tex,v 1.19 2008-04-30 03:53:30 remko Exp $
%
%	The GMT Documentation Project
%	Copyright 2000-2008.
%	Paul Wessel and Walter H. F. Smith
%------------------------------------------
%
\chapter{\gmt\ Map Projections}
\label{ch:6}

\GMT\ implements more than 30 different projections.  They all project the input coordinates
longitude and latitude to positions on a map.  In general, $x' = f(x,y,z)$ and $y' = g(x,y,z)$, where
$z$ is implicitly given as the radial vector length to the $(x,y)$ point on the chosen ellipsoid.  The functions $f$ and $g$ can be
quite nasty and we will refrain from presenting details in this document.  The interested read is referred to
{\it Snyder} [1987]\footnote{Snyder, J. P., 1987, Map Projections \- A Working Manual, U.S. Geological Survey Prof. Paper 1395.}.
We will mostly be using the \GMTprog{pscoast} command to demonstrate each of the projections.
\GMT\ map projections are grouped into four categories depending on the
nature of the projection.  The groups are

\begin{enumerate}
\item Conic map projections
\item Azimuthal map projections
\item Cylindrical map projections
\item Miscellaneous projections
\end{enumerate}

Because $x$ and $y$ are coupled we can only specify one plot-dimensional scale, typically
a map \emph{scale} (for lower-case map projection code) or a map \emph{width} (for upper-case
map projection code).  However, in some cases it would be more
practical to specify map \emph{height} instead of \emph{width}, while in other situations it would be nice
to set either the \emph{shortest} or \emph{longest} map dimension.  Users may select
these alternatives by appending a character code to their map dimension.  To specify map \emph{height},
append \textbf{h} to the given dimension; to select the minimum map dimension, append \textbf{-}, whereas you may
append \textbf{+} to select the maximum map dimension.  Without the modifier the map width is
selected by default.

In \GMT\ version 4.3.0 we noticed we ran out of the alphabet for 1-letter (and sometimes 2-letter) projection codes. To allow more flexibility, and to make it easier to remember the codes, we implemented the option to use the abbreviations used by the \progname{Proj4} mapping package. Since some of the \GMT\ projections are not in \progname{Proj4}, we invented some of our own as well. For a full list of both the old 1- and 2-letter codes, as well as the \progname{Proj4}-equivalents see the quick reference cards in Section~\ref{sec:purpose}. For example, \Opt{JM15c} and \Opt{JMerc/15c} have the same meaning.

\input{GMT_Chapter_6.1}
\input{GMT_Chapter_6.2}
\input{GMT_Chapter_6.3}
\input{GMT_Chapter_6.4}

%------------------------------------------
%	$Id: GMT_Chapter_6.tex,v 1.19 2008-04-30 03:53:30 remko Exp $
%
%	The GMT Documentation Project
%	Copyright 2000-2008.
%	Paul Wessel and Walter H. F. Smith
%------------------------------------------
%
\chapter{\gmt\ Map Projections}
\label{ch:6}

\GMT\ implements more than 30 different projections.  They all project the input coordinates
longitude and latitude to positions on a map.  In general, $x' = f(x,y,z)$ and $y' = g(x,y,z)$, where
$z$ is implicitly given as the radial vector length to the $(x,y)$ point on the chosen ellipsoid.  The functions $f$ and $g$ can be
quite nasty and we will refrain from presenting details in this document.  The interested read is referred to
{\it Snyder} [1987]\footnote{Snyder, J. P., 1987, Map Projections \- A Working Manual, U.S. Geological Survey Prof. Paper 1395.}.
We will mostly be using the \GMTprog{pscoast} command to demonstrate each of the projections.
\GMT\ map projections are grouped into four categories depending on the
nature of the projection.  The groups are

\begin{enumerate}
\item Conic map projections
\item Azimuthal map projections
\item Cylindrical map projections
\item Miscellaneous projections
\end{enumerate}

Because $x$ and $y$ are coupled we can only specify one plot-dimensional scale, typically
a map \emph{scale} (for lower-case map projection code) or a map \emph{width} (for upper-case
map projection code).  However, in some cases it would be more
practical to specify map \emph{height} instead of \emph{width}, while in other situations it would be nice
to set either the \emph{shortest} or \emph{longest} map dimension.  Users may select
these alternatives by appending a character code to their map dimension.  To specify map \emph{height},
append \textbf{h} to the given dimension; to select the minimum map dimension, append \textbf{-}, whereas you may
append \textbf{+} to select the maximum map dimension.  Without the modifier the map width is
selected by default.

In \GMT\ version 4.3.0 we noticed we ran out of the alphabet for 1-letter (and sometimes 2-letter) projection codes. To allow more flexibility, and to make it easier to remember the codes, we implemented the option to use the abbreviations used by the \progname{Proj4} mapping package. Since some of the \GMT\ projections are not in \progname{Proj4}, we invented some of our own as well. For a full list of both the old 1- and 2-letter codes, as well as the \progname{Proj4}-equivalents see the quick reference cards in Section~\ref{sec:purpose}. For example, \Opt{JM15c} and \Opt{JMerc/15c} have the same meaning.

\input{GMT_Chapter_6.1}
\input{GMT_Chapter_6.2}
\input{GMT_Chapter_6.3}
\input{GMT_Chapter_6.4}

%------------------------------------------
%	$Id: GMT_Chapter_6.tex,v 1.19 2008-04-30 03:53:30 remko Exp $
%
%	The GMT Documentation Project
%	Copyright 2000-2008.
%	Paul Wessel and Walter H. F. Smith
%------------------------------------------
%
\chapter{\gmt\ Map Projections}
\label{ch:6}

\GMT\ implements more than 30 different projections.  They all project the input coordinates
longitude and latitude to positions on a map.  In general, $x' = f(x,y,z)$ and $y' = g(x,y,z)$, where
$z$ is implicitly given as the radial vector length to the $(x,y)$ point on the chosen ellipsoid.  The functions $f$ and $g$ can be
quite nasty and we will refrain from presenting details in this document.  The interested read is referred to
{\it Snyder} [1987]\footnote{Snyder, J. P., 1987, Map Projections \- A Working Manual, U.S. Geological Survey Prof. Paper 1395.}.
We will mostly be using the \GMTprog{pscoast} command to demonstrate each of the projections.
\GMT\ map projections are grouped into four categories depending on the
nature of the projection.  The groups are

\begin{enumerate}
\item Conic map projections
\item Azimuthal map projections
\item Cylindrical map projections
\item Miscellaneous projections
\end{enumerate}

Because $x$ and $y$ are coupled we can only specify one plot-dimensional scale, typically
a map \emph{scale} (for lower-case map projection code) or a map \emph{width} (for upper-case
map projection code).  However, in some cases it would be more
practical to specify map \emph{height} instead of \emph{width}, while in other situations it would be nice
to set either the \emph{shortest} or \emph{longest} map dimension.  Users may select
these alternatives by appending a character code to their map dimension.  To specify map \emph{height},
append \textbf{h} to the given dimension; to select the minimum map dimension, append \textbf{-}, whereas you may
append \textbf{+} to select the maximum map dimension.  Without the modifier the map width is
selected by default.

In \GMT\ version 4.3.0 we noticed we ran out of the alphabet for 1-letter (and sometimes 2-letter) projection codes. To allow more flexibility, and to make it easier to remember the codes, we implemented the option to use the abbreviations used by the \progname{Proj4} mapping package. Since some of the \GMT\ projections are not in \progname{Proj4}, we invented some of our own as well. For a full list of both the old 1- and 2-letter codes, as well as the \progname{Proj4}-equivalents see the quick reference cards in Section~\ref{sec:purpose}. For example, \Opt{JM15c} and \Opt{JMerc/15c} have the same meaning.

\input{GMT_Chapter_6.1}
\input{GMT_Chapter_6.2}
\input{GMT_Chapter_6.3}
\input{GMT_Chapter_6.4}

%------------------------------------------
%	$Id: GMT_Chapter_6.tex,v 1.19 2008-04-30 03:53:30 remko Exp $
%
%	The GMT Documentation Project
%	Copyright 2000-2008.
%	Paul Wessel and Walter H. F. Smith
%------------------------------------------
%
\chapter{\gmt\ Map Projections}
\label{ch:6}

\GMT\ implements more than 30 different projections.  They all project the input coordinates
longitude and latitude to positions on a map.  In general, $x' = f(x,y,z)$ and $y' = g(x,y,z)$, where
$z$ is implicitly given as the radial vector length to the $(x,y)$ point on the chosen ellipsoid.  The functions $f$ and $g$ can be
quite nasty and we will refrain from presenting details in this document.  The interested read is referred to
{\it Snyder} [1987]\footnote{Snyder, J. P., 1987, Map Projections \- A Working Manual, U.S. Geological Survey Prof. Paper 1395.}.
We will mostly be using the \GMTprog{pscoast} command to demonstrate each of the projections.
\GMT\ map projections are grouped into four categories depending on the
nature of the projection.  The groups are

\begin{enumerate}
\item Conic map projections
\item Azimuthal map projections
\item Cylindrical map projections
\item Miscellaneous projections
\end{enumerate}

Because $x$ and $y$ are coupled we can only specify one plot-dimensional scale, typically
a map \emph{scale} (for lower-case map projection code) or a map \emph{width} (for upper-case
map projection code).  However, in some cases it would be more
practical to specify map \emph{height} instead of \emph{width}, while in other situations it would be nice
to set either the \emph{shortest} or \emph{longest} map dimension.  Users may select
these alternatives by appending a character code to their map dimension.  To specify map \emph{height},
append \textbf{h} to the given dimension; to select the minimum map dimension, append \textbf{-}, whereas you may
append \textbf{+} to select the maximum map dimension.  Without the modifier the map width is
selected by default.

In \GMT\ version 4.3.0 we noticed we ran out of the alphabet for 1-letter (and sometimes 2-letter) projection codes. To allow more flexibility, and to make it easier to remember the codes, we implemented the option to use the abbreviations used by the \progname{Proj4} mapping package. Since some of the \GMT\ projections are not in \progname{Proj4}, we invented some of our own as well. For a full list of both the old 1- and 2-letter codes, as well as the \progname{Proj4}-equivalents see the quick reference cards in Section~\ref{sec:purpose}. For example, \Opt{JM15c} and \Opt{JMerc/15c} have the same meaning.

\input{GMT_Chapter_6.1}
\input{GMT_Chapter_6.2}
\input{GMT_Chapter_6.3}
\input{GMT_Chapter_6.4}


%------------------------------------------
%	$Id: GMT_Chapter_6.tex,v 1.19 2008-04-30 03:53:30 remko Exp $
%
%	The GMT Documentation Project
%	Copyright 2000-2008.
%	Paul Wessel and Walter H. F. Smith
%------------------------------------------
%
\chapter{\gmt\ Map Projections}
\label{ch:6}

\GMT\ implements more than 30 different projections.  They all project the input coordinates
longitude and latitude to positions on a map.  In general, $x' = f(x,y,z)$ and $y' = g(x,y,z)$, where
$z$ is implicitly given as the radial vector length to the $(x,y)$ point on the chosen ellipsoid.  The functions $f$ and $g$ can be
quite nasty and we will refrain from presenting details in this document.  The interested read is referred to
{\it Snyder} [1987]\footnote{Snyder, J. P., 1987, Map Projections \- A Working Manual, U.S. Geological Survey Prof. Paper 1395.}.
We will mostly be using the \GMTprog{pscoast} command to demonstrate each of the projections.
\GMT\ map projections are grouped into four categories depending on the
nature of the projection.  The groups are

\begin{enumerate}
\item Conic map projections
\item Azimuthal map projections
\item Cylindrical map projections
\item Miscellaneous projections
\end{enumerate}

Because $x$ and $y$ are coupled we can only specify one plot-dimensional scale, typically
a map \emph{scale} (for lower-case map projection code) or a map \emph{width} (for upper-case
map projection code).  However, in some cases it would be more
practical to specify map \emph{height} instead of \emph{width}, while in other situations it would be nice
to set either the \emph{shortest} or \emph{longest} map dimension.  Users may select
these alternatives by appending a character code to their map dimension.  To specify map \emph{height},
append \textbf{h} to the given dimension; to select the minimum map dimension, append \textbf{-}, whereas you may
append \textbf{+} to select the maximum map dimension.  Without the modifier the map width is
selected by default.

In \GMT\ version 4.3.0 we noticed we ran out of the alphabet for 1-letter (and sometimes 2-letter) projection codes. To allow more flexibility, and to make it easier to remember the codes, we implemented the option to use the abbreviations used by the \progname{Proj4} mapping package. Since some of the \GMT\ projections are not in \progname{Proj4}, we invented some of our own as well. For a full list of both the old 1- and 2-letter codes, as well as the \progname{Proj4}-equivalents see the quick reference cards in Section~\ref{sec:purpose}. For example, \Opt{JM15c} and \Opt{JMerc/15c} have the same meaning.

%------------------------------------------
%	$Id: GMT_Chapter_6.tex,v 1.19 2008-04-30 03:53:30 remko Exp $
%
%	The GMT Documentation Project
%	Copyright 2000-2008.
%	Paul Wessel and Walter H. F. Smith
%------------------------------------------
%
\chapter{\gmt\ Map Projections}
\label{ch:6}

\GMT\ implements more than 30 different projections.  They all project the input coordinates
longitude and latitude to positions on a map.  In general, $x' = f(x,y,z)$ and $y' = g(x,y,z)$, where
$z$ is implicitly given as the radial vector length to the $(x,y)$ point on the chosen ellipsoid.  The functions $f$ and $g$ can be
quite nasty and we will refrain from presenting details in this document.  The interested read is referred to
{\it Snyder} [1987]\footnote{Snyder, J. P., 1987, Map Projections \- A Working Manual, U.S. Geological Survey Prof. Paper 1395.}.
We will mostly be using the \GMTprog{pscoast} command to demonstrate each of the projections.
\GMT\ map projections are grouped into four categories depending on the
nature of the projection.  The groups are

\begin{enumerate}
\item Conic map projections
\item Azimuthal map projections
\item Cylindrical map projections
\item Miscellaneous projections
\end{enumerate}

Because $x$ and $y$ are coupled we can only specify one plot-dimensional scale, typically
a map \emph{scale} (for lower-case map projection code) or a map \emph{width} (for upper-case
map projection code).  However, in some cases it would be more
practical to specify map \emph{height} instead of \emph{width}, while in other situations it would be nice
to set either the \emph{shortest} or \emph{longest} map dimension.  Users may select
these alternatives by appending a character code to their map dimension.  To specify map \emph{height},
append \textbf{h} to the given dimension; to select the minimum map dimension, append \textbf{-}, whereas you may
append \textbf{+} to select the maximum map dimension.  Without the modifier the map width is
selected by default.

In \GMT\ version 4.3.0 we noticed we ran out of the alphabet for 1-letter (and sometimes 2-letter) projection codes. To allow more flexibility, and to make it easier to remember the codes, we implemented the option to use the abbreviations used by the \progname{Proj4} mapping package. Since some of the \GMT\ projections are not in \progname{Proj4}, we invented some of our own as well. For a full list of both the old 1- and 2-letter codes, as well as the \progname{Proj4}-equivalents see the quick reference cards in Section~\ref{sec:purpose}. For example, \Opt{JM15c} and \Opt{JMerc/15c} have the same meaning.

\input{GMT_Chapter_6.1}
\input{GMT_Chapter_6.2}
\input{GMT_Chapter_6.3}
\input{GMT_Chapter_6.4}

%------------------------------------------
%	$Id: GMT_Chapter_6.tex,v 1.19 2008-04-30 03:53:30 remko Exp $
%
%	The GMT Documentation Project
%	Copyright 2000-2008.
%	Paul Wessel and Walter H. F. Smith
%------------------------------------------
%
\chapter{\gmt\ Map Projections}
\label{ch:6}

\GMT\ implements more than 30 different projections.  They all project the input coordinates
longitude and latitude to positions on a map.  In general, $x' = f(x,y,z)$ and $y' = g(x,y,z)$, where
$z$ is implicitly given as the radial vector length to the $(x,y)$ point on the chosen ellipsoid.  The functions $f$ and $g$ can be
quite nasty and we will refrain from presenting details in this document.  The interested read is referred to
{\it Snyder} [1987]\footnote{Snyder, J. P., 1987, Map Projections \- A Working Manual, U.S. Geological Survey Prof. Paper 1395.}.
We will mostly be using the \GMTprog{pscoast} command to demonstrate each of the projections.
\GMT\ map projections are grouped into four categories depending on the
nature of the projection.  The groups are

\begin{enumerate}
\item Conic map projections
\item Azimuthal map projections
\item Cylindrical map projections
\item Miscellaneous projections
\end{enumerate}

Because $x$ and $y$ are coupled we can only specify one plot-dimensional scale, typically
a map \emph{scale} (for lower-case map projection code) or a map \emph{width} (for upper-case
map projection code).  However, in some cases it would be more
practical to specify map \emph{height} instead of \emph{width}, while in other situations it would be nice
to set either the \emph{shortest} or \emph{longest} map dimension.  Users may select
these alternatives by appending a character code to their map dimension.  To specify map \emph{height},
append \textbf{h} to the given dimension; to select the minimum map dimension, append \textbf{-}, whereas you may
append \textbf{+} to select the maximum map dimension.  Without the modifier the map width is
selected by default.

In \GMT\ version 4.3.0 we noticed we ran out of the alphabet for 1-letter (and sometimes 2-letter) projection codes. To allow more flexibility, and to make it easier to remember the codes, we implemented the option to use the abbreviations used by the \progname{Proj4} mapping package. Since some of the \GMT\ projections are not in \progname{Proj4}, we invented some of our own as well. For a full list of both the old 1- and 2-letter codes, as well as the \progname{Proj4}-equivalents see the quick reference cards in Section~\ref{sec:purpose}. For example, \Opt{JM15c} and \Opt{JMerc/15c} have the same meaning.

\input{GMT_Chapter_6.1}
\input{GMT_Chapter_6.2}
\input{GMT_Chapter_6.3}
\input{GMT_Chapter_6.4}

%------------------------------------------
%	$Id: GMT_Chapter_6.tex,v 1.19 2008-04-30 03:53:30 remko Exp $
%
%	The GMT Documentation Project
%	Copyright 2000-2008.
%	Paul Wessel and Walter H. F. Smith
%------------------------------------------
%
\chapter{\gmt\ Map Projections}
\label{ch:6}

\GMT\ implements more than 30 different projections.  They all project the input coordinates
longitude and latitude to positions on a map.  In general, $x' = f(x,y,z)$ and $y' = g(x,y,z)$, where
$z$ is implicitly given as the radial vector length to the $(x,y)$ point on the chosen ellipsoid.  The functions $f$ and $g$ can be
quite nasty and we will refrain from presenting details in this document.  The interested read is referred to
{\it Snyder} [1987]\footnote{Snyder, J. P., 1987, Map Projections \- A Working Manual, U.S. Geological Survey Prof. Paper 1395.}.
We will mostly be using the \GMTprog{pscoast} command to demonstrate each of the projections.
\GMT\ map projections are grouped into four categories depending on the
nature of the projection.  The groups are

\begin{enumerate}
\item Conic map projections
\item Azimuthal map projections
\item Cylindrical map projections
\item Miscellaneous projections
\end{enumerate}

Because $x$ and $y$ are coupled we can only specify one plot-dimensional scale, typically
a map \emph{scale} (for lower-case map projection code) or a map \emph{width} (for upper-case
map projection code).  However, in some cases it would be more
practical to specify map \emph{height} instead of \emph{width}, while in other situations it would be nice
to set either the \emph{shortest} or \emph{longest} map dimension.  Users may select
these alternatives by appending a character code to their map dimension.  To specify map \emph{height},
append \textbf{h} to the given dimension; to select the minimum map dimension, append \textbf{-}, whereas you may
append \textbf{+} to select the maximum map dimension.  Without the modifier the map width is
selected by default.

In \GMT\ version 4.3.0 we noticed we ran out of the alphabet for 1-letter (and sometimes 2-letter) projection codes. To allow more flexibility, and to make it easier to remember the codes, we implemented the option to use the abbreviations used by the \progname{Proj4} mapping package. Since some of the \GMT\ projections are not in \progname{Proj4}, we invented some of our own as well. For a full list of both the old 1- and 2-letter codes, as well as the \progname{Proj4}-equivalents see the quick reference cards in Section~\ref{sec:purpose}. For example, \Opt{JM15c} and \Opt{JMerc/15c} have the same meaning.

\input{GMT_Chapter_6.1}
\input{GMT_Chapter_6.2}
\input{GMT_Chapter_6.3}
\input{GMT_Chapter_6.4}

%------------------------------------------
%	$Id: GMT_Chapter_6.tex,v 1.19 2008-04-30 03:53:30 remko Exp $
%
%	The GMT Documentation Project
%	Copyright 2000-2008.
%	Paul Wessel and Walter H. F. Smith
%------------------------------------------
%
\chapter{\gmt\ Map Projections}
\label{ch:6}

\GMT\ implements more than 30 different projections.  They all project the input coordinates
longitude and latitude to positions on a map.  In general, $x' = f(x,y,z)$ and $y' = g(x,y,z)$, where
$z$ is implicitly given as the radial vector length to the $(x,y)$ point on the chosen ellipsoid.  The functions $f$ and $g$ can be
quite nasty and we will refrain from presenting details in this document.  The interested read is referred to
{\it Snyder} [1987]\footnote{Snyder, J. P., 1987, Map Projections \- A Working Manual, U.S. Geological Survey Prof. Paper 1395.}.
We will mostly be using the \GMTprog{pscoast} command to demonstrate each of the projections.
\GMT\ map projections are grouped into four categories depending on the
nature of the projection.  The groups are

\begin{enumerate}
\item Conic map projections
\item Azimuthal map projections
\item Cylindrical map projections
\item Miscellaneous projections
\end{enumerate}

Because $x$ and $y$ are coupled we can only specify one plot-dimensional scale, typically
a map \emph{scale} (for lower-case map projection code) or a map \emph{width} (for upper-case
map projection code).  However, in some cases it would be more
practical to specify map \emph{height} instead of \emph{width}, while in other situations it would be nice
to set either the \emph{shortest} or \emph{longest} map dimension.  Users may select
these alternatives by appending a character code to their map dimension.  To specify map \emph{height},
append \textbf{h} to the given dimension; to select the minimum map dimension, append \textbf{-}, whereas you may
append \textbf{+} to select the maximum map dimension.  Without the modifier the map width is
selected by default.

In \GMT\ version 4.3.0 we noticed we ran out of the alphabet for 1-letter (and sometimes 2-letter) projection codes. To allow more flexibility, and to make it easier to remember the codes, we implemented the option to use the abbreviations used by the \progname{Proj4} mapping package. Since some of the \GMT\ projections are not in \progname{Proj4}, we invented some of our own as well. For a full list of both the old 1- and 2-letter codes, as well as the \progname{Proj4}-equivalents see the quick reference cards in Section~\ref{sec:purpose}. For example, \Opt{JM15c} and \Opt{JMerc/15c} have the same meaning.

\input{GMT_Chapter_6.1}
\input{GMT_Chapter_6.2}
\input{GMT_Chapter_6.3}
\input{GMT_Chapter_6.4}


%------------------------------------------
%	$Id: GMT_Chapter_6.tex,v 1.19 2008-04-30 03:53:30 remko Exp $
%
%	The GMT Documentation Project
%	Copyright 2000-2008.
%	Paul Wessel and Walter H. F. Smith
%------------------------------------------
%
\chapter{\gmt\ Map Projections}
\label{ch:6}

\GMT\ implements more than 30 different projections.  They all project the input coordinates
longitude and latitude to positions on a map.  In general, $x' = f(x,y,z)$ and $y' = g(x,y,z)$, where
$z$ is implicitly given as the radial vector length to the $(x,y)$ point on the chosen ellipsoid.  The functions $f$ and $g$ can be
quite nasty and we will refrain from presenting details in this document.  The interested read is referred to
{\it Snyder} [1987]\footnote{Snyder, J. P., 1987, Map Projections \- A Working Manual, U.S. Geological Survey Prof. Paper 1395.}.
We will mostly be using the \GMTprog{pscoast} command to demonstrate each of the projections.
\GMT\ map projections are grouped into four categories depending on the
nature of the projection.  The groups are

\begin{enumerate}
\item Conic map projections
\item Azimuthal map projections
\item Cylindrical map projections
\item Miscellaneous projections
\end{enumerate}

Because $x$ and $y$ are coupled we can only specify one plot-dimensional scale, typically
a map \emph{scale} (for lower-case map projection code) or a map \emph{width} (for upper-case
map projection code).  However, in some cases it would be more
practical to specify map \emph{height} instead of \emph{width}, while in other situations it would be nice
to set either the \emph{shortest} or \emph{longest} map dimension.  Users may select
these alternatives by appending a character code to their map dimension.  To specify map \emph{height},
append \textbf{h} to the given dimension; to select the minimum map dimension, append \textbf{-}, whereas you may
append \textbf{+} to select the maximum map dimension.  Without the modifier the map width is
selected by default.

In \GMT\ version 4.3.0 we noticed we ran out of the alphabet for 1-letter (and sometimes 2-letter) projection codes. To allow more flexibility, and to make it easier to remember the codes, we implemented the option to use the abbreviations used by the \progname{Proj4} mapping package. Since some of the \GMT\ projections are not in \progname{Proj4}, we invented some of our own as well. For a full list of both the old 1- and 2-letter codes, as well as the \progname{Proj4}-equivalents see the quick reference cards in Section~\ref{sec:purpose}. For example, \Opt{JM15c} and \Opt{JMerc/15c} have the same meaning.

%------------------------------------------
%	$Id: GMT_Chapter_6.tex,v 1.19 2008-04-30 03:53:30 remko Exp $
%
%	The GMT Documentation Project
%	Copyright 2000-2008.
%	Paul Wessel and Walter H. F. Smith
%------------------------------------------
%
\chapter{\gmt\ Map Projections}
\label{ch:6}

\GMT\ implements more than 30 different projections.  They all project the input coordinates
longitude and latitude to positions on a map.  In general, $x' = f(x,y,z)$ and $y' = g(x,y,z)$, where
$z$ is implicitly given as the radial vector length to the $(x,y)$ point on the chosen ellipsoid.  The functions $f$ and $g$ can be
quite nasty and we will refrain from presenting details in this document.  The interested read is referred to
{\it Snyder} [1987]\footnote{Snyder, J. P., 1987, Map Projections \- A Working Manual, U.S. Geological Survey Prof. Paper 1395.}.
We will mostly be using the \GMTprog{pscoast} command to demonstrate each of the projections.
\GMT\ map projections are grouped into four categories depending on the
nature of the projection.  The groups are

\begin{enumerate}
\item Conic map projections
\item Azimuthal map projections
\item Cylindrical map projections
\item Miscellaneous projections
\end{enumerate}

Because $x$ and $y$ are coupled we can only specify one plot-dimensional scale, typically
a map \emph{scale} (for lower-case map projection code) or a map \emph{width} (for upper-case
map projection code).  However, in some cases it would be more
practical to specify map \emph{height} instead of \emph{width}, while in other situations it would be nice
to set either the \emph{shortest} or \emph{longest} map dimension.  Users may select
these alternatives by appending a character code to their map dimension.  To specify map \emph{height},
append \textbf{h} to the given dimension; to select the minimum map dimension, append \textbf{-}, whereas you may
append \textbf{+} to select the maximum map dimension.  Without the modifier the map width is
selected by default.

In \GMT\ version 4.3.0 we noticed we ran out of the alphabet for 1-letter (and sometimes 2-letter) projection codes. To allow more flexibility, and to make it easier to remember the codes, we implemented the option to use the abbreviations used by the \progname{Proj4} mapping package. Since some of the \GMT\ projections are not in \progname{Proj4}, we invented some of our own as well. For a full list of both the old 1- and 2-letter codes, as well as the \progname{Proj4}-equivalents see the quick reference cards in Section~\ref{sec:purpose}. For example, \Opt{JM15c} and \Opt{JMerc/15c} have the same meaning.

\input{GMT_Chapter_6.1}
\input{GMT_Chapter_6.2}
\input{GMT_Chapter_6.3}
\input{GMT_Chapter_6.4}

%------------------------------------------
%	$Id: GMT_Chapter_6.tex,v 1.19 2008-04-30 03:53:30 remko Exp $
%
%	The GMT Documentation Project
%	Copyright 2000-2008.
%	Paul Wessel and Walter H. F. Smith
%------------------------------------------
%
\chapter{\gmt\ Map Projections}
\label{ch:6}

\GMT\ implements more than 30 different projections.  They all project the input coordinates
longitude and latitude to positions on a map.  In general, $x' = f(x,y,z)$ and $y' = g(x,y,z)$, where
$z$ is implicitly given as the radial vector length to the $(x,y)$ point on the chosen ellipsoid.  The functions $f$ and $g$ can be
quite nasty and we will refrain from presenting details in this document.  The interested read is referred to
{\it Snyder} [1987]\footnote{Snyder, J. P., 1987, Map Projections \- A Working Manual, U.S. Geological Survey Prof. Paper 1395.}.
We will mostly be using the \GMTprog{pscoast} command to demonstrate each of the projections.
\GMT\ map projections are grouped into four categories depending on the
nature of the projection.  The groups are

\begin{enumerate}
\item Conic map projections
\item Azimuthal map projections
\item Cylindrical map projections
\item Miscellaneous projections
\end{enumerate}

Because $x$ and $y$ are coupled we can only specify one plot-dimensional scale, typically
a map \emph{scale} (for lower-case map projection code) or a map \emph{width} (for upper-case
map projection code).  However, in some cases it would be more
practical to specify map \emph{height} instead of \emph{width}, while in other situations it would be nice
to set either the \emph{shortest} or \emph{longest} map dimension.  Users may select
these alternatives by appending a character code to their map dimension.  To specify map \emph{height},
append \textbf{h} to the given dimension; to select the minimum map dimension, append \textbf{-}, whereas you may
append \textbf{+} to select the maximum map dimension.  Without the modifier the map width is
selected by default.

In \GMT\ version 4.3.0 we noticed we ran out of the alphabet for 1-letter (and sometimes 2-letter) projection codes. To allow more flexibility, and to make it easier to remember the codes, we implemented the option to use the abbreviations used by the \progname{Proj4} mapping package. Since some of the \GMT\ projections are not in \progname{Proj4}, we invented some of our own as well. For a full list of both the old 1- and 2-letter codes, as well as the \progname{Proj4}-equivalents see the quick reference cards in Section~\ref{sec:purpose}. For example, \Opt{JM15c} and \Opt{JMerc/15c} have the same meaning.

\input{GMT_Chapter_6.1}
\input{GMT_Chapter_6.2}
\input{GMT_Chapter_6.3}
\input{GMT_Chapter_6.4}

%------------------------------------------
%	$Id: GMT_Chapter_6.tex,v 1.19 2008-04-30 03:53:30 remko Exp $
%
%	The GMT Documentation Project
%	Copyright 2000-2008.
%	Paul Wessel and Walter H. F. Smith
%------------------------------------------
%
\chapter{\gmt\ Map Projections}
\label{ch:6}

\GMT\ implements more than 30 different projections.  They all project the input coordinates
longitude and latitude to positions on a map.  In general, $x' = f(x,y,z)$ and $y' = g(x,y,z)$, where
$z$ is implicitly given as the radial vector length to the $(x,y)$ point on the chosen ellipsoid.  The functions $f$ and $g$ can be
quite nasty and we will refrain from presenting details in this document.  The interested read is referred to
{\it Snyder} [1987]\footnote{Snyder, J. P., 1987, Map Projections \- A Working Manual, U.S. Geological Survey Prof. Paper 1395.}.
We will mostly be using the \GMTprog{pscoast} command to demonstrate each of the projections.
\GMT\ map projections are grouped into four categories depending on the
nature of the projection.  The groups are

\begin{enumerate}
\item Conic map projections
\item Azimuthal map projections
\item Cylindrical map projections
\item Miscellaneous projections
\end{enumerate}

Because $x$ and $y$ are coupled we can only specify one plot-dimensional scale, typically
a map \emph{scale} (for lower-case map projection code) or a map \emph{width} (for upper-case
map projection code).  However, in some cases it would be more
practical to specify map \emph{height} instead of \emph{width}, while in other situations it would be nice
to set either the \emph{shortest} or \emph{longest} map dimension.  Users may select
these alternatives by appending a character code to their map dimension.  To specify map \emph{height},
append \textbf{h} to the given dimension; to select the minimum map dimension, append \textbf{-}, whereas you may
append \textbf{+} to select the maximum map dimension.  Without the modifier the map width is
selected by default.

In \GMT\ version 4.3.0 we noticed we ran out of the alphabet for 1-letter (and sometimes 2-letter) projection codes. To allow more flexibility, and to make it easier to remember the codes, we implemented the option to use the abbreviations used by the \progname{Proj4} mapping package. Since some of the \GMT\ projections are not in \progname{Proj4}, we invented some of our own as well. For a full list of both the old 1- and 2-letter codes, as well as the \progname{Proj4}-equivalents see the quick reference cards in Section~\ref{sec:purpose}. For example, \Opt{JM15c} and \Opt{JMerc/15c} have the same meaning.

\input{GMT_Chapter_6.1}
\input{GMT_Chapter_6.2}
\input{GMT_Chapter_6.3}
\input{GMT_Chapter_6.4}

%------------------------------------------
%	$Id: GMT_Chapter_6.tex,v 1.19 2008-04-30 03:53:30 remko Exp $
%
%	The GMT Documentation Project
%	Copyright 2000-2008.
%	Paul Wessel and Walter H. F. Smith
%------------------------------------------
%
\chapter{\gmt\ Map Projections}
\label{ch:6}

\GMT\ implements more than 30 different projections.  They all project the input coordinates
longitude and latitude to positions on a map.  In general, $x' = f(x,y,z)$ and $y' = g(x,y,z)$, where
$z$ is implicitly given as the radial vector length to the $(x,y)$ point on the chosen ellipsoid.  The functions $f$ and $g$ can be
quite nasty and we will refrain from presenting details in this document.  The interested read is referred to
{\it Snyder} [1987]\footnote{Snyder, J. P., 1987, Map Projections \- A Working Manual, U.S. Geological Survey Prof. Paper 1395.}.
We will mostly be using the \GMTprog{pscoast} command to demonstrate each of the projections.
\GMT\ map projections are grouped into four categories depending on the
nature of the projection.  The groups are

\begin{enumerate}
\item Conic map projections
\item Azimuthal map projections
\item Cylindrical map projections
\item Miscellaneous projections
\end{enumerate}

Because $x$ and $y$ are coupled we can only specify one plot-dimensional scale, typically
a map \emph{scale} (for lower-case map projection code) or a map \emph{width} (for upper-case
map projection code).  However, in some cases it would be more
practical to specify map \emph{height} instead of \emph{width}, while in other situations it would be nice
to set either the \emph{shortest} or \emph{longest} map dimension.  Users may select
these alternatives by appending a character code to their map dimension.  To specify map \emph{height},
append \textbf{h} to the given dimension; to select the minimum map dimension, append \textbf{-}, whereas you may
append \textbf{+} to select the maximum map dimension.  Without the modifier the map width is
selected by default.

In \GMT\ version 4.3.0 we noticed we ran out of the alphabet for 1-letter (and sometimes 2-letter) projection codes. To allow more flexibility, and to make it easier to remember the codes, we implemented the option to use the abbreviations used by the \progname{Proj4} mapping package. Since some of the \GMT\ projections are not in \progname{Proj4}, we invented some of our own as well. For a full list of both the old 1- and 2-letter codes, as well as the \progname{Proj4}-equivalents see the quick reference cards in Section~\ref{sec:purpose}. For example, \Opt{JM15c} and \Opt{JMerc/15c} have the same meaning.

\input{GMT_Chapter_6.1}
\input{GMT_Chapter_6.2}
\input{GMT_Chapter_6.3}
\input{GMT_Chapter_6.4}


%------------------------------------------
%	$Id: GMT_Chapter_6.tex,v 1.19 2008-04-30 03:53:30 remko Exp $
%
%	The GMT Documentation Project
%	Copyright 2000-2008.
%	Paul Wessel and Walter H. F. Smith
%------------------------------------------
%
\chapter{\gmt\ Map Projections}
\label{ch:6}

\GMT\ implements more than 30 different projections.  They all project the input coordinates
longitude and latitude to positions on a map.  In general, $x' = f(x,y,z)$ and $y' = g(x,y,z)$, where
$z$ is implicitly given as the radial vector length to the $(x,y)$ point on the chosen ellipsoid.  The functions $f$ and $g$ can be
quite nasty and we will refrain from presenting details in this document.  The interested read is referred to
{\it Snyder} [1987]\footnote{Snyder, J. P., 1987, Map Projections \- A Working Manual, U.S. Geological Survey Prof. Paper 1395.}.
We will mostly be using the \GMTprog{pscoast} command to demonstrate each of the projections.
\GMT\ map projections are grouped into four categories depending on the
nature of the projection.  The groups are

\begin{enumerate}
\item Conic map projections
\item Azimuthal map projections
\item Cylindrical map projections
\item Miscellaneous projections
\end{enumerate}

Because $x$ and $y$ are coupled we can only specify one plot-dimensional scale, typically
a map \emph{scale} (for lower-case map projection code) or a map \emph{width} (for upper-case
map projection code).  However, in some cases it would be more
practical to specify map \emph{height} instead of \emph{width}, while in other situations it would be nice
to set either the \emph{shortest} or \emph{longest} map dimension.  Users may select
these alternatives by appending a character code to their map dimension.  To specify map \emph{height},
append \textbf{h} to the given dimension; to select the minimum map dimension, append \textbf{-}, whereas you may
append \textbf{+} to select the maximum map dimension.  Without the modifier the map width is
selected by default.

In \GMT\ version 4.3.0 we noticed we ran out of the alphabet for 1-letter (and sometimes 2-letter) projection codes. To allow more flexibility, and to make it easier to remember the codes, we implemented the option to use the abbreviations used by the \progname{Proj4} mapping package. Since some of the \GMT\ projections are not in \progname{Proj4}, we invented some of our own as well. For a full list of both the old 1- and 2-letter codes, as well as the \progname{Proj4}-equivalents see the quick reference cards in Section~\ref{sec:purpose}. For example, \Opt{JM15c} and \Opt{JMerc/15c} have the same meaning.

%------------------------------------------
%	$Id: GMT_Chapter_6.tex,v 1.19 2008-04-30 03:53:30 remko Exp $
%
%	The GMT Documentation Project
%	Copyright 2000-2008.
%	Paul Wessel and Walter H. F. Smith
%------------------------------------------
%
\chapter{\gmt\ Map Projections}
\label{ch:6}

\GMT\ implements more than 30 different projections.  They all project the input coordinates
longitude and latitude to positions on a map.  In general, $x' = f(x,y,z)$ and $y' = g(x,y,z)$, where
$z$ is implicitly given as the radial vector length to the $(x,y)$ point on the chosen ellipsoid.  The functions $f$ and $g$ can be
quite nasty and we will refrain from presenting details in this document.  The interested read is referred to
{\it Snyder} [1987]\footnote{Snyder, J. P., 1987, Map Projections \- A Working Manual, U.S. Geological Survey Prof. Paper 1395.}.
We will mostly be using the \GMTprog{pscoast} command to demonstrate each of the projections.
\GMT\ map projections are grouped into four categories depending on the
nature of the projection.  The groups are

\begin{enumerate}
\item Conic map projections
\item Azimuthal map projections
\item Cylindrical map projections
\item Miscellaneous projections
\end{enumerate}

Because $x$ and $y$ are coupled we can only specify one plot-dimensional scale, typically
a map \emph{scale} (for lower-case map projection code) or a map \emph{width} (for upper-case
map projection code).  However, in some cases it would be more
practical to specify map \emph{height} instead of \emph{width}, while in other situations it would be nice
to set either the \emph{shortest} or \emph{longest} map dimension.  Users may select
these alternatives by appending a character code to their map dimension.  To specify map \emph{height},
append \textbf{h} to the given dimension; to select the minimum map dimension, append \textbf{-}, whereas you may
append \textbf{+} to select the maximum map dimension.  Without the modifier the map width is
selected by default.

In \GMT\ version 4.3.0 we noticed we ran out of the alphabet for 1-letter (and sometimes 2-letter) projection codes. To allow more flexibility, and to make it easier to remember the codes, we implemented the option to use the abbreviations used by the \progname{Proj4} mapping package. Since some of the \GMT\ projections are not in \progname{Proj4}, we invented some of our own as well. For a full list of both the old 1- and 2-letter codes, as well as the \progname{Proj4}-equivalents see the quick reference cards in Section~\ref{sec:purpose}. For example, \Opt{JM15c} and \Opt{JMerc/15c} have the same meaning.

\input{GMT_Chapter_6.1}
\input{GMT_Chapter_6.2}
\input{GMT_Chapter_6.3}
\input{GMT_Chapter_6.4}

%------------------------------------------
%	$Id: GMT_Chapter_6.tex,v 1.19 2008-04-30 03:53:30 remko Exp $
%
%	The GMT Documentation Project
%	Copyright 2000-2008.
%	Paul Wessel and Walter H. F. Smith
%------------------------------------------
%
\chapter{\gmt\ Map Projections}
\label{ch:6}

\GMT\ implements more than 30 different projections.  They all project the input coordinates
longitude and latitude to positions on a map.  In general, $x' = f(x,y,z)$ and $y' = g(x,y,z)$, where
$z$ is implicitly given as the radial vector length to the $(x,y)$ point on the chosen ellipsoid.  The functions $f$ and $g$ can be
quite nasty and we will refrain from presenting details in this document.  The interested read is referred to
{\it Snyder} [1987]\footnote{Snyder, J. P., 1987, Map Projections \- A Working Manual, U.S. Geological Survey Prof. Paper 1395.}.
We will mostly be using the \GMTprog{pscoast} command to demonstrate each of the projections.
\GMT\ map projections are grouped into four categories depending on the
nature of the projection.  The groups are

\begin{enumerate}
\item Conic map projections
\item Azimuthal map projections
\item Cylindrical map projections
\item Miscellaneous projections
\end{enumerate}

Because $x$ and $y$ are coupled we can only specify one plot-dimensional scale, typically
a map \emph{scale} (for lower-case map projection code) or a map \emph{width} (for upper-case
map projection code).  However, in some cases it would be more
practical to specify map \emph{height} instead of \emph{width}, while in other situations it would be nice
to set either the \emph{shortest} or \emph{longest} map dimension.  Users may select
these alternatives by appending a character code to their map dimension.  To specify map \emph{height},
append \textbf{h} to the given dimension; to select the minimum map dimension, append \textbf{-}, whereas you may
append \textbf{+} to select the maximum map dimension.  Without the modifier the map width is
selected by default.

In \GMT\ version 4.3.0 we noticed we ran out of the alphabet for 1-letter (and sometimes 2-letter) projection codes. To allow more flexibility, and to make it easier to remember the codes, we implemented the option to use the abbreviations used by the \progname{Proj4} mapping package. Since some of the \GMT\ projections are not in \progname{Proj4}, we invented some of our own as well. For a full list of both the old 1- and 2-letter codes, as well as the \progname{Proj4}-equivalents see the quick reference cards in Section~\ref{sec:purpose}. For example, \Opt{JM15c} and \Opt{JMerc/15c} have the same meaning.

\input{GMT_Chapter_6.1}
\input{GMT_Chapter_6.2}
\input{GMT_Chapter_6.3}
\input{GMT_Chapter_6.4}

%------------------------------------------
%	$Id: GMT_Chapter_6.tex,v 1.19 2008-04-30 03:53:30 remko Exp $
%
%	The GMT Documentation Project
%	Copyright 2000-2008.
%	Paul Wessel and Walter H. F. Smith
%------------------------------------------
%
\chapter{\gmt\ Map Projections}
\label{ch:6}

\GMT\ implements more than 30 different projections.  They all project the input coordinates
longitude and latitude to positions on a map.  In general, $x' = f(x,y,z)$ and $y' = g(x,y,z)$, where
$z$ is implicitly given as the radial vector length to the $(x,y)$ point on the chosen ellipsoid.  The functions $f$ and $g$ can be
quite nasty and we will refrain from presenting details in this document.  The interested read is referred to
{\it Snyder} [1987]\footnote{Snyder, J. P., 1987, Map Projections \- A Working Manual, U.S. Geological Survey Prof. Paper 1395.}.
We will mostly be using the \GMTprog{pscoast} command to demonstrate each of the projections.
\GMT\ map projections are grouped into four categories depending on the
nature of the projection.  The groups are

\begin{enumerate}
\item Conic map projections
\item Azimuthal map projections
\item Cylindrical map projections
\item Miscellaneous projections
\end{enumerate}

Because $x$ and $y$ are coupled we can only specify one plot-dimensional scale, typically
a map \emph{scale} (for lower-case map projection code) or a map \emph{width} (for upper-case
map projection code).  However, in some cases it would be more
practical to specify map \emph{height} instead of \emph{width}, while in other situations it would be nice
to set either the \emph{shortest} or \emph{longest} map dimension.  Users may select
these alternatives by appending a character code to their map dimension.  To specify map \emph{height},
append \textbf{h} to the given dimension; to select the minimum map dimension, append \textbf{-}, whereas you may
append \textbf{+} to select the maximum map dimension.  Without the modifier the map width is
selected by default.

In \GMT\ version 4.3.0 we noticed we ran out of the alphabet for 1-letter (and sometimes 2-letter) projection codes. To allow more flexibility, and to make it easier to remember the codes, we implemented the option to use the abbreviations used by the \progname{Proj4} mapping package. Since some of the \GMT\ projections are not in \progname{Proj4}, we invented some of our own as well. For a full list of both the old 1- and 2-letter codes, as well as the \progname{Proj4}-equivalents see the quick reference cards in Section~\ref{sec:purpose}. For example, \Opt{JM15c} and \Opt{JMerc/15c} have the same meaning.

\input{GMT_Chapter_6.1}
\input{GMT_Chapter_6.2}
\input{GMT_Chapter_6.3}
\input{GMT_Chapter_6.4}

%------------------------------------------
%	$Id: GMT_Chapter_6.tex,v 1.19 2008-04-30 03:53:30 remko Exp $
%
%	The GMT Documentation Project
%	Copyright 2000-2008.
%	Paul Wessel and Walter H. F. Smith
%------------------------------------------
%
\chapter{\gmt\ Map Projections}
\label{ch:6}

\GMT\ implements more than 30 different projections.  They all project the input coordinates
longitude and latitude to positions on a map.  In general, $x' = f(x,y,z)$ and $y' = g(x,y,z)$, where
$z$ is implicitly given as the radial vector length to the $(x,y)$ point on the chosen ellipsoid.  The functions $f$ and $g$ can be
quite nasty and we will refrain from presenting details in this document.  The interested read is referred to
{\it Snyder} [1987]\footnote{Snyder, J. P., 1987, Map Projections \- A Working Manual, U.S. Geological Survey Prof. Paper 1395.}.
We will mostly be using the \GMTprog{pscoast} command to demonstrate each of the projections.
\GMT\ map projections are grouped into four categories depending on the
nature of the projection.  The groups are

\begin{enumerate}
\item Conic map projections
\item Azimuthal map projections
\item Cylindrical map projections
\item Miscellaneous projections
\end{enumerate}

Because $x$ and $y$ are coupled we can only specify one plot-dimensional scale, typically
a map \emph{scale} (for lower-case map projection code) or a map \emph{width} (for upper-case
map projection code).  However, in some cases it would be more
practical to specify map \emph{height} instead of \emph{width}, while in other situations it would be nice
to set either the \emph{shortest} or \emph{longest} map dimension.  Users may select
these alternatives by appending a character code to their map dimension.  To specify map \emph{height},
append \textbf{h} to the given dimension; to select the minimum map dimension, append \textbf{-}, whereas you may
append \textbf{+} to select the maximum map dimension.  Without the modifier the map width is
selected by default.

In \GMT\ version 4.3.0 we noticed we ran out of the alphabet for 1-letter (and sometimes 2-letter) projection codes. To allow more flexibility, and to make it easier to remember the codes, we implemented the option to use the abbreviations used by the \progname{Proj4} mapping package. Since some of the \GMT\ projections are not in \progname{Proj4}, we invented some of our own as well. For a full list of both the old 1- and 2-letter codes, as well as the \progname{Proj4}-equivalents see the quick reference cards in Section~\ref{sec:purpose}. For example, \Opt{JM15c} and \Opt{JMerc/15c} have the same meaning.

\input{GMT_Chapter_6.1}
\input{GMT_Chapter_6.2}
\input{GMT_Chapter_6.3}
\input{GMT_Chapter_6.4}



%------------------------------------------
%	$Id: GMT_Chapter_6.tex,v 1.19 2008-04-30 03:53:30 remko Exp $
%
%	The GMT Documentation Project
%	Copyright 2000-2008.
%	Paul Wessel and Walter H. F. Smith
%------------------------------------------
%
\chapter{\gmt\ Map Projections}
\label{ch:6}

\GMT\ implements more than 30 different projections.  They all project the input coordinates
longitude and latitude to positions on a map.  In general, $x' = f(x,y,z)$ and $y' = g(x,y,z)$, where
$z$ is implicitly given as the radial vector length to the $(x,y)$ point on the chosen ellipsoid.  The functions $f$ and $g$ can be
quite nasty and we will refrain from presenting details in this document.  The interested read is referred to
{\it Snyder} [1987]\footnote{Snyder, J. P., 1987, Map Projections \- A Working Manual, U.S. Geological Survey Prof. Paper 1395.}.
We will mostly be using the \GMTprog{pscoast} command to demonstrate each of the projections.
\GMT\ map projections are grouped into four categories depending on the
nature of the projection.  The groups are

\begin{enumerate}
\item Conic map projections
\item Azimuthal map projections
\item Cylindrical map projections
\item Miscellaneous projections
\end{enumerate}

Because $x$ and $y$ are coupled we can only specify one plot-dimensional scale, typically
a map \emph{scale} (for lower-case map projection code) or a map \emph{width} (for upper-case
map projection code).  However, in some cases it would be more
practical to specify map \emph{height} instead of \emph{width}, while in other situations it would be nice
to set either the \emph{shortest} or \emph{longest} map dimension.  Users may select
these alternatives by appending a character code to their map dimension.  To specify map \emph{height},
append \textbf{h} to the given dimension; to select the minimum map dimension, append \textbf{-}, whereas you may
append \textbf{+} to select the maximum map dimension.  Without the modifier the map width is
selected by default.

In \GMT\ version 4.3.0 we noticed we ran out of the alphabet for 1-letter (and sometimes 2-letter) projection codes. To allow more flexibility, and to make it easier to remember the codes, we implemented the option to use the abbreviations used by the \progname{Proj4} mapping package. Since some of the \GMT\ projections are not in \progname{Proj4}, we invented some of our own as well. For a full list of both the old 1- and 2-letter codes, as well as the \progname{Proj4}-equivalents see the quick reference cards in Section~\ref{sec:purpose}. For example, \Opt{JM15c} and \Opt{JMerc/15c} have the same meaning.

%------------------------------------------
%	$Id: GMT_Chapter_6.tex,v 1.19 2008-04-30 03:53:30 remko Exp $
%
%	The GMT Documentation Project
%	Copyright 2000-2008.
%	Paul Wessel and Walter H. F. Smith
%------------------------------------------
%
\chapter{\gmt\ Map Projections}
\label{ch:6}

\GMT\ implements more than 30 different projections.  They all project the input coordinates
longitude and latitude to positions on a map.  In general, $x' = f(x,y,z)$ and $y' = g(x,y,z)$, where
$z$ is implicitly given as the radial vector length to the $(x,y)$ point on the chosen ellipsoid.  The functions $f$ and $g$ can be
quite nasty and we will refrain from presenting details in this document.  The interested read is referred to
{\it Snyder} [1987]\footnote{Snyder, J. P., 1987, Map Projections \- A Working Manual, U.S. Geological Survey Prof. Paper 1395.}.
We will mostly be using the \GMTprog{pscoast} command to demonstrate each of the projections.
\GMT\ map projections are grouped into four categories depending on the
nature of the projection.  The groups are

\begin{enumerate}
\item Conic map projections
\item Azimuthal map projections
\item Cylindrical map projections
\item Miscellaneous projections
\end{enumerate}

Because $x$ and $y$ are coupled we can only specify one plot-dimensional scale, typically
a map \emph{scale} (for lower-case map projection code) or a map \emph{width} (for upper-case
map projection code).  However, in some cases it would be more
practical to specify map \emph{height} instead of \emph{width}, while in other situations it would be nice
to set either the \emph{shortest} or \emph{longest} map dimension.  Users may select
these alternatives by appending a character code to their map dimension.  To specify map \emph{height},
append \textbf{h} to the given dimension; to select the minimum map dimension, append \textbf{-}, whereas you may
append \textbf{+} to select the maximum map dimension.  Without the modifier the map width is
selected by default.

In \GMT\ version 4.3.0 we noticed we ran out of the alphabet for 1-letter (and sometimes 2-letter) projection codes. To allow more flexibility, and to make it easier to remember the codes, we implemented the option to use the abbreviations used by the \progname{Proj4} mapping package. Since some of the \GMT\ projections are not in \progname{Proj4}, we invented some of our own as well. For a full list of both the old 1- and 2-letter codes, as well as the \progname{Proj4}-equivalents see the quick reference cards in Section~\ref{sec:purpose}. For example, \Opt{JM15c} and \Opt{JMerc/15c} have the same meaning.

%------------------------------------------
%	$Id: GMT_Chapter_6.tex,v 1.19 2008-04-30 03:53:30 remko Exp $
%
%	The GMT Documentation Project
%	Copyright 2000-2008.
%	Paul Wessel and Walter H. F. Smith
%------------------------------------------
%
\chapter{\gmt\ Map Projections}
\label{ch:6}

\GMT\ implements more than 30 different projections.  They all project the input coordinates
longitude and latitude to positions on a map.  In general, $x' = f(x,y,z)$ and $y' = g(x,y,z)$, where
$z$ is implicitly given as the radial vector length to the $(x,y)$ point on the chosen ellipsoid.  The functions $f$ and $g$ can be
quite nasty and we will refrain from presenting details in this document.  The interested read is referred to
{\it Snyder} [1987]\footnote{Snyder, J. P., 1987, Map Projections \- A Working Manual, U.S. Geological Survey Prof. Paper 1395.}.
We will mostly be using the \GMTprog{pscoast} command to demonstrate each of the projections.
\GMT\ map projections are grouped into four categories depending on the
nature of the projection.  The groups are

\begin{enumerate}
\item Conic map projections
\item Azimuthal map projections
\item Cylindrical map projections
\item Miscellaneous projections
\end{enumerate}

Because $x$ and $y$ are coupled we can only specify one plot-dimensional scale, typically
a map \emph{scale} (for lower-case map projection code) or a map \emph{width} (for upper-case
map projection code).  However, in some cases it would be more
practical to specify map \emph{height} instead of \emph{width}, while in other situations it would be nice
to set either the \emph{shortest} or \emph{longest} map dimension.  Users may select
these alternatives by appending a character code to their map dimension.  To specify map \emph{height},
append \textbf{h} to the given dimension; to select the minimum map dimension, append \textbf{-}, whereas you may
append \textbf{+} to select the maximum map dimension.  Without the modifier the map width is
selected by default.

In \GMT\ version 4.3.0 we noticed we ran out of the alphabet for 1-letter (and sometimes 2-letter) projection codes. To allow more flexibility, and to make it easier to remember the codes, we implemented the option to use the abbreviations used by the \progname{Proj4} mapping package. Since some of the \GMT\ projections are not in \progname{Proj4}, we invented some of our own as well. For a full list of both the old 1- and 2-letter codes, as well as the \progname{Proj4}-equivalents see the quick reference cards in Section~\ref{sec:purpose}. For example, \Opt{JM15c} and \Opt{JMerc/15c} have the same meaning.

\input{GMT_Chapter_6.1}
\input{GMT_Chapter_6.2}
\input{GMT_Chapter_6.3}
\input{GMT_Chapter_6.4}

%------------------------------------------
%	$Id: GMT_Chapter_6.tex,v 1.19 2008-04-30 03:53:30 remko Exp $
%
%	The GMT Documentation Project
%	Copyright 2000-2008.
%	Paul Wessel and Walter H. F. Smith
%------------------------------------------
%
\chapter{\gmt\ Map Projections}
\label{ch:6}

\GMT\ implements more than 30 different projections.  They all project the input coordinates
longitude and latitude to positions on a map.  In general, $x' = f(x,y,z)$ and $y' = g(x,y,z)$, where
$z$ is implicitly given as the radial vector length to the $(x,y)$ point on the chosen ellipsoid.  The functions $f$ and $g$ can be
quite nasty and we will refrain from presenting details in this document.  The interested read is referred to
{\it Snyder} [1987]\footnote{Snyder, J. P., 1987, Map Projections \- A Working Manual, U.S. Geological Survey Prof. Paper 1395.}.
We will mostly be using the \GMTprog{pscoast} command to demonstrate each of the projections.
\GMT\ map projections are grouped into four categories depending on the
nature of the projection.  The groups are

\begin{enumerate}
\item Conic map projections
\item Azimuthal map projections
\item Cylindrical map projections
\item Miscellaneous projections
\end{enumerate}

Because $x$ and $y$ are coupled we can only specify one plot-dimensional scale, typically
a map \emph{scale} (for lower-case map projection code) or a map \emph{width} (for upper-case
map projection code).  However, in some cases it would be more
practical to specify map \emph{height} instead of \emph{width}, while in other situations it would be nice
to set either the \emph{shortest} or \emph{longest} map dimension.  Users may select
these alternatives by appending a character code to their map dimension.  To specify map \emph{height},
append \textbf{h} to the given dimension; to select the minimum map dimension, append \textbf{-}, whereas you may
append \textbf{+} to select the maximum map dimension.  Without the modifier the map width is
selected by default.

In \GMT\ version 4.3.0 we noticed we ran out of the alphabet for 1-letter (and sometimes 2-letter) projection codes. To allow more flexibility, and to make it easier to remember the codes, we implemented the option to use the abbreviations used by the \progname{Proj4} mapping package. Since some of the \GMT\ projections are not in \progname{Proj4}, we invented some of our own as well. For a full list of both the old 1- and 2-letter codes, as well as the \progname{Proj4}-equivalents see the quick reference cards in Section~\ref{sec:purpose}. For example, \Opt{JM15c} and \Opt{JMerc/15c} have the same meaning.

\input{GMT_Chapter_6.1}
\input{GMT_Chapter_6.2}
\input{GMT_Chapter_6.3}
\input{GMT_Chapter_6.4}

%------------------------------------------
%	$Id: GMT_Chapter_6.tex,v 1.19 2008-04-30 03:53:30 remko Exp $
%
%	The GMT Documentation Project
%	Copyright 2000-2008.
%	Paul Wessel and Walter H. F. Smith
%------------------------------------------
%
\chapter{\gmt\ Map Projections}
\label{ch:6}

\GMT\ implements more than 30 different projections.  They all project the input coordinates
longitude and latitude to positions on a map.  In general, $x' = f(x,y,z)$ and $y' = g(x,y,z)$, where
$z$ is implicitly given as the radial vector length to the $(x,y)$ point on the chosen ellipsoid.  The functions $f$ and $g$ can be
quite nasty and we will refrain from presenting details in this document.  The interested read is referred to
{\it Snyder} [1987]\footnote{Snyder, J. P., 1987, Map Projections \- A Working Manual, U.S. Geological Survey Prof. Paper 1395.}.
We will mostly be using the \GMTprog{pscoast} command to demonstrate each of the projections.
\GMT\ map projections are grouped into four categories depending on the
nature of the projection.  The groups are

\begin{enumerate}
\item Conic map projections
\item Azimuthal map projections
\item Cylindrical map projections
\item Miscellaneous projections
\end{enumerate}

Because $x$ and $y$ are coupled we can only specify one plot-dimensional scale, typically
a map \emph{scale} (for lower-case map projection code) or a map \emph{width} (for upper-case
map projection code).  However, in some cases it would be more
practical to specify map \emph{height} instead of \emph{width}, while in other situations it would be nice
to set either the \emph{shortest} or \emph{longest} map dimension.  Users may select
these alternatives by appending a character code to their map dimension.  To specify map \emph{height},
append \textbf{h} to the given dimension; to select the minimum map dimension, append \textbf{-}, whereas you may
append \textbf{+} to select the maximum map dimension.  Without the modifier the map width is
selected by default.

In \GMT\ version 4.3.0 we noticed we ran out of the alphabet for 1-letter (and sometimes 2-letter) projection codes. To allow more flexibility, and to make it easier to remember the codes, we implemented the option to use the abbreviations used by the \progname{Proj4} mapping package. Since some of the \GMT\ projections are not in \progname{Proj4}, we invented some of our own as well. For a full list of both the old 1- and 2-letter codes, as well as the \progname{Proj4}-equivalents see the quick reference cards in Section~\ref{sec:purpose}. For example, \Opt{JM15c} and \Opt{JMerc/15c} have the same meaning.

\input{GMT_Chapter_6.1}
\input{GMT_Chapter_6.2}
\input{GMT_Chapter_6.3}
\input{GMT_Chapter_6.4}

%------------------------------------------
%	$Id: GMT_Chapter_6.tex,v 1.19 2008-04-30 03:53:30 remko Exp $
%
%	The GMT Documentation Project
%	Copyright 2000-2008.
%	Paul Wessel and Walter H. F. Smith
%------------------------------------------
%
\chapter{\gmt\ Map Projections}
\label{ch:6}

\GMT\ implements more than 30 different projections.  They all project the input coordinates
longitude and latitude to positions on a map.  In general, $x' = f(x,y,z)$ and $y' = g(x,y,z)$, where
$z$ is implicitly given as the radial vector length to the $(x,y)$ point on the chosen ellipsoid.  The functions $f$ and $g$ can be
quite nasty and we will refrain from presenting details in this document.  The interested read is referred to
{\it Snyder} [1987]\footnote{Snyder, J. P., 1987, Map Projections \- A Working Manual, U.S. Geological Survey Prof. Paper 1395.}.
We will mostly be using the \GMTprog{pscoast} command to demonstrate each of the projections.
\GMT\ map projections are grouped into four categories depending on the
nature of the projection.  The groups are

\begin{enumerate}
\item Conic map projections
\item Azimuthal map projections
\item Cylindrical map projections
\item Miscellaneous projections
\end{enumerate}

Because $x$ and $y$ are coupled we can only specify one plot-dimensional scale, typically
a map \emph{scale} (for lower-case map projection code) or a map \emph{width} (for upper-case
map projection code).  However, in some cases it would be more
practical to specify map \emph{height} instead of \emph{width}, while in other situations it would be nice
to set either the \emph{shortest} or \emph{longest} map dimension.  Users may select
these alternatives by appending a character code to their map dimension.  To specify map \emph{height},
append \textbf{h} to the given dimension; to select the minimum map dimension, append \textbf{-}, whereas you may
append \textbf{+} to select the maximum map dimension.  Without the modifier the map width is
selected by default.

In \GMT\ version 4.3.0 we noticed we ran out of the alphabet for 1-letter (and sometimes 2-letter) projection codes. To allow more flexibility, and to make it easier to remember the codes, we implemented the option to use the abbreviations used by the \progname{Proj4} mapping package. Since some of the \GMT\ projections are not in \progname{Proj4}, we invented some of our own as well. For a full list of both the old 1- and 2-letter codes, as well as the \progname{Proj4}-equivalents see the quick reference cards in Section~\ref{sec:purpose}. For example, \Opt{JM15c} and \Opt{JMerc/15c} have the same meaning.

\input{GMT_Chapter_6.1}
\input{GMT_Chapter_6.2}
\input{GMT_Chapter_6.3}
\input{GMT_Chapter_6.4}


%------------------------------------------
%	$Id: GMT_Chapter_6.tex,v 1.19 2008-04-30 03:53:30 remko Exp $
%
%	The GMT Documentation Project
%	Copyright 2000-2008.
%	Paul Wessel and Walter H. F. Smith
%------------------------------------------
%
\chapter{\gmt\ Map Projections}
\label{ch:6}

\GMT\ implements more than 30 different projections.  They all project the input coordinates
longitude and latitude to positions on a map.  In general, $x' = f(x,y,z)$ and $y' = g(x,y,z)$, where
$z$ is implicitly given as the radial vector length to the $(x,y)$ point on the chosen ellipsoid.  The functions $f$ and $g$ can be
quite nasty and we will refrain from presenting details in this document.  The interested read is referred to
{\it Snyder} [1987]\footnote{Snyder, J. P., 1987, Map Projections \- A Working Manual, U.S. Geological Survey Prof. Paper 1395.}.
We will mostly be using the \GMTprog{pscoast} command to demonstrate each of the projections.
\GMT\ map projections are grouped into four categories depending on the
nature of the projection.  The groups are

\begin{enumerate}
\item Conic map projections
\item Azimuthal map projections
\item Cylindrical map projections
\item Miscellaneous projections
\end{enumerate}

Because $x$ and $y$ are coupled we can only specify one plot-dimensional scale, typically
a map \emph{scale} (for lower-case map projection code) or a map \emph{width} (for upper-case
map projection code).  However, in some cases it would be more
practical to specify map \emph{height} instead of \emph{width}, while in other situations it would be nice
to set either the \emph{shortest} or \emph{longest} map dimension.  Users may select
these alternatives by appending a character code to their map dimension.  To specify map \emph{height},
append \textbf{h} to the given dimension; to select the minimum map dimension, append \textbf{-}, whereas you may
append \textbf{+} to select the maximum map dimension.  Without the modifier the map width is
selected by default.

In \GMT\ version 4.3.0 we noticed we ran out of the alphabet for 1-letter (and sometimes 2-letter) projection codes. To allow more flexibility, and to make it easier to remember the codes, we implemented the option to use the abbreviations used by the \progname{Proj4} mapping package. Since some of the \GMT\ projections are not in \progname{Proj4}, we invented some of our own as well. For a full list of both the old 1- and 2-letter codes, as well as the \progname{Proj4}-equivalents see the quick reference cards in Section~\ref{sec:purpose}. For example, \Opt{JM15c} and \Opt{JMerc/15c} have the same meaning.

%------------------------------------------
%	$Id: GMT_Chapter_6.tex,v 1.19 2008-04-30 03:53:30 remko Exp $
%
%	The GMT Documentation Project
%	Copyright 2000-2008.
%	Paul Wessel and Walter H. F. Smith
%------------------------------------------
%
\chapter{\gmt\ Map Projections}
\label{ch:6}

\GMT\ implements more than 30 different projections.  They all project the input coordinates
longitude and latitude to positions on a map.  In general, $x' = f(x,y,z)$ and $y' = g(x,y,z)$, where
$z$ is implicitly given as the radial vector length to the $(x,y)$ point on the chosen ellipsoid.  The functions $f$ and $g$ can be
quite nasty and we will refrain from presenting details in this document.  The interested read is referred to
{\it Snyder} [1987]\footnote{Snyder, J. P., 1987, Map Projections \- A Working Manual, U.S. Geological Survey Prof. Paper 1395.}.
We will mostly be using the \GMTprog{pscoast} command to demonstrate each of the projections.
\GMT\ map projections are grouped into four categories depending on the
nature of the projection.  The groups are

\begin{enumerate}
\item Conic map projections
\item Azimuthal map projections
\item Cylindrical map projections
\item Miscellaneous projections
\end{enumerate}

Because $x$ and $y$ are coupled we can only specify one plot-dimensional scale, typically
a map \emph{scale} (for lower-case map projection code) or a map \emph{width} (for upper-case
map projection code).  However, in some cases it would be more
practical to specify map \emph{height} instead of \emph{width}, while in other situations it would be nice
to set either the \emph{shortest} or \emph{longest} map dimension.  Users may select
these alternatives by appending a character code to their map dimension.  To specify map \emph{height},
append \textbf{h} to the given dimension; to select the minimum map dimension, append \textbf{-}, whereas you may
append \textbf{+} to select the maximum map dimension.  Without the modifier the map width is
selected by default.

In \GMT\ version 4.3.0 we noticed we ran out of the alphabet for 1-letter (and sometimes 2-letter) projection codes. To allow more flexibility, and to make it easier to remember the codes, we implemented the option to use the abbreviations used by the \progname{Proj4} mapping package. Since some of the \GMT\ projections are not in \progname{Proj4}, we invented some of our own as well. For a full list of both the old 1- and 2-letter codes, as well as the \progname{Proj4}-equivalents see the quick reference cards in Section~\ref{sec:purpose}. For example, \Opt{JM15c} and \Opt{JMerc/15c} have the same meaning.

\input{GMT_Chapter_6.1}
\input{GMT_Chapter_6.2}
\input{GMT_Chapter_6.3}
\input{GMT_Chapter_6.4}

%------------------------------------------
%	$Id: GMT_Chapter_6.tex,v 1.19 2008-04-30 03:53:30 remko Exp $
%
%	The GMT Documentation Project
%	Copyright 2000-2008.
%	Paul Wessel and Walter H. F. Smith
%------------------------------------------
%
\chapter{\gmt\ Map Projections}
\label{ch:6}

\GMT\ implements more than 30 different projections.  They all project the input coordinates
longitude and latitude to positions on a map.  In general, $x' = f(x,y,z)$ and $y' = g(x,y,z)$, where
$z$ is implicitly given as the radial vector length to the $(x,y)$ point on the chosen ellipsoid.  The functions $f$ and $g$ can be
quite nasty and we will refrain from presenting details in this document.  The interested read is referred to
{\it Snyder} [1987]\footnote{Snyder, J. P., 1987, Map Projections \- A Working Manual, U.S. Geological Survey Prof. Paper 1395.}.
We will mostly be using the \GMTprog{pscoast} command to demonstrate each of the projections.
\GMT\ map projections are grouped into four categories depending on the
nature of the projection.  The groups are

\begin{enumerate}
\item Conic map projections
\item Azimuthal map projections
\item Cylindrical map projections
\item Miscellaneous projections
\end{enumerate}

Because $x$ and $y$ are coupled we can only specify one plot-dimensional scale, typically
a map \emph{scale} (for lower-case map projection code) or a map \emph{width} (for upper-case
map projection code).  However, in some cases it would be more
practical to specify map \emph{height} instead of \emph{width}, while in other situations it would be nice
to set either the \emph{shortest} or \emph{longest} map dimension.  Users may select
these alternatives by appending a character code to their map dimension.  To specify map \emph{height},
append \textbf{h} to the given dimension; to select the minimum map dimension, append \textbf{-}, whereas you may
append \textbf{+} to select the maximum map dimension.  Without the modifier the map width is
selected by default.

In \GMT\ version 4.3.0 we noticed we ran out of the alphabet for 1-letter (and sometimes 2-letter) projection codes. To allow more flexibility, and to make it easier to remember the codes, we implemented the option to use the abbreviations used by the \progname{Proj4} mapping package. Since some of the \GMT\ projections are not in \progname{Proj4}, we invented some of our own as well. For a full list of both the old 1- and 2-letter codes, as well as the \progname{Proj4}-equivalents see the quick reference cards in Section~\ref{sec:purpose}. For example, \Opt{JM15c} and \Opt{JMerc/15c} have the same meaning.

\input{GMT_Chapter_6.1}
\input{GMT_Chapter_6.2}
\input{GMT_Chapter_6.3}
\input{GMT_Chapter_6.4}

%------------------------------------------
%	$Id: GMT_Chapter_6.tex,v 1.19 2008-04-30 03:53:30 remko Exp $
%
%	The GMT Documentation Project
%	Copyright 2000-2008.
%	Paul Wessel and Walter H. F. Smith
%------------------------------------------
%
\chapter{\gmt\ Map Projections}
\label{ch:6}

\GMT\ implements more than 30 different projections.  They all project the input coordinates
longitude and latitude to positions on a map.  In general, $x' = f(x,y,z)$ and $y' = g(x,y,z)$, where
$z$ is implicitly given as the radial vector length to the $(x,y)$ point on the chosen ellipsoid.  The functions $f$ and $g$ can be
quite nasty and we will refrain from presenting details in this document.  The interested read is referred to
{\it Snyder} [1987]\footnote{Snyder, J. P., 1987, Map Projections \- A Working Manual, U.S. Geological Survey Prof. Paper 1395.}.
We will mostly be using the \GMTprog{pscoast} command to demonstrate each of the projections.
\GMT\ map projections are grouped into four categories depending on the
nature of the projection.  The groups are

\begin{enumerate}
\item Conic map projections
\item Azimuthal map projections
\item Cylindrical map projections
\item Miscellaneous projections
\end{enumerate}

Because $x$ and $y$ are coupled we can only specify one plot-dimensional scale, typically
a map \emph{scale} (for lower-case map projection code) or a map \emph{width} (for upper-case
map projection code).  However, in some cases it would be more
practical to specify map \emph{height} instead of \emph{width}, while in other situations it would be nice
to set either the \emph{shortest} or \emph{longest} map dimension.  Users may select
these alternatives by appending a character code to their map dimension.  To specify map \emph{height},
append \textbf{h} to the given dimension; to select the minimum map dimension, append \textbf{-}, whereas you may
append \textbf{+} to select the maximum map dimension.  Without the modifier the map width is
selected by default.

In \GMT\ version 4.3.0 we noticed we ran out of the alphabet for 1-letter (and sometimes 2-letter) projection codes. To allow more flexibility, and to make it easier to remember the codes, we implemented the option to use the abbreviations used by the \progname{Proj4} mapping package. Since some of the \GMT\ projections are not in \progname{Proj4}, we invented some of our own as well. For a full list of both the old 1- and 2-letter codes, as well as the \progname{Proj4}-equivalents see the quick reference cards in Section~\ref{sec:purpose}. For example, \Opt{JM15c} and \Opt{JMerc/15c} have the same meaning.

\input{GMT_Chapter_6.1}
\input{GMT_Chapter_6.2}
\input{GMT_Chapter_6.3}
\input{GMT_Chapter_6.4}

%------------------------------------------
%	$Id: GMT_Chapter_6.tex,v 1.19 2008-04-30 03:53:30 remko Exp $
%
%	The GMT Documentation Project
%	Copyright 2000-2008.
%	Paul Wessel and Walter H. F. Smith
%------------------------------------------
%
\chapter{\gmt\ Map Projections}
\label{ch:6}

\GMT\ implements more than 30 different projections.  They all project the input coordinates
longitude and latitude to positions on a map.  In general, $x' = f(x,y,z)$ and $y' = g(x,y,z)$, where
$z$ is implicitly given as the radial vector length to the $(x,y)$ point on the chosen ellipsoid.  The functions $f$ and $g$ can be
quite nasty and we will refrain from presenting details in this document.  The interested read is referred to
{\it Snyder} [1987]\footnote{Snyder, J. P., 1987, Map Projections \- A Working Manual, U.S. Geological Survey Prof. Paper 1395.}.
We will mostly be using the \GMTprog{pscoast} command to demonstrate each of the projections.
\GMT\ map projections are grouped into four categories depending on the
nature of the projection.  The groups are

\begin{enumerate}
\item Conic map projections
\item Azimuthal map projections
\item Cylindrical map projections
\item Miscellaneous projections
\end{enumerate}

Because $x$ and $y$ are coupled we can only specify one plot-dimensional scale, typically
a map \emph{scale} (for lower-case map projection code) or a map \emph{width} (for upper-case
map projection code).  However, in some cases it would be more
practical to specify map \emph{height} instead of \emph{width}, while in other situations it would be nice
to set either the \emph{shortest} or \emph{longest} map dimension.  Users may select
these alternatives by appending a character code to their map dimension.  To specify map \emph{height},
append \textbf{h} to the given dimension; to select the minimum map dimension, append \textbf{-}, whereas you may
append \textbf{+} to select the maximum map dimension.  Without the modifier the map width is
selected by default.

In \GMT\ version 4.3.0 we noticed we ran out of the alphabet for 1-letter (and sometimes 2-letter) projection codes. To allow more flexibility, and to make it easier to remember the codes, we implemented the option to use the abbreviations used by the \progname{Proj4} mapping package. Since some of the \GMT\ projections are not in \progname{Proj4}, we invented some of our own as well. For a full list of both the old 1- and 2-letter codes, as well as the \progname{Proj4}-equivalents see the quick reference cards in Section~\ref{sec:purpose}. For example, \Opt{JM15c} and \Opt{JMerc/15c} have the same meaning.

\input{GMT_Chapter_6.1}
\input{GMT_Chapter_6.2}
\input{GMT_Chapter_6.3}
\input{GMT_Chapter_6.4}


%------------------------------------------
%	$Id: GMT_Chapter_6.tex,v 1.19 2008-04-30 03:53:30 remko Exp $
%
%	The GMT Documentation Project
%	Copyright 2000-2008.
%	Paul Wessel and Walter H. F. Smith
%------------------------------------------
%
\chapter{\gmt\ Map Projections}
\label{ch:6}

\GMT\ implements more than 30 different projections.  They all project the input coordinates
longitude and latitude to positions on a map.  In general, $x' = f(x,y,z)$ and $y' = g(x,y,z)$, where
$z$ is implicitly given as the radial vector length to the $(x,y)$ point on the chosen ellipsoid.  The functions $f$ and $g$ can be
quite nasty and we will refrain from presenting details in this document.  The interested read is referred to
{\it Snyder} [1987]\footnote{Snyder, J. P., 1987, Map Projections \- A Working Manual, U.S. Geological Survey Prof. Paper 1395.}.
We will mostly be using the \GMTprog{pscoast} command to demonstrate each of the projections.
\GMT\ map projections are grouped into four categories depending on the
nature of the projection.  The groups are

\begin{enumerate}
\item Conic map projections
\item Azimuthal map projections
\item Cylindrical map projections
\item Miscellaneous projections
\end{enumerate}

Because $x$ and $y$ are coupled we can only specify one plot-dimensional scale, typically
a map \emph{scale} (for lower-case map projection code) or a map \emph{width} (for upper-case
map projection code).  However, in some cases it would be more
practical to specify map \emph{height} instead of \emph{width}, while in other situations it would be nice
to set either the \emph{shortest} or \emph{longest} map dimension.  Users may select
these alternatives by appending a character code to their map dimension.  To specify map \emph{height},
append \textbf{h} to the given dimension; to select the minimum map dimension, append \textbf{-}, whereas you may
append \textbf{+} to select the maximum map dimension.  Without the modifier the map width is
selected by default.

In \GMT\ version 4.3.0 we noticed we ran out of the alphabet for 1-letter (and sometimes 2-letter) projection codes. To allow more flexibility, and to make it easier to remember the codes, we implemented the option to use the abbreviations used by the \progname{Proj4} mapping package. Since some of the \GMT\ projections are not in \progname{Proj4}, we invented some of our own as well. For a full list of both the old 1- and 2-letter codes, as well as the \progname{Proj4}-equivalents see the quick reference cards in Section~\ref{sec:purpose}. For example, \Opt{JM15c} and \Opt{JMerc/15c} have the same meaning.

%------------------------------------------
%	$Id: GMT_Chapter_6.tex,v 1.19 2008-04-30 03:53:30 remko Exp $
%
%	The GMT Documentation Project
%	Copyright 2000-2008.
%	Paul Wessel and Walter H. F. Smith
%------------------------------------------
%
\chapter{\gmt\ Map Projections}
\label{ch:6}

\GMT\ implements more than 30 different projections.  They all project the input coordinates
longitude and latitude to positions on a map.  In general, $x' = f(x,y,z)$ and $y' = g(x,y,z)$, where
$z$ is implicitly given as the radial vector length to the $(x,y)$ point on the chosen ellipsoid.  The functions $f$ and $g$ can be
quite nasty and we will refrain from presenting details in this document.  The interested read is referred to
{\it Snyder} [1987]\footnote{Snyder, J. P., 1987, Map Projections \- A Working Manual, U.S. Geological Survey Prof. Paper 1395.}.
We will mostly be using the \GMTprog{pscoast} command to demonstrate each of the projections.
\GMT\ map projections are grouped into four categories depending on the
nature of the projection.  The groups are

\begin{enumerate}
\item Conic map projections
\item Azimuthal map projections
\item Cylindrical map projections
\item Miscellaneous projections
\end{enumerate}

Because $x$ and $y$ are coupled we can only specify one plot-dimensional scale, typically
a map \emph{scale} (for lower-case map projection code) or a map \emph{width} (for upper-case
map projection code).  However, in some cases it would be more
practical to specify map \emph{height} instead of \emph{width}, while in other situations it would be nice
to set either the \emph{shortest} or \emph{longest} map dimension.  Users may select
these alternatives by appending a character code to their map dimension.  To specify map \emph{height},
append \textbf{h} to the given dimension; to select the minimum map dimension, append \textbf{-}, whereas you may
append \textbf{+} to select the maximum map dimension.  Without the modifier the map width is
selected by default.

In \GMT\ version 4.3.0 we noticed we ran out of the alphabet for 1-letter (and sometimes 2-letter) projection codes. To allow more flexibility, and to make it easier to remember the codes, we implemented the option to use the abbreviations used by the \progname{Proj4} mapping package. Since some of the \GMT\ projections are not in \progname{Proj4}, we invented some of our own as well. For a full list of both the old 1- and 2-letter codes, as well as the \progname{Proj4}-equivalents see the quick reference cards in Section~\ref{sec:purpose}. For example, \Opt{JM15c} and \Opt{JMerc/15c} have the same meaning.

\input{GMT_Chapter_6.1}
\input{GMT_Chapter_6.2}
\input{GMT_Chapter_6.3}
\input{GMT_Chapter_6.4}

%------------------------------------------
%	$Id: GMT_Chapter_6.tex,v 1.19 2008-04-30 03:53:30 remko Exp $
%
%	The GMT Documentation Project
%	Copyright 2000-2008.
%	Paul Wessel and Walter H. F. Smith
%------------------------------------------
%
\chapter{\gmt\ Map Projections}
\label{ch:6}

\GMT\ implements more than 30 different projections.  They all project the input coordinates
longitude and latitude to positions on a map.  In general, $x' = f(x,y,z)$ and $y' = g(x,y,z)$, where
$z$ is implicitly given as the radial vector length to the $(x,y)$ point on the chosen ellipsoid.  The functions $f$ and $g$ can be
quite nasty and we will refrain from presenting details in this document.  The interested read is referred to
{\it Snyder} [1987]\footnote{Snyder, J. P., 1987, Map Projections \- A Working Manual, U.S. Geological Survey Prof. Paper 1395.}.
We will mostly be using the \GMTprog{pscoast} command to demonstrate each of the projections.
\GMT\ map projections are grouped into four categories depending on the
nature of the projection.  The groups are

\begin{enumerate}
\item Conic map projections
\item Azimuthal map projections
\item Cylindrical map projections
\item Miscellaneous projections
\end{enumerate}

Because $x$ and $y$ are coupled we can only specify one plot-dimensional scale, typically
a map \emph{scale} (for lower-case map projection code) or a map \emph{width} (for upper-case
map projection code).  However, in some cases it would be more
practical to specify map \emph{height} instead of \emph{width}, while in other situations it would be nice
to set either the \emph{shortest} or \emph{longest} map dimension.  Users may select
these alternatives by appending a character code to their map dimension.  To specify map \emph{height},
append \textbf{h} to the given dimension; to select the minimum map dimension, append \textbf{-}, whereas you may
append \textbf{+} to select the maximum map dimension.  Without the modifier the map width is
selected by default.

In \GMT\ version 4.3.0 we noticed we ran out of the alphabet for 1-letter (and sometimes 2-letter) projection codes. To allow more flexibility, and to make it easier to remember the codes, we implemented the option to use the abbreviations used by the \progname{Proj4} mapping package. Since some of the \GMT\ projections are not in \progname{Proj4}, we invented some of our own as well. For a full list of both the old 1- and 2-letter codes, as well as the \progname{Proj4}-equivalents see the quick reference cards in Section~\ref{sec:purpose}. For example, \Opt{JM15c} and \Opt{JMerc/15c} have the same meaning.

\input{GMT_Chapter_6.1}
\input{GMT_Chapter_6.2}
\input{GMT_Chapter_6.3}
\input{GMT_Chapter_6.4}

%------------------------------------------
%	$Id: GMT_Chapter_6.tex,v 1.19 2008-04-30 03:53:30 remko Exp $
%
%	The GMT Documentation Project
%	Copyright 2000-2008.
%	Paul Wessel and Walter H. F. Smith
%------------------------------------------
%
\chapter{\gmt\ Map Projections}
\label{ch:6}

\GMT\ implements more than 30 different projections.  They all project the input coordinates
longitude and latitude to positions on a map.  In general, $x' = f(x,y,z)$ and $y' = g(x,y,z)$, where
$z$ is implicitly given as the radial vector length to the $(x,y)$ point on the chosen ellipsoid.  The functions $f$ and $g$ can be
quite nasty and we will refrain from presenting details in this document.  The interested read is referred to
{\it Snyder} [1987]\footnote{Snyder, J. P., 1987, Map Projections \- A Working Manual, U.S. Geological Survey Prof. Paper 1395.}.
We will mostly be using the \GMTprog{pscoast} command to demonstrate each of the projections.
\GMT\ map projections are grouped into four categories depending on the
nature of the projection.  The groups are

\begin{enumerate}
\item Conic map projections
\item Azimuthal map projections
\item Cylindrical map projections
\item Miscellaneous projections
\end{enumerate}

Because $x$ and $y$ are coupled we can only specify one plot-dimensional scale, typically
a map \emph{scale} (for lower-case map projection code) or a map \emph{width} (for upper-case
map projection code).  However, in some cases it would be more
practical to specify map \emph{height} instead of \emph{width}, while in other situations it would be nice
to set either the \emph{shortest} or \emph{longest} map dimension.  Users may select
these alternatives by appending a character code to their map dimension.  To specify map \emph{height},
append \textbf{h} to the given dimension; to select the minimum map dimension, append \textbf{-}, whereas you may
append \textbf{+} to select the maximum map dimension.  Without the modifier the map width is
selected by default.

In \GMT\ version 4.3.0 we noticed we ran out of the alphabet for 1-letter (and sometimes 2-letter) projection codes. To allow more flexibility, and to make it easier to remember the codes, we implemented the option to use the abbreviations used by the \progname{Proj4} mapping package. Since some of the \GMT\ projections are not in \progname{Proj4}, we invented some of our own as well. For a full list of both the old 1- and 2-letter codes, as well as the \progname{Proj4}-equivalents see the quick reference cards in Section~\ref{sec:purpose}. For example, \Opt{JM15c} and \Opt{JMerc/15c} have the same meaning.

\input{GMT_Chapter_6.1}
\input{GMT_Chapter_6.2}
\input{GMT_Chapter_6.3}
\input{GMT_Chapter_6.4}

%------------------------------------------
%	$Id: GMT_Chapter_6.tex,v 1.19 2008-04-30 03:53:30 remko Exp $
%
%	The GMT Documentation Project
%	Copyright 2000-2008.
%	Paul Wessel and Walter H. F. Smith
%------------------------------------------
%
\chapter{\gmt\ Map Projections}
\label{ch:6}

\GMT\ implements more than 30 different projections.  They all project the input coordinates
longitude and latitude to positions on a map.  In general, $x' = f(x,y,z)$ and $y' = g(x,y,z)$, where
$z$ is implicitly given as the radial vector length to the $(x,y)$ point on the chosen ellipsoid.  The functions $f$ and $g$ can be
quite nasty and we will refrain from presenting details in this document.  The interested read is referred to
{\it Snyder} [1987]\footnote{Snyder, J. P., 1987, Map Projections \- A Working Manual, U.S. Geological Survey Prof. Paper 1395.}.
We will mostly be using the \GMTprog{pscoast} command to demonstrate each of the projections.
\GMT\ map projections are grouped into four categories depending on the
nature of the projection.  The groups are

\begin{enumerate}
\item Conic map projections
\item Azimuthal map projections
\item Cylindrical map projections
\item Miscellaneous projections
\end{enumerate}

Because $x$ and $y$ are coupled we can only specify one plot-dimensional scale, typically
a map \emph{scale} (for lower-case map projection code) or a map \emph{width} (for upper-case
map projection code).  However, in some cases it would be more
practical to specify map \emph{height} instead of \emph{width}, while in other situations it would be nice
to set either the \emph{shortest} or \emph{longest} map dimension.  Users may select
these alternatives by appending a character code to their map dimension.  To specify map \emph{height},
append \textbf{h} to the given dimension; to select the minimum map dimension, append \textbf{-}, whereas you may
append \textbf{+} to select the maximum map dimension.  Without the modifier the map width is
selected by default.

In \GMT\ version 4.3.0 we noticed we ran out of the alphabet for 1-letter (and sometimes 2-letter) projection codes. To allow more flexibility, and to make it easier to remember the codes, we implemented the option to use the abbreviations used by the \progname{Proj4} mapping package. Since some of the \GMT\ projections are not in \progname{Proj4}, we invented some of our own as well. For a full list of both the old 1- and 2-letter codes, as well as the \progname{Proj4}-equivalents see the quick reference cards in Section~\ref{sec:purpose}. For example, \Opt{JM15c} and \Opt{JMerc/15c} have the same meaning.

\input{GMT_Chapter_6.1}
\input{GMT_Chapter_6.2}
\input{GMT_Chapter_6.3}
\input{GMT_Chapter_6.4}


%------------------------------------------
%	$Id: GMT_Chapter_6.tex,v 1.19 2008-04-30 03:53:30 remko Exp $
%
%	The GMT Documentation Project
%	Copyright 2000-2008.
%	Paul Wessel and Walter H. F. Smith
%------------------------------------------
%
\chapter{\gmt\ Map Projections}
\label{ch:6}

\GMT\ implements more than 30 different projections.  They all project the input coordinates
longitude and latitude to positions on a map.  In general, $x' = f(x,y,z)$ and $y' = g(x,y,z)$, where
$z$ is implicitly given as the radial vector length to the $(x,y)$ point on the chosen ellipsoid.  The functions $f$ and $g$ can be
quite nasty and we will refrain from presenting details in this document.  The interested read is referred to
{\it Snyder} [1987]\footnote{Snyder, J. P., 1987, Map Projections \- A Working Manual, U.S. Geological Survey Prof. Paper 1395.}.
We will mostly be using the \GMTprog{pscoast} command to demonstrate each of the projections.
\GMT\ map projections are grouped into four categories depending on the
nature of the projection.  The groups are

\begin{enumerate}
\item Conic map projections
\item Azimuthal map projections
\item Cylindrical map projections
\item Miscellaneous projections
\end{enumerate}

Because $x$ and $y$ are coupled we can only specify one plot-dimensional scale, typically
a map \emph{scale} (for lower-case map projection code) or a map \emph{width} (for upper-case
map projection code).  However, in some cases it would be more
practical to specify map \emph{height} instead of \emph{width}, while in other situations it would be nice
to set either the \emph{shortest} or \emph{longest} map dimension.  Users may select
these alternatives by appending a character code to their map dimension.  To specify map \emph{height},
append \textbf{h} to the given dimension; to select the minimum map dimension, append \textbf{-}, whereas you may
append \textbf{+} to select the maximum map dimension.  Without the modifier the map width is
selected by default.

In \GMT\ version 4.3.0 we noticed we ran out of the alphabet for 1-letter (and sometimes 2-letter) projection codes. To allow more flexibility, and to make it easier to remember the codes, we implemented the option to use the abbreviations used by the \progname{Proj4} mapping package. Since some of the \GMT\ projections are not in \progname{Proj4}, we invented some of our own as well. For a full list of both the old 1- and 2-letter codes, as well as the \progname{Proj4}-equivalents see the quick reference cards in Section~\ref{sec:purpose}. For example, \Opt{JM15c} and \Opt{JMerc/15c} have the same meaning.

%------------------------------------------
%	$Id: GMT_Chapter_6.tex,v 1.19 2008-04-30 03:53:30 remko Exp $
%
%	The GMT Documentation Project
%	Copyright 2000-2008.
%	Paul Wessel and Walter H. F. Smith
%------------------------------------------
%
\chapter{\gmt\ Map Projections}
\label{ch:6}

\GMT\ implements more than 30 different projections.  They all project the input coordinates
longitude and latitude to positions on a map.  In general, $x' = f(x,y,z)$ and $y' = g(x,y,z)$, where
$z$ is implicitly given as the radial vector length to the $(x,y)$ point on the chosen ellipsoid.  The functions $f$ and $g$ can be
quite nasty and we will refrain from presenting details in this document.  The interested read is referred to
{\it Snyder} [1987]\footnote{Snyder, J. P., 1987, Map Projections \- A Working Manual, U.S. Geological Survey Prof. Paper 1395.}.
We will mostly be using the \GMTprog{pscoast} command to demonstrate each of the projections.
\GMT\ map projections are grouped into four categories depending on the
nature of the projection.  The groups are

\begin{enumerate}
\item Conic map projections
\item Azimuthal map projections
\item Cylindrical map projections
\item Miscellaneous projections
\end{enumerate}

Because $x$ and $y$ are coupled we can only specify one plot-dimensional scale, typically
a map \emph{scale} (for lower-case map projection code) or a map \emph{width} (for upper-case
map projection code).  However, in some cases it would be more
practical to specify map \emph{height} instead of \emph{width}, while in other situations it would be nice
to set either the \emph{shortest} or \emph{longest} map dimension.  Users may select
these alternatives by appending a character code to their map dimension.  To specify map \emph{height},
append \textbf{h} to the given dimension; to select the minimum map dimension, append \textbf{-}, whereas you may
append \textbf{+} to select the maximum map dimension.  Without the modifier the map width is
selected by default.

In \GMT\ version 4.3.0 we noticed we ran out of the alphabet for 1-letter (and sometimes 2-letter) projection codes. To allow more flexibility, and to make it easier to remember the codes, we implemented the option to use the abbreviations used by the \progname{Proj4} mapping package. Since some of the \GMT\ projections are not in \progname{Proj4}, we invented some of our own as well. For a full list of both the old 1- and 2-letter codes, as well as the \progname{Proj4}-equivalents see the quick reference cards in Section~\ref{sec:purpose}. For example, \Opt{JM15c} and \Opt{JMerc/15c} have the same meaning.

\input{GMT_Chapter_6.1}
\input{GMT_Chapter_6.2}
\input{GMT_Chapter_6.3}
\input{GMT_Chapter_6.4}

%------------------------------------------
%	$Id: GMT_Chapter_6.tex,v 1.19 2008-04-30 03:53:30 remko Exp $
%
%	The GMT Documentation Project
%	Copyright 2000-2008.
%	Paul Wessel and Walter H. F. Smith
%------------------------------------------
%
\chapter{\gmt\ Map Projections}
\label{ch:6}

\GMT\ implements more than 30 different projections.  They all project the input coordinates
longitude and latitude to positions on a map.  In general, $x' = f(x,y,z)$ and $y' = g(x,y,z)$, where
$z$ is implicitly given as the radial vector length to the $(x,y)$ point on the chosen ellipsoid.  The functions $f$ and $g$ can be
quite nasty and we will refrain from presenting details in this document.  The interested read is referred to
{\it Snyder} [1987]\footnote{Snyder, J. P., 1987, Map Projections \- A Working Manual, U.S. Geological Survey Prof. Paper 1395.}.
We will mostly be using the \GMTprog{pscoast} command to demonstrate each of the projections.
\GMT\ map projections are grouped into four categories depending on the
nature of the projection.  The groups are

\begin{enumerate}
\item Conic map projections
\item Azimuthal map projections
\item Cylindrical map projections
\item Miscellaneous projections
\end{enumerate}

Because $x$ and $y$ are coupled we can only specify one plot-dimensional scale, typically
a map \emph{scale} (for lower-case map projection code) or a map \emph{width} (for upper-case
map projection code).  However, in some cases it would be more
practical to specify map \emph{height} instead of \emph{width}, while in other situations it would be nice
to set either the \emph{shortest} or \emph{longest} map dimension.  Users may select
these alternatives by appending a character code to their map dimension.  To specify map \emph{height},
append \textbf{h} to the given dimension; to select the minimum map dimension, append \textbf{-}, whereas you may
append \textbf{+} to select the maximum map dimension.  Without the modifier the map width is
selected by default.

In \GMT\ version 4.3.0 we noticed we ran out of the alphabet for 1-letter (and sometimes 2-letter) projection codes. To allow more flexibility, and to make it easier to remember the codes, we implemented the option to use the abbreviations used by the \progname{Proj4} mapping package. Since some of the \GMT\ projections are not in \progname{Proj4}, we invented some of our own as well. For a full list of both the old 1- and 2-letter codes, as well as the \progname{Proj4}-equivalents see the quick reference cards in Section~\ref{sec:purpose}. For example, \Opt{JM15c} and \Opt{JMerc/15c} have the same meaning.

\input{GMT_Chapter_6.1}
\input{GMT_Chapter_6.2}
\input{GMT_Chapter_6.3}
\input{GMT_Chapter_6.4}

%------------------------------------------
%	$Id: GMT_Chapter_6.tex,v 1.19 2008-04-30 03:53:30 remko Exp $
%
%	The GMT Documentation Project
%	Copyright 2000-2008.
%	Paul Wessel and Walter H. F. Smith
%------------------------------------------
%
\chapter{\gmt\ Map Projections}
\label{ch:6}

\GMT\ implements more than 30 different projections.  They all project the input coordinates
longitude and latitude to positions on a map.  In general, $x' = f(x,y,z)$ and $y' = g(x,y,z)$, where
$z$ is implicitly given as the radial vector length to the $(x,y)$ point on the chosen ellipsoid.  The functions $f$ and $g$ can be
quite nasty and we will refrain from presenting details in this document.  The interested read is referred to
{\it Snyder} [1987]\footnote{Snyder, J. P., 1987, Map Projections \- A Working Manual, U.S. Geological Survey Prof. Paper 1395.}.
We will mostly be using the \GMTprog{pscoast} command to demonstrate each of the projections.
\GMT\ map projections are grouped into four categories depending on the
nature of the projection.  The groups are

\begin{enumerate}
\item Conic map projections
\item Azimuthal map projections
\item Cylindrical map projections
\item Miscellaneous projections
\end{enumerate}

Because $x$ and $y$ are coupled we can only specify one plot-dimensional scale, typically
a map \emph{scale} (for lower-case map projection code) or a map \emph{width} (for upper-case
map projection code).  However, in some cases it would be more
practical to specify map \emph{height} instead of \emph{width}, while in other situations it would be nice
to set either the \emph{shortest} or \emph{longest} map dimension.  Users may select
these alternatives by appending a character code to their map dimension.  To specify map \emph{height},
append \textbf{h} to the given dimension; to select the minimum map dimension, append \textbf{-}, whereas you may
append \textbf{+} to select the maximum map dimension.  Without the modifier the map width is
selected by default.

In \GMT\ version 4.3.0 we noticed we ran out of the alphabet for 1-letter (and sometimes 2-letter) projection codes. To allow more flexibility, and to make it easier to remember the codes, we implemented the option to use the abbreviations used by the \progname{Proj4} mapping package. Since some of the \GMT\ projections are not in \progname{Proj4}, we invented some of our own as well. For a full list of both the old 1- and 2-letter codes, as well as the \progname{Proj4}-equivalents see the quick reference cards in Section~\ref{sec:purpose}. For example, \Opt{JM15c} and \Opt{JMerc/15c} have the same meaning.

\input{GMT_Chapter_6.1}
\input{GMT_Chapter_6.2}
\input{GMT_Chapter_6.3}
\input{GMT_Chapter_6.4}

%------------------------------------------
%	$Id: GMT_Chapter_6.tex,v 1.19 2008-04-30 03:53:30 remko Exp $
%
%	The GMT Documentation Project
%	Copyright 2000-2008.
%	Paul Wessel and Walter H. F. Smith
%------------------------------------------
%
\chapter{\gmt\ Map Projections}
\label{ch:6}

\GMT\ implements more than 30 different projections.  They all project the input coordinates
longitude and latitude to positions on a map.  In general, $x' = f(x,y,z)$ and $y' = g(x,y,z)$, where
$z$ is implicitly given as the radial vector length to the $(x,y)$ point on the chosen ellipsoid.  The functions $f$ and $g$ can be
quite nasty and we will refrain from presenting details in this document.  The interested read is referred to
{\it Snyder} [1987]\footnote{Snyder, J. P., 1987, Map Projections \- A Working Manual, U.S. Geological Survey Prof. Paper 1395.}.
We will mostly be using the \GMTprog{pscoast} command to demonstrate each of the projections.
\GMT\ map projections are grouped into four categories depending on the
nature of the projection.  The groups are

\begin{enumerate}
\item Conic map projections
\item Azimuthal map projections
\item Cylindrical map projections
\item Miscellaneous projections
\end{enumerate}

Because $x$ and $y$ are coupled we can only specify one plot-dimensional scale, typically
a map \emph{scale} (for lower-case map projection code) or a map \emph{width} (for upper-case
map projection code).  However, in some cases it would be more
practical to specify map \emph{height} instead of \emph{width}, while in other situations it would be nice
to set either the \emph{shortest} or \emph{longest} map dimension.  Users may select
these alternatives by appending a character code to their map dimension.  To specify map \emph{height},
append \textbf{h} to the given dimension; to select the minimum map dimension, append \textbf{-}, whereas you may
append \textbf{+} to select the maximum map dimension.  Without the modifier the map width is
selected by default.

In \GMT\ version 4.3.0 we noticed we ran out of the alphabet for 1-letter (and sometimes 2-letter) projection codes. To allow more flexibility, and to make it easier to remember the codes, we implemented the option to use the abbreviations used by the \progname{Proj4} mapping package. Since some of the \GMT\ projections are not in \progname{Proj4}, we invented some of our own as well. For a full list of both the old 1- and 2-letter codes, as well as the \progname{Proj4}-equivalents see the quick reference cards in Section~\ref{sec:purpose}. For example, \Opt{JM15c} and \Opt{JMerc/15c} have the same meaning.

\input{GMT_Chapter_6.1}
\input{GMT_Chapter_6.2}
\input{GMT_Chapter_6.3}
\input{GMT_Chapter_6.4}



%------------------------------------------
%	$Id: GMT_Chapter_6.tex,v 1.19 2008-04-30 03:53:30 remko Exp $
%
%	The GMT Documentation Project
%	Copyright 2000-2008.
%	Paul Wessel and Walter H. F. Smith
%------------------------------------------
%
\chapter{\gmt\ Map Projections}
\label{ch:6}

\GMT\ implements more than 30 different projections.  They all project the input coordinates
longitude and latitude to positions on a map.  In general, $x' = f(x,y,z)$ and $y' = g(x,y,z)$, where
$z$ is implicitly given as the radial vector length to the $(x,y)$ point on the chosen ellipsoid.  The functions $f$ and $g$ can be
quite nasty and we will refrain from presenting details in this document.  The interested read is referred to
{\it Snyder} [1987]\footnote{Snyder, J. P., 1987, Map Projections \- A Working Manual, U.S. Geological Survey Prof. Paper 1395.}.
We will mostly be using the \GMTprog{pscoast} command to demonstrate each of the projections.
\GMT\ map projections are grouped into four categories depending on the
nature of the projection.  The groups are

\begin{enumerate}
\item Conic map projections
\item Azimuthal map projections
\item Cylindrical map projections
\item Miscellaneous projections
\end{enumerate}

Because $x$ and $y$ are coupled we can only specify one plot-dimensional scale, typically
a map \emph{scale} (for lower-case map projection code) or a map \emph{width} (for upper-case
map projection code).  However, in some cases it would be more
practical to specify map \emph{height} instead of \emph{width}, while in other situations it would be nice
to set either the \emph{shortest} or \emph{longest} map dimension.  Users may select
these alternatives by appending a character code to their map dimension.  To specify map \emph{height},
append \textbf{h} to the given dimension; to select the minimum map dimension, append \textbf{-}, whereas you may
append \textbf{+} to select the maximum map dimension.  Without the modifier the map width is
selected by default.

In \GMT\ version 4.3.0 we noticed we ran out of the alphabet for 1-letter (and sometimes 2-letter) projection codes. To allow more flexibility, and to make it easier to remember the codes, we implemented the option to use the abbreviations used by the \progname{Proj4} mapping package. Since some of the \GMT\ projections are not in \progname{Proj4}, we invented some of our own as well. For a full list of both the old 1- and 2-letter codes, as well as the \progname{Proj4}-equivalents see the quick reference cards in Section~\ref{sec:purpose}. For example, \Opt{JM15c} and \Opt{JMerc/15c} have the same meaning.

%------------------------------------------
%	$Id: GMT_Chapter_6.tex,v 1.19 2008-04-30 03:53:30 remko Exp $
%
%	The GMT Documentation Project
%	Copyright 2000-2008.
%	Paul Wessel and Walter H. F. Smith
%------------------------------------------
%
\chapter{\gmt\ Map Projections}
\label{ch:6}

\GMT\ implements more than 30 different projections.  They all project the input coordinates
longitude and latitude to positions on a map.  In general, $x' = f(x,y,z)$ and $y' = g(x,y,z)$, where
$z$ is implicitly given as the radial vector length to the $(x,y)$ point on the chosen ellipsoid.  The functions $f$ and $g$ can be
quite nasty and we will refrain from presenting details in this document.  The interested read is referred to
{\it Snyder} [1987]\footnote{Snyder, J. P., 1987, Map Projections \- A Working Manual, U.S. Geological Survey Prof. Paper 1395.}.
We will mostly be using the \GMTprog{pscoast} command to demonstrate each of the projections.
\GMT\ map projections are grouped into four categories depending on the
nature of the projection.  The groups are

\begin{enumerate}
\item Conic map projections
\item Azimuthal map projections
\item Cylindrical map projections
\item Miscellaneous projections
\end{enumerate}

Because $x$ and $y$ are coupled we can only specify one plot-dimensional scale, typically
a map \emph{scale} (for lower-case map projection code) or a map \emph{width} (for upper-case
map projection code).  However, in some cases it would be more
practical to specify map \emph{height} instead of \emph{width}, while in other situations it would be nice
to set either the \emph{shortest} or \emph{longest} map dimension.  Users may select
these alternatives by appending a character code to their map dimension.  To specify map \emph{height},
append \textbf{h} to the given dimension; to select the minimum map dimension, append \textbf{-}, whereas you may
append \textbf{+} to select the maximum map dimension.  Without the modifier the map width is
selected by default.

In \GMT\ version 4.3.0 we noticed we ran out of the alphabet for 1-letter (and sometimes 2-letter) projection codes. To allow more flexibility, and to make it easier to remember the codes, we implemented the option to use the abbreviations used by the \progname{Proj4} mapping package. Since some of the \GMT\ projections are not in \progname{Proj4}, we invented some of our own as well. For a full list of both the old 1- and 2-letter codes, as well as the \progname{Proj4}-equivalents see the quick reference cards in Section~\ref{sec:purpose}. For example, \Opt{JM15c} and \Opt{JMerc/15c} have the same meaning.

%------------------------------------------
%	$Id: GMT_Chapter_6.tex,v 1.19 2008-04-30 03:53:30 remko Exp $
%
%	The GMT Documentation Project
%	Copyright 2000-2008.
%	Paul Wessel and Walter H. F. Smith
%------------------------------------------
%
\chapter{\gmt\ Map Projections}
\label{ch:6}

\GMT\ implements more than 30 different projections.  They all project the input coordinates
longitude and latitude to positions on a map.  In general, $x' = f(x,y,z)$ and $y' = g(x,y,z)$, where
$z$ is implicitly given as the radial vector length to the $(x,y)$ point on the chosen ellipsoid.  The functions $f$ and $g$ can be
quite nasty and we will refrain from presenting details in this document.  The interested read is referred to
{\it Snyder} [1987]\footnote{Snyder, J. P., 1987, Map Projections \- A Working Manual, U.S. Geological Survey Prof. Paper 1395.}.
We will mostly be using the \GMTprog{pscoast} command to demonstrate each of the projections.
\GMT\ map projections are grouped into four categories depending on the
nature of the projection.  The groups are

\begin{enumerate}
\item Conic map projections
\item Azimuthal map projections
\item Cylindrical map projections
\item Miscellaneous projections
\end{enumerate}

Because $x$ and $y$ are coupled we can only specify one plot-dimensional scale, typically
a map \emph{scale} (for lower-case map projection code) or a map \emph{width} (for upper-case
map projection code).  However, in some cases it would be more
practical to specify map \emph{height} instead of \emph{width}, while in other situations it would be nice
to set either the \emph{shortest} or \emph{longest} map dimension.  Users may select
these alternatives by appending a character code to their map dimension.  To specify map \emph{height},
append \textbf{h} to the given dimension; to select the minimum map dimension, append \textbf{-}, whereas you may
append \textbf{+} to select the maximum map dimension.  Without the modifier the map width is
selected by default.

In \GMT\ version 4.3.0 we noticed we ran out of the alphabet for 1-letter (and sometimes 2-letter) projection codes. To allow more flexibility, and to make it easier to remember the codes, we implemented the option to use the abbreviations used by the \progname{Proj4} mapping package. Since some of the \GMT\ projections are not in \progname{Proj4}, we invented some of our own as well. For a full list of both the old 1- and 2-letter codes, as well as the \progname{Proj4}-equivalents see the quick reference cards in Section~\ref{sec:purpose}. For example, \Opt{JM15c} and \Opt{JMerc/15c} have the same meaning.

\input{GMT_Chapter_6.1}
\input{GMT_Chapter_6.2}
\input{GMT_Chapter_6.3}
\input{GMT_Chapter_6.4}

%------------------------------------------
%	$Id: GMT_Chapter_6.tex,v 1.19 2008-04-30 03:53:30 remko Exp $
%
%	The GMT Documentation Project
%	Copyright 2000-2008.
%	Paul Wessel and Walter H. F. Smith
%------------------------------------------
%
\chapter{\gmt\ Map Projections}
\label{ch:6}

\GMT\ implements more than 30 different projections.  They all project the input coordinates
longitude and latitude to positions on a map.  In general, $x' = f(x,y,z)$ and $y' = g(x,y,z)$, where
$z$ is implicitly given as the radial vector length to the $(x,y)$ point on the chosen ellipsoid.  The functions $f$ and $g$ can be
quite nasty and we will refrain from presenting details in this document.  The interested read is referred to
{\it Snyder} [1987]\footnote{Snyder, J. P., 1987, Map Projections \- A Working Manual, U.S. Geological Survey Prof. Paper 1395.}.
We will mostly be using the \GMTprog{pscoast} command to demonstrate each of the projections.
\GMT\ map projections are grouped into four categories depending on the
nature of the projection.  The groups are

\begin{enumerate}
\item Conic map projections
\item Azimuthal map projections
\item Cylindrical map projections
\item Miscellaneous projections
\end{enumerate}

Because $x$ and $y$ are coupled we can only specify one plot-dimensional scale, typically
a map \emph{scale} (for lower-case map projection code) or a map \emph{width} (for upper-case
map projection code).  However, in some cases it would be more
practical to specify map \emph{height} instead of \emph{width}, while in other situations it would be nice
to set either the \emph{shortest} or \emph{longest} map dimension.  Users may select
these alternatives by appending a character code to their map dimension.  To specify map \emph{height},
append \textbf{h} to the given dimension; to select the minimum map dimension, append \textbf{-}, whereas you may
append \textbf{+} to select the maximum map dimension.  Without the modifier the map width is
selected by default.

In \GMT\ version 4.3.0 we noticed we ran out of the alphabet for 1-letter (and sometimes 2-letter) projection codes. To allow more flexibility, and to make it easier to remember the codes, we implemented the option to use the abbreviations used by the \progname{Proj4} mapping package. Since some of the \GMT\ projections are not in \progname{Proj4}, we invented some of our own as well. For a full list of both the old 1- and 2-letter codes, as well as the \progname{Proj4}-equivalents see the quick reference cards in Section~\ref{sec:purpose}. For example, \Opt{JM15c} and \Opt{JMerc/15c} have the same meaning.

\input{GMT_Chapter_6.1}
\input{GMT_Chapter_6.2}
\input{GMT_Chapter_6.3}
\input{GMT_Chapter_6.4}

%------------------------------------------
%	$Id: GMT_Chapter_6.tex,v 1.19 2008-04-30 03:53:30 remko Exp $
%
%	The GMT Documentation Project
%	Copyright 2000-2008.
%	Paul Wessel and Walter H. F. Smith
%------------------------------------------
%
\chapter{\gmt\ Map Projections}
\label{ch:6}

\GMT\ implements more than 30 different projections.  They all project the input coordinates
longitude and latitude to positions on a map.  In general, $x' = f(x,y,z)$ and $y' = g(x,y,z)$, where
$z$ is implicitly given as the radial vector length to the $(x,y)$ point on the chosen ellipsoid.  The functions $f$ and $g$ can be
quite nasty and we will refrain from presenting details in this document.  The interested read is referred to
{\it Snyder} [1987]\footnote{Snyder, J. P., 1987, Map Projections \- A Working Manual, U.S. Geological Survey Prof. Paper 1395.}.
We will mostly be using the \GMTprog{pscoast} command to demonstrate each of the projections.
\GMT\ map projections are grouped into four categories depending on the
nature of the projection.  The groups are

\begin{enumerate}
\item Conic map projections
\item Azimuthal map projections
\item Cylindrical map projections
\item Miscellaneous projections
\end{enumerate}

Because $x$ and $y$ are coupled we can only specify one plot-dimensional scale, typically
a map \emph{scale} (for lower-case map projection code) or a map \emph{width} (for upper-case
map projection code).  However, in some cases it would be more
practical to specify map \emph{height} instead of \emph{width}, while in other situations it would be nice
to set either the \emph{shortest} or \emph{longest} map dimension.  Users may select
these alternatives by appending a character code to their map dimension.  To specify map \emph{height},
append \textbf{h} to the given dimension; to select the minimum map dimension, append \textbf{-}, whereas you may
append \textbf{+} to select the maximum map dimension.  Without the modifier the map width is
selected by default.

In \GMT\ version 4.3.0 we noticed we ran out of the alphabet for 1-letter (and sometimes 2-letter) projection codes. To allow more flexibility, and to make it easier to remember the codes, we implemented the option to use the abbreviations used by the \progname{Proj4} mapping package. Since some of the \GMT\ projections are not in \progname{Proj4}, we invented some of our own as well. For a full list of both the old 1- and 2-letter codes, as well as the \progname{Proj4}-equivalents see the quick reference cards in Section~\ref{sec:purpose}. For example, \Opt{JM15c} and \Opt{JMerc/15c} have the same meaning.

\input{GMT_Chapter_6.1}
\input{GMT_Chapter_6.2}
\input{GMT_Chapter_6.3}
\input{GMT_Chapter_6.4}

%------------------------------------------
%	$Id: GMT_Chapter_6.tex,v 1.19 2008-04-30 03:53:30 remko Exp $
%
%	The GMT Documentation Project
%	Copyright 2000-2008.
%	Paul Wessel and Walter H. F. Smith
%------------------------------------------
%
\chapter{\gmt\ Map Projections}
\label{ch:6}

\GMT\ implements more than 30 different projections.  They all project the input coordinates
longitude and latitude to positions on a map.  In general, $x' = f(x,y,z)$ and $y' = g(x,y,z)$, where
$z$ is implicitly given as the radial vector length to the $(x,y)$ point on the chosen ellipsoid.  The functions $f$ and $g$ can be
quite nasty and we will refrain from presenting details in this document.  The interested read is referred to
{\it Snyder} [1987]\footnote{Snyder, J. P., 1987, Map Projections \- A Working Manual, U.S. Geological Survey Prof. Paper 1395.}.
We will mostly be using the \GMTprog{pscoast} command to demonstrate each of the projections.
\GMT\ map projections are grouped into four categories depending on the
nature of the projection.  The groups are

\begin{enumerate}
\item Conic map projections
\item Azimuthal map projections
\item Cylindrical map projections
\item Miscellaneous projections
\end{enumerate}

Because $x$ and $y$ are coupled we can only specify one plot-dimensional scale, typically
a map \emph{scale} (for lower-case map projection code) or a map \emph{width} (for upper-case
map projection code).  However, in some cases it would be more
practical to specify map \emph{height} instead of \emph{width}, while in other situations it would be nice
to set either the \emph{shortest} or \emph{longest} map dimension.  Users may select
these alternatives by appending a character code to their map dimension.  To specify map \emph{height},
append \textbf{h} to the given dimension; to select the minimum map dimension, append \textbf{-}, whereas you may
append \textbf{+} to select the maximum map dimension.  Without the modifier the map width is
selected by default.

In \GMT\ version 4.3.0 we noticed we ran out of the alphabet for 1-letter (and sometimes 2-letter) projection codes. To allow more flexibility, and to make it easier to remember the codes, we implemented the option to use the abbreviations used by the \progname{Proj4} mapping package. Since some of the \GMT\ projections are not in \progname{Proj4}, we invented some of our own as well. For a full list of both the old 1- and 2-letter codes, as well as the \progname{Proj4}-equivalents see the quick reference cards in Section~\ref{sec:purpose}. For example, \Opt{JM15c} and \Opt{JMerc/15c} have the same meaning.

\input{GMT_Chapter_6.1}
\input{GMT_Chapter_6.2}
\input{GMT_Chapter_6.3}
\input{GMT_Chapter_6.4}


%------------------------------------------
%	$Id: GMT_Chapter_6.tex,v 1.19 2008-04-30 03:53:30 remko Exp $
%
%	The GMT Documentation Project
%	Copyright 2000-2008.
%	Paul Wessel and Walter H. F. Smith
%------------------------------------------
%
\chapter{\gmt\ Map Projections}
\label{ch:6}

\GMT\ implements more than 30 different projections.  They all project the input coordinates
longitude and latitude to positions on a map.  In general, $x' = f(x,y,z)$ and $y' = g(x,y,z)$, where
$z$ is implicitly given as the radial vector length to the $(x,y)$ point on the chosen ellipsoid.  The functions $f$ and $g$ can be
quite nasty and we will refrain from presenting details in this document.  The interested read is referred to
{\it Snyder} [1987]\footnote{Snyder, J. P., 1987, Map Projections \- A Working Manual, U.S. Geological Survey Prof. Paper 1395.}.
We will mostly be using the \GMTprog{pscoast} command to demonstrate each of the projections.
\GMT\ map projections are grouped into four categories depending on the
nature of the projection.  The groups are

\begin{enumerate}
\item Conic map projections
\item Azimuthal map projections
\item Cylindrical map projections
\item Miscellaneous projections
\end{enumerate}

Because $x$ and $y$ are coupled we can only specify one plot-dimensional scale, typically
a map \emph{scale} (for lower-case map projection code) or a map \emph{width} (for upper-case
map projection code).  However, in some cases it would be more
practical to specify map \emph{height} instead of \emph{width}, while in other situations it would be nice
to set either the \emph{shortest} or \emph{longest} map dimension.  Users may select
these alternatives by appending a character code to their map dimension.  To specify map \emph{height},
append \textbf{h} to the given dimension; to select the minimum map dimension, append \textbf{-}, whereas you may
append \textbf{+} to select the maximum map dimension.  Without the modifier the map width is
selected by default.

In \GMT\ version 4.3.0 we noticed we ran out of the alphabet for 1-letter (and sometimes 2-letter) projection codes. To allow more flexibility, and to make it easier to remember the codes, we implemented the option to use the abbreviations used by the \progname{Proj4} mapping package. Since some of the \GMT\ projections are not in \progname{Proj4}, we invented some of our own as well. For a full list of both the old 1- and 2-letter codes, as well as the \progname{Proj4}-equivalents see the quick reference cards in Section~\ref{sec:purpose}. For example, \Opt{JM15c} and \Opt{JMerc/15c} have the same meaning.

%------------------------------------------
%	$Id: GMT_Chapter_6.tex,v 1.19 2008-04-30 03:53:30 remko Exp $
%
%	The GMT Documentation Project
%	Copyright 2000-2008.
%	Paul Wessel and Walter H. F. Smith
%------------------------------------------
%
\chapter{\gmt\ Map Projections}
\label{ch:6}

\GMT\ implements more than 30 different projections.  They all project the input coordinates
longitude and latitude to positions on a map.  In general, $x' = f(x,y,z)$ and $y' = g(x,y,z)$, where
$z$ is implicitly given as the radial vector length to the $(x,y)$ point on the chosen ellipsoid.  The functions $f$ and $g$ can be
quite nasty and we will refrain from presenting details in this document.  The interested read is referred to
{\it Snyder} [1987]\footnote{Snyder, J. P., 1987, Map Projections \- A Working Manual, U.S. Geological Survey Prof. Paper 1395.}.
We will mostly be using the \GMTprog{pscoast} command to demonstrate each of the projections.
\GMT\ map projections are grouped into four categories depending on the
nature of the projection.  The groups are

\begin{enumerate}
\item Conic map projections
\item Azimuthal map projections
\item Cylindrical map projections
\item Miscellaneous projections
\end{enumerate}

Because $x$ and $y$ are coupled we can only specify one plot-dimensional scale, typically
a map \emph{scale} (for lower-case map projection code) or a map \emph{width} (for upper-case
map projection code).  However, in some cases it would be more
practical to specify map \emph{height} instead of \emph{width}, while in other situations it would be nice
to set either the \emph{shortest} or \emph{longest} map dimension.  Users may select
these alternatives by appending a character code to their map dimension.  To specify map \emph{height},
append \textbf{h} to the given dimension; to select the minimum map dimension, append \textbf{-}, whereas you may
append \textbf{+} to select the maximum map dimension.  Without the modifier the map width is
selected by default.

In \GMT\ version 4.3.0 we noticed we ran out of the alphabet for 1-letter (and sometimes 2-letter) projection codes. To allow more flexibility, and to make it easier to remember the codes, we implemented the option to use the abbreviations used by the \progname{Proj4} mapping package. Since some of the \GMT\ projections are not in \progname{Proj4}, we invented some of our own as well. For a full list of both the old 1- and 2-letter codes, as well as the \progname{Proj4}-equivalents see the quick reference cards in Section~\ref{sec:purpose}. For example, \Opt{JM15c} and \Opt{JMerc/15c} have the same meaning.

\input{GMT_Chapter_6.1}
\input{GMT_Chapter_6.2}
\input{GMT_Chapter_6.3}
\input{GMT_Chapter_6.4}

%------------------------------------------
%	$Id: GMT_Chapter_6.tex,v 1.19 2008-04-30 03:53:30 remko Exp $
%
%	The GMT Documentation Project
%	Copyright 2000-2008.
%	Paul Wessel and Walter H. F. Smith
%------------------------------------------
%
\chapter{\gmt\ Map Projections}
\label{ch:6}

\GMT\ implements more than 30 different projections.  They all project the input coordinates
longitude and latitude to positions on a map.  In general, $x' = f(x,y,z)$ and $y' = g(x,y,z)$, where
$z$ is implicitly given as the radial vector length to the $(x,y)$ point on the chosen ellipsoid.  The functions $f$ and $g$ can be
quite nasty and we will refrain from presenting details in this document.  The interested read is referred to
{\it Snyder} [1987]\footnote{Snyder, J. P., 1987, Map Projections \- A Working Manual, U.S. Geological Survey Prof. Paper 1395.}.
We will mostly be using the \GMTprog{pscoast} command to demonstrate each of the projections.
\GMT\ map projections are grouped into four categories depending on the
nature of the projection.  The groups are

\begin{enumerate}
\item Conic map projections
\item Azimuthal map projections
\item Cylindrical map projections
\item Miscellaneous projections
\end{enumerate}

Because $x$ and $y$ are coupled we can only specify one plot-dimensional scale, typically
a map \emph{scale} (for lower-case map projection code) or a map \emph{width} (for upper-case
map projection code).  However, in some cases it would be more
practical to specify map \emph{height} instead of \emph{width}, while in other situations it would be nice
to set either the \emph{shortest} or \emph{longest} map dimension.  Users may select
these alternatives by appending a character code to their map dimension.  To specify map \emph{height},
append \textbf{h} to the given dimension; to select the minimum map dimension, append \textbf{-}, whereas you may
append \textbf{+} to select the maximum map dimension.  Without the modifier the map width is
selected by default.

In \GMT\ version 4.3.0 we noticed we ran out of the alphabet for 1-letter (and sometimes 2-letter) projection codes. To allow more flexibility, and to make it easier to remember the codes, we implemented the option to use the abbreviations used by the \progname{Proj4} mapping package. Since some of the \GMT\ projections are not in \progname{Proj4}, we invented some of our own as well. For a full list of both the old 1- and 2-letter codes, as well as the \progname{Proj4}-equivalents see the quick reference cards in Section~\ref{sec:purpose}. For example, \Opt{JM15c} and \Opt{JMerc/15c} have the same meaning.

\input{GMT_Chapter_6.1}
\input{GMT_Chapter_6.2}
\input{GMT_Chapter_6.3}
\input{GMT_Chapter_6.4}

%------------------------------------------
%	$Id: GMT_Chapter_6.tex,v 1.19 2008-04-30 03:53:30 remko Exp $
%
%	The GMT Documentation Project
%	Copyright 2000-2008.
%	Paul Wessel and Walter H. F. Smith
%------------------------------------------
%
\chapter{\gmt\ Map Projections}
\label{ch:6}

\GMT\ implements more than 30 different projections.  They all project the input coordinates
longitude and latitude to positions on a map.  In general, $x' = f(x,y,z)$ and $y' = g(x,y,z)$, where
$z$ is implicitly given as the radial vector length to the $(x,y)$ point on the chosen ellipsoid.  The functions $f$ and $g$ can be
quite nasty and we will refrain from presenting details in this document.  The interested read is referred to
{\it Snyder} [1987]\footnote{Snyder, J. P., 1987, Map Projections \- A Working Manual, U.S. Geological Survey Prof. Paper 1395.}.
We will mostly be using the \GMTprog{pscoast} command to demonstrate each of the projections.
\GMT\ map projections are grouped into four categories depending on the
nature of the projection.  The groups are

\begin{enumerate}
\item Conic map projections
\item Azimuthal map projections
\item Cylindrical map projections
\item Miscellaneous projections
\end{enumerate}

Because $x$ and $y$ are coupled we can only specify one plot-dimensional scale, typically
a map \emph{scale} (for lower-case map projection code) or a map \emph{width} (for upper-case
map projection code).  However, in some cases it would be more
practical to specify map \emph{height} instead of \emph{width}, while in other situations it would be nice
to set either the \emph{shortest} or \emph{longest} map dimension.  Users may select
these alternatives by appending a character code to their map dimension.  To specify map \emph{height},
append \textbf{h} to the given dimension; to select the minimum map dimension, append \textbf{-}, whereas you may
append \textbf{+} to select the maximum map dimension.  Without the modifier the map width is
selected by default.

In \GMT\ version 4.3.0 we noticed we ran out of the alphabet for 1-letter (and sometimes 2-letter) projection codes. To allow more flexibility, and to make it easier to remember the codes, we implemented the option to use the abbreviations used by the \progname{Proj4} mapping package. Since some of the \GMT\ projections are not in \progname{Proj4}, we invented some of our own as well. For a full list of both the old 1- and 2-letter codes, as well as the \progname{Proj4}-equivalents see the quick reference cards in Section~\ref{sec:purpose}. For example, \Opt{JM15c} and \Opt{JMerc/15c} have the same meaning.

\input{GMT_Chapter_6.1}
\input{GMT_Chapter_6.2}
\input{GMT_Chapter_6.3}
\input{GMT_Chapter_6.4}

%------------------------------------------
%	$Id: GMT_Chapter_6.tex,v 1.19 2008-04-30 03:53:30 remko Exp $
%
%	The GMT Documentation Project
%	Copyright 2000-2008.
%	Paul Wessel and Walter H. F. Smith
%------------------------------------------
%
\chapter{\gmt\ Map Projections}
\label{ch:6}

\GMT\ implements more than 30 different projections.  They all project the input coordinates
longitude and latitude to positions on a map.  In general, $x' = f(x,y,z)$ and $y' = g(x,y,z)$, where
$z$ is implicitly given as the radial vector length to the $(x,y)$ point on the chosen ellipsoid.  The functions $f$ and $g$ can be
quite nasty and we will refrain from presenting details in this document.  The interested read is referred to
{\it Snyder} [1987]\footnote{Snyder, J. P., 1987, Map Projections \- A Working Manual, U.S. Geological Survey Prof. Paper 1395.}.
We will mostly be using the \GMTprog{pscoast} command to demonstrate each of the projections.
\GMT\ map projections are grouped into four categories depending on the
nature of the projection.  The groups are

\begin{enumerate}
\item Conic map projections
\item Azimuthal map projections
\item Cylindrical map projections
\item Miscellaneous projections
\end{enumerate}

Because $x$ and $y$ are coupled we can only specify one plot-dimensional scale, typically
a map \emph{scale} (for lower-case map projection code) or a map \emph{width} (for upper-case
map projection code).  However, in some cases it would be more
practical to specify map \emph{height} instead of \emph{width}, while in other situations it would be nice
to set either the \emph{shortest} or \emph{longest} map dimension.  Users may select
these alternatives by appending a character code to their map dimension.  To specify map \emph{height},
append \textbf{h} to the given dimension; to select the minimum map dimension, append \textbf{-}, whereas you may
append \textbf{+} to select the maximum map dimension.  Without the modifier the map width is
selected by default.

In \GMT\ version 4.3.0 we noticed we ran out of the alphabet for 1-letter (and sometimes 2-letter) projection codes. To allow more flexibility, and to make it easier to remember the codes, we implemented the option to use the abbreviations used by the \progname{Proj4} mapping package. Since some of the \GMT\ projections are not in \progname{Proj4}, we invented some of our own as well. For a full list of both the old 1- and 2-letter codes, as well as the \progname{Proj4}-equivalents see the quick reference cards in Section~\ref{sec:purpose}. For example, \Opt{JM15c} and \Opt{JMerc/15c} have the same meaning.

\input{GMT_Chapter_6.1}
\input{GMT_Chapter_6.2}
\input{GMT_Chapter_6.3}
\input{GMT_Chapter_6.4}


%------------------------------------------
%	$Id: GMT_Chapter_6.tex,v 1.19 2008-04-30 03:53:30 remko Exp $
%
%	The GMT Documentation Project
%	Copyright 2000-2008.
%	Paul Wessel and Walter H. F. Smith
%------------------------------------------
%
\chapter{\gmt\ Map Projections}
\label{ch:6}

\GMT\ implements more than 30 different projections.  They all project the input coordinates
longitude and latitude to positions on a map.  In general, $x' = f(x,y,z)$ and $y' = g(x,y,z)$, where
$z$ is implicitly given as the radial vector length to the $(x,y)$ point on the chosen ellipsoid.  The functions $f$ and $g$ can be
quite nasty and we will refrain from presenting details in this document.  The interested read is referred to
{\it Snyder} [1987]\footnote{Snyder, J. P., 1987, Map Projections \- A Working Manual, U.S. Geological Survey Prof. Paper 1395.}.
We will mostly be using the \GMTprog{pscoast} command to demonstrate each of the projections.
\GMT\ map projections are grouped into four categories depending on the
nature of the projection.  The groups are

\begin{enumerate}
\item Conic map projections
\item Azimuthal map projections
\item Cylindrical map projections
\item Miscellaneous projections
\end{enumerate}

Because $x$ and $y$ are coupled we can only specify one plot-dimensional scale, typically
a map \emph{scale} (for lower-case map projection code) or a map \emph{width} (for upper-case
map projection code).  However, in some cases it would be more
practical to specify map \emph{height} instead of \emph{width}, while in other situations it would be nice
to set either the \emph{shortest} or \emph{longest} map dimension.  Users may select
these alternatives by appending a character code to their map dimension.  To specify map \emph{height},
append \textbf{h} to the given dimension; to select the minimum map dimension, append \textbf{-}, whereas you may
append \textbf{+} to select the maximum map dimension.  Without the modifier the map width is
selected by default.

In \GMT\ version 4.3.0 we noticed we ran out of the alphabet for 1-letter (and sometimes 2-letter) projection codes. To allow more flexibility, and to make it easier to remember the codes, we implemented the option to use the abbreviations used by the \progname{Proj4} mapping package. Since some of the \GMT\ projections are not in \progname{Proj4}, we invented some of our own as well. For a full list of both the old 1- and 2-letter codes, as well as the \progname{Proj4}-equivalents see the quick reference cards in Section~\ref{sec:purpose}. For example, \Opt{JM15c} and \Opt{JMerc/15c} have the same meaning.

%------------------------------------------
%	$Id: GMT_Chapter_6.tex,v 1.19 2008-04-30 03:53:30 remko Exp $
%
%	The GMT Documentation Project
%	Copyright 2000-2008.
%	Paul Wessel and Walter H. F. Smith
%------------------------------------------
%
\chapter{\gmt\ Map Projections}
\label{ch:6}

\GMT\ implements more than 30 different projections.  They all project the input coordinates
longitude and latitude to positions on a map.  In general, $x' = f(x,y,z)$ and $y' = g(x,y,z)$, where
$z$ is implicitly given as the radial vector length to the $(x,y)$ point on the chosen ellipsoid.  The functions $f$ and $g$ can be
quite nasty and we will refrain from presenting details in this document.  The interested read is referred to
{\it Snyder} [1987]\footnote{Snyder, J. P., 1987, Map Projections \- A Working Manual, U.S. Geological Survey Prof. Paper 1395.}.
We will mostly be using the \GMTprog{pscoast} command to demonstrate each of the projections.
\GMT\ map projections are grouped into four categories depending on the
nature of the projection.  The groups are

\begin{enumerate}
\item Conic map projections
\item Azimuthal map projections
\item Cylindrical map projections
\item Miscellaneous projections
\end{enumerate}

Because $x$ and $y$ are coupled we can only specify one plot-dimensional scale, typically
a map \emph{scale} (for lower-case map projection code) or a map \emph{width} (for upper-case
map projection code).  However, in some cases it would be more
practical to specify map \emph{height} instead of \emph{width}, while in other situations it would be nice
to set either the \emph{shortest} or \emph{longest} map dimension.  Users may select
these alternatives by appending a character code to their map dimension.  To specify map \emph{height},
append \textbf{h} to the given dimension; to select the minimum map dimension, append \textbf{-}, whereas you may
append \textbf{+} to select the maximum map dimension.  Without the modifier the map width is
selected by default.

In \GMT\ version 4.3.0 we noticed we ran out of the alphabet for 1-letter (and sometimes 2-letter) projection codes. To allow more flexibility, and to make it easier to remember the codes, we implemented the option to use the abbreviations used by the \progname{Proj4} mapping package. Since some of the \GMT\ projections are not in \progname{Proj4}, we invented some of our own as well. For a full list of both the old 1- and 2-letter codes, as well as the \progname{Proj4}-equivalents see the quick reference cards in Section~\ref{sec:purpose}. For example, \Opt{JM15c} and \Opt{JMerc/15c} have the same meaning.

\input{GMT_Chapter_6.1}
\input{GMT_Chapter_6.2}
\input{GMT_Chapter_6.3}
\input{GMT_Chapter_6.4}

%------------------------------------------
%	$Id: GMT_Chapter_6.tex,v 1.19 2008-04-30 03:53:30 remko Exp $
%
%	The GMT Documentation Project
%	Copyright 2000-2008.
%	Paul Wessel and Walter H. F. Smith
%------------------------------------------
%
\chapter{\gmt\ Map Projections}
\label{ch:6}

\GMT\ implements more than 30 different projections.  They all project the input coordinates
longitude and latitude to positions on a map.  In general, $x' = f(x,y,z)$ and $y' = g(x,y,z)$, where
$z$ is implicitly given as the radial vector length to the $(x,y)$ point on the chosen ellipsoid.  The functions $f$ and $g$ can be
quite nasty and we will refrain from presenting details in this document.  The interested read is referred to
{\it Snyder} [1987]\footnote{Snyder, J. P., 1987, Map Projections \- A Working Manual, U.S. Geological Survey Prof. Paper 1395.}.
We will mostly be using the \GMTprog{pscoast} command to demonstrate each of the projections.
\GMT\ map projections are grouped into four categories depending on the
nature of the projection.  The groups are

\begin{enumerate}
\item Conic map projections
\item Azimuthal map projections
\item Cylindrical map projections
\item Miscellaneous projections
\end{enumerate}

Because $x$ and $y$ are coupled we can only specify one plot-dimensional scale, typically
a map \emph{scale} (for lower-case map projection code) or a map \emph{width} (for upper-case
map projection code).  However, in some cases it would be more
practical to specify map \emph{height} instead of \emph{width}, while in other situations it would be nice
to set either the \emph{shortest} or \emph{longest} map dimension.  Users may select
these alternatives by appending a character code to their map dimension.  To specify map \emph{height},
append \textbf{h} to the given dimension; to select the minimum map dimension, append \textbf{-}, whereas you may
append \textbf{+} to select the maximum map dimension.  Without the modifier the map width is
selected by default.

In \GMT\ version 4.3.0 we noticed we ran out of the alphabet for 1-letter (and sometimes 2-letter) projection codes. To allow more flexibility, and to make it easier to remember the codes, we implemented the option to use the abbreviations used by the \progname{Proj4} mapping package. Since some of the \GMT\ projections are not in \progname{Proj4}, we invented some of our own as well. For a full list of both the old 1- and 2-letter codes, as well as the \progname{Proj4}-equivalents see the quick reference cards in Section~\ref{sec:purpose}. For example, \Opt{JM15c} and \Opt{JMerc/15c} have the same meaning.

\input{GMT_Chapter_6.1}
\input{GMT_Chapter_6.2}
\input{GMT_Chapter_6.3}
\input{GMT_Chapter_6.4}

%------------------------------------------
%	$Id: GMT_Chapter_6.tex,v 1.19 2008-04-30 03:53:30 remko Exp $
%
%	The GMT Documentation Project
%	Copyright 2000-2008.
%	Paul Wessel and Walter H. F. Smith
%------------------------------------------
%
\chapter{\gmt\ Map Projections}
\label{ch:6}

\GMT\ implements more than 30 different projections.  They all project the input coordinates
longitude and latitude to positions on a map.  In general, $x' = f(x,y,z)$ and $y' = g(x,y,z)$, where
$z$ is implicitly given as the radial vector length to the $(x,y)$ point on the chosen ellipsoid.  The functions $f$ and $g$ can be
quite nasty and we will refrain from presenting details in this document.  The interested read is referred to
{\it Snyder} [1987]\footnote{Snyder, J. P., 1987, Map Projections \- A Working Manual, U.S. Geological Survey Prof. Paper 1395.}.
We will mostly be using the \GMTprog{pscoast} command to demonstrate each of the projections.
\GMT\ map projections are grouped into four categories depending on the
nature of the projection.  The groups are

\begin{enumerate}
\item Conic map projections
\item Azimuthal map projections
\item Cylindrical map projections
\item Miscellaneous projections
\end{enumerate}

Because $x$ and $y$ are coupled we can only specify one plot-dimensional scale, typically
a map \emph{scale} (for lower-case map projection code) or a map \emph{width} (for upper-case
map projection code).  However, in some cases it would be more
practical to specify map \emph{height} instead of \emph{width}, while in other situations it would be nice
to set either the \emph{shortest} or \emph{longest} map dimension.  Users may select
these alternatives by appending a character code to their map dimension.  To specify map \emph{height},
append \textbf{h} to the given dimension; to select the minimum map dimension, append \textbf{-}, whereas you may
append \textbf{+} to select the maximum map dimension.  Without the modifier the map width is
selected by default.

In \GMT\ version 4.3.0 we noticed we ran out of the alphabet for 1-letter (and sometimes 2-letter) projection codes. To allow more flexibility, and to make it easier to remember the codes, we implemented the option to use the abbreviations used by the \progname{Proj4} mapping package. Since some of the \GMT\ projections are not in \progname{Proj4}, we invented some of our own as well. For a full list of both the old 1- and 2-letter codes, as well as the \progname{Proj4}-equivalents see the quick reference cards in Section~\ref{sec:purpose}. For example, \Opt{JM15c} and \Opt{JMerc/15c} have the same meaning.

\input{GMT_Chapter_6.1}
\input{GMT_Chapter_6.2}
\input{GMT_Chapter_6.3}
\input{GMT_Chapter_6.4}

%------------------------------------------
%	$Id: GMT_Chapter_6.tex,v 1.19 2008-04-30 03:53:30 remko Exp $
%
%	The GMT Documentation Project
%	Copyright 2000-2008.
%	Paul Wessel and Walter H. F. Smith
%------------------------------------------
%
\chapter{\gmt\ Map Projections}
\label{ch:6}

\GMT\ implements more than 30 different projections.  They all project the input coordinates
longitude and latitude to positions on a map.  In general, $x' = f(x,y,z)$ and $y' = g(x,y,z)$, where
$z$ is implicitly given as the radial vector length to the $(x,y)$ point on the chosen ellipsoid.  The functions $f$ and $g$ can be
quite nasty and we will refrain from presenting details in this document.  The interested read is referred to
{\it Snyder} [1987]\footnote{Snyder, J. P., 1987, Map Projections \- A Working Manual, U.S. Geological Survey Prof. Paper 1395.}.
We will mostly be using the \GMTprog{pscoast} command to demonstrate each of the projections.
\GMT\ map projections are grouped into four categories depending on the
nature of the projection.  The groups are

\begin{enumerate}
\item Conic map projections
\item Azimuthal map projections
\item Cylindrical map projections
\item Miscellaneous projections
\end{enumerate}

Because $x$ and $y$ are coupled we can only specify one plot-dimensional scale, typically
a map \emph{scale} (for lower-case map projection code) or a map \emph{width} (for upper-case
map projection code).  However, in some cases it would be more
practical to specify map \emph{height} instead of \emph{width}, while in other situations it would be nice
to set either the \emph{shortest} or \emph{longest} map dimension.  Users may select
these alternatives by appending a character code to their map dimension.  To specify map \emph{height},
append \textbf{h} to the given dimension; to select the minimum map dimension, append \textbf{-}, whereas you may
append \textbf{+} to select the maximum map dimension.  Without the modifier the map width is
selected by default.

In \GMT\ version 4.3.0 we noticed we ran out of the alphabet for 1-letter (and sometimes 2-letter) projection codes. To allow more flexibility, and to make it easier to remember the codes, we implemented the option to use the abbreviations used by the \progname{Proj4} mapping package. Since some of the \GMT\ projections are not in \progname{Proj4}, we invented some of our own as well. For a full list of both the old 1- and 2-letter codes, as well as the \progname{Proj4}-equivalents see the quick reference cards in Section~\ref{sec:purpose}. For example, \Opt{JM15c} and \Opt{JMerc/15c} have the same meaning.

\input{GMT_Chapter_6.1}
\input{GMT_Chapter_6.2}
\input{GMT_Chapter_6.3}
\input{GMT_Chapter_6.4}


%------------------------------------------
%	$Id: GMT_Chapter_6.tex,v 1.19 2008-04-30 03:53:30 remko Exp $
%
%	The GMT Documentation Project
%	Copyright 2000-2008.
%	Paul Wessel and Walter H. F. Smith
%------------------------------------------
%
\chapter{\gmt\ Map Projections}
\label{ch:6}

\GMT\ implements more than 30 different projections.  They all project the input coordinates
longitude and latitude to positions on a map.  In general, $x' = f(x,y,z)$ and $y' = g(x,y,z)$, where
$z$ is implicitly given as the radial vector length to the $(x,y)$ point on the chosen ellipsoid.  The functions $f$ and $g$ can be
quite nasty and we will refrain from presenting details in this document.  The interested read is referred to
{\it Snyder} [1987]\footnote{Snyder, J. P., 1987, Map Projections \- A Working Manual, U.S. Geological Survey Prof. Paper 1395.}.
We will mostly be using the \GMTprog{pscoast} command to demonstrate each of the projections.
\GMT\ map projections are grouped into four categories depending on the
nature of the projection.  The groups are

\begin{enumerate}
\item Conic map projections
\item Azimuthal map projections
\item Cylindrical map projections
\item Miscellaneous projections
\end{enumerate}

Because $x$ and $y$ are coupled we can only specify one plot-dimensional scale, typically
a map \emph{scale} (for lower-case map projection code) or a map \emph{width} (for upper-case
map projection code).  However, in some cases it would be more
practical to specify map \emph{height} instead of \emph{width}, while in other situations it would be nice
to set either the \emph{shortest} or \emph{longest} map dimension.  Users may select
these alternatives by appending a character code to their map dimension.  To specify map \emph{height},
append \textbf{h} to the given dimension; to select the minimum map dimension, append \textbf{-}, whereas you may
append \textbf{+} to select the maximum map dimension.  Without the modifier the map width is
selected by default.

In \GMT\ version 4.3.0 we noticed we ran out of the alphabet for 1-letter (and sometimes 2-letter) projection codes. To allow more flexibility, and to make it easier to remember the codes, we implemented the option to use the abbreviations used by the \progname{Proj4} mapping package. Since some of the \GMT\ projections are not in \progname{Proj4}, we invented some of our own as well. For a full list of both the old 1- and 2-letter codes, as well as the \progname{Proj4}-equivalents see the quick reference cards in Section~\ref{sec:purpose}. For example, \Opt{JM15c} and \Opt{JMerc/15c} have the same meaning.

%------------------------------------------
%	$Id: GMT_Chapter_6.tex,v 1.19 2008-04-30 03:53:30 remko Exp $
%
%	The GMT Documentation Project
%	Copyright 2000-2008.
%	Paul Wessel and Walter H. F. Smith
%------------------------------------------
%
\chapter{\gmt\ Map Projections}
\label{ch:6}

\GMT\ implements more than 30 different projections.  They all project the input coordinates
longitude and latitude to positions on a map.  In general, $x' = f(x,y,z)$ and $y' = g(x,y,z)$, where
$z$ is implicitly given as the radial vector length to the $(x,y)$ point on the chosen ellipsoid.  The functions $f$ and $g$ can be
quite nasty and we will refrain from presenting details in this document.  The interested read is referred to
{\it Snyder} [1987]\footnote{Snyder, J. P., 1987, Map Projections \- A Working Manual, U.S. Geological Survey Prof. Paper 1395.}.
We will mostly be using the \GMTprog{pscoast} command to demonstrate each of the projections.
\GMT\ map projections are grouped into four categories depending on the
nature of the projection.  The groups are

\begin{enumerate}
\item Conic map projections
\item Azimuthal map projections
\item Cylindrical map projections
\item Miscellaneous projections
\end{enumerate}

Because $x$ and $y$ are coupled we can only specify one plot-dimensional scale, typically
a map \emph{scale} (for lower-case map projection code) or a map \emph{width} (for upper-case
map projection code).  However, in some cases it would be more
practical to specify map \emph{height} instead of \emph{width}, while in other situations it would be nice
to set either the \emph{shortest} or \emph{longest} map dimension.  Users may select
these alternatives by appending a character code to their map dimension.  To specify map \emph{height},
append \textbf{h} to the given dimension; to select the minimum map dimension, append \textbf{-}, whereas you may
append \textbf{+} to select the maximum map dimension.  Without the modifier the map width is
selected by default.

In \GMT\ version 4.3.0 we noticed we ran out of the alphabet for 1-letter (and sometimes 2-letter) projection codes. To allow more flexibility, and to make it easier to remember the codes, we implemented the option to use the abbreviations used by the \progname{Proj4} mapping package. Since some of the \GMT\ projections are not in \progname{Proj4}, we invented some of our own as well. For a full list of both the old 1- and 2-letter codes, as well as the \progname{Proj4}-equivalents see the quick reference cards in Section~\ref{sec:purpose}. For example, \Opt{JM15c} and \Opt{JMerc/15c} have the same meaning.

\input{GMT_Chapter_6.1}
\input{GMT_Chapter_6.2}
\input{GMT_Chapter_6.3}
\input{GMT_Chapter_6.4}

%------------------------------------------
%	$Id: GMT_Chapter_6.tex,v 1.19 2008-04-30 03:53:30 remko Exp $
%
%	The GMT Documentation Project
%	Copyright 2000-2008.
%	Paul Wessel and Walter H. F. Smith
%------------------------------------------
%
\chapter{\gmt\ Map Projections}
\label{ch:6}

\GMT\ implements more than 30 different projections.  They all project the input coordinates
longitude and latitude to positions on a map.  In general, $x' = f(x,y,z)$ and $y' = g(x,y,z)$, where
$z$ is implicitly given as the radial vector length to the $(x,y)$ point on the chosen ellipsoid.  The functions $f$ and $g$ can be
quite nasty and we will refrain from presenting details in this document.  The interested read is referred to
{\it Snyder} [1987]\footnote{Snyder, J. P., 1987, Map Projections \- A Working Manual, U.S. Geological Survey Prof. Paper 1395.}.
We will mostly be using the \GMTprog{pscoast} command to demonstrate each of the projections.
\GMT\ map projections are grouped into four categories depending on the
nature of the projection.  The groups are

\begin{enumerate}
\item Conic map projections
\item Azimuthal map projections
\item Cylindrical map projections
\item Miscellaneous projections
\end{enumerate}

Because $x$ and $y$ are coupled we can only specify one plot-dimensional scale, typically
a map \emph{scale} (for lower-case map projection code) or a map \emph{width} (for upper-case
map projection code).  However, in some cases it would be more
practical to specify map \emph{height} instead of \emph{width}, while in other situations it would be nice
to set either the \emph{shortest} or \emph{longest} map dimension.  Users may select
these alternatives by appending a character code to their map dimension.  To specify map \emph{height},
append \textbf{h} to the given dimension; to select the minimum map dimension, append \textbf{-}, whereas you may
append \textbf{+} to select the maximum map dimension.  Without the modifier the map width is
selected by default.

In \GMT\ version 4.3.0 we noticed we ran out of the alphabet for 1-letter (and sometimes 2-letter) projection codes. To allow more flexibility, and to make it easier to remember the codes, we implemented the option to use the abbreviations used by the \progname{Proj4} mapping package. Since some of the \GMT\ projections are not in \progname{Proj4}, we invented some of our own as well. For a full list of both the old 1- and 2-letter codes, as well as the \progname{Proj4}-equivalents see the quick reference cards in Section~\ref{sec:purpose}. For example, \Opt{JM15c} and \Opt{JMerc/15c} have the same meaning.

\input{GMT_Chapter_6.1}
\input{GMT_Chapter_6.2}
\input{GMT_Chapter_6.3}
\input{GMT_Chapter_6.4}

%------------------------------------------
%	$Id: GMT_Chapter_6.tex,v 1.19 2008-04-30 03:53:30 remko Exp $
%
%	The GMT Documentation Project
%	Copyright 2000-2008.
%	Paul Wessel and Walter H. F. Smith
%------------------------------------------
%
\chapter{\gmt\ Map Projections}
\label{ch:6}

\GMT\ implements more than 30 different projections.  They all project the input coordinates
longitude and latitude to positions on a map.  In general, $x' = f(x,y,z)$ and $y' = g(x,y,z)$, where
$z$ is implicitly given as the radial vector length to the $(x,y)$ point on the chosen ellipsoid.  The functions $f$ and $g$ can be
quite nasty and we will refrain from presenting details in this document.  The interested read is referred to
{\it Snyder} [1987]\footnote{Snyder, J. P., 1987, Map Projections \- A Working Manual, U.S. Geological Survey Prof. Paper 1395.}.
We will mostly be using the \GMTprog{pscoast} command to demonstrate each of the projections.
\GMT\ map projections are grouped into four categories depending on the
nature of the projection.  The groups are

\begin{enumerate}
\item Conic map projections
\item Azimuthal map projections
\item Cylindrical map projections
\item Miscellaneous projections
\end{enumerate}

Because $x$ and $y$ are coupled we can only specify one plot-dimensional scale, typically
a map \emph{scale} (for lower-case map projection code) or a map \emph{width} (for upper-case
map projection code).  However, in some cases it would be more
practical to specify map \emph{height} instead of \emph{width}, while in other situations it would be nice
to set either the \emph{shortest} or \emph{longest} map dimension.  Users may select
these alternatives by appending a character code to their map dimension.  To specify map \emph{height},
append \textbf{h} to the given dimension; to select the minimum map dimension, append \textbf{-}, whereas you may
append \textbf{+} to select the maximum map dimension.  Without the modifier the map width is
selected by default.

In \GMT\ version 4.3.0 we noticed we ran out of the alphabet for 1-letter (and sometimes 2-letter) projection codes. To allow more flexibility, and to make it easier to remember the codes, we implemented the option to use the abbreviations used by the \progname{Proj4} mapping package. Since some of the \GMT\ projections are not in \progname{Proj4}, we invented some of our own as well. For a full list of both the old 1- and 2-letter codes, as well as the \progname{Proj4}-equivalents see the quick reference cards in Section~\ref{sec:purpose}. For example, \Opt{JM15c} and \Opt{JMerc/15c} have the same meaning.

\input{GMT_Chapter_6.1}
\input{GMT_Chapter_6.2}
\input{GMT_Chapter_6.3}
\input{GMT_Chapter_6.4}

%------------------------------------------
%	$Id: GMT_Chapter_6.tex,v 1.19 2008-04-30 03:53:30 remko Exp $
%
%	The GMT Documentation Project
%	Copyright 2000-2008.
%	Paul Wessel and Walter H. F. Smith
%------------------------------------------
%
\chapter{\gmt\ Map Projections}
\label{ch:6}

\GMT\ implements more than 30 different projections.  They all project the input coordinates
longitude and latitude to positions on a map.  In general, $x' = f(x,y,z)$ and $y' = g(x,y,z)$, where
$z$ is implicitly given as the radial vector length to the $(x,y)$ point on the chosen ellipsoid.  The functions $f$ and $g$ can be
quite nasty and we will refrain from presenting details in this document.  The interested read is referred to
{\it Snyder} [1987]\footnote{Snyder, J. P., 1987, Map Projections \- A Working Manual, U.S. Geological Survey Prof. Paper 1395.}.
We will mostly be using the \GMTprog{pscoast} command to demonstrate each of the projections.
\GMT\ map projections are grouped into four categories depending on the
nature of the projection.  The groups are

\begin{enumerate}
\item Conic map projections
\item Azimuthal map projections
\item Cylindrical map projections
\item Miscellaneous projections
\end{enumerate}

Because $x$ and $y$ are coupled we can only specify one plot-dimensional scale, typically
a map \emph{scale} (for lower-case map projection code) or a map \emph{width} (for upper-case
map projection code).  However, in some cases it would be more
practical to specify map \emph{height} instead of \emph{width}, while in other situations it would be nice
to set either the \emph{shortest} or \emph{longest} map dimension.  Users may select
these alternatives by appending a character code to their map dimension.  To specify map \emph{height},
append \textbf{h} to the given dimension; to select the minimum map dimension, append \textbf{-}, whereas you may
append \textbf{+} to select the maximum map dimension.  Without the modifier the map width is
selected by default.

In \GMT\ version 4.3.0 we noticed we ran out of the alphabet for 1-letter (and sometimes 2-letter) projection codes. To allow more flexibility, and to make it easier to remember the codes, we implemented the option to use the abbreviations used by the \progname{Proj4} mapping package. Since some of the \GMT\ projections are not in \progname{Proj4}, we invented some of our own as well. For a full list of both the old 1- and 2-letter codes, as well as the \progname{Proj4}-equivalents see the quick reference cards in Section~\ref{sec:purpose}. For example, \Opt{JM15c} and \Opt{JMerc/15c} have the same meaning.

\input{GMT_Chapter_6.1}
\input{GMT_Chapter_6.2}
\input{GMT_Chapter_6.3}
\input{GMT_Chapter_6.4}



%------------------------------------------
%	$Id: GMT_Chapter_6.tex,v 1.19 2008-04-30 03:53:30 remko Exp $
%
%	The GMT Documentation Project
%	Copyright 2000-2008.
%	Paul Wessel and Walter H. F. Smith
%------------------------------------------
%
\chapter{\gmt\ Map Projections}
\label{ch:6}

\GMT\ implements more than 30 different projections.  They all project the input coordinates
longitude and latitude to positions on a map.  In general, $x' = f(x,y,z)$ and $y' = g(x,y,z)$, where
$z$ is implicitly given as the radial vector length to the $(x,y)$ point on the chosen ellipsoid.  The functions $f$ and $g$ can be
quite nasty and we will refrain from presenting details in this document.  The interested read is referred to
{\it Snyder} [1987]\footnote{Snyder, J. P., 1987, Map Projections \- A Working Manual, U.S. Geological Survey Prof. Paper 1395.}.
We will mostly be using the \GMTprog{pscoast} command to demonstrate each of the projections.
\GMT\ map projections are grouped into four categories depending on the
nature of the projection.  The groups are

\begin{enumerate}
\item Conic map projections
\item Azimuthal map projections
\item Cylindrical map projections
\item Miscellaneous projections
\end{enumerate}

Because $x$ and $y$ are coupled we can only specify one plot-dimensional scale, typically
a map \emph{scale} (for lower-case map projection code) or a map \emph{width} (for upper-case
map projection code).  However, in some cases it would be more
practical to specify map \emph{height} instead of \emph{width}, while in other situations it would be nice
to set either the \emph{shortest} or \emph{longest} map dimension.  Users may select
these alternatives by appending a character code to their map dimension.  To specify map \emph{height},
append \textbf{h} to the given dimension; to select the minimum map dimension, append \textbf{-}, whereas you may
append \textbf{+} to select the maximum map dimension.  Without the modifier the map width is
selected by default.

In \GMT\ version 4.3.0 we noticed we ran out of the alphabet for 1-letter (and sometimes 2-letter) projection codes. To allow more flexibility, and to make it easier to remember the codes, we implemented the option to use the abbreviations used by the \progname{Proj4} mapping package. Since some of the \GMT\ projections are not in \progname{Proj4}, we invented some of our own as well. For a full list of both the old 1- and 2-letter codes, as well as the \progname{Proj4}-equivalents see the quick reference cards in Section~\ref{sec:purpose}. For example, \Opt{JM15c} and \Opt{JMerc/15c} have the same meaning.

%------------------------------------------
%	$Id: GMT_Chapter_6.tex,v 1.19 2008-04-30 03:53:30 remko Exp $
%
%	The GMT Documentation Project
%	Copyright 2000-2008.
%	Paul Wessel and Walter H. F. Smith
%------------------------------------------
%
\chapter{\gmt\ Map Projections}
\label{ch:6}

\GMT\ implements more than 30 different projections.  They all project the input coordinates
longitude and latitude to positions on a map.  In general, $x' = f(x,y,z)$ and $y' = g(x,y,z)$, where
$z$ is implicitly given as the radial vector length to the $(x,y)$ point on the chosen ellipsoid.  The functions $f$ and $g$ can be
quite nasty and we will refrain from presenting details in this document.  The interested read is referred to
{\it Snyder} [1987]\footnote{Snyder, J. P., 1987, Map Projections \- A Working Manual, U.S. Geological Survey Prof. Paper 1395.}.
We will mostly be using the \GMTprog{pscoast} command to demonstrate each of the projections.
\GMT\ map projections are grouped into four categories depending on the
nature of the projection.  The groups are

\begin{enumerate}
\item Conic map projections
\item Azimuthal map projections
\item Cylindrical map projections
\item Miscellaneous projections
\end{enumerate}

Because $x$ and $y$ are coupled we can only specify one plot-dimensional scale, typically
a map \emph{scale} (for lower-case map projection code) or a map \emph{width} (for upper-case
map projection code).  However, in some cases it would be more
practical to specify map \emph{height} instead of \emph{width}, while in other situations it would be nice
to set either the \emph{shortest} or \emph{longest} map dimension.  Users may select
these alternatives by appending a character code to their map dimension.  To specify map \emph{height},
append \textbf{h} to the given dimension; to select the minimum map dimension, append \textbf{-}, whereas you may
append \textbf{+} to select the maximum map dimension.  Without the modifier the map width is
selected by default.

In \GMT\ version 4.3.0 we noticed we ran out of the alphabet for 1-letter (and sometimes 2-letter) projection codes. To allow more flexibility, and to make it easier to remember the codes, we implemented the option to use the abbreviations used by the \progname{Proj4} mapping package. Since some of the \GMT\ projections are not in \progname{Proj4}, we invented some of our own as well. For a full list of both the old 1- and 2-letter codes, as well as the \progname{Proj4}-equivalents see the quick reference cards in Section~\ref{sec:purpose}. For example, \Opt{JM15c} and \Opt{JMerc/15c} have the same meaning.

%------------------------------------------
%	$Id: GMT_Chapter_6.tex,v 1.19 2008-04-30 03:53:30 remko Exp $
%
%	The GMT Documentation Project
%	Copyright 2000-2008.
%	Paul Wessel and Walter H. F. Smith
%------------------------------------------
%
\chapter{\gmt\ Map Projections}
\label{ch:6}

\GMT\ implements more than 30 different projections.  They all project the input coordinates
longitude and latitude to positions on a map.  In general, $x' = f(x,y,z)$ and $y' = g(x,y,z)$, where
$z$ is implicitly given as the radial vector length to the $(x,y)$ point on the chosen ellipsoid.  The functions $f$ and $g$ can be
quite nasty and we will refrain from presenting details in this document.  The interested read is referred to
{\it Snyder} [1987]\footnote{Snyder, J. P., 1987, Map Projections \- A Working Manual, U.S. Geological Survey Prof. Paper 1395.}.
We will mostly be using the \GMTprog{pscoast} command to demonstrate each of the projections.
\GMT\ map projections are grouped into four categories depending on the
nature of the projection.  The groups are

\begin{enumerate}
\item Conic map projections
\item Azimuthal map projections
\item Cylindrical map projections
\item Miscellaneous projections
\end{enumerate}

Because $x$ and $y$ are coupled we can only specify one plot-dimensional scale, typically
a map \emph{scale} (for lower-case map projection code) or a map \emph{width} (for upper-case
map projection code).  However, in some cases it would be more
practical to specify map \emph{height} instead of \emph{width}, while in other situations it would be nice
to set either the \emph{shortest} or \emph{longest} map dimension.  Users may select
these alternatives by appending a character code to their map dimension.  To specify map \emph{height},
append \textbf{h} to the given dimension; to select the minimum map dimension, append \textbf{-}, whereas you may
append \textbf{+} to select the maximum map dimension.  Without the modifier the map width is
selected by default.

In \GMT\ version 4.3.0 we noticed we ran out of the alphabet for 1-letter (and sometimes 2-letter) projection codes. To allow more flexibility, and to make it easier to remember the codes, we implemented the option to use the abbreviations used by the \progname{Proj4} mapping package. Since some of the \GMT\ projections are not in \progname{Proj4}, we invented some of our own as well. For a full list of both the old 1- and 2-letter codes, as well as the \progname{Proj4}-equivalents see the quick reference cards in Section~\ref{sec:purpose}. For example, \Opt{JM15c} and \Opt{JMerc/15c} have the same meaning.

\input{GMT_Chapter_6.1}
\input{GMT_Chapter_6.2}
\input{GMT_Chapter_6.3}
\input{GMT_Chapter_6.4}

%------------------------------------------
%	$Id: GMT_Chapter_6.tex,v 1.19 2008-04-30 03:53:30 remko Exp $
%
%	The GMT Documentation Project
%	Copyright 2000-2008.
%	Paul Wessel and Walter H. F. Smith
%------------------------------------------
%
\chapter{\gmt\ Map Projections}
\label{ch:6}

\GMT\ implements more than 30 different projections.  They all project the input coordinates
longitude and latitude to positions on a map.  In general, $x' = f(x,y,z)$ and $y' = g(x,y,z)$, where
$z$ is implicitly given as the radial vector length to the $(x,y)$ point on the chosen ellipsoid.  The functions $f$ and $g$ can be
quite nasty and we will refrain from presenting details in this document.  The interested read is referred to
{\it Snyder} [1987]\footnote{Snyder, J. P., 1987, Map Projections \- A Working Manual, U.S. Geological Survey Prof. Paper 1395.}.
We will mostly be using the \GMTprog{pscoast} command to demonstrate each of the projections.
\GMT\ map projections are grouped into four categories depending on the
nature of the projection.  The groups are

\begin{enumerate}
\item Conic map projections
\item Azimuthal map projections
\item Cylindrical map projections
\item Miscellaneous projections
\end{enumerate}

Because $x$ and $y$ are coupled we can only specify one plot-dimensional scale, typically
a map \emph{scale} (for lower-case map projection code) or a map \emph{width} (for upper-case
map projection code).  However, in some cases it would be more
practical to specify map \emph{height} instead of \emph{width}, while in other situations it would be nice
to set either the \emph{shortest} or \emph{longest} map dimension.  Users may select
these alternatives by appending a character code to their map dimension.  To specify map \emph{height},
append \textbf{h} to the given dimension; to select the minimum map dimension, append \textbf{-}, whereas you may
append \textbf{+} to select the maximum map dimension.  Without the modifier the map width is
selected by default.

In \GMT\ version 4.3.0 we noticed we ran out of the alphabet for 1-letter (and sometimes 2-letter) projection codes. To allow more flexibility, and to make it easier to remember the codes, we implemented the option to use the abbreviations used by the \progname{Proj4} mapping package. Since some of the \GMT\ projections are not in \progname{Proj4}, we invented some of our own as well. For a full list of both the old 1- and 2-letter codes, as well as the \progname{Proj4}-equivalents see the quick reference cards in Section~\ref{sec:purpose}. For example, \Opt{JM15c} and \Opt{JMerc/15c} have the same meaning.

\input{GMT_Chapter_6.1}
\input{GMT_Chapter_6.2}
\input{GMT_Chapter_6.3}
\input{GMT_Chapter_6.4}

%------------------------------------------
%	$Id: GMT_Chapter_6.tex,v 1.19 2008-04-30 03:53:30 remko Exp $
%
%	The GMT Documentation Project
%	Copyright 2000-2008.
%	Paul Wessel and Walter H. F. Smith
%------------------------------------------
%
\chapter{\gmt\ Map Projections}
\label{ch:6}

\GMT\ implements more than 30 different projections.  They all project the input coordinates
longitude and latitude to positions on a map.  In general, $x' = f(x,y,z)$ and $y' = g(x,y,z)$, where
$z$ is implicitly given as the radial vector length to the $(x,y)$ point on the chosen ellipsoid.  The functions $f$ and $g$ can be
quite nasty and we will refrain from presenting details in this document.  The interested read is referred to
{\it Snyder} [1987]\footnote{Snyder, J. P., 1987, Map Projections \- A Working Manual, U.S. Geological Survey Prof. Paper 1395.}.
We will mostly be using the \GMTprog{pscoast} command to demonstrate each of the projections.
\GMT\ map projections are grouped into four categories depending on the
nature of the projection.  The groups are

\begin{enumerate}
\item Conic map projections
\item Azimuthal map projections
\item Cylindrical map projections
\item Miscellaneous projections
\end{enumerate}

Because $x$ and $y$ are coupled we can only specify one plot-dimensional scale, typically
a map \emph{scale} (for lower-case map projection code) or a map \emph{width} (for upper-case
map projection code).  However, in some cases it would be more
practical to specify map \emph{height} instead of \emph{width}, while in other situations it would be nice
to set either the \emph{shortest} or \emph{longest} map dimension.  Users may select
these alternatives by appending a character code to their map dimension.  To specify map \emph{height},
append \textbf{h} to the given dimension; to select the minimum map dimension, append \textbf{-}, whereas you may
append \textbf{+} to select the maximum map dimension.  Without the modifier the map width is
selected by default.

In \GMT\ version 4.3.0 we noticed we ran out of the alphabet for 1-letter (and sometimes 2-letter) projection codes. To allow more flexibility, and to make it easier to remember the codes, we implemented the option to use the abbreviations used by the \progname{Proj4} mapping package. Since some of the \GMT\ projections are not in \progname{Proj4}, we invented some of our own as well. For a full list of both the old 1- and 2-letter codes, as well as the \progname{Proj4}-equivalents see the quick reference cards in Section~\ref{sec:purpose}. For example, \Opt{JM15c} and \Opt{JMerc/15c} have the same meaning.

\input{GMT_Chapter_6.1}
\input{GMT_Chapter_6.2}
\input{GMT_Chapter_6.3}
\input{GMT_Chapter_6.4}

%------------------------------------------
%	$Id: GMT_Chapter_6.tex,v 1.19 2008-04-30 03:53:30 remko Exp $
%
%	The GMT Documentation Project
%	Copyright 2000-2008.
%	Paul Wessel and Walter H. F. Smith
%------------------------------------------
%
\chapter{\gmt\ Map Projections}
\label{ch:6}

\GMT\ implements more than 30 different projections.  They all project the input coordinates
longitude and latitude to positions on a map.  In general, $x' = f(x,y,z)$ and $y' = g(x,y,z)$, where
$z$ is implicitly given as the radial vector length to the $(x,y)$ point on the chosen ellipsoid.  The functions $f$ and $g$ can be
quite nasty and we will refrain from presenting details in this document.  The interested read is referred to
{\it Snyder} [1987]\footnote{Snyder, J. P., 1987, Map Projections \- A Working Manual, U.S. Geological Survey Prof. Paper 1395.}.
We will mostly be using the \GMTprog{pscoast} command to demonstrate each of the projections.
\GMT\ map projections are grouped into four categories depending on the
nature of the projection.  The groups are

\begin{enumerate}
\item Conic map projections
\item Azimuthal map projections
\item Cylindrical map projections
\item Miscellaneous projections
\end{enumerate}

Because $x$ and $y$ are coupled we can only specify one plot-dimensional scale, typically
a map \emph{scale} (for lower-case map projection code) or a map \emph{width} (for upper-case
map projection code).  However, in some cases it would be more
practical to specify map \emph{height} instead of \emph{width}, while in other situations it would be nice
to set either the \emph{shortest} or \emph{longest} map dimension.  Users may select
these alternatives by appending a character code to their map dimension.  To specify map \emph{height},
append \textbf{h} to the given dimension; to select the minimum map dimension, append \textbf{-}, whereas you may
append \textbf{+} to select the maximum map dimension.  Without the modifier the map width is
selected by default.

In \GMT\ version 4.3.0 we noticed we ran out of the alphabet for 1-letter (and sometimes 2-letter) projection codes. To allow more flexibility, and to make it easier to remember the codes, we implemented the option to use the abbreviations used by the \progname{Proj4} mapping package. Since some of the \GMT\ projections are not in \progname{Proj4}, we invented some of our own as well. For a full list of both the old 1- and 2-letter codes, as well as the \progname{Proj4}-equivalents see the quick reference cards in Section~\ref{sec:purpose}. For example, \Opt{JM15c} and \Opt{JMerc/15c} have the same meaning.

\input{GMT_Chapter_6.1}
\input{GMT_Chapter_6.2}
\input{GMT_Chapter_6.3}
\input{GMT_Chapter_6.4}


%------------------------------------------
%	$Id: GMT_Chapter_6.tex,v 1.19 2008-04-30 03:53:30 remko Exp $
%
%	The GMT Documentation Project
%	Copyright 2000-2008.
%	Paul Wessel and Walter H. F. Smith
%------------------------------------------
%
\chapter{\gmt\ Map Projections}
\label{ch:6}

\GMT\ implements more than 30 different projections.  They all project the input coordinates
longitude and latitude to positions on a map.  In general, $x' = f(x,y,z)$ and $y' = g(x,y,z)$, where
$z$ is implicitly given as the radial vector length to the $(x,y)$ point on the chosen ellipsoid.  The functions $f$ and $g$ can be
quite nasty and we will refrain from presenting details in this document.  The interested read is referred to
{\it Snyder} [1987]\footnote{Snyder, J. P., 1987, Map Projections \- A Working Manual, U.S. Geological Survey Prof. Paper 1395.}.
We will mostly be using the \GMTprog{pscoast} command to demonstrate each of the projections.
\GMT\ map projections are grouped into four categories depending on the
nature of the projection.  The groups are

\begin{enumerate}
\item Conic map projections
\item Azimuthal map projections
\item Cylindrical map projections
\item Miscellaneous projections
\end{enumerate}

Because $x$ and $y$ are coupled we can only specify one plot-dimensional scale, typically
a map \emph{scale} (for lower-case map projection code) or a map \emph{width} (for upper-case
map projection code).  However, in some cases it would be more
practical to specify map \emph{height} instead of \emph{width}, while in other situations it would be nice
to set either the \emph{shortest} or \emph{longest} map dimension.  Users may select
these alternatives by appending a character code to their map dimension.  To specify map \emph{height},
append \textbf{h} to the given dimension; to select the minimum map dimension, append \textbf{-}, whereas you may
append \textbf{+} to select the maximum map dimension.  Without the modifier the map width is
selected by default.

In \GMT\ version 4.3.0 we noticed we ran out of the alphabet for 1-letter (and sometimes 2-letter) projection codes. To allow more flexibility, and to make it easier to remember the codes, we implemented the option to use the abbreviations used by the \progname{Proj4} mapping package. Since some of the \GMT\ projections are not in \progname{Proj4}, we invented some of our own as well. For a full list of both the old 1- and 2-letter codes, as well as the \progname{Proj4}-equivalents see the quick reference cards in Section~\ref{sec:purpose}. For example, \Opt{JM15c} and \Opt{JMerc/15c} have the same meaning.

%------------------------------------------
%	$Id: GMT_Chapter_6.tex,v 1.19 2008-04-30 03:53:30 remko Exp $
%
%	The GMT Documentation Project
%	Copyright 2000-2008.
%	Paul Wessel and Walter H. F. Smith
%------------------------------------------
%
\chapter{\gmt\ Map Projections}
\label{ch:6}

\GMT\ implements more than 30 different projections.  They all project the input coordinates
longitude and latitude to positions on a map.  In general, $x' = f(x,y,z)$ and $y' = g(x,y,z)$, where
$z$ is implicitly given as the radial vector length to the $(x,y)$ point on the chosen ellipsoid.  The functions $f$ and $g$ can be
quite nasty and we will refrain from presenting details in this document.  The interested read is referred to
{\it Snyder} [1987]\footnote{Snyder, J. P., 1987, Map Projections \- A Working Manual, U.S. Geological Survey Prof. Paper 1395.}.
We will mostly be using the \GMTprog{pscoast} command to demonstrate each of the projections.
\GMT\ map projections are grouped into four categories depending on the
nature of the projection.  The groups are

\begin{enumerate}
\item Conic map projections
\item Azimuthal map projections
\item Cylindrical map projections
\item Miscellaneous projections
\end{enumerate}

Because $x$ and $y$ are coupled we can only specify one plot-dimensional scale, typically
a map \emph{scale} (for lower-case map projection code) or a map \emph{width} (for upper-case
map projection code).  However, in some cases it would be more
practical to specify map \emph{height} instead of \emph{width}, while in other situations it would be nice
to set either the \emph{shortest} or \emph{longest} map dimension.  Users may select
these alternatives by appending a character code to their map dimension.  To specify map \emph{height},
append \textbf{h} to the given dimension; to select the minimum map dimension, append \textbf{-}, whereas you may
append \textbf{+} to select the maximum map dimension.  Without the modifier the map width is
selected by default.

In \GMT\ version 4.3.0 we noticed we ran out of the alphabet for 1-letter (and sometimes 2-letter) projection codes. To allow more flexibility, and to make it easier to remember the codes, we implemented the option to use the abbreviations used by the \progname{Proj4} mapping package. Since some of the \GMT\ projections are not in \progname{Proj4}, we invented some of our own as well. For a full list of both the old 1- and 2-letter codes, as well as the \progname{Proj4}-equivalents see the quick reference cards in Section~\ref{sec:purpose}. For example, \Opt{JM15c} and \Opt{JMerc/15c} have the same meaning.

\input{GMT_Chapter_6.1}
\input{GMT_Chapter_6.2}
\input{GMT_Chapter_6.3}
\input{GMT_Chapter_6.4}

%------------------------------------------
%	$Id: GMT_Chapter_6.tex,v 1.19 2008-04-30 03:53:30 remko Exp $
%
%	The GMT Documentation Project
%	Copyright 2000-2008.
%	Paul Wessel and Walter H. F. Smith
%------------------------------------------
%
\chapter{\gmt\ Map Projections}
\label{ch:6}

\GMT\ implements more than 30 different projections.  They all project the input coordinates
longitude and latitude to positions on a map.  In general, $x' = f(x,y,z)$ and $y' = g(x,y,z)$, where
$z$ is implicitly given as the radial vector length to the $(x,y)$ point on the chosen ellipsoid.  The functions $f$ and $g$ can be
quite nasty and we will refrain from presenting details in this document.  The interested read is referred to
{\it Snyder} [1987]\footnote{Snyder, J. P., 1987, Map Projections \- A Working Manual, U.S. Geological Survey Prof. Paper 1395.}.
We will mostly be using the \GMTprog{pscoast} command to demonstrate each of the projections.
\GMT\ map projections are grouped into four categories depending on the
nature of the projection.  The groups are

\begin{enumerate}
\item Conic map projections
\item Azimuthal map projections
\item Cylindrical map projections
\item Miscellaneous projections
\end{enumerate}

Because $x$ and $y$ are coupled we can only specify one plot-dimensional scale, typically
a map \emph{scale} (for lower-case map projection code) or a map \emph{width} (for upper-case
map projection code).  However, in some cases it would be more
practical to specify map \emph{height} instead of \emph{width}, while in other situations it would be nice
to set either the \emph{shortest} or \emph{longest} map dimension.  Users may select
these alternatives by appending a character code to their map dimension.  To specify map \emph{height},
append \textbf{h} to the given dimension; to select the minimum map dimension, append \textbf{-}, whereas you may
append \textbf{+} to select the maximum map dimension.  Without the modifier the map width is
selected by default.

In \GMT\ version 4.3.0 we noticed we ran out of the alphabet for 1-letter (and sometimes 2-letter) projection codes. To allow more flexibility, and to make it easier to remember the codes, we implemented the option to use the abbreviations used by the \progname{Proj4} mapping package. Since some of the \GMT\ projections are not in \progname{Proj4}, we invented some of our own as well. For a full list of both the old 1- and 2-letter codes, as well as the \progname{Proj4}-equivalents see the quick reference cards in Section~\ref{sec:purpose}. For example, \Opt{JM15c} and \Opt{JMerc/15c} have the same meaning.

\input{GMT_Chapter_6.1}
\input{GMT_Chapter_6.2}
\input{GMT_Chapter_6.3}
\input{GMT_Chapter_6.4}

%------------------------------------------
%	$Id: GMT_Chapter_6.tex,v 1.19 2008-04-30 03:53:30 remko Exp $
%
%	The GMT Documentation Project
%	Copyright 2000-2008.
%	Paul Wessel and Walter H. F. Smith
%------------------------------------------
%
\chapter{\gmt\ Map Projections}
\label{ch:6}

\GMT\ implements more than 30 different projections.  They all project the input coordinates
longitude and latitude to positions on a map.  In general, $x' = f(x,y,z)$ and $y' = g(x,y,z)$, where
$z$ is implicitly given as the radial vector length to the $(x,y)$ point on the chosen ellipsoid.  The functions $f$ and $g$ can be
quite nasty and we will refrain from presenting details in this document.  The interested read is referred to
{\it Snyder} [1987]\footnote{Snyder, J. P., 1987, Map Projections \- A Working Manual, U.S. Geological Survey Prof. Paper 1395.}.
We will mostly be using the \GMTprog{pscoast} command to demonstrate each of the projections.
\GMT\ map projections are grouped into four categories depending on the
nature of the projection.  The groups are

\begin{enumerate}
\item Conic map projections
\item Azimuthal map projections
\item Cylindrical map projections
\item Miscellaneous projections
\end{enumerate}

Because $x$ and $y$ are coupled we can only specify one plot-dimensional scale, typically
a map \emph{scale} (for lower-case map projection code) or a map \emph{width} (for upper-case
map projection code).  However, in some cases it would be more
practical to specify map \emph{height} instead of \emph{width}, while in other situations it would be nice
to set either the \emph{shortest} or \emph{longest} map dimension.  Users may select
these alternatives by appending a character code to their map dimension.  To specify map \emph{height},
append \textbf{h} to the given dimension; to select the minimum map dimension, append \textbf{-}, whereas you may
append \textbf{+} to select the maximum map dimension.  Without the modifier the map width is
selected by default.

In \GMT\ version 4.3.0 we noticed we ran out of the alphabet for 1-letter (and sometimes 2-letter) projection codes. To allow more flexibility, and to make it easier to remember the codes, we implemented the option to use the abbreviations used by the \progname{Proj4} mapping package. Since some of the \GMT\ projections are not in \progname{Proj4}, we invented some of our own as well. For a full list of both the old 1- and 2-letter codes, as well as the \progname{Proj4}-equivalents see the quick reference cards in Section~\ref{sec:purpose}. For example, \Opt{JM15c} and \Opt{JMerc/15c} have the same meaning.

\input{GMT_Chapter_6.1}
\input{GMT_Chapter_6.2}
\input{GMT_Chapter_6.3}
\input{GMT_Chapter_6.4}

%------------------------------------------
%	$Id: GMT_Chapter_6.tex,v 1.19 2008-04-30 03:53:30 remko Exp $
%
%	The GMT Documentation Project
%	Copyright 2000-2008.
%	Paul Wessel and Walter H. F. Smith
%------------------------------------------
%
\chapter{\gmt\ Map Projections}
\label{ch:6}

\GMT\ implements more than 30 different projections.  They all project the input coordinates
longitude and latitude to positions on a map.  In general, $x' = f(x,y,z)$ and $y' = g(x,y,z)$, where
$z$ is implicitly given as the radial vector length to the $(x,y)$ point on the chosen ellipsoid.  The functions $f$ and $g$ can be
quite nasty and we will refrain from presenting details in this document.  The interested read is referred to
{\it Snyder} [1987]\footnote{Snyder, J. P., 1987, Map Projections \- A Working Manual, U.S. Geological Survey Prof. Paper 1395.}.
We will mostly be using the \GMTprog{pscoast} command to demonstrate each of the projections.
\GMT\ map projections are grouped into four categories depending on the
nature of the projection.  The groups are

\begin{enumerate}
\item Conic map projections
\item Azimuthal map projections
\item Cylindrical map projections
\item Miscellaneous projections
\end{enumerate}

Because $x$ and $y$ are coupled we can only specify one plot-dimensional scale, typically
a map \emph{scale} (for lower-case map projection code) or a map \emph{width} (for upper-case
map projection code).  However, in some cases it would be more
practical to specify map \emph{height} instead of \emph{width}, while in other situations it would be nice
to set either the \emph{shortest} or \emph{longest} map dimension.  Users may select
these alternatives by appending a character code to their map dimension.  To specify map \emph{height},
append \textbf{h} to the given dimension; to select the minimum map dimension, append \textbf{-}, whereas you may
append \textbf{+} to select the maximum map dimension.  Without the modifier the map width is
selected by default.

In \GMT\ version 4.3.0 we noticed we ran out of the alphabet for 1-letter (and sometimes 2-letter) projection codes. To allow more flexibility, and to make it easier to remember the codes, we implemented the option to use the abbreviations used by the \progname{Proj4} mapping package. Since some of the \GMT\ projections are not in \progname{Proj4}, we invented some of our own as well. For a full list of both the old 1- and 2-letter codes, as well as the \progname{Proj4}-equivalents see the quick reference cards in Section~\ref{sec:purpose}. For example, \Opt{JM15c} and \Opt{JMerc/15c} have the same meaning.

\input{GMT_Chapter_6.1}
\input{GMT_Chapter_6.2}
\input{GMT_Chapter_6.3}
\input{GMT_Chapter_6.4}


%------------------------------------------
%	$Id: GMT_Chapter_6.tex,v 1.19 2008-04-30 03:53:30 remko Exp $
%
%	The GMT Documentation Project
%	Copyright 2000-2008.
%	Paul Wessel and Walter H. F. Smith
%------------------------------------------
%
\chapter{\gmt\ Map Projections}
\label{ch:6}

\GMT\ implements more than 30 different projections.  They all project the input coordinates
longitude and latitude to positions on a map.  In general, $x' = f(x,y,z)$ and $y' = g(x,y,z)$, where
$z$ is implicitly given as the radial vector length to the $(x,y)$ point on the chosen ellipsoid.  The functions $f$ and $g$ can be
quite nasty and we will refrain from presenting details in this document.  The interested read is referred to
{\it Snyder} [1987]\footnote{Snyder, J. P., 1987, Map Projections \- A Working Manual, U.S. Geological Survey Prof. Paper 1395.}.
We will mostly be using the \GMTprog{pscoast} command to demonstrate each of the projections.
\GMT\ map projections are grouped into four categories depending on the
nature of the projection.  The groups are

\begin{enumerate}
\item Conic map projections
\item Azimuthal map projections
\item Cylindrical map projections
\item Miscellaneous projections
\end{enumerate}

Because $x$ and $y$ are coupled we can only specify one plot-dimensional scale, typically
a map \emph{scale} (for lower-case map projection code) or a map \emph{width} (for upper-case
map projection code).  However, in some cases it would be more
practical to specify map \emph{height} instead of \emph{width}, while in other situations it would be nice
to set either the \emph{shortest} or \emph{longest} map dimension.  Users may select
these alternatives by appending a character code to their map dimension.  To specify map \emph{height},
append \textbf{h} to the given dimension; to select the minimum map dimension, append \textbf{-}, whereas you may
append \textbf{+} to select the maximum map dimension.  Without the modifier the map width is
selected by default.

In \GMT\ version 4.3.0 we noticed we ran out of the alphabet for 1-letter (and sometimes 2-letter) projection codes. To allow more flexibility, and to make it easier to remember the codes, we implemented the option to use the abbreviations used by the \progname{Proj4} mapping package. Since some of the \GMT\ projections are not in \progname{Proj4}, we invented some of our own as well. For a full list of both the old 1- and 2-letter codes, as well as the \progname{Proj4}-equivalents see the quick reference cards in Section~\ref{sec:purpose}. For example, \Opt{JM15c} and \Opt{JMerc/15c} have the same meaning.

%------------------------------------------
%	$Id: GMT_Chapter_6.tex,v 1.19 2008-04-30 03:53:30 remko Exp $
%
%	The GMT Documentation Project
%	Copyright 2000-2008.
%	Paul Wessel and Walter H. F. Smith
%------------------------------------------
%
\chapter{\gmt\ Map Projections}
\label{ch:6}

\GMT\ implements more than 30 different projections.  They all project the input coordinates
longitude and latitude to positions on a map.  In general, $x' = f(x,y,z)$ and $y' = g(x,y,z)$, where
$z$ is implicitly given as the radial vector length to the $(x,y)$ point on the chosen ellipsoid.  The functions $f$ and $g$ can be
quite nasty and we will refrain from presenting details in this document.  The interested read is referred to
{\it Snyder} [1987]\footnote{Snyder, J. P., 1987, Map Projections \- A Working Manual, U.S. Geological Survey Prof. Paper 1395.}.
We will mostly be using the \GMTprog{pscoast} command to demonstrate each of the projections.
\GMT\ map projections are grouped into four categories depending on the
nature of the projection.  The groups are

\begin{enumerate}
\item Conic map projections
\item Azimuthal map projections
\item Cylindrical map projections
\item Miscellaneous projections
\end{enumerate}

Because $x$ and $y$ are coupled we can only specify one plot-dimensional scale, typically
a map \emph{scale} (for lower-case map projection code) or a map \emph{width} (for upper-case
map projection code).  However, in some cases it would be more
practical to specify map \emph{height} instead of \emph{width}, while in other situations it would be nice
to set either the \emph{shortest} or \emph{longest} map dimension.  Users may select
these alternatives by appending a character code to their map dimension.  To specify map \emph{height},
append \textbf{h} to the given dimension; to select the minimum map dimension, append \textbf{-}, whereas you may
append \textbf{+} to select the maximum map dimension.  Without the modifier the map width is
selected by default.

In \GMT\ version 4.3.0 we noticed we ran out of the alphabet for 1-letter (and sometimes 2-letter) projection codes. To allow more flexibility, and to make it easier to remember the codes, we implemented the option to use the abbreviations used by the \progname{Proj4} mapping package. Since some of the \GMT\ projections are not in \progname{Proj4}, we invented some of our own as well. For a full list of both the old 1- and 2-letter codes, as well as the \progname{Proj4}-equivalents see the quick reference cards in Section~\ref{sec:purpose}. For example, \Opt{JM15c} and \Opt{JMerc/15c} have the same meaning.

\input{GMT_Chapter_6.1}
\input{GMT_Chapter_6.2}
\input{GMT_Chapter_6.3}
\input{GMT_Chapter_6.4}

%------------------------------------------
%	$Id: GMT_Chapter_6.tex,v 1.19 2008-04-30 03:53:30 remko Exp $
%
%	The GMT Documentation Project
%	Copyright 2000-2008.
%	Paul Wessel and Walter H. F. Smith
%------------------------------------------
%
\chapter{\gmt\ Map Projections}
\label{ch:6}

\GMT\ implements more than 30 different projections.  They all project the input coordinates
longitude and latitude to positions on a map.  In general, $x' = f(x,y,z)$ and $y' = g(x,y,z)$, where
$z$ is implicitly given as the radial vector length to the $(x,y)$ point on the chosen ellipsoid.  The functions $f$ and $g$ can be
quite nasty and we will refrain from presenting details in this document.  The interested read is referred to
{\it Snyder} [1987]\footnote{Snyder, J. P., 1987, Map Projections \- A Working Manual, U.S. Geological Survey Prof. Paper 1395.}.
We will mostly be using the \GMTprog{pscoast} command to demonstrate each of the projections.
\GMT\ map projections are grouped into four categories depending on the
nature of the projection.  The groups are

\begin{enumerate}
\item Conic map projections
\item Azimuthal map projections
\item Cylindrical map projections
\item Miscellaneous projections
\end{enumerate}

Because $x$ and $y$ are coupled we can only specify one plot-dimensional scale, typically
a map \emph{scale} (for lower-case map projection code) or a map \emph{width} (for upper-case
map projection code).  However, in some cases it would be more
practical to specify map \emph{height} instead of \emph{width}, while in other situations it would be nice
to set either the \emph{shortest} or \emph{longest} map dimension.  Users may select
these alternatives by appending a character code to their map dimension.  To specify map \emph{height},
append \textbf{h} to the given dimension; to select the minimum map dimension, append \textbf{-}, whereas you may
append \textbf{+} to select the maximum map dimension.  Without the modifier the map width is
selected by default.

In \GMT\ version 4.3.0 we noticed we ran out of the alphabet for 1-letter (and sometimes 2-letter) projection codes. To allow more flexibility, and to make it easier to remember the codes, we implemented the option to use the abbreviations used by the \progname{Proj4} mapping package. Since some of the \GMT\ projections are not in \progname{Proj4}, we invented some of our own as well. For a full list of both the old 1- and 2-letter codes, as well as the \progname{Proj4}-equivalents see the quick reference cards in Section~\ref{sec:purpose}. For example, \Opt{JM15c} and \Opt{JMerc/15c} have the same meaning.

\input{GMT_Chapter_6.1}
\input{GMT_Chapter_6.2}
\input{GMT_Chapter_6.3}
\input{GMT_Chapter_6.4}

%------------------------------------------
%	$Id: GMT_Chapter_6.tex,v 1.19 2008-04-30 03:53:30 remko Exp $
%
%	The GMT Documentation Project
%	Copyright 2000-2008.
%	Paul Wessel and Walter H. F. Smith
%------------------------------------------
%
\chapter{\gmt\ Map Projections}
\label{ch:6}

\GMT\ implements more than 30 different projections.  They all project the input coordinates
longitude and latitude to positions on a map.  In general, $x' = f(x,y,z)$ and $y' = g(x,y,z)$, where
$z$ is implicitly given as the radial vector length to the $(x,y)$ point on the chosen ellipsoid.  The functions $f$ and $g$ can be
quite nasty and we will refrain from presenting details in this document.  The interested read is referred to
{\it Snyder} [1987]\footnote{Snyder, J. P., 1987, Map Projections \- A Working Manual, U.S. Geological Survey Prof. Paper 1395.}.
We will mostly be using the \GMTprog{pscoast} command to demonstrate each of the projections.
\GMT\ map projections are grouped into four categories depending on the
nature of the projection.  The groups are

\begin{enumerate}
\item Conic map projections
\item Azimuthal map projections
\item Cylindrical map projections
\item Miscellaneous projections
\end{enumerate}

Because $x$ and $y$ are coupled we can only specify one plot-dimensional scale, typically
a map \emph{scale} (for lower-case map projection code) or a map \emph{width} (for upper-case
map projection code).  However, in some cases it would be more
practical to specify map \emph{height} instead of \emph{width}, while in other situations it would be nice
to set either the \emph{shortest} or \emph{longest} map dimension.  Users may select
these alternatives by appending a character code to their map dimension.  To specify map \emph{height},
append \textbf{h} to the given dimension; to select the minimum map dimension, append \textbf{-}, whereas you may
append \textbf{+} to select the maximum map dimension.  Without the modifier the map width is
selected by default.

In \GMT\ version 4.3.0 we noticed we ran out of the alphabet for 1-letter (and sometimes 2-letter) projection codes. To allow more flexibility, and to make it easier to remember the codes, we implemented the option to use the abbreviations used by the \progname{Proj4} mapping package. Since some of the \GMT\ projections are not in \progname{Proj4}, we invented some of our own as well. For a full list of both the old 1- and 2-letter codes, as well as the \progname{Proj4}-equivalents see the quick reference cards in Section~\ref{sec:purpose}. For example, \Opt{JM15c} and \Opt{JMerc/15c} have the same meaning.

\input{GMT_Chapter_6.1}
\input{GMT_Chapter_6.2}
\input{GMT_Chapter_6.3}
\input{GMT_Chapter_6.4}

%------------------------------------------
%	$Id: GMT_Chapter_6.tex,v 1.19 2008-04-30 03:53:30 remko Exp $
%
%	The GMT Documentation Project
%	Copyright 2000-2008.
%	Paul Wessel and Walter H. F. Smith
%------------------------------------------
%
\chapter{\gmt\ Map Projections}
\label{ch:6}

\GMT\ implements more than 30 different projections.  They all project the input coordinates
longitude and latitude to positions on a map.  In general, $x' = f(x,y,z)$ and $y' = g(x,y,z)$, where
$z$ is implicitly given as the radial vector length to the $(x,y)$ point on the chosen ellipsoid.  The functions $f$ and $g$ can be
quite nasty and we will refrain from presenting details in this document.  The interested read is referred to
{\it Snyder} [1987]\footnote{Snyder, J. P., 1987, Map Projections \- A Working Manual, U.S. Geological Survey Prof. Paper 1395.}.
We will mostly be using the \GMTprog{pscoast} command to demonstrate each of the projections.
\GMT\ map projections are grouped into four categories depending on the
nature of the projection.  The groups are

\begin{enumerate}
\item Conic map projections
\item Azimuthal map projections
\item Cylindrical map projections
\item Miscellaneous projections
\end{enumerate}

Because $x$ and $y$ are coupled we can only specify one plot-dimensional scale, typically
a map \emph{scale} (for lower-case map projection code) or a map \emph{width} (for upper-case
map projection code).  However, in some cases it would be more
practical to specify map \emph{height} instead of \emph{width}, while in other situations it would be nice
to set either the \emph{shortest} or \emph{longest} map dimension.  Users may select
these alternatives by appending a character code to their map dimension.  To specify map \emph{height},
append \textbf{h} to the given dimension; to select the minimum map dimension, append \textbf{-}, whereas you may
append \textbf{+} to select the maximum map dimension.  Without the modifier the map width is
selected by default.

In \GMT\ version 4.3.0 we noticed we ran out of the alphabet for 1-letter (and sometimes 2-letter) projection codes. To allow more flexibility, and to make it easier to remember the codes, we implemented the option to use the abbreviations used by the \progname{Proj4} mapping package. Since some of the \GMT\ projections are not in \progname{Proj4}, we invented some of our own as well. For a full list of both the old 1- and 2-letter codes, as well as the \progname{Proj4}-equivalents see the quick reference cards in Section~\ref{sec:purpose}. For example, \Opt{JM15c} and \Opt{JMerc/15c} have the same meaning.

\input{GMT_Chapter_6.1}
\input{GMT_Chapter_6.2}
\input{GMT_Chapter_6.3}
\input{GMT_Chapter_6.4}


%------------------------------------------
%	$Id: GMT_Chapter_6.tex,v 1.19 2008-04-30 03:53:30 remko Exp $
%
%	The GMT Documentation Project
%	Copyright 2000-2008.
%	Paul Wessel and Walter H. F. Smith
%------------------------------------------
%
\chapter{\gmt\ Map Projections}
\label{ch:6}

\GMT\ implements more than 30 different projections.  They all project the input coordinates
longitude and latitude to positions on a map.  In general, $x' = f(x,y,z)$ and $y' = g(x,y,z)$, where
$z$ is implicitly given as the radial vector length to the $(x,y)$ point on the chosen ellipsoid.  The functions $f$ and $g$ can be
quite nasty and we will refrain from presenting details in this document.  The interested read is referred to
{\it Snyder} [1987]\footnote{Snyder, J. P., 1987, Map Projections \- A Working Manual, U.S. Geological Survey Prof. Paper 1395.}.
We will mostly be using the \GMTprog{pscoast} command to demonstrate each of the projections.
\GMT\ map projections are grouped into four categories depending on the
nature of the projection.  The groups are

\begin{enumerate}
\item Conic map projections
\item Azimuthal map projections
\item Cylindrical map projections
\item Miscellaneous projections
\end{enumerate}

Because $x$ and $y$ are coupled we can only specify one plot-dimensional scale, typically
a map \emph{scale} (for lower-case map projection code) or a map \emph{width} (for upper-case
map projection code).  However, in some cases it would be more
practical to specify map \emph{height} instead of \emph{width}, while in other situations it would be nice
to set either the \emph{shortest} or \emph{longest} map dimension.  Users may select
these alternatives by appending a character code to their map dimension.  To specify map \emph{height},
append \textbf{h} to the given dimension; to select the minimum map dimension, append \textbf{-}, whereas you may
append \textbf{+} to select the maximum map dimension.  Without the modifier the map width is
selected by default.

In \GMT\ version 4.3.0 we noticed we ran out of the alphabet for 1-letter (and sometimes 2-letter) projection codes. To allow more flexibility, and to make it easier to remember the codes, we implemented the option to use the abbreviations used by the \progname{Proj4} mapping package. Since some of the \GMT\ projections are not in \progname{Proj4}, we invented some of our own as well. For a full list of both the old 1- and 2-letter codes, as well as the \progname{Proj4}-equivalents see the quick reference cards in Section~\ref{sec:purpose}. For example, \Opt{JM15c} and \Opt{JMerc/15c} have the same meaning.

%------------------------------------------
%	$Id: GMT_Chapter_6.tex,v 1.19 2008-04-30 03:53:30 remko Exp $
%
%	The GMT Documentation Project
%	Copyright 2000-2008.
%	Paul Wessel and Walter H. F. Smith
%------------------------------------------
%
\chapter{\gmt\ Map Projections}
\label{ch:6}

\GMT\ implements more than 30 different projections.  They all project the input coordinates
longitude and latitude to positions on a map.  In general, $x' = f(x,y,z)$ and $y' = g(x,y,z)$, where
$z$ is implicitly given as the radial vector length to the $(x,y)$ point on the chosen ellipsoid.  The functions $f$ and $g$ can be
quite nasty and we will refrain from presenting details in this document.  The interested read is referred to
{\it Snyder} [1987]\footnote{Snyder, J. P., 1987, Map Projections \- A Working Manual, U.S. Geological Survey Prof. Paper 1395.}.
We will mostly be using the \GMTprog{pscoast} command to demonstrate each of the projections.
\GMT\ map projections are grouped into four categories depending on the
nature of the projection.  The groups are

\begin{enumerate}
\item Conic map projections
\item Azimuthal map projections
\item Cylindrical map projections
\item Miscellaneous projections
\end{enumerate}

Because $x$ and $y$ are coupled we can only specify one plot-dimensional scale, typically
a map \emph{scale} (for lower-case map projection code) or a map \emph{width} (for upper-case
map projection code).  However, in some cases it would be more
practical to specify map \emph{height} instead of \emph{width}, while in other situations it would be nice
to set either the \emph{shortest} or \emph{longest} map dimension.  Users may select
these alternatives by appending a character code to their map dimension.  To specify map \emph{height},
append \textbf{h} to the given dimension; to select the minimum map dimension, append \textbf{-}, whereas you may
append \textbf{+} to select the maximum map dimension.  Without the modifier the map width is
selected by default.

In \GMT\ version 4.3.0 we noticed we ran out of the alphabet for 1-letter (and sometimes 2-letter) projection codes. To allow more flexibility, and to make it easier to remember the codes, we implemented the option to use the abbreviations used by the \progname{Proj4} mapping package. Since some of the \GMT\ projections are not in \progname{Proj4}, we invented some of our own as well. For a full list of both the old 1- and 2-letter codes, as well as the \progname{Proj4}-equivalents see the quick reference cards in Section~\ref{sec:purpose}. For example, \Opt{JM15c} and \Opt{JMerc/15c} have the same meaning.

\input{GMT_Chapter_6.1}
\input{GMT_Chapter_6.2}
\input{GMT_Chapter_6.3}
\input{GMT_Chapter_6.4}

%------------------------------------------
%	$Id: GMT_Chapter_6.tex,v 1.19 2008-04-30 03:53:30 remko Exp $
%
%	The GMT Documentation Project
%	Copyright 2000-2008.
%	Paul Wessel and Walter H. F. Smith
%------------------------------------------
%
\chapter{\gmt\ Map Projections}
\label{ch:6}

\GMT\ implements more than 30 different projections.  They all project the input coordinates
longitude and latitude to positions on a map.  In general, $x' = f(x,y,z)$ and $y' = g(x,y,z)$, where
$z$ is implicitly given as the radial vector length to the $(x,y)$ point on the chosen ellipsoid.  The functions $f$ and $g$ can be
quite nasty and we will refrain from presenting details in this document.  The interested read is referred to
{\it Snyder} [1987]\footnote{Snyder, J. P., 1987, Map Projections \- A Working Manual, U.S. Geological Survey Prof. Paper 1395.}.
We will mostly be using the \GMTprog{pscoast} command to demonstrate each of the projections.
\GMT\ map projections are grouped into four categories depending on the
nature of the projection.  The groups are

\begin{enumerate}
\item Conic map projections
\item Azimuthal map projections
\item Cylindrical map projections
\item Miscellaneous projections
\end{enumerate}

Because $x$ and $y$ are coupled we can only specify one plot-dimensional scale, typically
a map \emph{scale} (for lower-case map projection code) or a map \emph{width} (for upper-case
map projection code).  However, in some cases it would be more
practical to specify map \emph{height} instead of \emph{width}, while in other situations it would be nice
to set either the \emph{shortest} or \emph{longest} map dimension.  Users may select
these alternatives by appending a character code to their map dimension.  To specify map \emph{height},
append \textbf{h} to the given dimension; to select the minimum map dimension, append \textbf{-}, whereas you may
append \textbf{+} to select the maximum map dimension.  Without the modifier the map width is
selected by default.

In \GMT\ version 4.3.0 we noticed we ran out of the alphabet for 1-letter (and sometimes 2-letter) projection codes. To allow more flexibility, and to make it easier to remember the codes, we implemented the option to use the abbreviations used by the \progname{Proj4} mapping package. Since some of the \GMT\ projections are not in \progname{Proj4}, we invented some of our own as well. For a full list of both the old 1- and 2-letter codes, as well as the \progname{Proj4}-equivalents see the quick reference cards in Section~\ref{sec:purpose}. For example, \Opt{JM15c} and \Opt{JMerc/15c} have the same meaning.

\input{GMT_Chapter_6.1}
\input{GMT_Chapter_6.2}
\input{GMT_Chapter_6.3}
\input{GMT_Chapter_6.4}

%------------------------------------------
%	$Id: GMT_Chapter_6.tex,v 1.19 2008-04-30 03:53:30 remko Exp $
%
%	The GMT Documentation Project
%	Copyright 2000-2008.
%	Paul Wessel and Walter H. F. Smith
%------------------------------------------
%
\chapter{\gmt\ Map Projections}
\label{ch:6}

\GMT\ implements more than 30 different projections.  They all project the input coordinates
longitude and latitude to positions on a map.  In general, $x' = f(x,y,z)$ and $y' = g(x,y,z)$, where
$z$ is implicitly given as the radial vector length to the $(x,y)$ point on the chosen ellipsoid.  The functions $f$ and $g$ can be
quite nasty and we will refrain from presenting details in this document.  The interested read is referred to
{\it Snyder} [1987]\footnote{Snyder, J. P., 1987, Map Projections \- A Working Manual, U.S. Geological Survey Prof. Paper 1395.}.
We will mostly be using the \GMTprog{pscoast} command to demonstrate each of the projections.
\GMT\ map projections are grouped into four categories depending on the
nature of the projection.  The groups are

\begin{enumerate}
\item Conic map projections
\item Azimuthal map projections
\item Cylindrical map projections
\item Miscellaneous projections
\end{enumerate}

Because $x$ and $y$ are coupled we can only specify one plot-dimensional scale, typically
a map \emph{scale} (for lower-case map projection code) or a map \emph{width} (for upper-case
map projection code).  However, in some cases it would be more
practical to specify map \emph{height} instead of \emph{width}, while in other situations it would be nice
to set either the \emph{shortest} or \emph{longest} map dimension.  Users may select
these alternatives by appending a character code to their map dimension.  To specify map \emph{height},
append \textbf{h} to the given dimension; to select the minimum map dimension, append \textbf{-}, whereas you may
append \textbf{+} to select the maximum map dimension.  Without the modifier the map width is
selected by default.

In \GMT\ version 4.3.0 we noticed we ran out of the alphabet for 1-letter (and sometimes 2-letter) projection codes. To allow more flexibility, and to make it easier to remember the codes, we implemented the option to use the abbreviations used by the \progname{Proj4} mapping package. Since some of the \GMT\ projections are not in \progname{Proj4}, we invented some of our own as well. For a full list of both the old 1- and 2-letter codes, as well as the \progname{Proj4}-equivalents see the quick reference cards in Section~\ref{sec:purpose}. For example, \Opt{JM15c} and \Opt{JMerc/15c} have the same meaning.

\input{GMT_Chapter_6.1}
\input{GMT_Chapter_6.2}
\input{GMT_Chapter_6.3}
\input{GMT_Chapter_6.4}

%------------------------------------------
%	$Id: GMT_Chapter_6.tex,v 1.19 2008-04-30 03:53:30 remko Exp $
%
%	The GMT Documentation Project
%	Copyright 2000-2008.
%	Paul Wessel and Walter H. F. Smith
%------------------------------------------
%
\chapter{\gmt\ Map Projections}
\label{ch:6}

\GMT\ implements more than 30 different projections.  They all project the input coordinates
longitude and latitude to positions on a map.  In general, $x' = f(x,y,z)$ and $y' = g(x,y,z)$, where
$z$ is implicitly given as the radial vector length to the $(x,y)$ point on the chosen ellipsoid.  The functions $f$ and $g$ can be
quite nasty and we will refrain from presenting details in this document.  The interested read is referred to
{\it Snyder} [1987]\footnote{Snyder, J. P., 1987, Map Projections \- A Working Manual, U.S. Geological Survey Prof. Paper 1395.}.
We will mostly be using the \GMTprog{pscoast} command to demonstrate each of the projections.
\GMT\ map projections are grouped into four categories depending on the
nature of the projection.  The groups are

\begin{enumerate}
\item Conic map projections
\item Azimuthal map projections
\item Cylindrical map projections
\item Miscellaneous projections
\end{enumerate}

Because $x$ and $y$ are coupled we can only specify one plot-dimensional scale, typically
a map \emph{scale} (for lower-case map projection code) or a map \emph{width} (for upper-case
map projection code).  However, in some cases it would be more
practical to specify map \emph{height} instead of \emph{width}, while in other situations it would be nice
to set either the \emph{shortest} or \emph{longest} map dimension.  Users may select
these alternatives by appending a character code to their map dimension.  To specify map \emph{height},
append \textbf{h} to the given dimension; to select the minimum map dimension, append \textbf{-}, whereas you may
append \textbf{+} to select the maximum map dimension.  Without the modifier the map width is
selected by default.

In \GMT\ version 4.3.0 we noticed we ran out of the alphabet for 1-letter (and sometimes 2-letter) projection codes. To allow more flexibility, and to make it easier to remember the codes, we implemented the option to use the abbreviations used by the \progname{Proj4} mapping package. Since some of the \GMT\ projections are not in \progname{Proj4}, we invented some of our own as well. For a full list of both the old 1- and 2-letter codes, as well as the \progname{Proj4}-equivalents see the quick reference cards in Section~\ref{sec:purpose}. For example, \Opt{JM15c} and \Opt{JMerc/15c} have the same meaning.

\input{GMT_Chapter_6.1}
\input{GMT_Chapter_6.2}
\input{GMT_Chapter_6.3}
\input{GMT_Chapter_6.4}



