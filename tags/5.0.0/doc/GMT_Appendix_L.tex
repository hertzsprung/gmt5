%------------------------------------------
%	$Id$
%
%	The GMT Documentation Project
%	Copyright (c) 2000-2012.
%	P. Wessel, W. H. F. Smith, R. Scharroo, and J. Luis
%------------------------------------------
%
\chapter{\gmt\ on non-\UNIX\ platforms}
\label{app:L}
\thispagestyle{headings}

\section{Introduction}
\index{GMT@\GMT!on non-\UNIX\ platforms}
\index{GMT@\GMT!PCs running Interix}
\index{GMT@\GMT!PCs running Linux}

While \GMT\ can be ported to non-\UNIX\ systems such as
Windows, it is also true that one of the
strengths of \GMT\ lies its symbiotic relationship with
\UNIX.  We therefore recommend that \GMT\ be installed in
a POSIX-compliant \UNIX\ environment such as traditional \UNIX-systems, Linux,
or Mac OS X.  If abandoning your non-\UNIX\ operating system
is not an option, consider one of these solutions:

\begin{description}
\item [WINDOWS:] Choose among these three possibilities:

\begin{enumerate}

\item Install \GMT\ under Cygwin (A GNU port to Windows). 

\item Install \GMT\ under SFU (Windows Services for \UNIX); a free download from
Microsoft\footnote{Microsoft Services for \UNIX\ is formerly known as Interix, in the distant past known as OpenNT.}.

\item Install \GMT\ in Windows using Microsoft C/C++ or other
compilers.  Unlike the first two, this option will not provide you with any
\UNIX\ tools so you will be limited to what you can do with
DOS batch files.

\index{GMT@\GMT!compile with Microsoft C/C++}

\end{enumerate}

\end{description}

\section{Cygwin and \gmt}
\index{GMT@\GMT!under Cygwin|(}
\index{Cygwin|(}

Because \GMT\ works best in conjugation with \UNIX\ tools we
suggest you install \GMT\ using the Cygwin product from
Cygnus (now assimilated by Redhat, Inc.).  This free version works on any Windows version
and it comes with both the Bourne Again shell \progname{bash} and the \progname{tcsh}.
You also have access to most standard GNU development tools such
as compilers and text processing tools (\progname{awk},
\progname{grep}, \progname{sed}, etc.).  Note that executables prepared for Windows
will also run under Cygwin.

Follow the instructions on the Cygwin page\footnote{cygwin.com} on how
to install the package; note you must explicitly add all the development tool
packages (e.g., \progname{gcc} etc) as the basic installation does not include them by default.
Once you are up and running under Cygwin, you may install \GMT\ 
the same way you do under any other \UNIX\ platform by either
running the automated install via \progname{install\_gmt} or manually
running configure first, then type make all.  If you install via the
web form, make sure you save the parameter file without DOS CR/LF
endings.  Use \progname{dos2unix} to get rid of those if need be.

Finally, from Cygwin's User Guide: By default, no Cygwin program
can allocate more than 384 MB of memory (program and data).
You should not need to change this default in most circumstances. However, if you need
to use more real or virtual memory in your machine you may add an entry in either the
\textbf{HKEY\_LOCAL\_MACHINE} (to change the limit for all users) or \textbf{HKEY\_CURRENT\_USER} (for just
the current user) section of the registry.  Add the DWORD value \textbf{heap\_chunk\_in\_mb} and set
it to the desired memory limit in decimal Mb.  It is preferred to do this in Cygwin
using the \progname{regtool} program included in the Cygwin package. (For more information about
\progname{regtool} or the other Cygwin utilities, see the Section called Cygwin Utilities in
Chapter 3 of the Cygwin's User Guide or use the \-\-help option of each utility.) You should always be careful
when using \progname{regtool} since damaging your system registry can result in an unusable system.
This example sets the local machine memory limit to 1024 Mb:
\begin{verbatim}
regtool -i set /HKLM/Software/Cygnus\ Solutions/Cygwin/heap_chunk_in_mb 1024
regtool -v list /HKLM/Software/Cygnus\ Solutions/Cygwin
\end{verbatim}
For more installation details see the general README file.

\index{GMT@\GMT!under Cygwin|)}
\index{Cygwin|)}

\section{SFU and \gmt}
\index{GMT@\GMT!under SFU|(}
\index{SFU (Windows Services for \UNIX)|(}

SFU\footnote{See www.microsoft.com/technet/interopmigration/unix/sfu for details.} is also similar to Cygwin
in that it provides precompiled \UNIX\ tools for DOS/WIN32,
including the \progname{sh} and \progname{csh} shells.  However, our experience has been negative in that
extra tools need to be installed and it is a painful and error-prone process; we cannot recommend it.
\index{GMT@\GMT!under SFU|)}
\index{SFU (Windows Services for \UNIX)|)}

\section{WIN32/64 and \gmt}
\index{GMT@\GMT!under Win32|(}
\index{GMT@\GMT!under Win64|(}
\index{Win32 and \GMT|(}
\index{Win64 and \GMT|(}

\GMT\ will compile and install using the Microsoft Visual C/C++
compiler.  We expect other WIN32/64 C compilers to give similar
results (e.g., the Intel compiler).  Since \progname{configure} cannot be run you
must manually rename \filename{gmt\_notposix.h.in} to
\filename{gmt\_notposix.h}.  The netCDF home page gives full
information on how to compile and install netCDF; precompiled
libraries are also available.  At present we simply have a lame \filename{gmtinstall.bat}
file that compiles the entire \GMT\
package, and \filename{gmtsuppl.bat} which compiles most of the
supplemental programs.  If you just need to run \GMT\ and do not want to mess with compilations,
get the precompiled binaries from the \GMT\ ftp sites.

\index{GMT@\GMT!under Win32|)}
\index{GMT@\GMT!under Win64|)}
\index{Win32 and \GMT|)}
\index{Win64 and \GMT|)}

\section{Mac OS and \gmt}
\index{GMT@\GMT!under Mac OS}
\index{Mac OS and \GMT}

\GMT\ will install directly under Mac OS X which is Unix-compliant.
