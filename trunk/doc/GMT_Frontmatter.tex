%------------------------------------------
%	$Id: GMT_Frontmatter.tex,v 1.10 2003-01-14 02:38:03 pwessel Exp $
%
%	The GMT Documentation Project
%	Copyright 2000-2003.
%	Paul Wessel and Walter H. F. Smith
%------------------------------------------
%

%----------------------ACKNOWLEDGMENTS--------------------------

\chapter*{Acknowledgments}\index{Acknowledgments}
\addcontentsline{toc}{chapter}{Acknowledgments}

The Generic Mapping Tools (\GMT) could not have been designed without
the generous support of several people.  We gratefully acknowledge
A. B. Watts and W. F. Haxby for supporting our efforts on the original
version 1.0 while we were their graduate students at Lamont-Doherty
Earth Observatory.  Doug Shearer and Roger Davis patiently answered
many of our questions over e-mail.  The subroutine \texttt{gaussj} was
written and supplied by Bill Menke, L-DEO.
Further development of versions 2.0 and 2.1 at SOEST would not have
been possible without the support from the Hawaii Institute of
Geophysics and School of Ocean and Earth Science and Technology
Post-Doctoral Fellowship program to Paul Wessel.  Walter H. F. Smith
gratefully acknowledges the generous support of the C. H. and I. M.
Green Foundation for Earth Sciences at the Institute of Geophysics
and Planetary Physics, Scripps Institution of Oceanography, University
of California at San Diego.
\GMT\ versions 3.0--3.4 owe their existence to grants EAR-93-02272, OCE-95-29431,
and OCE-0082552 from the National Science Foundation, which we
gratefully acknowledge.

We would like to acknowledge the feedback we have received from many
of the users of earlier versions.  Many of these suggestions have
been implemented, and the bug reports have been useful in providing
more robust programs.  Specifically, we would like to thank William
Weibel, Ameet Raval, Manfred Brands, Angel Li, Andrew Macrae, John
Lillibridge, Richard Signell, Michael Barck, Alex Madon, Ben Horner-Johnson,
Georg Schwarz, Lloyd Parkes, David Townsend, and many others for
advice on how to make \GMT\ portable to DEC, SGI, HP, IBM, Apple, and
NEXT workstations.  John Lillibridge provided example 11.  William
Yip helped translate \GMT\ to POSIX ANSI C and incorporate netCDF 3, 
Allen Cogbill provided OS/2 patches for EMX, and Hanno von Lom helped
resolve problems with DLL libraries for Win32.

\begin{flushright}
Honolulu, HI and Silver Spring, MD, January 2003
\end{flushright}
\vspace{\baselineskip}

\begin{center}
\small
\begin{tabular}{lr}
\epsfig{figure=eps/Wessel.eps}	&	\epsfig{figure=eps/Smith.eps} \\
\normalsize \htmladdnormallink{{\textbf {Dr. P\aa l (Paul) Wessel}}}{http://www.soest.hawaii.edu/wessel} & \normalsize \htmladdnormallink{{\textbf {Dr. Walter H. F. Smith}}}{mailto:walter@raptor.grdl.noaa.gov} \\
Professor & Geophysicist \\
\htmladdnormallink{Department of Geology and Geophysics}{http://www.soest.hawaii.edu/GG} & \htmladdnormallink{Laboratory for Satellite Altimetry}{http://www.grdl.noaa.gov} \\
\htmladdnormallink{School of Ocean and Earth Science and Technology}{http://www.soest.hawaii.edu} & \htmladdnormallink{National Oceanographic Data Center}{http://www.nodc.noaa.gov} \\
\htmladdnormallink{University of Hawaii at Manoa}{http://www.hawaii.edu} & \htmladdnormallink{National Oceanic and Atmospheric Administration}{http://www.noaa.gov} \\
\normalsize
\end{tabular}
\end{center}

% HORIZONTAL MUGSHOTS FOR LATEX

%\begin{latexonly}
%\noindent
%\frame{\epsfig{figure=eps/Wessel.eps}}
%\hfill
%\frame{\epsfig{figure=eps/Smith.eps}}
%\par
%\vspace{0.5\baselineskip}
%\noindent
%\htmladdnormallink{{\textbf {Dr. P\aa l (Paul) Wessel}}}{http://www.soest.hawaii.edu/wessel}
%\hfill \htmladdnormallink{{\textbf {Dr. Walter H. F. Smith}}}{mailto:walter@raptor.grdl.noaa.gov} \\
%\small
%\noindent
%Professor \hfill Geophysicist \\
%Department of Geology and Geophysics \hfill Laboratory for Satellite Altimetry \\
%School of Ocean and Earth Science and Technology \hfill National Oceanographic Data Center \\
%University of Hawaii at Manoa \hfill National Oceanic and Atmospheric Administration \\
%\normalsize
%\end{latexonly}

% VERTICAL MUGSHOTS FOR HTML

%\begin{htmlonly}
%\noindent
%\epsfig{figure=eps/Wessel.eps}
%\par
%\vspace{0.5\baselineskip}
%\noindent
%\htmladdnormallink{{\textbf {Dr. P\aa l (Paul) Wessel}}}{http://www.soest.hawaii.edu/wessel} \\
%\small
%\noindent
%Professor \\
%\htmladdnormallink{Department of Geology and Geophysics}{http://www.soest.hawaii.edu/GG} \\
%\htmladdnormallink{School of Ocean and Earth Science and Technology}{http://www.soest.hawaii.edu} \\
%\htmladdnormallink{University of Hawaii at Manoa}{http://www.hawaii.edu} \\
%\normalsize

%\noindent
%\frame{\epsfig{figure=eps/Smith.eps}}
%\par
%\vspace{0.5\baselineskip}
%\noindent
%\htmladdnormallink{{\textbf {Dr. Walter H. F. Smith}}}{mailto:walter@raptor.grdl.noaa.gov} \\
%\small
%\noindent
%Geophysicist \\
%\htmladdnormallink{Laboratory for Satellite Altimetry}{http://www.grdl.noaa.gov} \\
%\htmladdnormallink{National Oceanographic Data Center}{http://www.nodc.noaa.gov} \\
%\htmladdnormallink{National Oceanic and Atmospheric Administration}{http://www.noaa.gov} \\
%\normalsize

%\end{htmlonly}


%------------------------GMT DOCUMENTATION PROJECT---------------------------

\chapter*{The GMT Documentation Project}
\addcontentsline{toc}{chapter}{The GMT Documentation Project}

Starting with \GMT\ version 3.2, all \GMT\ documentation was
converted from Microsoft \progname{Word} to \LaTeX\ files.
This step was taken for a number of reasons:

\begin{enumerate}

\item Having all the documentation source available in
ASCII format makes it easier to access by several
\GMT\ developers working on different platforms in 
different countries.

\item \GMT\ scripts can now be included directly into the text
so that the documentation is automatically up-to-date
when scripts are modified.

\item All figures are generated on the fly and included as
\GMT\ EPS files which thus are always up-to-date.

\item It is easy to convert the \LaTeX\ files to other
formats, such as HTML, SGML, and \PS.

\item The whole task of assembling the pieces, be it generating
figures or extracting text portions from the master archive under
SCCS control, is automated by a simple cshell script.

\item Only free software are used to maintain the \GMT\ Documentation.

\end{enumerate}

Because this transition was undertaken by a complete \LaTeX\ novice
it is likely that the layout will change with time.  It is also
likely that errors have crept into the document.  Please send email
to \htmladdnormallink{the \GMT\ team}{mailto:gmt-team@hawaii.edu}
if you find any.

%--------------------------------REMINDER------------------------------------

\chapter*{A Reminder}
\addcontentsline{toc}{chapter}{A Reminder}

If you feel it is appropriate, you may consider paying us back by
citing our {\it EOS} articles on \GMT\ (and perhaps also our Geophysics
article on the \GMT\ program \GMTprog{surface}) when you publish papers
containing results or illustrations obtained using \GMT.  The EOS
articles on \GMT\ are \\
\index{EOS article}%
\index{Article!in EOS}%

\begin{itemize}

\item{Wessel, P., and W. H. F. Smith, New, improved version of Generic
Mapping Tools released, {\it EOS Trans. Amer. Geophys. U.}, vol. 79
(47), pp. 579, 1998.}

\item{Wessel, P., and W. H. F. Smith, New version of the Generic
Mapping Tools released, {\it EOS Trans. Amer. Geophys. U.}, vol. 76
(33), pp. 329, 1995.}

\item{Wessel, P., and W. H. F. Smith, New version of the Generic
Mapping Tools released, {\it EOS Trans. Amer. Geophys. U. electronic
supplement}, \htmladdnormallink{http://www.agu.org/eos\_elec/95154e.html}
{http://www.agu.org/eos\_elec/95154e.html}, 1995.}

\item{Wessel, P., and W. H. F. Smith, Free software helps map and
display data, {\it EOS Trans. Amer. Geophys. U.}, vol. 72 (41),
pp. 441, 445-446, 1991.}

\end{itemize}

The article in {\it Geophysics} on surface is
\index{Geophysics article}%
\index{Article!in Geophysics}%
\begin{itemize}

\item{Smith, W. H. F., and P. Wessel, Gridding with continuous
curvature splines in tension, {\it Geophysics}, vol. 55 (3), pp.
293-305, 1990.}

\end{itemize}

A few \GMT\ users take the time to write us letters, telling us of the
difference \GMT\ is making in their work.  We appreciate receiving these
letters.  On days when we wonder why we ever released \GMT\ we pull
these letters out and read them.  Seriously, as financial support for
\GMT\ depends on how well we can ``sell'' the idea to funding agencies and
our superiors, letter-writing is one area where \GMT\ users can affect
such decisions by supporting the \GMT\ project. 

%-------------------------COPYRIGHT----------------------------------

\chapter*{Copyright and Caveat Emptor!}
\index{Copyright}
\addcontentsline{toc}{chapter}{Copyright and Caveat Emptor!}

\begin{center}
Copyright \copyright 1991-- 2003 by Paul Wessel and Walter H. F. Smith
\end{center}

\vspace{\baselineskip}

The Generic Mapping Tools (\GMT) is free software; you can redistribute
it and/or modify it under the terms of the GNU General Public License
as published by the Free Software Foundation. \\

The \GMT\ package is distributed in the hope that it will be useful, but
WITHOUT ANY WARRANTY; without even the implied warranty of
MERCHANTABILITY or FITNESS FOR A PARTICULAR PURPOSE.  See the
file \filename{COPYING} in the \GMT\ directory or the
\htmladdnormallinkfoot{GNU General Public License}{http://www.gnu.org/copyleft/gpl.html}
for more details. \\

Permission is granted to make and distribute verbatim copies of this
manual provided that the copyright notice and these paragraphs are
preserved on all copies.  The \GMT\ package may be included in a bundled
distribution of software for which a reasonable fee may be charged. \\

The Generic Mapping Tools (\GMT) does not come with any warranties,
nor is it guaranteed to work on your computer.  The user assumes full
responsibility for the use of this system. In particular, the School of
Ocean and Earth Science and Technology, the National Oceanic and
Atmospheric Administration, the National Science Foundation,
Paul Wessel, Walter H. F. Smith, or any other individuals involved in
the design and maintenance of \GMT\ are NOT responsible for any damage
that may follow from correct \emph{or} incorrect use of these programs.

%----------------------TYPOGRAPHIC CONVENTIONS------------------------

\chapter*{Typographic conventions}
\index{Typographic conventions}
\addcontentsline{toc}{chapter}{Typographic conventions}

In reading this documentation, the following provides a summary of
the typographic conventions used in this document.

\begin{enumerate}

\item User input and \GMT\ or \UNIX\ commands are indicated by
using the \texttt{typewriter} type style, e.g.,
\texttt{chmod +x job03.csh}.

\item The names of \GMT\ programs are indicated by the
\textsf{\textbf{bold, sans serif}} type style, e.g.,
we plot text with \textsf{\textbf{pstext}}.

\begin{latexonly}
\item The names of other programs are indicated by the
\textsl{\textbf{bold, slanted}} type style, e.g.,
\textsl{\textbf{grep}}.
\end{latexonly}
\begin{htmlonly}
\item The names of other programs are indicated by the
\textit{italic} type style, e.g., \textit{grep}.
\end{htmlonly}

\item File names are indicated by the \underline{underline}
type style, e.g., \underline{gmt.h}.

\end{enumerate}
