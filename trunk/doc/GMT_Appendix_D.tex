%------------------------------------------
%	$Id: GMT_Appendix_D.tex,v 1.33 2011-04-07 00:55:30 guru Exp $
%
%	The GMT Documentation Project
%	Copyright (c) 2000-2011.
%	P. Wessel, W. H. F. Smith, R. Scharroo, and J. Luis
%------------------------------------------
%
\chapter{Availability of \gmt\ and related code}
\label{app:D}
\index{GMT@\GMT!obtaining}
\thispagestyle{headings}

\section{Source distribution}
All the source code, support data, PDF
and HTML versions of all documentation (including \UNIX\
manual pages) can be obtained by anonymous
ftp from several mirror sites.  We also maintain a \GMT\
page on the World Wide Web (http://\GMTSITE);
see this page for installation directions 
which allow for a simplified, automatic install procedure
(for the purchase of CD-R and DVD-R media, see \htmladdnormallink{http://www.geoware-online.com}{http://www.geoware-online.com}.)

The \GMT\ compressed tar archives requires \progname{bzip2} to expand.  If this utility
is not installed on your system, you must obtained it by your system's package manager
or install it separately\footnote{http://www.bzip.org}.
The GMT archives are as follows:

\begin{description}

\item[gmt-\GMTDOCVERSION.tar.bz2] Contains all \GMT\ and supplemental source code needed for compilation, support files
	needed at run-time (cpt files, symbols and \PS\ patterns), and all documentation
	(man pages, Cookbook and Technical Reference, and the tutorial), the data files
	used in the tutorial, and all the shell scripts and support data used in the Cookbook section.

\item[gshhs-\GSHHSVERSION.tar.bz2] Contains all resolutions (full, high, intermediate,
low, and crude) of the GSHHS coastline database.  Required to run \GMT.

\end{description}

\index{netCDF, obtaining}

The netCDF library that makes up the backbone of the grid file
i/o operations can be obtained from Unidata by downloading he file
\filename{netcdf.tar.Z} from the anonymous FTP directory of
\underline{unidata.ucar.edu}.

\index{GMT@\GMT!subversion install}
\section{Install via subversion}

The \GMT\ development tree can be installed via subversion.  Simply run
\begin{verbatim}
	svn checkout svn://pohaku.soest.hawaii.edu/gmt5
\end{verbatim}

\index{GMT@\GMT!binaries for Win32}
\section{Pre-compiled Executables}

For Windows users who just want executables we have three Windows installers available.  Choose one
of the first two and \emph{optionally} the third:

\begin{description}

\item[gmt-\GMTDOCVERSION\_install32.exe] The 32-bit install with all \GMT\ executables (including supplements),
the netCDF DLL, the complete set of GSHHS coastlines, rivers, and borders, the example batch scripts and data, and all documentation in HTML format.

\item[gmt-\GMTDOCVERSION\_install64.exe] The 64-bit install with all \GMT\ executables (including supplements),
the netCDF DLL, the complete set of GSHHS coastlines, rivers, and borders, the example batch scripts and data, and all documentation in HTML format.

\item[gmt-\GMTDOCVERSION\_pdf\_install.exe] Installer for the optional \GMT\ documentation in PDF format.

\end{description}
Usually, only one of the 32- or 64-bit installers will be needed.

