%------------------------------------------
%       $Id: GMT_Appendix_B.tex,v 1.16 2005-12-18 15:56:05 remko Exp $
%
%       The GMT Documentation Project
%       Copyright 2000-2006.
%       Paul Wessel and Walter H. F. Smith
%------------------------------------------
%

\chapter{\gmt\ file formats}
\label{app:B}
\thispagestyle{headings}

\section{Table data} 
\index{Table!format}

These files have {\it N} records which have {\it M} fields each.
Most programs can read multicolumn files, but require that the
{\it x} [and {\it y}] variable(s) be stored in the 1st [and 2nd]
column (There are, however, some exceptions to this rule, such as
\GMTprog{filter1d} and \GMTprog{sample1d}).  \GMT\ can read both ASCII and
binary table data.

\subsection{ASCII tables}
\index{Table!format!ASCII}
\index{Table!multisegment}
\subsubsection{Optional file header records}
The first data record may be preceded by 1 or more header records.
When using such files, make sure to use the \Opt{H} option and set
the parameter {\bf N\_HEADER\_RECS} in the \filename{.gmtdefaults4} file
(System default is 1 header record if \Opt{H} is set; you may also
use \Opt{H}{\it nrecs} directly).  Fields
within a record must be separated by spaces, tabs, or commas.
Each field can be an integer or floating-point number or a geographic
coordinate string using the [+$|$-]dd[:mm[:ss]][W$|$S$|$N$|$E$|$w$|$s$|$n$|$e] format.
Thus, 12:30:44.5W, 17.5S, 1:00:05, and 200:45E are all valid input strings.
\subsubsection{Optional segment header records}
When dealing with time- or ({\it x,y})-series it is usually
convenient to have each profile in separate files.
However, this may sometimes prove impractical due to large numbers
of profiles.  An example is files of digitized lineations where
the number of individual features may range into the thousands.
One file per feature would in this case be unreasonable and
furthermore clog up the directory. \GMT\ provides a mechanism for
keeping more than one profile in a file.  Such files are called
\emph{multiple segment files} and are identical to the ones
just outlined except that they have subheaders interspersed with
data records that signal the start of a segment.
The subheaders may be of any format, but all must have the same
character in the first column.  When using such files, you must
specify the \Opt{M} option.  The unique character is by default
'$>$', but you can override that by appending your chosen character
to the M option.  E.g., \Opt{MH} will look for subheaders starting
with H, whereas \Opt{M'*'} will check for asterisks (The quotes
are necessary since * has special meaning to \UNIX).  Some programs
such as \GMTprog{psxy} will examine the subheaders to see if they
contain \Opt{W} and \Opt{G} options for specifying pen and
fill attributes for individual segments, \Opt{Z} to change
color via a cpt-file, or \Opt{L} for label specifications.  These settings
(and occasionally others) will override the corresponding command line options.

\subsection{Binary tables}
\index{Table!format!binary}

\GMT\ programs also support binary tables to speed up input-output
for i/o-intensive tasks like gridding and preprocessing.  Files
may have no header (hence the \Opt{H} option cannot be used)
and all data must either be single or double precision (no mixing
allowed).  Multiple segment files are allowed (\Opt{M}) and the
segment headers are assumed to be records where all the fields equal 
NaN.  Flags appended to \Opt{M} are ignored.  The format and number
of fields are specified with the
\Opt{b} option.  Thus, for input you may set \Opt{bi}[{\bf s}][{\it n}],
where {\bf s} designates single precision (default is {\bf d} for double) and
{\it n} is the number of fields.  For output, use \Opt{bo}[{\bf s}] 
(the programs know how many columns to write, unless you use \Opt{M} in which
case we need to know the number of output columns up front).  If you need
to swap the byte-order on either input or output you must use upper case
{\bf S} or {\bf D} instead.

\section{Grid files} 

\subsection{NetCDF files}
\index{grdfile!formats!netCDF|(}

By default, \GMT\ stores 2-D grids as COARDS-compliant netCDF files.
COARDS (Cooperative Ocean/Atmosphere Research Data Service) is a convention used by
many agencies distributing gridded data for ocean and atmosphere research. Sticking to
this convention allows \GMT\ to read gridded data provided by other institutes
and other programs. Conversely, other general domain programs
will be able to read grids created by \GMT.

The netCDF grdfile in \GMT\ has several attributes (See Table~\ref{tbl:netcdf-format})
to describe the content. The routine
that deals with netCDF grdfiles is sufficiently flexible so that grdfiles slightly deviating
from the standards used by \GMT\ can also be read.

\begin{table}
\centering
\begin{tabular}{|l|l|} \hline
\multicolumn{1}{|c}{\emph{Attribute}}   &       \multicolumn{1}{|c|}{\emph{Description}}        \\ \hline
\multicolumn{2}{|c|}{\emph{Global attributes}} \\ \hline
Conventions			& COARDS (optional) \\ \hline
title				& Title (optional) \\ \hline
source				& How file was created (optional) \\ \hline
node\_offset			& 0 for grid line registration (default), 1 for pixel registration \\ \hline
\multicolumn{2}{|c|}{\emph{$x$- and $y$-variable attributes}} \\ \hline
long\_name			& Variable name (and unit) (default: ``x'' and ``y'') \\ \hline
actual\_range			& Minimum and maximum $x$ and $y$ of region; if absent \\
(or valid\_range)		& the first and last $x$- and $y$-values are queried \\ \hline
\multicolumn{2}{|c|}{\emph{$z$-variable attributes}} \\ \hline
long\_name			& Variable name (and unit) (default: ``z'') \\ \hline
scale\_factor			& Factor to multiply $z$ with (default: 1) \\ \hline
add\_offset			& Offset to add to scaled $z$ (default: 0) \\ \hline
actual\_range			& Minimum and maximum $z$ (optional) \\ \hline
\_FillValue			& Value associated with missing data points; if absent an\\
(or missing\_value)		& approriate default value is assumed, depending on data type. \\ \hline
\end{tabular} 
\caption{Attributes of default \gmt\ gridded file in COARDS-compliant netCDF format.}
\label{tbl:netcdf-format}
\end{table}

By default, the first 2-D or 3-D variable in a netCDF file will by read as the $z$ variable
and the coordinate axes $x$ and $y$ will be determined from the dimensions of the $z$ variable.
\GMT\ will recognise whether the $y$ (latitude) variable increases or decreases. Both forms of
data storage are handled appropriately. When the $z$ variable is 3-dimensional, such as
is the case of longitude-latitude-depth grids, only the first ``layer'' will be loaded.
For more information on the use of COARDS-compliant netCDF files, read Section~\ref{sec:netcdf}.

\index{grdfile!formats!netCDF|)}

\GMT\ also allows other formats to be read.  In addition to
the default netCDF format it can use binary floating points, short
integers, bytes, and bits, as well as 8-bit Sun rasterfiles (colormap
ignored).   Additional formats may be used by supplying read/write
functions and linking these with the \GMT\ libraries.
The source file \filename{gmt\_customio.c} has the information that
programmers will need to augment \GMT\ to read custom grdfiles.  We
anticipate that the number of pre-programmed formats will increase as
enterprising users implement what they need.  See Section~\ref{sec:grdformats}
for more information.

\subsection{Grid line and Pixel registration} 
\index{grdfile!registration|(}

Scanline format means that the data are stored in rows ({\it y} = constant)
going from the ``top'' ($y = y_{max}$ (north)) to the ``bottom''
($y = y_{min}$ (south)).  Data within each row are ordered from
``left'' ($x = x_{min}$ (west)) to ``right'' ($x = x_{max}$
(east)).  The {\it node\_offset} signals how the nodes are laid out.
The grid is always defined as the intersections of all {\it x}
( \( x = x_{min}, x_{min} + x_{inc}, x_{min} + 2 \cdot x_{inc}, \ldots, x_{max} \) )
and {\it y} ( \( y = y_{min}, y_{min} + y_{inc}, y_{min} + 2 \cdot y_{inc}, \ldots, y_{max} \) )
lines.  The two scenarios differ in which area each data point
represents.
The default registration in \GMT\ is grid line registration.  Most
programs can handle both types, and for some programs like \GMTprog{grdimage}
a pixel registered file makes more sense.  Utility programs like
\GMTprog{grdsample} and \GMTprog{grdproject} will allow you to convert
from one format to the other.

\subsubsection{Grid line registration}
\index{grdfile!registration!grid line|(}

In this registration, the nodes are centered on the grid line
intersections and the data points represent the average value
in a cell of dimensions ($x_{inc} \cdot y_{inc}$) centered on the
nodes (Figure~\ref{fig:GMT_gridreg}).
In the case of grid line registration the number of nodes are
related to region and grid spacing by \\

\[ \begin{array}{ccl} 
nx & =  &       (x_{max} - x_{min}) / x_{inc} + 1       \\ 
ny & =  &       (y_{max} - y_{min}) / y_{inc} + 1
\end{array} \]

which for the example in Figure~\ref{fig:GMT_gridreg} yields $nx = ny = 4$.

\GMTfig[h]{GMT_gridreg}{Grid line registration of data nodes.}

\index{grdfile!registration!grid line|)}

\subsubsection{Pixel registration}
\index{grdfile!registration!pixel|(}

Here, the nodes are centered in the grid cells, i.e., the areas
between grid lines, and the data points represent the average
values within each cell (Figure~\ref{fig:GMT_pixelreg}.
In the case of pixel registration the number of nodes are related
to region and grid spacing by \\

\[ \begin{array}{ccl} 
nx & =  &       (x_{max} - x_{min}) / x_{inc}   \\ 
ny & =  &       (y_{max} - y_{min}) / y_{inc}
\end{array} \]

Thus, given the same region (\Opt{R}), the pixel registered grids have one less
column and one less row than the grid line registered grids; here we
find $nx = ny = 3$.

\GMTfig[H]{GMT_pixelreg}{Pixel registration of data nodes.}

\index{grdfile!registration!pixel|)}
\index{grdfile!registration|)}

\subsection{Boundary Conditions for operations on grids} 
\index{grdfile!boundary conditions|(}

\GMT\ has the option to specify boundary conditions in some programs
that operate on grids (\GMTprog{grdsample} \Opt{L}; \GMTprog{grdgradient}
\Opt{L}; \GMTprog{grdtrack} \Opt{L}; \GMTprog{nearneighbor} \Opt{L};
\GMTprog{grdview} \Opt{L}).  The boundary conditions come into play
when interpolating or computing derivatives near the limits of the
region covered by the grid. The \emph{default} boundary
conditions used are those which are ``natural'' for the boundary
of a minimum curvature interpolating surface.
If the user knows that the data are periodic in {\it x} (and/or
{\it y}), or that the data cover a sphere with {\it x},{\it y}
representing {\it longitude},{\it latitude}, then there are better
choices for the boundary conditions.
Periodic conditions on {\it x} (and/or {\it y}) are chosen by
specifying {\it x} (and/or {\it y}) as the boundary condition flags;
global spherical cases are specified using the {\it g} (geographical)
flag.  Behavior of these conditions is as follows:

\begin{description}
\index{grdfile!boundary conditions!periodic}

\item[Periodic] conditions on {\it x} indicate that the data are
periodic in the distance ($x_{max} - x_{min}$) and thus repeat
values after every $N = (x_{max} - x_{min})/x_{inc}$.  Note that
this implies that in a grid-registered file the values in the first
and last columns are equal, since these are located at $x = x_{min}$
and $x = x_{max}$, and there are {\it N} + 1 columns in the file.
This is not the case in a pixel-registered file, where there are only
{\it N} and the first and last columns are located at
$x_{min} + x_{inc}/2$ and $x_{max} - x_{inc}/2$.  If {\it y} is
periodic all the same holds for {\it y}.

\item[Geographical] conditions indicate the following:
\index{grdfile!boundary conditions!geographical}

\begin{enumerate}

\item If $(x_{max} - x_{min}) \geq 360$ and also 180 modulo $x_{inc} = 0$
then a periodic condition is used on {\it x} with a period of 360;
else a default condition is used on the {\it x} boundaries.

\item If condition 1 is true and also $y_{max} = 90$ then a
``north pole condition'' is used at $y_{max}$, else a default
condition is used there.

\item If condition 1 is true and also $y_{min} = -90$ then a
``south pole condition'' is used at $y_{min}$, else a default
condition is used there.

\end{enumerate} 

``Pole conditions'' use a 180\DS\ phase-shift of the data,
requiring 180 modulo $x_{inc} = 0$.

\item[Default] boundary conditions are
\index{grdfile!boundary conditions!default}

\[ \nabla^2 f = \frac{\partial}{\partial n} \nabla^2 f = 0 \] 

on the boundary, where $f(x, y)$ is represented by the values in
the grid file, and $\partial/\partial n$ is the derivative in the direction normal to a
boundary, and

\[ \nabla^2 = \left(\frac{\partial^2}{\partial x^2} + \frac{\partial^2}{\partial y^2}\right) \]

is the two-dimensional Laplacian operator.

\end{description}
\index{grdfile!boundary conditions|)}

\subsection{Native binary grid files}
\index{grdfile!native binary|(}

The old style native grid file format that was common in earlier version of \GMT\ is still
supported, although the use of netCDF files is strongly recommended.
The file starts with a header containing a number of attributes defining the content.
The \GMTprog{grdedit} utility program will allow you to edit parts of
the header of an existing grdfile.  The attributes listed in Table~\ref{tbl:grdheader}
are contained within the header record in the order given (except the $z$-array which
is not part of the header structure, but makes up the rest of the file).

\begin{table}
\centering
\begin{tabular}{|l|l|} \hline
\multicolumn{1}{|c}{\emph{Parameter}}   &       \multicolumn{1}{|c|}{\emph{Description}}        \\ \hline
{\bf int} {\it nx}                      &       Number of nodes in the $x$-dimension    \\ \hline
{\bf int} {\it ny}                      &       Number of nodes in the y-dimension      \\ \hline
{\bf int} {\it node\_offset}    &       0 for grid line registration, 1 for pixel registration  \\ \hline
{\bf double} {\it x\_min}               &       Minimum $x$-value of region     \\ \hline
{\bf double} {\it x\_max}               &       Maximum $x$-value of region  \\ \hline
{\bf double} {\it y\_min}               &       Minimum $y$-value of region  \\ \hline
{\bf double} {\it y\_max}               &       Maximum $y$-value of region  \\ \hline
{\bf double} {\it z\_min}               &       Minimum $z$-value in data set  \\ \hline
{\bf double} {\it z\_max}               &       Maximum $z$-value in data set  \\ \hline
{\bf double} {\it x\_inc}               &       Node spacing in $x$-dimension  \\ \hline
{\bf double} {\it y\_inc}               &       Node spacing in $y$-dimension  \\ \hline
{\bf double} {\it z\_scale\_factor}     &       Factor to multiply $z$-values after read  \\ \hline
{\bf double} {\it z\_add\_offset}       &       Offset to add to scaled $z$-values  \\ \hline
{\bf char} {\it x\_units}[80]   &       Units of the $x$-dimension      \\ \hline
{\bf char} {\it y\_units}[80]   &       Units of the $y$-dimension      \\ \hline
{\bf char} {\it z\_units}[80]   &       Units of the $z$-dimension      \\ \hline 
{\bf char} {\it title}[80]      &       Descriptive title of the data set       \\ \hline
{\bf char} {\it command}[320]   &       Command line that produced the grdfile  \\ \hline
{\bf char} {\it remark}[160]    &       Any additional comments \\ \hline \hline
{\bf TYPE} {\it z}[nx*ny]      &       1-D array with $z$-values in scanline format \\ \hline
\end{tabular}
\caption{\gmt\ gridded file header record. {\bf TYPE} can be {\bf char}, {\bf short}, {\bf int}, {\bf float}, or {\bf
double}.}
\label{tbl:grdheader}
\end{table}
\index{grdfile!native binary|)}

\section{Sun raster files}
\index{Rasterfile!format}

The Sun raster file format consists of a header followed by a series
of unsigned 1-byte integers that represents the bit-pattern.  Bits
are scanline oriented, and each row must contain an even number of
bytes.  The predefined 1-bit
patterns in \GMT\ have dimensions of 64 by 64, but other sizes will be
accepted when using the \Opt{Gp$|$P} option.  The Sun header structure
is outline in Table~\ref{tbl:sunheader}.

\begin{table}[H]
\centering

\begin{tabular}{|l|l|}  \hline
\multicolumn{1}{|c}{\emph{Parameter}}   &       \multicolumn{1}{|c|}{\emph{Description}}        \\ \hline
{\bf int} {\it ras\_magic}      &       Magic number  \\ \hline
{\bf int} {\it ras\_width}      &       Width (pixels) of image  \\ \hline
{\bf int} {\it ras\_height}     &       Height (pixels) of image  \\ \hline
{\bf int} {\it ras\_depth}      &       Depth (1, 8, 24, 32 bits) of pixel  \\ \hline
{\bf int} {\it ras\_length}     &       Length (bytes) of image  \\ \hline
{\bf int} {\it ras\_type}       &       Type of file; see RT\_* below  \\ \hline
{\bf int} {\it ras\_maptype}    &       Type of colormap; see RMT\_* below  \\ \hline
{\bf int} {\it ras\_maplength}  &       Length (bytes) of following map  \\ \hline
\end{tabular}

\caption{Structure of a Sun rasterfile.}
\label{tbl:sunheader}
\end{table} 

After the header, the color map (if {\it ras\_maptype} is not RMT\_NONE)
follows for {\it ras\_maplength} bytes, followed by an image of
{\it ras\_length} bytes.  Some related definitions are given in Table~\ref{tbl:sundef}.  

\begin{table}[H]
\index{Rasterfile!definitions}
\centering
\begin{tabular}{|l|l|}  \hline

\multicolumn{1}{|c}{\emph{Macro name}}  &       \multicolumn{1}{|c|}{\emph{Description}}        \\ \hline
RAS\_MAGIC      &       0x59a66a95  \\ \hline
RT\_STANDARD    &       1 (Raw pixrect image in 68000 byte order)  \\ \hline
RT\_BYTE\_ENCODED       &       2 (Run-length compression of bytes)  \\ \hline
RT\_FORMAT\_RGB &       3 ([X]RGB instead of [X]BGR)  \\ \hline
RMT\_NONE       &       0 (ras\_maplength is expected to be 0)  \\ \hline
RMT\_EQUAL\_RGB &       1 (red[ras\_maplength/3],green[],blue[])  \\ \hline

\end{tabular}

\caption{Sun macro definitions relevant to rasterfiles.}
\label{tbl:sundef}
\end{table} 

Numerous public-domain programs exist, such as \progname{xv} and
\progname{convert} (in the ImageMagick package), that will translate
between various rasterfile formats such as tiff, gif, jpeg, and
Sun raster.  Raster patterns may be created with \GMT\ plotting
tools by generating \PS\ plots that can be rasterized
by \progname{ghostscript} and translated into the right raster format.
