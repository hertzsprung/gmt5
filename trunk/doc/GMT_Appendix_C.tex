%------------------------------------------
%	$Id: GMT_Appendix_C.tex,v 1.10 2005-07-12 04:13:24 pwessel Exp $
%
%	The GMT Documentation Project
%	Copyright 2000-2005.
%	Paul Wessel and Walter H. F. Smith
%------------------------------------------
%

\chapter{Making \gmt\ Encapsulated \PS\ Files}
\index{PostScript@\PS!encapsulated (EPS)}
\index{EPS file}
\thispagestyle{headings}

\GMT\ can produce both freeform \PS\ files and the more
restricted Encapsulated \PS\ files (EPS).  The former is
intended to be sent to a printer or \PS\ previewer, while
the latter is indended to be included in another document
(but should also be able to print and preview).  You
control what kind of \PS\ that \GMT\ produces by manipulating
the {\bf PAPER\_MEDIA} parameter (see the \GMTprog{gmtdefaults} man
page for how this is accomplished).  Note that a freeform \PS\
file may contain special operators (such as \texttt{Setpagedevice})
that is specific to printers (e.g., selection of paper tray).
Some previewers (among them, Sun's \progname{pageview}) do not
understand these valid instructions and may fail to image the file.
If this is your situation you should choose another viewer (we
recommend \progname{ghostview}) or select EPS output instead.

However, there is much confusion over what an EPS file is
and if other programs can read it.  Much of this has to do
with the claim by some software manufacturers that their
programs can read and edit EPS files.  We used to get much
mail from people asking us to let \GMT\ produce EPS files
that can be read, e.g., by Adobe \progname{Illustrator}.
This was a limitation of early versions of Adobe \progname{Illustrator} and
similar programs, not \GMT!  Since then, Adobe
\progname{Illustrator} and other programs have
improved their abilities to parse freeform \PS\ such as that produced
by \GMT, but problems seem to occasionally reappear.

An EPS file that is to be placed into another application
(such as a text document) need to have correct bounding-box
parameters.  These are found in the \PS\ Document
Comment \%\%BoundingBox.  Applications that generate EPS
files should set these parameters correctly.  Because \GMT\
makes the \PS\ files on the fly, often with several
overlays, it is not possible to do so accurately.  However,
\GMT\ does make an effort to ensure that the boundingbox is
large enough to contain the entire composite plot\footnote{In contrast,
regular \GMT\ \PS\ files simply have
a \%\%BoundingBox that equal the size of the chosen paper.}.
Therefore, if you need a ``tight'' boundingbox you need to post-process
your \PS\ file.  There are several ways in which this
can be accomplished.

\begin{itemize}

\item Programs such as Adobe \progname{Illustrator}, Aldus
\progname{Freehand}, and Corel \progname{Draw} will allow you
to edit the boundingbox graphically.

\item A command-line alternative is to use freely-available
program \progname{epstool} from the makers of Aladdin
\progname{ghostscript}.  Running \\ 

\indent \texttt{epstool -c -b yourplot.ps} \\ 

should give a tight BoundingBox; \progname{epstool} assumes the plot
is page size and not a huge poster.

\item Another option is to use \progname{ps2epsi} which also comes
with the \progname{ghostscript} package.  Running \\ 

\indent \texttt{ps2epsi myplot.ps myplot.eps} \\

should also do the trick.

\end{itemize}

If you do not want to modify the illustration but just
include it in a text document: Many word processors (such
as Microsoft \progname{Word} and Corel \progname{WordPerfect})
will let you include
a \PS\ file that you may place but not edit.  You will not
be able to view the figure on-screen, but it will print
correctly.  All illustrations in this \GMT\ documentation were
\GMT-produced \PS\ files that were converted to EPS files
using \progname{ps2epsi} and then included into a \LaTeX\ document. 

These examples do not constitute endorsements of the products
mentioned above; they only represent our limited experience
with the problem.  For other solutions and further help,
please post messages to
\htmladdnormallink{gmt-help@hawaii.edu}{mailto:gmt-help@hawaii.edu}.
