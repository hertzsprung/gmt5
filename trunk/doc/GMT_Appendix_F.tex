%------------------------------------------
%	$Id: GMT_Appendix_F.tex,v 1.1.1.1 2000-12-28 01:23:45 gmt Exp $
%
%	The GMT Documentation Project
%	Copyright 2000-2001.
%	Paul Wessel and Walter H. F. Smith
%------------------------------------------
%
\chapter{Chart of octal codes for characters}
\index{Characters!octal}
\index{Octal characters}
\thispagestyle{headings}

\GMTfig[h]{GMT_App_F_1}{Octal codes and corresponding symbols for standard fonts}

The characters and their octal codes in the reencoded standard fonts
are shown in Figure~\ref{fig:GMT_App_F_1}.  The chart for
the Symbol (\GMT\ font number 12) character
sets are presented in Figure~\ref{fig:GMT_App_F_2} below.  Gray
areas signify codes reserved for
control characters.  The octal code is obtained by appending the
column value to the $\backslash$?? value, e.g., $\partial$ is
$\backslash$266 in the Symbol font.  In order to use all the extended
characters you need to set {\bf WANT\_EURO\_FONT} to true in your
\filename{.gmtdefaults} file\footnote{By default, this is true if you chose
SI units and false if you chose US units during the installation.}.
\index{Symbol font}
\index{Font!symbol}

\GMTfig[h]{GMT_App_F_2}{Octal codes and corresponding symbols for the Symbol font}

The Pifont ZapfDingbats is available as \GMT\ font number 34 and
can be used for special symbols not listed above.  The various
symbols are illustrated in Figure~\ref{fig:GMT_App_F_3}.

\GMTfig[h]{GMT_App_F_3}{Octal codes and corresponding symbols for ZapfDingbats font}
