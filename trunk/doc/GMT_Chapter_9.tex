%------------------------------------------
%	$Id: GMT_Chapter_9.tex,v 1.8 2010-10-01 23:58:02 guru Exp $
%
%	The GMT Documentation Project
%	Copyright 2000-2010.
%	Paul Wessel and Walter H. F. Smith
%------------------------------------------
%
\chapter{Mailing lists, updates, and bug reports}
\label{ch:9}
\index{Mailinglists}
\index{GMT@\GMT!Mailinglists}
\thispagestyle{headings}

Most public-domain (and even commercial) software comes
with bugs, and the speed with which such bugs are detected
and removed depends to a large degree on the willingness
of the user community to report these to us in a useful
manner.  When your car breaks down, simply telling the
mechanic that it doesn't work will hardly speed up the
repair or cut back costs!  Therefore, we ask that if
you detect a bug, first make sure that it in fact is a
bug and not a user error.  Then, send us email about the
problem.  Be sure to include all the information necessary
for us to recreate the situation in which the bug occurred.
This will include the full command line used and, if
possible, the data file used by the program.  Send the
bug-reports to
\htmladdnormallink{gmt-help@lists.hawaii.edu}{mailto:gmt-help@lists.hawaii.edu}.
We will try to fix bugs as soon as our schedules permit and
inform users about the bug and availability of updated code
(See Appendix~\ref{app:D}).

Two electronic mailing lists are available to which
users may subscribe. 
\htmladdnormallink{gmt-group@lists.hawaii.edu}{mailto:gmt-group@lists.hawaii.edu}
and is primarily a way for us to notify the users when bugs
have been fixed or when new updates have been installed in
the ftp directory (See Appendix~\ref{app:D}).  We also maintain another list
(\htmladdnormallink{gmt-help@lists.hawaii.edu}{mailto:gmt-help@lists.hawaii.edu}) which interested users may
subscribe to.  It basically provides a forum for \GMT\ users
to exchange ideas and ask questions about \GMT\ usage,
installation and portability, etc. Please use this utility
rather than sending questions directly to us personally.
We hope you appreciate that we simply do not have time to be
everybody's personal \GMT\ tutor.

The electronic mailing lists are maintained automatically
by a program.  To subscribe to one or both of the lists,
send a message to
\htmladdnormallink{listserv@lists.hawaii.edu}{mailto:listserv@lists.hawaii.edu}
containing the command(s):

\vspace{\baselineskip} 

subscribe gmt-group $<$your full name, not email address$>$ \\

subscribe gmt-help $<$your full name, not email address$>$ \\

(Do not type the angular brackets $<$$>$).  You may also
register electronically via the \GMT\ home web
page (\GMTSITE).  For information
on what commands you may send, send a message containing
the word help.  You must interact with the listserver to be
added to or removed from the mailing lists!  We strongly recommend that you
at least subscribe to gmt-group since this is how we can notify
you of future updates and bug-fixes.  Most new users will
also benefit from having the other forum (gmt-help) as they
struggle to realign their sense of logic with that of \GMT.
While anybody may post messages to gmt-help, access to gmt-group
is restricted to minimize net traffic.  Any message sent to
gmt-group will be intercepted by the \GMT\ manager who will
determine if the message is important enough to cause thousands
of mailtools to go BEEP.  Communication with other \GMT\ users
should go via gmt-help.  Finally, all \GMT\ information is provided
online at the main \GMT\ home page in Hawaii, i.e.,
\GMTSITE.  Changes to \GMT\ will also be posted
on this page.  The main \GMT\ page has links to the official
\GMT\ ftp sites around the world.
