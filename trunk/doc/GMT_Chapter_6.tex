%------------------------------------------
%	$$
%
%	The GMT Documentation Project
%	Copyright 2000-2001.
%	Paul Wessel and Walter H. F. Smith
%------------------------------------------
%
\section{Non-map Projections}

The linear projection comes in four flavors: linear, log$_{10}$,
power (or exponential), and time (calendar).  The projection for the {\it y}-axis can be set
independently from the {\it x}-axis.  Two subsets of linear will be discussed
separately; these are a polar (cylindrical) projection and a linear projection applied to
geographic coordinates (with a 360 degree periodicity in the $x$-coordinate).  We will show examples
of all of these projections using dummy data sets created with
\GMTprog{gmtmath}, a ``Reverse Polish Notation'' (RPN) calculator that
operates on or creates table data: 
\index{Reverse Polish Notation (RPN)}
\index{RPN (Reverse Polish Notation)}
\input{scripts/GMT_dummydata}

\subsection{Cartesian Linear Projection (\Opt{Jx} \Opt{JX})}
\index{Projection!linear|(}
\index{Projection!linear!Cartesian|(}
\index{Cartesian linear projection|(}
\index{Linear projection|(}
\index{\Opt{Jx} \Opt{JX} (Non-map projections)|(}

Selection of this transformation will result in a linear scaling
of the input coordinates.  The projection is defined by stating \\

$\bullet$ scale in inches/unit (\Opt{Jx}) or axis
length in inches (\Opt{JX}) \\

If the {\it y}-scale or {\it y}-axis length is different from that of
the {\it x}-axis (which is most often the case), separate the two
scales (or lengths) by a slash, e.g., \Opt{Jx}0.1i/0.5i or \Opt{JX}8i/5i. 
Thus, our $y = \sqrt{x}$ data sets will plot as shown
in Figure~\ref{fig:GMT_linear}.

\GMTfig{GMT_linear}{Linear transformation of coordinates}

The complete commands given to produce this plot were 

\input{scripts/GMT_linear}

Normally, the user's {\it x}-values will increase to the right
and the {\it y}-values will increase upwards.  It should be noted
that in many situations it is desirable to have the direction of
positive coordinates be reversed.  For example, when plotting
depth on the {\it y}-axis it makes more sense to have the positive
direction downwards.  All that is required to reverse the sense of
positive direction is to supply a negative scale (or axis length).
\index{Projection!linear|)}
\index{Projection!linear!Cartesian|)}
\index{Cartesian linear projection|)}
\index{Linear projection|)}

\subsection{Logarithmic projection}
\index{Projection!logarithmic|(}
\index{Logarithmic projection|(}

\GMTfig[h]{GMT_log}{Logarithmic transformation of $x$-coordinates}

The log$_{10}$ transformation is selected by appending an {\bf l}
(lower case L) immediately following the scale (or axis length)
value.  Hence, to produce a plot in which the {\it x}-axis is
logarithmic (the {\it y}-axis remains linear), try 

\input{scripts/GMT_log} 

\par Note that if {\it x}- and {\it y}-scaling are different and
a log$_{10}$-log$_{10}$ plot is desired, the {\bf l} must be
appended twice: Once after the {\it x}-scale (before the /) and
once after the {\it y}-scale. 
\index{Projection!logarithmic|)}
\index{Logarithmic projection|)}

\subsection{Power projection}
\index{Projection!power (exponential)|(}
\index{Power (exponential) projection|(}

\GMTfig[h]{GMT_pow}{Exponential or power transformation of $x$-coordinates}

This projection allows us to display {\it x$^a$} versus {\it y$^b$}.
While {\it a} and {\it b} can be any values, we will select {\it a}
= 0.5 and {\it b} = 1 which means we will plot {\it y} versus $\sqrt{x}$.
We indicate this scaling by appending a {\bf p} (lower case P) followed
by the desired exponent, in our case 0.5.  Since {\it b} = 1 we do not
need to specify {\bf p}1 since it is identical to the linear scaling.
Thus our command becomes

\input{scripts/GMT_pow} 
\index{Projection!power (exponential)|)}
\index{Power (exponential) projection|)}

\subsection{Geographical linear projection}
\label{sec:linear}
\index{Projection!linear!geographic|(}
\index{Geographic Linear projection|(}

\GMTfig{GMT_linear_d}{Linear transformation of map coordinates}

While these linear projections are primarily designed for generic
{\it x},{\it y} data, it is sometimes necessary to plot geographical
data in a linear projection.  This poses a problem since longitudes
have a 360\DS\ periodicity.  \GMT\ therefore needs to be informed
that it has been given geographical data although a linear projection
has been chosen.  We do so by appending a {\bf d} (for degrees) to
the end of the \Opt{Jx} (or \Opt{JX}) option.  As an example, we
want to plot a crude world map centered on 125\DS E.  Our command will be 

\input{scripts/GMT_linear_d} 

\noindent
with the result reproduced in Figure~\ref{fig:GMT_linear_d}.

\index{Projection!linear!geographic|)}
\index{Geographic Linear projection|)}
\index{\Opt{Jx} \Opt{JX} (Non-map projections)|)}

\subsection{Linear Projection with Polar ($\theta, r$)
Coordinates (\Opt{Jp } \Opt{JP})}
\index{Projection!polar ($\theta, r$)|(}
\index{Polar ($\theta, r$) projection|(}
\index{\Opt{Jp} \Opt{JP} (Polar ($\theta, r$) projections)|(}

\GMTfig{GMT_polar}{Polar (Cylindrical) transformation of
($\theta, r$) coordinates}

In many applications the data is better described in polar or
cylindrical ({\it $\theta$}, {\it r}) coordinates rather than
the usual Cartesian coordinates ({\it x}, {\it y}).  The
relationship between the Cartesian and polar coordinates are
described by $x = r \cdot \cos{\theta}, y = r \cdot \sin{\theta}$.
The polar transformation is simply defined by providing \\

\begin{description}

\item[$\bullet$] scale in inches/unit (\Opt{Jp}) or full width of plot in inches (\Opt{JP})
\item[$\bullet$] Optionally, insert {\bf a} after {\bf p$|$P} to indicate CW azimuths rather than CCW directions
\item[$\bullet$] Optionally, append /$origin$ in degrees to indicate an angular offset [0]

\end{description}

As an example of this projection we will create a gridded data set
in polar coordinates $z(\theta, r) = r^2 \cdot \cos{4\theta}$
using \GMTprog{grdmath}, a RPN calculator that operates on or
creates grdfiles.

\input{scripts/GMT_polar} 

We used \GMTprog{grdcontour} to make a contour map of this data.  Because
the data file only contains values with $2 \leq r \leq 4$, a donut
shaped plot appears in Figure~\ref{fig:GMT_polar}.\
\index{Projection!polar ($\theta, r$)|)}
\index{Polar ($\theta, r$) projection|)}
\index{\Opt{Jp} \Opt{JP} (Polar ($\theta, r$) projections)|)}

\subsection{Linear projection with Calendar Time Coordinates}
\label{sec:time}
\index{Projection!linear!calendar|(}
\index{Calendar Linear projection|(}


\index{Projection!linear!calendar|)}
\index{Calendar Linear projection|)}
