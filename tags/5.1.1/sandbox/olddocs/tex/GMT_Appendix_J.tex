%------------------------------------------
%	$Id$
%
%	The GMT Documentation Project
%	Copyright (c) 2000-2013.
%	P. Wessel, W. H. F. Smith, R. Scharroo, J. Luis and F. Wobbe
%------------------------------------------
%
\chapter{The \gmt\ High-Resolution Coastline Data}
\label{app:K}
\index{GMT@\GMT!coastlines|(}
\index{Coastlines!preprocessing|(}

Starting with version 3.0, \GMT\ use a completely new coastline
database and the \GMTprog{pscoast} utility was been completely
rewritten to handle the new file format.  Many users have asked
us why it has taken so long for \GMT\ to use a high-resolution
coastline database; after all, such data have been available in
the public domain for years.  To answer such questions we will
take you along the road that starts with these public domain
data sets and ends up with the database used by \GMT.

\section{Selecting the right data} 
\index{World Data Bank II}
\index{CIA Data Bank}
\index{World Vector Shoreline}
\index{WVS}
\index{WDBII}

There are two well-known public-domain data sets that could be
used for this purpose.  Once is known as the World Data Bank II
or CIA Data Bank (WDB) and contains coastlines, lakes, political
boundaries, and rivers.  The other, the World Vector Shoreline
(WVS) only contains shorelines between saltwater and land (i.e.,
no lakes).  It turns out that the WVS data is far superior to the
WDB data as far as data quality goes, but as noted it lacks lakes,
not to mention rivers and borders.  We decided to use the WVS
whenever possible and supplement it with WDB data.  We got these
data over the Internet; they are also available on CD-ROM from
the National Geophysical Data Center in Boulder, Colorado\footnote{
www.ngdc.noaa.gov}.

\section{Format required by \gmt} 

In order to paint continents or oceans it is necessary that the
coastline data be organized in polygons that may be filled.
Simple line segments can be used to draw the coastline, but for
painting polygons are required.  Both the WVS and WDB data
consists of unsorted line segments: there is no information
included that tells you which segments belong to the same
polygon (e.g., Australia should be one large polygon).
In addition, polygons enclosing land must be differentiated from
polygons enclosing lakes since they will need different paint.
Finally, we want \GMTprog{pscoast} to be flexible enough that it can
paint the land \emph{or} the oceans \emph{or} both.
If just land (or oceans) is selected we do not want to paint
those areas that are not land (or oceans) since previous plot
programs may have drawn in those areas.  Thus, we will need to
combine polygons into new polygons that lend themselves to fill
land (or oceans) only (Note that older versions of \GMTprog{pscoast}
always painted lakes and wiped out whatever was plotted beneath).

\section{The long and winding road} 

The WVS and WDB together represent more than 100 Mb of binary
data and something like 20 million data points.  Hence, it
becomes obvious that any manipulation of these data must be
automated.  For instance, the reasonable requirement that no
coastline should cross another coastline becomes a complicated
processing step.

\begin{enumerate} 

\item To begin, we first made sure that all data were ``clean'',
i.e., that there were no outliers and bad points.  We had to
write several programs to ensure data consistency and remove
``spikes'' and bad points from the raw data.  Also, crossing
segments were automatically ``trimmed'' provided only
a few points had to be deleted.  A few hundred more complicated
cases had to be examined semi-manually.

\item Programs were written to examine all the loose segments
and determine which segments should be joined to produce
polygons.  Because not all segments joined exactly (there were
non-zero gaps between some segments) we had to find all possible
combinations and choose the simplest combinations.
The WVS segments joined to produce more than 200,000 polygons,
the largest being the Africa-Eurasia polygon which has 1.4
million points.  The WDB data resulted in a smaller data base
($\sim$25\% of WVS).

\item We now needed to combine the WVS and WDB data bases.
The main problem here is that we have duplicates of polygons:
most of the features in WVS are also in WDB.  However, because
the resolution of the data differ it is nontrivial to figure
out which polygons in WDB to include and which ones to ignore.
We used two techniques to address this problem.
First, we looked for crossovers between all possible pairs of
polygons.  Because of the crossover processing in step 1 above we know
that there are no remaining crossovers within WVS and WDB; thus
any crossovers would be between WVS and WDB polygons.  Crossovers
could mean two things: (1) A slightly misplaced WDB polygon
crosses a more accurate WVS polygon, both representing the same
geographic feature, or (2) a misplaced WDB polygon (e.g., a small
coastal lake) crosses the accurate WVS shoreline.  We distinguished
between these cases by comparing the area and centroid of the two
polygons.  In almost all cases it was obvious when we had
duplicates; a few cases had to be checked manually.  Second,
on many occasions the WDB duplicate polygon did not cross its
WVS counterpart but was either entirely inside or outside the
WVS polygon.  In those cases we relied on the area-centroid tests.

\item While the largest polygons were easy to identify by visual
inspection, the majority remain unidentified.  Since it is
important to know whether a polygon is a continent or a small
pond inside an island inside a lake we wrote programs that would
determine the hierarchical level of each polygon.  Here, level~=~1
represents ocean/land boundaries, 2 is land/lakes borders, 3 is
lakes/islands-in-lakes, and 4 is islands-in-lakes/ponds-in-islands-in-lakes.
Level 4 was the highest level encountered in the data.
To automatically determine the hierarchical levels we wrote
programs that would compare all possible pairs of polygons
and find how many polygons a given polygon was inside.  Because
of the size and number of the polygons such programs would
typically run for 3 days on a Sparc-2 workstation.

\item Once we know what type a polygon is we can enforce a
common ``orientation'' for all polygons. We arranged them so
that when you move along a polygon from beginning to end, your
left hand is pointing toward ``land''.  At this step we also
computed the area of all polygons since we would like the
option to plot only features that are bigger than a minimum
area to be specified by the user.

\item Obviously, if you need to make a map of Denmark then
you do not want to read the entire 1.4 million points making
up the Africa-Eurasia polygon.  Furthermore, most plotting
devices will not let you paint and fill a polygon of that size
due to memory restrictions.  Hence, we need to partition the
polygons so that smaller subsets can be accessed rapidly.
Likewise, if you want to plot a world map on a letter-size paper
there is no need to plot 10 million data points as most of them
will plot several times on the same pixel and the operation
would take a very long time to complete.  We chose to make 5
versions on the database, corresponding to different resolutions.
The decimation was carried out using the Douglas-Peucker (DP)
line-reduction algorithm\footnote{Douglas, D.H., and T. K. Peucker,
1973, Algorithms for the reduction of the number of points
required to represent a digitized line or its caricature,
\emph{Canadian Cartographer}, 10, 112--122.}.  We chose the
cutoffs so that each subset was approximately 20\% the size of
the next higher resolution.  The five resolutions are called
\textbf{f}ull, \textbf{h}igh, \textbf{i}ntermediate, \textbf{l}ow, and
\textbf{c}rude; they are accessed in \GMTprog{pscoast}, \GMTprog{gmtselect},
and \GMTprog{grdlandmask} with the \Opt{D} option\footnote{ The full
and high resolution files are in separate archives because of their
size.  Not all users may need these files as the intermediate data
set is better than the data provided with version 2.1.4.}.  For each of
these 5 data sets (\textbf{f}, \textbf{h}, \textbf{i}, \textbf{l}, \textbf{c})
we specified an equidistant grid (1\DS, 2\DS, 5\DS,
10\DS, 20\DS) and split all polygons into line-segments
that each fit inside one of the many boxes defined by these grid
lines.  Thus, to paint the entire continent of Australia we
instead paint many smaller polygons made up of these line
segments and gridlines.  Some book-keeping has to be done since
we need to know which parent polygon these smaller pieces came
from in order to prescribe the correct paint or ignore if the
feature is smaller than the cutoff specified by the user.  The
resulting segment coordinates were then scaled to fit in short
integer format to preserve precision and written in netCDF format
for ultimate portability across hardware platforms\footnote{
If you need complete polygons in a simpler format, see the article
on GSHHG\index{GSHHG} (Wessel, P., and W. H. F. Smith, 1996, A Global, self-consistent, hierarchical, high-resolution shoreline database,
\emph{J. Geophys. Res. 101}, 8741--8743).}.

\item While we are now back to a file of line-segments we are in
a much better position to create smaller polygons for painting.
Two problems must be overcome to correctly paint an area:

\begin{itemize}

\item We must be able to join line segments and grid cell borders
into meaningful polygons; how we do this will depend on whether
we want to paint the land or the oceans.

\item We want to nest the polygons so that no paint falls on areas
that are ``wet'' (or ``dry''); e.g., if a grid cell completely on
land contains a lake with a small island, we do not want to paint
the lake and then draw the island, but paint the annulus or ``donut''
that is represented by the land and lake, and then plot the island.

\end{itemize}

\GMT\ uses a polygon-assembly routine that carries out these
tasks on the fly.
\index{GMT@\GMT!coastlines|)}
\index{Coastlines!preprocessing|)}

\end{enumerate} 

\section{The Five Resolutions} 

We will demonstrate the power of the new database by starting with
a regional hemisphere map centered near Papua New Guinea and zoom
in on a specified point.  The map regions will be specified in
projected km from the projection center, e.g., we may want the
map to go from \mbox{-2000} km to \mbox{+2000} km in the longitudinal
and the latitudinal direction.
However, \GMT\ programs expects degrees in the \Opt{R} option that
specifies the desired region.  Given the chosen map projection we
can automate this process by using a simple shell function that we
call \progname{getbox}.

Also, as we zoom in on the projection center we want to draw the
outline of the next map region on the plot.  To do that we need
the geographical coordinates of the four corners of the region
rectangle.  Again, we automate this task by adding the simple
function \progname{getrect} from \progname{doc/scripts/functions.sh}.

\script[.]{functions}

\subsection{The crude resolution (\Opt{Dc})} 
\index{Coastlines!resolution!crude|(}

We begin with an azimuthal equidistant map of the hemisphere
centered on 130\DS 21'E, 0\DS 12'S, which is slightly west
of New Guinea, near the Strait of Dampier.  The edges of the
map are all 9000 km true distance from the projection center.
At this scale (and for global maps) the crude resolution data
will usually be adequate to capture the main geographic features.
To avoid cluttering the map with insignificant detail we only
plot features (i.e., polygons) that exceed 500 km$^2$ in area.
Smaller features would only occupy a few pixels on the plot and
make the map look ``dirty''.  We also add national borders to
the plot.  The crude database is heavily decimated and simplified
by the DP-routine: The total file size of the coastlines, rivers,
and borders database is only 283 kbytes.  The plot is produced by the
script:

\script{GMT_App_K_1} 

\GMTfig{GMT_App_K_1}{Map using the crude resolution coastline data.}

Here, we use the \GMTdef{MAP\_ANNOT\_OBLIQUE} bit flags to achieve
horizontal annotations and set \GMTdef{MAP\_ANNOT\_MIN\_SPACING} to suppress some longitudinal
annotations near the S pole that otherwise would overprint.
The box indicates the outline of the next map.

\index{Coastlines!resolution!crude|)}

\subsection{The low resolution (\Opt{Dl})} 
\index{Coastlines!resolution!low|(}

We have now reduced the map area by zooming in on the map center.
Now, the edges of the map are all 2000 km true distance from
the projection center.  At this scale we choose the low resolution
data that faithfully reproduce the dominant geographic features
in the region.  We cut back on minor features less than 100 km$^2$
in area.  We still add national borders to the plot.  The low
database is less decimated and simplified by the DP-routine: The
total file size of the coastlines, rivers, and borders combined
grows to 907 kbytes; it is the default resolution in \GMT.  The
plot is generated by the script:

\script{GMT_App_K_2} 

\GMTfig{GMT_App_K_2}{Map using the low resolution coastline data.}

\index{Coastlines!resolution!low|)}

\subsection{The intermediate resolution (\Opt{Di})} 
\index{Coastlines!resolution!intermediate|(}

We continue to zoom in on the map center.  In this map, the
edges of the map are all 500 km true distance from the projection
center.  We abandon the low resolution data set as it would look
too jagged at this scale and instead employ the intermediate
resolution data that faithfully reproduce the dominant geographic
features in the region.  This time, we ignore features less than
20 km$^2$ in area.  Although the script still asks for national
borders none exist within our region.  The intermediate database
is moderately decimated and simplified by the DP-routine: The
combined file size of the coastlines, rivers, and borders now
exceeds 3.35 Mbytes.  The plot is generated by the script:

\script{GMT_App_K_3} 

\GMTfig{GMT_App_K_3}{Map using the intermediate resolution coastline data.  We
have added a compass rose just because we have the power to do so.}

\index{Coastlines!resolution!intermediate|)}

\subsection{The high resolution (\Opt{Dh})} 
\index{Coastlines!resolution!high|(}

The relentless zooming continues!  Now, the edges of the map
are all 100 km true distance from the projection center.  We
step up to the high resolution data set as it is needed to
accurately portray the detailed geographic features within the
region.  Because of the small scale we only ignore features less
than 1 km$^2$ in area.  The high resolution database has undergone
minor decimation and simplification by the DP-routine: The
combined file size of the coastlines, rivers, and borders now
swells to 12.3 Mbytes.  The map and the final outline box are
generated by these commands:

\script{GMT_App_K_4} 

\GMTfig{GMT_App_K_4}{Map using the high resolution coastline data.}

\index{Coastlines!resolution!high|)}

\subsection{The full resolution (\Opt{Df})} 
\index{Coastlines!resolution!full|(}

We now arrive at our final plot, which shows a detailed view of
the western side of the small island of Waigeo.  The map area
is approximately 40 by 40 km.  We call upon the full resolution
data set to portray the richness of geographic detail within this
region; no features are ignored.  The full resolution has
undergone no decimation and it shows: The combined file size of
the coastlines, rivers, and borders totals a (once considered hefty) 55.9 Mbytes.
Our final map is reproduced by the single command:

\script{GMT_App_K_5} 

\GMTfig{GMT_App_K_5}{Map using the full resolution coastline data.}

We hope you will study these examples to enable you to make
efficient and wise use of this vast data set.
\index{Coastlines!resolution!full|)}
