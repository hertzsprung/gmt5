%------------------------------------------
%	$Id$
%
%	The GMT Documentation Project
%	Copyright (c) 2000-2012.
%	P. Wessel, W. H. F. Smith, R. Scharroo, and J. Luis
%------------------------------------------
%
\chapter{Introduction}
\label{ch:2}
\thispagestyle{headings}

Most scientists are familiar with the sequence:
\emph{raw data $\rightarrow$ processing $\rightarrow$ final illustration}.
In order to finalize papers for submission to scientific journals,
prepare proposals, and create overheads and slides for various
presentations, many scientists spend large amounts of time and
money to create camera-ready figures.  This process can be tedious
and is often done manually, since available commercial or in-house
software usually can do only part of the job.  To expedite this
process we introduce the Generic Mapping Tools (\GMT\ for short),
which is a free\footnote{See GNU General Public License
(www.gnu.org/copyleft/gpl.html) for terms on
redistribution and modifications.}, software package that can be used
to manipulate columns of tabular data, time-series, and gridded
data sets, and display these data in a variety of forms ranging
from simple \emph{x}-\emph{y} plots to maps and color, perspective,
and shaded-relief illustrations.  \GMT\ uses the \PS\
page description language [\emph{Adobe Systems Inc.}, 1990]\index{PostScript@\PS}.  With \PS, multiple plot
files can easily be superimposed to create arbitrarily complex
images in gray tones or 24-bit true color.  Line drawings, bitmapped
images, and text can be easily combined in one illustration.
\PS\ plot files are device-independent: The same file
can be printed at 300 dots per inch (dpi) on an ordinary laserwriter
or at 2470 dpi on a phototypesetter when ultimate quality is needed.
\GMT\ software is written as a set of \UNIX\ tools\footnote{The
tools can also be installed on other platforms (see Appendix~\ref{app:L}).}
and is totally self-contained and fully documented.  The system is offered free
of charge and is distributed over the computer
network (Internet) [\emph{Wessel and Smith, 1991; 1995a,b; 1998}].

The original version 1.0 of \GMT\ was released in the summer of 1988
when the authors were graduate students at Lamont-Doherty Earth
Observatory\index{Lamont-Doherty Earth Observatory}\index{L-DEO} of Columbia University.
During our tenure as graduate
students, L-DEO changed its computing environment to a distributed
network of \UNIX\ workstations, and we wrote \GMT\ to run in this
environment.  It became a success at L-DEO, and soon spread to
numerous other institutions in the US, Canada, Europe, and Japan.
The current version benefits from the many suggestions
contributed by users of the earlier versions, and now includes more
than 50 tools, more than 30 projections, and many other new, more
flexible features.  \GMT\ provides scientists with a variety of
tools for data manipulation and display, including routines to sample,
filter, compute spectral estimates, and determine trends in time
series, grid or triangulate arbitrarily spaced data, perform
mathematical operations (including filtering) on 2-D data sets
both in the space and frequency domain, sample surfaces along
arbitrary tracks or onto a new grid, calculate volumes, and find
trend surfaces.  The plotting programs will let the user make linear,
log$_{10}$, and \emph{x$^a$}--\emph{y$^b$} diagrams, polar and
rectangular histograms, maps with filled continents and coastlines
choosing from many common map projections, contour plots, mesh plots,
monochrome or color images, and artificially illuminated
shaded-relief and 3-D perspective illustrations. 

\index{ANSI C compliant}
\index{POSIX compliant}
\index{Y2K compliant}
\index{Compliance!POSIX}
\index{Compliance!ANSI C}
\index{Compliance!Y2K}
\GMT\ is written in the highly portable ANSI C \index{ANSI C} programming language
[\emph{Kernighan and Ritchie}, 1988], is fully POSIX compliant\index{POSIX}
[\emph{Lewine}, 1991], has no Year 2000 problems\index{Year 2000 compliant}, and may be used
with any hardware running some flavor of \UNIX, possibly with minor
modifications.  In writing \GMT, we have followed the modular
design philosophy of \UNIX: The \emph{raw data $\rightarrow$  processing $\rightarrow$
final illustration} flow is broken down to a series of elementary
steps; each step is accomplished by a separate \GMT\ or \UNIX\ tool.
This modular approach brings several benefits: (1) only a few
programs are needed, (2) each program is small and easy to update
and maintain, (3) each step is independent of the previous step
and the data type and can therefore be used in a variety of
applications, and (4) the programs can be chained together in
shell scripts or with pipes, thereby creating a process tailored
to do a user-specific task.  The  decoupling of the data retrieval
step from the subsequent massage and plotting is particularly
important, since each institution will typically have its own
data base formats.  To use \GMT\ with custom data bases, one has
only to write a data extraction tool which will put out data in a
form readable by \GMT\ (discussed below).  After writing the extractor,
all other \GMT\ modules will work as they are. 

\GMT\ makes full use of the \PS\ page description language, and can produce color illustrations
if a color \PS\ device is available.  One does not
necessarily have to have access to a top-of-the-line color printer
to take advantage of the color capabilities offered by \GMT: Several
companies offer imaging services where the customer provides a
\PS\ plot file and gets color slides or hardcopies in return.
Furthermore, general-purpose \PS\ raster image processors
(RIPs) are now becoming available, letting the user create raster images
from \PS\ and plot these bitmaps on raster devices like computer
screens, dot-matrix printers, large format raster plotters, and film
writers\footnote{One public-domain RIP is \progname{ghostscript},
available from www.gnu.org.}.
Because the publication costs of color illustrations are high,
\GMT\ offers 90 common bit and hachure patterns, including many geologic
map symbol types, as well as complete graytone shading operations.
Additional bit and hachure patterns may also be designed by the user.
With these tools, it is possible to generate publication-ready
monochrome originals on a common laserwriter. 

\GMT\ is thoroughly documented and comes with a technical reference and
cookbook which explains the purpose of the package and its many features,
and provides numerous examples to help new users quickly become familiar
with the operation and philosophy of the system.  The cookbook contains
the shell scripts that were used for each example; \PS\
files of each illustration are also provided.  All programs have
individual manual pages which can be installed as part of the on-line
documentation under the \UNIX\ \progname{man} utility or as web pages.  In addition, the
programs offer friendly help messages which make them essentially
self-teaching -- if a user enters invalid or ambiguous command arguments,
the program will print a warning to the screen with a synopsis of the
valid arguments.  All the documentation is available for web browsing
and may be installed at the user's site.

The processing and display routines within \GMT\ are completely general
and will handle any (\emph{x,y}) or (\emph{x,y,z}) data as input.
For many purposes the (\emph{x,y}) coordinates will be (longitude,
latitude) but in most cases they could equally well be any other
variables (e.g., wavelength, power spectral density).  Since the \GMT\
plot tools will map these (\emph{x,y}) coordinates to positions on a
plot or map using a variety of transformations (linear, log-log, and
several map projections), they can be used with any data that are
given by two or three coordinates.  In order to simplify and standardize
input and output, \GMT\ uses two file formats only.  Arbitrary sequences
of (\emph{x,y}) or (\emph{x,y,z}) data are read from multi-column ASCII
tables, i.e., each file consists of several records, in which each
coordinate is confined to a separate column\footnote{Programs now also
allow for fast, binary multicolumn file i/o.}.  This format is
straightforward and allows the user to perform almost any simple
(or complicated) reformatting or processing task using standard
\UNIX\ utilities such as \progname{cut}, \progname{paste}, \progname{grep},
\progname{sed} and \progname{awk}.
Two-dimensional data that have been sampled on an equidistant grid are
read and written by \GMT\ in a binary grid file using the functions
provided with the netCDF\index{netCDF} library (a free, public-domain software
library available separately from UCAR, the University Corporation
of Atmospheric Research [\emph{Treinish and Gough}, 1987]).  This XDR\index{XDR}
(External Data Representation) based format is architecture independent,
which allows the user to transfer the binary data files from one
computer system to another\footnote{While the netCDF format is the default,
other formats are also possible, including user-defined formats.}.
\GMT\ contains programs that will read ASCII
(\emph{x,y,z}) files and produce grid files.  One such program,
\GMTprog{surface}, includes new modifications to the gridding algorithm
developed by \emph{Smith and Wessel} [1990] using continuous splines
in tension. 

Most of the programs will produce some form of output, which falls
into four categories.  Several of the programs may produce more than
one of these types of output:\par 

\begin{enumerate}

\item{1-D ASCII Tables --- For example, a ($x,y$) series may be filtered and
the filtered values output.  ASCII output is written to the standard output stream.} 

\item{2-D binary (netCDF or user-defined) grid files -- Programs that grid
ASCII ($x,y,z$) data or operate on existing grid files produce this type of output.} 

\item{\PS\ -- The plotting programs all use the \PS\
page description language to define plots.  These commands are stored as ASCII
text and can be edited should you want to customize the plot beyond the options
available in the programs themselves.} 

\item{Reports -- Several \GMT\ programs read input files and report statistics
and other information.  Nearly all programs have an optional ``verbose''
operation, which reports on the progress of computation.  All programs feature
usage messages, which prompt the user if incorrect commands have been given.
Such text is written to the standard error stream and can therefore be
separated from ASCII table output.} 

\end{enumerate} 

\GMT\ is available over the Internet at no charge.  To obtain a copy, read
the relevant information on the \GMT\ home page \GMTSITE,
or email
\htmladdnormallink{listserv@hawaii.edu}{mailto:listserv@hawaii.edu}
a note containing the single message \\ 
\index{GMT@\GMT!Mailinglists}
\index{Mailinglists}
\index{GMT@\GMT!on CD-ROM}
\index{GMT@\GMT!obtaining}
\index{GMT@\GMT!home page}

\textbf{information gmt-group} \\

The listserver will mail you back a shell-script that you may run to obtain
all necessary programs, libraries, and support data.  After you obtain the
\GMT\ archive, you will find that it contains information on how to install
\GMT\ on your hardware platform and how to obtain additional files that you
may need or want.  The archive also contains a license agreement and
registration file.  We also maintain two electronic mailing lists you may
subscribe to in order to stay informed about bug fixes and upgrades (See
Chapter~\ref{ch:7}).

For those without net-access that need to obtain \GMT: Geoware
\htmladdnormallink{(http://www.geoware-online.com)}{http://www.geoware-online.com}
makes and distributes CD-R and DVD-R media with the \GMT\ package, compatible supplements, and
several Gb of useful Earth and ocean science data sets.  For more information send e-mail to
\htmladdnormallink{geoware@geoware-online.com}{mailto:geoware@geoware-online.com}. 

\GMT\ has served a multitude of scientists very well, and their responses
have prompted us to develop these programs even further.  It is our
hope that the new version will satisfy these users and attract new
users as well.  We present this system to the community in order to
promote sharing of research software among investigators in the US
and abroad. 

\section*{References}

\begin{enumerate}
\item Kernighan, B. W., and D. M. Ritchie, \emph{The C programming language},
2nd edition, p. 272, Prentice-Hall, Englewood Cliffs, New Jersey, 1988.

\item Adobe Systems Inc., \emph{PostScript Language Reference Manual},
2nd edition, p. 764, Addison-Wesley, Reading, Massachusetts, 1990.

\item Lewine, D., POSIX programmer's guide, 1st edition, p. 607, O'Reilly
\& Associates, Sebastopol, California, 1991. 

\item Treinish, L. A., and M. L. Gough, A software package for the
data-independent management of multidimensional data,
\emph{EOS trans. AGU, 68, }633--635, 1987.

\item Smith, W. H. F., and P. Wessel, Gridding with continuous curvature
splines in tension, \emph{Geophysics, 55, }293--305, 1990. 

\item Wessel, P., and W. H. F. Smith, New, improved version of Generic
Mapping Tools released, \emph{EOS trans. AGU}, 79, 579, 1998. 

\item Wessel, P., and W. H. F. Smith, New version of the Generic
Mapping Tools released, \emph{EOS trans. AGU}, 76, 329, 1995a. 

\item Wessel, P., and W. H. F. Smith, New version of the Generic
Mapping Tools released, \emph{EOS electronic supplement, }
http://www.agu.org/eos\_elec/95154e.html, 1995b. 

\item Wessel, P., and W. H. F. Smith, Free software helps map and
display data, \emph{EOS trans. AGU}, 72, 441 \& 445--446, 1991. 

\end{enumerate}
