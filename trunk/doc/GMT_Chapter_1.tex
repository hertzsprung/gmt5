%------------------------------------------
%	$Id$
%
%	The GMT Documentation Project
%	Copyright (c) 2000-2012.
%	P. Wessel, W. H. F. Smith, R. Scharroo, J. Luis and F. Wobbe
%------------------------------------------
%
\chapter{Preface} 
\label{ch:1}

While \GMT\ has served the map-making and data processing needs of scientists since 1988\footnote{Version
1.0 was then informally released at the Lamont-Doherty Earth Observatory.}, the current global use was
heralded by the first official release in \emph{EOS Trans. AGU} in the fall of 1991.  Since then,
\GMT\ has grown to become a standard tool for many users, particularly in the Earth and Ocean Sciences.
Development has at times been rapid, and numerous releases have seen the light of day since the early
versions.  For a detailed history of the changes from release to release, see file \filename{ChangeLog}
in the main \GMT\ directory.  For a nightly snapshot of ongoing activity, see the online
\htmladdnormallink{ChangeLog}{http://\GMTSITE/gmt/gmt_changelog.php} page.

The success of \GMT\ is to a large degree due to the input of the user community. In fact, most of the
capabilities and options in \GMT\ programs originated as user requests.
We would like to hear from you should you have any suggestions for future enhancements and modification.
Please send your comments to the
\htmladdnormallink{GMT help list}{mailto:gmt-help@lists.hawaii.edu}.

\section{What is new in \gmt\ 5.x?}

\GMT\ 5 represents a new branch of \GMT\ development that preserves the capabilities of the previous
versions while adding new tricks to many of the tools.  Furthermore, we have added system-wide
capabilities for handling PDF transparency, dealing with GIS aspatial data, and eliminated the need
for the \Opt{m} option.  Our \PS\ library \GMTprog{PSL} has seen a complete rewrite as well and produce
shorter and more compact \PS.
However, the big news is for developers who wish to leverage \GMT\ in their
own applications.  We have completely revamped the code base so that high-level \GMT\ functionality
is now accessible via \GMT\ ``modules''.  These are high-level functions named after their corresponding
programs (.e.g., \GMTfunc{GMT\_grdimage}) that contains all of the functionality of that program within
the function.  While currently callable from C/C++ only, we have built several of the Matlab interface
modules as well and will soon start on the Python version.  Developers should consult the GMT API Documentation for more details.

Most of the \GMT\ default parameters have changed named in order to group parameters into logical groups
and to use more consistent naming.  Some new default parameters have been added as well, such as
\textbf{MAP\_ANNOT\_ORTHO}, which controls whether axes annotations for Cartesian plots are horizontal or
orthogonal to the individual axes.

Because of the default name changes and other command-line changes (such as making \Opt{m} obsolete), we
recommend that users of \GMT\ 4 consider learning the new rules and defaults.  However, to ease the
transition to \GMT\ 5 you may use the \-\-enable-compat switch when running configure , thus allowing the use
of many obsolete default names and command switches (you will receive a warning instead).

\subsection{Overview of \gmt\ 5.0.0 [Jan-1, 2013]}

This version shares the same bug fixes applied to \GMT\ 4.5.9, released Jan 1, 2013.
Several new programs have been added; some have been promoted from earlier supplements:

\begin{description}
	\item [\GMTprog{gmt2kml}]: A \GMTprog{psxy}-like tool to produce KML overlays for Google Earth.
	\item [\GMTprog{gmtdp}]: A line-reduction tool for coastlines and similar lines.
	\item [\GMTprog{gmtstitch}]: Join individual lines whose end points match within given tolerance.
	\item [\GMTprog{gmtwhich}]: Return the full path to specified data files.
	\item [\GMTprog{kml2gmt}]: Extract GMT data tables from Google Earth KML files.
	\item [\GMTprog{gmtspatial}]: Perform geospatial operations on lines and polygons.
	\item [\GMTprog{gmtvector}]: Perform basic vector manipulation in 2-D and 3-D.
\end{description}

\noindent
Below is a list of improvements that affect several \gmt\ programs equally:

\begin{enumerate}
	\item All programs now use consistent, standardized choices for plot dimension units (\textbf{c}m, \textbf{i}nch, or
	\textbf{p}oint;
		we no longer consider \textbf{m}eter a plot length unit), and actual distances (choose spherical arc lengths in \textbf{d}egree,
		\textbf{m}inute, and \textbf{s}econd [was \textbf{c}], or distances in m\textbf{e}ter [Default], \textbf{f}oot [new], 
		\textbf{k}m, \textbf{M}ile [was sometimes \textbf{i} or \textbf{m}], \textbf{n}autical mile, and s\textbf{u}rvey foot [new]).
	\item Programs that read data tables can now process multi-segment tables automatically.  This means
		programs that did not have this capability (e.g., \GMTprog{filter1d}) now can filter segments
		separately; consequently, there is no longer a \Opt{m} option.
	\item Programs that read data tables can now process the aspatial metadata in OGR/GMT files with the new \Opt{a} option.
		These are produced by \progname{ogr2ogr} (a GDAL tool) when selecting the -f ``GMT'' output format.  See Appendix Q
		for an explanation of the OGR/GMT file format.  Because all GIS information is encoded via \GMT\ comment lines
		these files can also be used in \GMT\ 4 (the GIS metadata is simply skipped).
	\item Programs that read data tables can control which columns to read and in what order with the new \Opt{i} option.
	\item Programs that write data tables can control which columns to write and in what order with the new \Opt{o} option.
	\item Programs that write data tables can specify a custom binary format using the enhanced \Opt{b} option.
	\item Programs that read data tables can control how records with NaNs are handled with the new \Opt{s} option.
	\item Programs that read grids can use new common option \Opt{n} to control grid interpolation settings and boundary conditions.
	\item Programs that read grids can now handle Arc/Info float binary files (GRIDFLOAT) and ESRI .hdr formats.
	\item Programs that read grids now set boundary conditions to aid further processing.  If a subset then the
		boundary conditions are taken from the surrounding grid values.
	\item There is new \gmt\ defaults parameters that control which algorithms to use for Fourier transforms (GMT\_FFT) and triangulation (GMT\_TRIANGULATE).
	\item Great circle distance approximations can now be fine-tuned via new \gmt\ default parameters PROJ\_MEAN\_RADIUS and PROJ\_AUX\_LATITUDE.
		Geodesics are now even more accurate by using the Vincenty [1975] algorithm instead of Rudoe's method.
	\item New parameter EXTRAPOLATE\_VAL controls what splines should do if requested to extrapolate beyond the given data domain.
	\item \GMT\ 5 only produces \PS\ and no longer has a setting for EPS.  We made this decision since a) our EPS effort was always
		very approximate (no consideration of font metrics. etc.) and often wrong, and b) \GMTprog{ps2raster} handles it exactly.
	\item The \Opt{B} option can now handle irregular and custom annotations (see Section \ref{sec:custaxes}).
		It also has an automatic mode which will select optimal intervals.  The 3-D base maps can now have horizontal
		gridlines on xz and yz back walls.
	\item The \Opt{R} option may now accept a leading unit which implies the given coordinates are projected map coordinates and should be
		replaced with the corresponding geographic coordinates given the specified map projection.  For linear projections
		such units imply a simple unit conversion for the given coordinates (e.g., km to meter).
	\item Introduced \Opt{fp}[{\it unit}] which allows data input to be in projected
		values, e.g., UTM coordinates while \Opt{Ju} is given.
	\item All plot programs can take a new \Opt{p} option for perspective view from infinity.  In \GMT\ 4, only some
		programs could do this (e.g., \GMTprog{pscoast}) and it took a program-specific option, typically \Opt{E} and
		sometimes an option \Opt{Z} would be needed as well.  This information is now all passed via \Opt{p} and
		applies across all \GMT\ plotting programs.
	\item All plot programs can take a new \Opt{t} option to modify the PDF transparency level.  However, as \PS\ has
		no provision for transparency you can only see the effect if you convert it to PDF.
	\item All text can now optionally be filled with patterns and/or drawn with outline pens.  In the past, only
		\GMTprog{pstext} could plot outline fonts via \Opt{S}\emph{pen}.  Now, any text can be an outline text
		by manipulating the corresponding FONT defaults (e.g., \textbf{FONT\_TITLE}).
	\item All color or fill specifications may append @\emph{transparency} to change the PDF transparency level for that item.
		See \Opt{t} for limitations on how to visualize this transparency.
\end{enumerate}

\noindent
Here is a list of recent enhancements to specific programs:

\begin{enumerate}
	\item \GMTprog{blockmedian} added \Opt{Er}[-] to return as last column the record number that
		gave the median value.  For ties, we return the record number
		of the higher data value unless \Opt{Er}- is given (return lower).
		Added \Opt{Es} to read and output source id for median value.
	\item \GMTprog{blockmode} added \Opt{Er}[-] but for modal value.
		Added \Opt{Es} to read and output source id for modal value.
	\item \GMTprog{gmtconvert} now has optional PCRE (regular expression) support.
	\item \GMTprog{gmtmath} with Opt{N}{\it ncol} and input files will add extra blank columns, if needed.
	\item \GMTprog{grdblend} can take list of grids on the command line and blend, and now has more blend choices.  Grids no
		longer have to have same registration or spacing.
	\item \GMTprog{grdfilter} can now do spherical filtering (with wrap around longitudes and over poles) for non-global grids.
		We have also begun implementing Open MP threads to speed up calculations on multi-core machines.  We have
		added rectangular filtering and automatic resampling to input resolution for high-pass filters.  There is also
		\Opt{Ff}{\it weightgrd} which reads the gridfile {\it weightgrd} for a custom Cartesian
		 grid convolution.  The {\it weightgrd} must have odd dimensions.  Similarly added \Opt{Fo}{\it opgrd}
		for operators (via coefficients in the grdfile {\it opgrd}) whose weight sum is zero (hence we do not sum
		and divide the convolution by the weight sum).
	\item \GMTprog{grdinfo} now has modifier \Opt{Ts}{\it dz} which returns a symmetrical range about zero.
	\item \GMTprog{grdmask} has new option \Opt{Ni}$|$I$|$p$|$P to set inside of polygons to the polygon IDs.
		These may come from OGR aspatial values, segment head \Opt{L}ID,
		or a running number, starting at a specified origin [0].  Now correctly handles polygons with perimeters and holes.
		Added z as possible radius value in \Opt{S} which means read radii from 3rd input column.
	\item \GMTprog{grdmath} added operator SUM which adds up all non-NaN entries and returns a grid with all nodes set to the sum constant.
	\item \GMTprog{grdtrack} has a new \Opt{C}, \Opt{D} options to automatically create an equidistant set of cross-sectional
		profiles given input line segments; one or more grids can then be sampled at these locations.  Also added \Opt{S}
		which stack cross-profiles generated with \Opt{C}.  Finally, \Opt{N}
		will not skip points that are outside the grid domain but return NaN as sampled value.
	\item \GMTprog{mapproject} has a new \Opt{N} option to do geodetic/geocentric conversions; it combines with \Opt{I}
		for inverse conversions.  Also, we have extended \Opt{A} to accept \Opt{A}\textbf{o}$|$\textbf{O} to compute line orientations (-90/90).
	\item \GMTprog{makecpt} and \GMTprog{grd2cpt} has a new \Opt{F} option to specify output color representation, e.g.,
		to output the CPT table in h-s-v format despite originally being given in r/g/b.
	\item \GMTprog{minmax} has new option \Opt{A} to select what group to report on (all input, per file, or per segment).
	\item \GMTprog{gmtconvert} has new option \Opt{Q} to select a particular segment number.
	\item \GMTprog{gmtmath} and \GMTprog{grdmath} now support simple replacement macros via user files
		\filename{.grdmath} and \filename{.gmtmath}.  This mechanism works by replacing the macro name
		with the equivalent arguments in the program argument lists.
	\item \GMTprog{grdvolume} has enhanced \Opt{T}, now \Opt{T}[\textbf{c}$|$\textbf{h}] for ORS estimates based on max curvature or height.
	\item \GMTprog{project} has added \Opt{G}...[+] so if + is appended we get a segment header with information about the pole for the circle.
	\item \GMTprog{ps2raster} has added a \Opt{TF} option to create multi-page PDF files.
	\item \GMTprog{pscontour} now similar to \GMTprog{grdcontour} in the options it takes, e.g., \Opt{C} in particular.
		In \GMT\ 4, the program could only read a CPT file and not take a specific contour interval.
	\item \GMTprog{psrose} has added \Opt{Wv}{\it pen} to specify pen for vector (in \Opt{C}).
		Added \Opt{Zu} to set all radii to unity (i.e., for analysis of angles only).
	\item \GMTprog{psscale} has a new option \Opt{T} that paints a rectangle behind the color bar.
	\item \GMTprog{pstext} has enhanced \Opt{DJ} option to shorten diagonal offsets by $\sqrt{2}$ to
		maintain the same distance from point to annotation.
	\item \GMTprog{psxy.c} and \GMTprog{psxyz.c} can take symbol modifier \textbf{+s}\emph{scale}[\emph{unit}][/\emph{origin}][{ \bf l}]
		in \Opt{S} to adjust scales read from files.  This is used when you have data in the third column that should be
		used for symbol size but they need to be offset (by \emph{origin}) and scaled by (\emph{scale}) first; \textbf{l} means
		take the logarithm of the data column first.  Also, the custom symbol macro languages has been expanded considerably
		to allow for complicated, multi-parameter symbols; see Appendix N for details.
		Finally, allow the base for bars and columns optionally to be read from data file by using not specifying the base value.
	\item \GMTprog{pstext.c} can take simplified input via new option \Opt{F} to set fixed font (including size), angle, and justification.  If
		these parameters are fixed for all the text strings then the input can simply be \emph{x y text}.
	\item \GMTprog{triangulate} now offers \Opt{S} to write triangle polygons and can handle 2-column input if \Opt{Z} is given.
	\item \GMTprog{xyz2grd} now also offers \Opt{Am} (mean, the default) and \Opt{Ar} (rms).
	
\end{enumerate}

\noindent
Several supplements have new feature as well:

\begin{enumerate}
	\item \GMTprog{mgd77/mgd77convert.c} added \Opt{C} option to assemble *.mgd77 files from *.h77/*.a77 pairs.
	\item The spotter programs can now read Gplates rotations directly as well as write this format.
		Now, \GMTprog{rotconverter} can extract plate circuit rotations on-the-fly from the Gplates rotation file.
\end{enumerate}

\section{Incompatibilities between \gmt\ 5.x and \gmt\ 4.x}

As features are added and bugs are discovered, it is occasionally necessary to break
the established syntax of a \gmt\ program option, such as when the intent of the option
is non-unique due to a modifier key being the same as a distance unit indicator.  Other times we see a greatly improved
commonality across similar options by making minor adjustments.  However, we are aware that
such changes may cause grief and trouble with established scripts and the habits of many \gmt\ users.
To alleviate this situation we have introduced a configuration that allows \gmt\ to
tolerate and process obsolete program syntax (to the extent possible).  To activate you must
make sure {\bf GMT\_COMPAT} is not set to ``no'' in your \filename{ConfigUser.cmake} file.  When not
running in compatibility mode any obsolete syntax will be considered as errors.  We recommend
that users with prior \gmt\ 4 experience run \gmt\ 5 in compatibility mode, heed the warnings
about obsolete syntax, and correct their scripts or habits accordingly.  When this transition
has been successfully navigated it is better to turn compatibility mode off and leave the
past behind.  Occasionally, users will supply an ancient \gmt\ 3 syntax which
may have worked in \gmt\ 4 but is not tolerated in \gmt\ 5.

Here are a list of known incompatibilities that are correctly processed correctly with a warning under
compatibility mode:
\begin{enumerate}
	\item \gmt\ {\bf default names}: We have organized the default parameters logically by group and
		renamed several to be easier to remember and to group.  Old and new names can be
		found in Tables~\ref{tbl:obsoletedefs1} and \ref{tbl:obsoletedefs2}.  In addition, a few defaults are no longer
		recognized, such as N\_COPIES, PS\_COPIES, DOTS\_PR\_INCH, GMT\_CPTDIR, PS\_DPI, and PS\_EPS, TRANSPARENCY.
		This also means the old common option \Opt{c} for specifying \PS\ copies is no longer available.
	\item {\bf Units}: The unit abbreviation for arc seconds is finally {\bf s} instead of {\bf c},
		with the same change for upper case in some clock format statements.
	\item {\bf Contour labels}: The modifiers \Mod{k}{\it fontcolor} and \Mod{s}{\it fontsize} are obsolete, now being part of \Mod{f}{\it font}.
	\item {\bf Ellipsoids}: Assigning {\bf PROJ\_ELLIPSOID} a file name is deprecated, use comma-separated parameters {\it a, f$^{-1}$} instead.
	\item {\bf Custom symbol macros:} Circle macro symbol {\bf C} is deprecated; use {\bf c} instead.
	\item {\bf Map scale}: Used by \GMTprog{psbasemap} and others.  Here, the unit {\bf m} is deprecated; use {\bf M} for statute miles.
	\item {\bf 3-D perspective}: Some programs used a combination of \Opt{E}, \Opt{Z} to set up a 3-D perspective
		view, but these options were not universal.  The new 3-D perspective in \gmt\ 5 means you
		instead use the common option \Opt{p} to configure the 3-D projection.
	\item {\bf Pixel vs. gridline registration:} Some programs used to have a local \Opt{F} to turn on pixel registration;
		now this is a common option \Opt{r}.
	\item {\bf Table file headers}: For consistency with other common i/o options we now use \Opt{h} instead of \Opt{H}.
	\item {\bf Segment headers}: These are now automatically detected and hence there is no longer a \Opt{m} (or the older \Opt{M} option).
	\item {\bf Front symbol}: The syntax for the front symbol has changed from \Opt{Sf}{\it spacing/size}[\Mod{d}][{\Mod{t}][:{\it offset}]
		to \Opt{Sf}{\it spacing/size}[{\bf +r+l}][{\bf +f+t+s+c+b}][\Mod{o}{\it offset}].
	\item {\bf Vector symbol}: With the introduction of geo-vectors there are three kinds of vectors that can be drawn:
		Cartesian (straight) vectors with \Opt{Sv} or \Opt{SV}, geo-vectors (great circles) with \Opt{S=}, and circular vectors with \Opt{Sm}.
		These are all composed of a line (controlled by pen settings) and 0--2 arrow heads (control by fill and outline settings).
		Many modifiers common to all arrows have been introduced using the {\bf +key}[{\it arg}] format.  The {\it size}
		of a vector refers to the length of its head; all other quantities are given via modifiers (which have sensible default values).
		In particular, giving size as {\it vectorwidth/headlength/headwidth} is deprecated.
		See the \GMTprog{psxy} man page for a clear description of all modifiers.
	\item \GMTprog{blockmean}: The \Opt{S} and \Opt{Sz} options are deprecated; use \Opt{Ss} instead.
	\item \GMTprog{filter1d}: The \Opt{N}{\it ncol/tcol} option is deprecated; use \Opt{N}{\it tcol} instead
		as we automatically determine the number of columns in the file.
	\item \GMTprog{gmtconvert}: \Opt{F} is deprecated; use common option \Opt{o} instead.
	\item \GMTprog{gmtdefaults}: \Opt{L} is deprecated; this is now the default behavior.
	\item \GMTprog{gmtmath}: \Opt{F} is deprecated; use common option \Opt{o} instead.
	\item \GMTprog{gmtselect}: \Opt{Cf} is deprecated; use common specification format \Opt{C-} instead. Also,
		\Opt{N}...{\bf o} is deprecated; use \Opt{E} instead.
	\item \GMTprog{grd2xyz}: \Opt{E} is deprecated as the ESRI ASCII exchange format is now detected automatically.
	\item \GMTprog{grdcontour}: \Opt{m} is deprecated as segment headers are handled automatically.
	\item \GMTprog{grdfft}: \Opt{M} is deprecated; use common option \Opt{fg} instead.
	\item \GMTprog{grdgradient}: \Opt{L} is deprecated; use common option \Opt{n} instead.  Also,
		\Opt{M} is deprecated; use common option \Opt{fg} instead.
	\item \GMTprog{grdlandmask}: \Opt{N}...{\bf o} is deprecated; use \Opt{E} instead.
	\item \GMTprog{grdimage}: \Opt{S} is deprecated; use \Opt{n}{\it mode}[\Mod{a}][\Mod{t}{\it threshold}] instead.
	\item \GMTprog{grdmath}: LDIST and PDIST now return distances in spherical degrees; while in \gmt\ 4 it returned km; use DEG2KM for conversion, if needed.
	\item \GMTprog{grdproject}: \Opt{S} is deprecated; use \Opt{n}{\it mode}[\Mod{a}][\Mod{t}{\it threshold}] instead.
		Also, \Opt{N} is deprecated; use \Opt{D} instead.
	\item \GMTprog{grdsample}: \Opt{Q} is deprecated; use \Opt{n}{\it mode}[\Mod{a}][\Mod{t}{\it threshold}] instead.
		Also, \Opt{L} is deprecated; use common option \Opt{n} instead, and \Opt{N}{\it nx>/<ny} is deprecated; use \Opt{I}{\it nx+>/<ny+} instead.
	\item \GMTprog{grdtrack}: \Opt{Q} is deprecated; use \Opt{n}{\it mode}[\Mod{a}][\Mod{t}{\it threshold}] instead.
		Also, \Opt{L} is deprecated; use common option \Opt{n} instead, and \Opt{S} is deprecated; use common option \Opt{sa} instead.
	\item \GMTprog{grdvector}: \Opt{E} is deprecated; use the vector modifier \Mod{jc} as well as the general vector specifications discussed earlier.
	\item \GMTprog{grdview}: \Opt{L} is deprecated; use common option \Opt{n} instead.
	\item \GMTprog{nearneighbor}: \Opt{L} is deprecated; use common option \Opt{n} instead.
	\item \GMTprog{project}: \Opt{D} is deprecated; use {-}{-}FORMAT\_GEO\_OUT instead.
	\item \GMTprog{psbasemap}: \Opt{G} is deprecated; specify canvas color via \Opt{B} modifier \Mod{g}{\it color}.
	\item \GMTprog{pscoast}: \Opt{m} is deprecated and have reverted to \Opt{M} for selecting data output instead of plotting.
	\item \GMTprog{pscontour}: \Opt{T}{\it indexfile} is deprecated; use \Opt{Q}{\it indexfile}.
	\item \GMTprog{pshistogram}: \Opt{T}{\it col} is deprecated; use common option \Opt{i} instead.
	\item \GMTprog{pslegend}: Paragraph text header flag > is deprecated; use P instead.
	\item \GMTprog{psmask}: \Opt{D}...\Mod{n}{\it min} is deprecated; use \Opt{Q} instead.
	\item \GMTprog{psrose}: Old vector specifications in Option \Opt{M} are deprecated; see new explanations.
	\item \GMTprog{pstext}: \Opt{m} is deprecated; use \Opt{M} to indicate paragraph mode.  Also,
		\Opt{S} is deprecated as fonts attributes are now specified via the font itself.
	\item \GMTprog{pswiggle}: \Opt{D} is deprecated; use common option \Opt{g} to indicate data gaps.  Also,
		\Opt{N} is deprecated as all fills are set via the \Opt{G} option.
	\item \GMTprog{psxy}: Old vector specifications in Option \Opt{S} are deprecated; see new explanations.
	\item \GMTprog{psxyz}: Old vector specifications in Option \Opt{S} are deprecated; see new explanations.
	\item \GMTprog{splitxyz}: \Opt{G} is deprecated; use common option \Opt{g} to indicate data gaps.  Also,
		\Opt{M} is deprecated; use common option \Opt{fg} instead.
	\item \GMTprog{triangulate}: \Opt{m} is deprecated; use \Opt{M} to output triangle vertices.
	\item \GMTprog{xyz2grd}: \Opt{E} is deprecated as the ESRI ASCII exchange format is one of our recognized formats. Also,
		\Opt{A} (no arguments) is deprecated; use \Opt{Az} instead.
	\item \GMTprog{dbase/grdraster}: The H{\it skip} field in \filename{grdraster.info} is no longer
		expected as we automatically determine if a raster has a \gmt\ header.  Also, to
		output {\it x,y,z} triplets instead of writing a grid now requires \Opt{T}.
	\item \GMTprog{img/img2grd}: \Opt{m}{\it inc} is deprecated; use \Opt{I}{\it inc} instead.
	\item \GMTprog{meca/psvelo}: Old vector specifications are deprecated; see new explanations.
	\item \GMTprog{mgd77/mgd77convert}: \Opt{4} is deprecated; use \Opt{D} instead.
	\item \GMTprog{mgd77/mgd77list}: The unit {\bf m} is deprecated; use {\bf M} for statute miles.
	\item \GMTprog{mgd77/mgd77manage}: The unit {\bf m} is deprecated; use {\bf M} for statute miles.
		The \Opt{Q} is deprecated; use \Opt{n}{\it mode}[\Mod{a}][\Mod{t}{\it threshold}] instead
	\item \GMTprog{mgd77/mgd77path}: \Opt{P} is deprecated (clashes with \gmt\ common options); use \Opt{A} instead.
	\item \GMTprog{spotter/backtracker}: \Opt{C} is deprecated as stage vs. finite rotations are detected automatically.
	\item \GMTprog{spotter/grdrotater}: \Opt{C} is deprecated as stage vs. finite rotations are detected automatically.
		Also, \Opt{T}{\it lon/lat/angle} is now set via \Opt{e}{\it lon/lat/angle}.
	\item \GMTprog{spotter/grdspotter}: \Opt{C} is deprecated as stage vs. finite rotations are detected automatically.
	\item \GMTprog{spotter/hotpotter}: \Opt{C} is deprecated as stage vs. finite rotations are detected automatically.
	\item \GMTprog{spotter/originator}: \Opt{C} is deprecated as stage vs. finite rotations are detected automatically.
	\item \GMTprog{spotter/rotconverter}: \Opt{Ff} selection is deprecated, use \Opt{Ft} instead.
	\item \GMTprog{x2sys/x2sys\_datalist}: The unit {\bf m} is deprecated; use {\bf M} for statute miles.
\end{enumerate}

\begin{table}[H]
\centering
\begin{tabular}{|l|l|} \hline
\emph{Old Name}	& \emph{New Name} \\ \hline
INPUT\_CLOCK\_FORMAT		&	FORMAT\_CLOCK\_IN \\ \hline
INPUT\_DATE\_FORMAT		&	FORMAT\_DATE\_IN \\ \hline
OUTPUT\_CLOCK\_FORMAT		&	FORMAT\_CLOCK\_OUT \\ \hline
OUTPUT\_DATE\_FORMAT		&	FORMAT\_DATE\_OUT \\ \hline
OUTPUT\_CLOCK\_FORMAT		&	FORMAT\_CLOCK\_OUT \\ \hline
OUTPUT\_DEGREE\_FORMAT		&	FORMAT\_GEO\_OUT \\ \hline
PLOT\_CLOCK\_FORMAT		&	FORMAT\_CLOCK\_MAP \\ \hline
PLOT\_DATE\_FORMAT		&	FORMAT\_DATE\_MAP \\ \hline
PLOT\_DEGREE\_FORMAT		&	FORMAT\_GEO\_MAP \\ \hline
TIME\_FORMAT\_PRIMARY		&	FORMAT\_TIME\_PRIMARY\_MAP \\ \hline
TIME\_FORMAT\_SECONDARY		&	FORMAT\_TIME\_SECONDARY\_MAP \\ \hline
D\_FORMAT			&	FORMAT\_FLOAT\_OUT \\ \hline
UNIX\_TIME\_FORMAT		&	FORMAT\_TIME\_LOGO \\ \hline
ANNOT\_FONT\_PRIMARY		&	FONT\_ANNOT\_PRIMARY \\ \hline
ANNOT\_FONT\_SECONDARY		&	FONT\_ANNOT\_SECONDARY \\ \hline
HEADER\_FONT			&	FONT\_TITLE \\ \hline
LABEL\_FONT			&	FONT\_LABEL \\ \hline
ANNOT\_FONT\_SIZE\_PRIMARY	&	FONT\_ANNOT\_PRIMARY \\ \hline
ANNOT\_FONT\_SIZE\_SECONDARY	&	FONT\_ANNOT\_SECONDARY \\ \hline
HEADER\_FONT\_SIZE		&	FONT\_TITLE \\ \hline
ANNOT\_OFFSET\_PRIMARY		&	MAP\_ANNOT\_OFFSET\_PRIMARY \\ \hline
ANNOT\_OFFSET\_SECONDARY	&	MAP\_ANNOT\_OFFSET\_SECONDARY \\ \hline
OBLIQUE\_ANNOTATION		&	MAP\_ANNOT\_OBLIQUE \\ \hline
ANNOT\_MIN\_ANGLE		&	MAP\_ANNOT\_MIN\_SPACING \\ \hline
Y\_AXIS\_TYPE			&	MAP\_ANNOT\_ORTHO \\ \hline
DEGREE\_SYMBOL			&	MAP\_DEGREE\_SYMBOL \\ \hline
BASEMAP\_AXES			&	MAP\_FRAME\_AXES \\ \hline
BASEMAP\_FRAME\_RGB		&	MAP\_DEFAULT\_PEN \\ \hline
FRAME\_PEN			&	MAP\_FRAME\_PEN \\ \hline
BASEMAP\_TYPE			&	MAP\_FRAME\_TYPE \\ \hline
FRAME\_WIDTH			&	MAP\_FRAME\_WIDTH \\ \hline
GRID\_CROSS\_SIZE\_PRIMARY	&	MAP\_GRID\_CROSS\_SIZE\_PRIMARY \\ \hline
GRID\_CROSS\_SIZE\_SECONDARY	&	MAP\_GRID\_CROSS\_SIZE\_SECONDARY \\ \hline
GRID\_PEN\_PRIMARY		&	MAP\_GRID\_PEN\_PRIMARY \\ \hline
GRID\_PEN\_SECONDARY		&	MAP\_GRID\_PEN\_SECONDARY \\ \hline
LABEL\_OFFSET			&	MAP\_LABEL\_OFFSET \\ \hline
LINE\_STEP			&	MAP\_LINE\_STEP \\ \hline
UNIX\_TIME			&	MAP\_LOGO \\ \hline
UNIX\_TIME\_POS			&	MAP\_LOGO\_POS \\ \hline
X\_ORIGIN			&	MAP\_ORIGIN\_X \\ \hline
Y\_ORIGIN			&	MAP\_ORIGIN\_Y \\ \hline
POLAR\_CAP			&	MAP\_POLAR\_CAP \\ \hline
TICK\_LENGTH			&	MAP\_TICK\_LENGTH\_PRIMARY|SECONDARY \\ \hline
TICK\_PEN			&	MAP\_TICK\_PEN\_PRIMARY|SECONDARY \\ \hline
\end{tabular}
\caption{Renaming of some \gmt\ default parameters.}
\label{tbl:obsoletedefs1}
\end{table}
\begin{table}[H]
\centering
\begin{tabular}{|l|l|} \hline
\emph{Old Name}	& \emph{New Name} \\ \hline
HEADER\_OFFSET			&	MAP\_TITLE\_OFFSET \\ \hline
VECTOR\_SHAPE			&	MAP\_VECTOR\_SHAPE \\ \hline
HSV\_MIN\_SATURATION		&	COLOR\_HSV\_MIN\_S \\ \hline
HSV\_MAX\_SATURATION		&	COLOR\_HSV\_MAX\_S \\ \hline
HSV\_MIN\_VALUE			&	COLOR\_HSV\_MIN\_V \\ \hline
HSV\_MAX\_VALUE			&	COLOR\_HSV\_MAX\_V \\ \hline
CHAR\_ENCODING			&	PS\_CHAR\_ENCODING \\ \hline
PS\_COLOR			&	COLOR\_HSV\_MAX\_V \\ \hline
PAGE\_COLOR			&	PS\_PAGE\_COLOR \\ \hline
PAGE\_ORIENTATION		&	PS\_PAGE\_ORIENTATION \\ \hline
PAPER\_MEDIA			&	PS\_MEDIA \\ \hline
GLOBAL\_X\_SCALE		&	PS\_SCALE\_X \\ \hline
GLOBAL\_Y\_SCALE		&	PS\_SCALE\_X \\ \hline
FIELD\_DELIMITER		&	IO\_COL\_SEPARATOR \\ \hline
GRIDFILE\_FORMAT		&	IO\_GRIDFILE\_FORMAT \\ \hline
GRIDFILE\_SHORTHAND		&	IO\_GRIDFILE\_SHORTHAND \\ \hline
NAN\_RECORDS			&	IO\_NAN\_RECORDS \\ \hline
XY\_TOGGLE			&	IO\_LONLAT\_TOGGLE \\ \hline
ELLIPSOID			&	PROJ\_ELLIPSOID \\ \hline
MEASURE\_UNIT			&	PROJ\_LENGTH\_UNIT \\ \hline
MAP\_SCALE\_FACTOR		&	PROJ\_SCALE\_FACTOR \\ \hline
HISTORY				&	GMT\_HISTORY \\ \hline
INTERPOLANT			&	GMT\_INTERPOLANT \\ \hline
VERBOSE				&	GMT\_VERBOSE \\ \hline
INTERPOLANT			&	GMT\_INTERPOLANT \\ \hline
WANT\_LEAP\_SECONDS		&	TIME\_LEAP\_SECONDS \\ \hline
Y2K\_OFFSET\_YEAR		&	TIME\_Y2K\_OFFSET\_YEAR \\ \hline
INTERPOLANT			&	GMT\_INTERPOLANT \\ \hline
\end{tabular}
\caption{Renaming of some \gmt\ default parameters.}
\label{tbl:obsoletedefs2}
\end{table}
