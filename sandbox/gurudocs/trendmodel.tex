%	$Id$
% Internal documentation for what will happen with trend-modeling in GMT 5.2
% Paul Wessel, Feb 5, 2015.
\documentclass[12pt,letterpaper,margin=0.5in]{report}
\usepackage{times}
\usepackage{graphicx}
\usepackage{breqn}
\usepackage[margin=0.5in]{geometry}
\usepackage{lscape}
\textheight = 9 in
\topmargin = -1 in
\begin{document}

\section*{IMPROVING TREND-MODELING OPTIONS IN GMT 5.2}

We seek to enhance the three trend-fitting modules in GMT: {\it trend1d, trend2d}, and {\it grdtrend}.
To do this we need to extend the {\bf --N} option to accommodate more flexible trend specifications
while retaining backwards compatibility with GMT 5 and earlier syntax.  The existing syntax is
\\ \\
{\bf --N}[{\bf f}]{\it n}[{\bf r}]
\\ \\
where the optional {\bf f} selects Fourier rather than the default polynomial bases (currently only available in {\it trend1d}), $n$ sets the degree of
the model, and the optional {\bf r} selects robust fitting.

\section*{1. Specifying a trend model in 1-D (trend1d)}

The proposed revised syntax for 1-D will be
\\ \\
{\bf --N}[{\bf p}$|${\bf P}$|${\bf f}$|${\bf F}$|${\bf c}$|${\bf C}$|${\bf s}$|${\bf S}]{\it n}[{\bf +l}$X$][{\bf +o}$x_0$][{\bf +r}]
\\ \\
with new choices for specifying basis types and new modifiers that override default settings for key parameters; these are discussed
in more detail when they are introduced below.
The general 1-D trend model we wish to fit is defined as
\begin{equation}
y(x) = \sum_{i = 0}^n p_i x^i + \sum_{i = 0}^m  f_i \cos \left [\omega_i (x-x_0) - \phi_i \right ],
\label{1D}
\end{equation}
where the Fourier frequencies are $\omega_i = 2 \pi i / X$.  Here, $x_0$ is the coordinate origin (by default the minimum $x$ input coordinate unless overruled by
a {\bf +o}$x_0$ modifier), and $X$ is the fundamental period (or wavelength) of the signal (by default the range of the input data unless overruled by a {\bf +l}$X$ modifier). Note that the polynomial as given herein is only used for clarity; the solution actually uses Chebychev polynomials for stability.
We will assume that $x$ (and $y$ below) have been corrected for their respective origins before we do the fitting and thus simply
use $x$ below to represent $(x-x_0)$.  As written this model is nonlinear in $\phi_i$; we rewrite the Fourier terms using trigonometric identities:
\begin{equation}
f_i \cos \left [\omega_i x - \phi_i \right ] =  c_i \cos \omega_i x  + s_i \sin \omega_i x .
\end{equation}
We recover the original model parameters via $\phi_i = \tan^{-1}(s_i/c_i)$ and $f_i = \sqrt{c_i^2 + s_i^2}$.
\par
We want to select this entire sum (\ref{1D}) or just portions of it.  The new syntax would allow one or more comma-separated specifications like these:
\begin{enumerate}
	\item {\bf p}{\it n} will include all $n+1$ terms of the full polynomial sum in (1) [i.e., the old {\bf --N}{\it n}].
	\item {\bf P}{\it n} (or {\bf x}{\it n}, or {\bf xx...x}) will only include the single term $x^n$.
	\item {\bf f}{\it m} will include all $2m$ terms in the full Fourier sum (1), i.e., both the cosine and sine components [The old {\bf --Nf}{\it m}].
	\item {\bf F}{\it m} will only include 2 items: the sine and cosine terms for the $m$'th frequency.
	\item {\bf c}{\it m} will include only the cosine series ($m$ terms) in (1).
	\item {\bf C}{\it m} will only include the single cosine term for the $m$'th frequency.
	\item {\bf s}{\it m} will include only the sine series ($m$ terms) in (1).
	\item {\bf S}{\it m} will only include the single sine term for the $m$'th frequency.
\end{enumerate}
Thus, to recreate (\ref{1D}) exactly in {\it trend1d} we would specify {\bf --Np}{\it n},{\bf f}{\it m}, for some values of {\it n} and {\it m}.  Note:
\begin{enumerate}
	\item While the Fourier series starts at $i = 0$, the sine term vanishes and the cosine term becomes an intercept.
	\item If both polynomial and Fourier terms are selected and both include an intercept then we start the Fourier sum at $i = 1$ instead (i.e., the intercept will normally be considered part of the polynomial series).
\end{enumerate}
Some 1-D examples of arguments to {\bf --N} and the equations they imply:
\begin{enumerate}
	\item {\bf p}3 means $p_0 + p_1x + p_2x^2 + p_3x^3$.
	\item {\bf P}3,{\bf f}1 means $p_3x^3 + c_1\cos \omega_1 x + s_1 \sin \omega_1 x$.
	\item {\bf f}1 means $c_0 + c_1\cos \omega_1 x + s_1 \sin \omega_1 x$.
	\item {\bf p}1,{\bf f}1 means $p_0 + p_1x + c_1\cos \omega_1 x + s_1 \sin \omega_1 x$.
	\item {\bf x}0,{\bf x}2 means $p_0 + p_2x^2$.
\end{enumerate}

\section*{2. Specifying a trend model in 2-D (trend2d, grdtrend)}

The proposed revised syntax for 2-D is similar:
\\ \\
{\bf --N}[{\bf p}$|${\bf P}$|${\bf f}$|${\bf F}$|${\bf c}$|${\bf C}$|${\bf s}$|${\bf S}$|${\bf x}$|${\bf X}$|${\bf y}$|${\bf Y}][{\bf x}]{\it n}[{\bf y}{\it m}][{\bf +l}$X/Y$][{\bf +o}$x_0/y_0$][{\bf +r}]
\\ \\
Here, the general model is
\begin{equation}
z(x,y) = \sum_{i = 0}^n \sum_{j = 0}^n p_{ij} x^iy^j + \sum_{i = 0}^m \sum_{j = 0}^m f_{ij} \cos \left [\omega_i^x x \
	- \phi_i^x \right ] \cos \left [\omega_j^y y - \phi_j^y \right ] .
\label{2D}
\end{equation}
with the obvious extension to include $y$-parameters.
This model is nonlinear in $\phi_i^x$ and $\phi_i^y$ so we linearize the Fourier terms, i.e.,
\begin{dmath}
f_{ij} \cos \left [\omega_i^x x- \phi_i^x \right ] \cos \left [\omega_j^y y - \phi_j^y \right ] =
a_{ij} \cos \omega_i^x x \cos \omega_j^y y + b_{ij} \cos \omega_i^x x \sin \omega_j^y y + \\
c_{ij} \sin \omega_i^x x \cos \omega_j^y y + d_{ij} \sin \omega_i^x x \sin \omega_j^y y
\end{dmath}
and we recover the original parameters via $\phi_i^x = \tan^{-1}(c_{ij}/a_{ij})$ and $\phi_i^y = \tan^{-1}(b_{ij}/a_{ij})$
for the two phases and $f_{ij} = \sqrt{a_{ij}^2 + b_{ij}^2 + c_{ij}^2 + f_{ij}^2}$ for the single amplitude.
As for the 1-D case
we wish to be able to specify subsets of this general model in (\ref{2D}).  Here are the possible arguments to {\bf --N}, alone or comma-separated:
\begin{enumerate}
	\item {\bf p}{\it n} will include the $n^2$ terms of the full polynomial sum in (1).
	\item {\bf P}{\it n} will only retain the $n(n+1)/2$ terms in the sum for which $i+j \leq n$.  [The old {\bf --N}{\it n}]
	\item {\bf x}{\it n} will only include the sum over the $x^n$-series (i.e., it implicitly sets $j = 0$ in (\ref{2D})).
	\item {\bf X}{\it n} will only include the term $x^n$.
	\item {\bf y}{\it n} will only include the sum over the $y^n$-series (i.e., it implicitly sets $i = 0$ in (\ref{2D})).
	\item {\bf Y}{\it n} will only include the term $y^n$.
	\item {\bf x}{\it n}{\bf y}{\it m} (or {\bf xx...xyy...y} will only include the single term $x^ny^m$.
	\item {\bf f}{\it m} will include all $4m^2$ terms of the full Fourier sum in (\ref{2D}), i.e., four terms (4) for each frequency.
	\item {\bf F}{\it m} will only include the four sine and cosine terms for the $m$'th frequency.
	\item {\bf c}{\it m} will include only the cosine product series ($m$ terms) in (4), i.e., let phases equal zero.
	\item {\bf C}{\it m} will only include the cosine product for the $m$'th frequency.
	\item {\bf s}{\it m} will include only the sine product series ($m$ terms) in (4), i.e., let phases equal 90.
	\item {\bf S}{\it m} will only include the sine product for the $m$'th frequency.
	\item {\bf fx}{\it m} will ignore y-terms and just include the $2m$ terms of the full 1-D Fourier sum in (\ref{1D}) for $x$, i.e., both the cosine and sine components.
	\item {\bf fy}{\it m} will ignore x-terms and just include the $2m$ terms of the full 1-D Fourier sum in (\ref{1D}) for $y$, i.e., both the cosine and sine components.
	\item {\bf Fx}{\it m} will only include the cosine term for the $m$'th frequency in $x$.
	\item {\bf Fy}{\it m} will only include the cosine term for the $m$'th frequency in $y$.
	\item {\bf cx}{\it m} will include only the cosine series ($m$ terms) in (1) in $x$.
	\item {\bf cy}{\it m} will include only the cosine series ($m$ terms) in (1) in $y$.
	\item {\bf Cx}{\it m} will only include the single cosine term for the $m$'th frequency in $x$.
	\item {\bf Cy}{\it m} will only include the single cosine term for the $m$'th frequency in $y$.
	\item {\bf sx}{\it m} will include only the sine series ($m$ terms) in (1) in $x$.
	\item {\bf sy}{\it m} will include only the sine series ($m$ terms) in (1) in $y$.
	\item {\bf Sx}{\it m} will only include the single sine term for the $m$'th frequency in $x$.
	\item {\bf Sy}{\it m} will only include the single sine term for the $m$'th frequency in $y$.
\end{enumerate}
Some 2-D examples of arguments to {\bf --N} and the equations they imply:
\begin{enumerate}
	\item {\bf p}1 means $p_0 + p_1 x + p_2 y + p_3xy$ (bilinear).
	\item {\bf P}3 means the full 10-term 3rd degree polynomial used in trend2d -N10.
	\item {\bf p}3 means the full 16-term bicubic.
	\item {\bf s}3 means the 4-term sum over powers of x.
	\item {\bf f}1 means $c_0 + a_1\cos \omega^x_1 x \cos \omega ^y_1 y + a_2\cos \omega^x_1 x \sin \omega ^y_1 y + a_3\sin \omega^x_1 x \cos \omega ^y_1 y + a_4\sin \omega^x_1 x \sin \omega ^y_1 y$.
	\item {\bf X}2,{\bf Sy}2 means $c_0 x^2 + c_1 \sin \omega^y_2 y$
\end{enumerate}

\section*{3. Backwards Compatibility}

\subsection*{3.1 trend1d}
There is one-to-one correspondence between the old syntax and its equivalent
in the proposal.  Options like {\bf --N}[{\bf f}]{\it n}[{\bf r}] will be translated to {\bf --N}[{\bf f}$|${\bf p}]{\it n}[{\bf +r}], i.e.,
we simply add the {\bf p} basis indicator if none were given.

\subsection*{3.2 trend2d, grdtrend}
In the
2-D case there were never any Fourier option, so the only ambiguity arises for options
with no basis function indicator.  When such arguments are found we assume it represent
the old syntax and the order {\it n} means the first {\it n}
terms in the fixed polynomial
\begin{equation}
z(x,y) = m_1 + m_2 x + m_3 y + m_4 xy + m_5 x^2 + m_6 y^2 + m_7 x^3 + m_8 x^2y + m_9 xy^2 + m_{10} y^3.
\label{eq:2Dold}
\end{equation}
We may then translate a plain {\bf --N}{\it n}[{\bf r}] option, provided
{\it n} is in the 1--10 range, by substituting the new syntax that is equivalent to the original legacy intent (\ref{eq:2Dold}). Below, {\it n} is the item number:
\begin{enumerate}
	\item {\bf --NP}{\it 0}[{\bf +r}]
	\item {\bf --Nx}{\it 1}[{\bf +r}]
	\item {\bf --NP}{\it 1}[{\bf +r}]
	\item {\bf --NP}{\it 1},{\bf xy}[{\bf +r}]
	\item {\bf --NP}{\it 1},{\bf xy},{\bf xx}[{\bf +r}]
	\item {\bf --NP}{\it 2}[{\bf +r}]
	\item {\bf --NP}{\it 2},{\bf xxx}[{\bf +r}]
	\item {\bf --NP}{\it 2},{\bf xxx},{\bf xxy}[{\bf +r}]
	\item {\bf --NP}{\it 2},{\bf xxx},{\bf xxy},{\bf xyy}[{\bf +r}]
	\item {\bf --NP}{\it 3}[{\bf +r}]
\end{enumerate}
The accommodations ensures 100\% backwards compatibility with previous versions.
\end{document}
