%------------------------------------------
%	$Id: GMT_Appendix_H.tex,v 1.17 2009-01-09 04:02:32 guru Exp $
%
%	The GMT Documentation Project
%	Copyright 2000-2009.
%	Paul Wessel and Walter H. F. Smith
%------------------------------------------
%
\chapter{Problems with display of \gmt\ \PS}
\label{app:H}
\thispagestyle{headings}

\GMT\ creates valid (so far as we know) Adobe \PS\
Level 1.  It does not use operators specific to Level 2 and
should therefore produce output that will print on old as well
as new \PS\ printers\footnote{Note, however, that the \Opt{Q} option in \GMTprog{grdimage}
will exercise a \PS\ Level 3 feature called colormasking.}.  Sometimes unexpected things
happen when \GMT\ output is sent to certain printers or displays.
This section lists some things we have learned from experience,
and some work-arounds.  Note that many of these lessons are now rather old so hopefully
these workarounds no longer apply to anybody...

\section{\PS\ driver bugs}
\index{PostScript@\PS!driver bugs}

When you try to display a \PS\ file on a device,
such as a printer or your screen, then a program called a
\PS\ device driver has to compute which device
pixels should receive which colors (black or white in the case
of a simple laser printer) in order to display the file.  At
this stage, certain device-dependent things may happen.  These
are not limitations of \GMT\ or \PS, but of the
particular display device.  The following bugs are known to us
based on our experiences:

\begin{enumerate}
\index{PostScript@\PS!Sun SPARCprinter bug}

\item 	Early versions of the Sun SPARCprinter software
caused linewidth-dependent path displacement.  We reported
this bug and it has been fixed in newer versions of the software.
Try using \GMTprog{psxy} to draw $y = f(x)$ twice, once with a
thin pen (\Opt{W}1) and once with a fat pen (\Opt{W}10);
if they do not plot on top of each other, you have this kind
of bug and need new software.  The problem may also show up
when you plot a mixture of solid and dashed (or dotted) lines
of various pen thickness

\index{PostScript@\PS!HP Laserjet 4M bug}

\item The first version of the HP Laserjet 4M (prior to Aug--93)
had bugs in the driver program.  The old one was
\PS\ SIMM, part number C2080-60001; the new one
is called \PS\ SIMM, part number C2080-60002.
You need to get this one plugged into your printer if you have
an HP LaserJet 4M.

\index{PostScript@\PS!limitations|(}

\item Apple Laserwriters with the older versions of Apple's
\PS\ driver will give the error ``limitcheck''
and fail to plot when they encounter a path exceeding about
1000--1500 points.  Try to get a newer driver from Apple, but
if you can't do that, set the parameter MAX\_L1\_PATH to
1000--1500 or even smaller in the file \filename{src/pslib\_inc.h}
and recompile \GMT.  The number of points in a \PS\
path can be arbitrarily large, in principle; \GMT\ will only
create paths longer than MAX\_L1\_PATH if the path represents
a filled polygon or clipping path.  Line-drawings (no fill)
will be split so that no segment exceeds MAX\_L1\_PATH.
This means \GMTprog{psxy} \Opt{G} will issue a warning when you
plot a polygon with more than MAX\_L1\_PATH points in it.  It is
then your responsibility to split the large polygon into several
smaller segments.  If \GMTprog{pscoast} gives such warnings and the
file fails to plot you may have to select one of the lower
resolution databases  The path limitation exemplified by these
Apple printers is what makes the higher-resolution coastlines
for \GMTprog{pscoast} non-trivial:  such coastlines have to be
organized so that fill operations do not generate excessively
large paths.  Some HP \PS\ cartridges for the
Laserjet III also have trouble with paths exceeding 1500
points; they may successfully print the file, but it can take
all night!
\index{PostScript@\PS!limitations|)}

\index{PostScript@\PS!Sun pageview}

\item 8-bit color screen displays (and programs which use only
8-bits, even on 24-bit monitors, such as Sun's \progname{pageview} under
OpenWindows) may not dither cleverly, and so the color they show you
may not resemble the color your \PS\ file is asking
for.  Therefore, if you choose colors you like on the screen,
you may be surprised to find that your plot looks different on
the hardcopy printer or film writer.  The only thing you can
do is be aware of this, and make some test cases on your hardcopy
devices and compare them with the screen, until you get used
to this effect.  (Each hardcopy device is also a little
different, and so you will eventually find that you want to
tune your color choices for each device.)  The rgb color cube
in example 11 may help.

\item Some versions of Sun's OpenWindows program \progname{pageview}
have only a limited number of colors available; the number
can be increased somewhat by starting \progname{openwin} with the
option ``\texttt{openwin -cubesize large}''.

\item Finally, \progname{pageview} seem to have problems understanding
the \texttt{setpagedevice} operator.  We recommend you only use
\progname{pageview} on EPS files or use \progname{ghostview} instead.

\index{PostScript@\PS!CMYK and RGB|(}

\item Many color hardcopy devices use CMYK color systems. \GMT\
\PS\ uses RGB (even if your cpt files are using HSV).
The three coordinates of RGB space can be mapped into three
coordinates in CMY space, and in theory K (black) is superfluous.
But it is hard to get CMY inks to mix into a good black or gray,
so these printers supply a black ink as well, hence CMYK.  The
\PS\ driver for a CMYK printer should be smart
enough to compute what portion of CMY can be drawn in K, and
use K for this and remove it from CMY; however, some of them
aren't.

\item In early releases of \GMT\ we always used the \PS\
command \texttt{r g b setrgbcolor} to specify colors, even if the color
happened to be a shade of gray ($r=g=b$) or black ($r=g=b=0$).  One
of our users found that black came out muddy brown when he used
\progname{FreedomOfPress} to make a Versatec plot of a \GMT\ map.
He found that if he used the \PS\ command \texttt{g setgray} (where $g$
is a graylevel) then the problem went away.
Apparently, his installation of \progname{FreedomOfPress} uses only CMY with
the command \texttt{setrgbcolor}, and so \texttt{0 0 0 setrgbcolor}
tries to make black out of CMY instead of K.  To fix this, in
release 2.1 of \GMT\ we changed some routines in \filename{pslib.c}
to check if ($r=g$ and $r=b$), in which case \texttt{g setgray} is
used instead of \texttt{r g b setrgbcolor}.

\item Recent experience with some Tektronix Phaser printers and
with commercial printing shops has shown that this substitution
creates problems precisely opposite of the problems our Versatec
user has.  The Tektronix and commercial (we think it was a Scitex)
machines do not use K when you say \texttt{0 setgray} but they do when
you say \texttt{0 0 0 setrgbcolor}.  We believe that these problems are
likely to disappear as the various software developers make their
codes more robust.  Note that this is not a fault with \GMT:
$r = g = b = 0$ means black and should plot that way.
Thus, the \GMT\ source code as shipped to you checks whether $r=g$
and $r=b$, in which case it uses \texttt{setgray}, else \texttt{setrgbcolor}.
If your gray tones are not being drawn with K, you have two
work-around options: (1) edit the source for \filename{pslib.c}
or (2) edit your \PS\ file and try using \texttt{setrgbcolor}
in all cases.  The simplest way to do so is to redefine the
\texttt{setgray} operator to use \texttt{setrgbcolor}.
Insert the line \\

\indent \texttt{/setgray  \{dup dup setrgbcolor\} def} \\

immediately following the first line in the file (starts with
\%!PS.) 

\item Some color film writers are very sensitive to the brand
of film.  If black doesn't look black on your color slides, try
a different film.
\index{PostScript@\PS!CMYK and RGB|)}

\end{enumerate}

\section{Resolution and dots per inch}
\index{PostScript@\PS!resolution and dpi}

The parameter {\bf DOTS\_PR\_INCH} can be set by the user through
the \filename{.gmtdefaults4} file or \GMTprog{gmtset}.  By default
it is equal to the value in the \filename{gmt\_defaults.h}
file, which is supplied with 300 when you get \GMT\ from us.
This seems a good size for most applications, but should ideally
reflect the resolution of your hardcopy device (most laserwriters
have at least 300 dpi, hence our default value).  \GMT\ computes what the
plot should look like in double precision floating point
coordinates, and then converts these to integer coordinates at
{\bf DOTS\_PR\_INCH} resolution.  This helps us find out that certain
points in a path lie on top of other points, and we can remove
these, making smaller paths.  Small paths are important for the
laserwriter bugs above, and also to make fill operations compute
faster.  Some users have set their {\bf DOTS\_PR\_INCH} to very large
numbers.  This only makes the \PS\ output bigger
without affecting the appearance of the plot.  However, if you
want to make a plot which fits on a page at first, and then
later magnify this same \PS\ file to a huge size,
the higher DPI is important.  Your data may not have the higher
resolution but on certain devices the edges of fonts will not
look crisp if they are not drawn with an effective resolution
of 300 dpi or so.  Beware of making an excessively large path.
Note that if you change dpi the linewidths produced by your
\Opt{W} options will change, unless you have appended {\bf p}
for linewidth in points.

\section{European characters}
\index{Text!European}
\index{Characters!European}
Note for users of \progname{pageview} in Sun OpenWindows: \GMT\ now
offers some octal escape sequences to load European alphabet
characters in text strings (see Section~\ref{sec:escape}).  When
this feature is enabled, the header on \GMT\ \PS\ output includes
a section defining special fonts.  The definition is added to
the header whether or not your plot actually uses the fonts.

Users who view their \GMT\ \PS\ output using
\progname{pageview} in OpenWindows on Sun computers or user older
laserwriters may have difficulties with the European font
definition.  If your installation of OpenWindows followed
a space-saving suggestion of Sun, you may have excluded the
European fonts, in which case \progname{pageview} will fail
to render your plot.

Ask your system administrator about this, or run this simple
test: (1) View a \GMT\ \PS\ file with \progname{pageview}.
If it comes up OK, you will be fine.  If it comes up blank,
open the ``Edit PostScript'' button and examine the lower
window for error messages.  (The European font problem generates
lots of error messages in this window).  (2)  Verify that the
\PS\ file is OK, by sending it to a laserwriter
and making sure it comes out.  (3)  If the \PS\
file is OK but it chokes \progname{pageview}, then edit the \PS\
file, cutting out everything between the lines: \\

\noindent
\%\%\%\%\% START OF EUROPEAN FONT DEFINITION \%\%\%\%\% \\
$<$bunch of definitions$>$ \\
\%\%\%\%\% END OF EUROPEAN FONT DEFINITION \%\%\%\%\% \\

Now try \progname{pageview} on the edited version.  If it now comes
up, you have a limited subset of OpenWindows installed.  If
you discover that these fonts cause you trouble, then you can
edit your \filename{.gmtdefaults4} file to set {\bf CHAR\_ENCODING} = Standard,
which will suppress the printing of this definition in the
\GMT\ \PS\ header.  You can
make output which will be viewable in \progname{pageview} without
any editing.  However, you would have to reset this to TRUE
before attempting to use European fonts, and then the output will
become un-\progname{pageview}-able again.  If you try to
concatenate segments of \GMT\ \PS\ made with and without the
European fonts enabled, then you may find that you have problems,
either with the definition, or because you ask for something
not defined.

\section{Hints}
\index{PostScript@\PS!\GMT\ hints}

When making images and perspective views of large amounts of
data, the \GMT\ programs can take some time to run, the resulting
\PS\ files can be very large, and the time to display
the plot can be long.  Fine tuning a plot script can take lots
of trial and error.  We recommend using \GMTprog{grdsample} to make
a low resolution version of the data files you are plotting, and
practice with that, so it is faster; when the script is perfect,
use the full-resolution data files.  We often begin building a
script using only \GMTprog{psbasemap} or \GMTprog{pscoast} to get
the various plots oriented correctly on the page; once this works
we replace the \GMTprog{psbasemap} calls with the actually desired
\GMT\ programs.

If you want to make color shaded relief images and you haven't
had much experience with it, here is a good first cut at the
problem:  Set your {\bf COLOR\_MODEL} to HSV using \GMTprog{gmtset}.  Use
\GMTprog{makecpt} or \GMTprog{grd2cpt} to make a continuous color
palette spanning the range of your data.  Use the \Opt{Nt}
option on \GMTprog{grdgradient}.  Try the result, and then play with
the tuning of the \filename{.gmtdefaults4}, the cpt file, and
the gradient file.
