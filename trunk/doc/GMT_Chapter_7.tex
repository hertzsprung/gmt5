%------------------------------------------
%	$Id: GMT_Chapter_7.tex,v 1.5 2002-02-26 22:26:14 pwessel Exp $
%
%	The GMT Documentation Project
%	Copyright 2000-2002.
%	Paul Wessel and Walter H. F. Smith
%------------------------------------------
%
\chapter{Cook-book}
\thispagestyle{headings}

In this section we will be giving several examples of
typical usage of \GMT\ programs.  In general, we will
start with a raw data set, manipulate the numbers in
various ways, then display the results in diagram or
map view.  The resulting plots will have in common that
they are all made up of simpler plots that have been
overlaid to create a complex illustration.  We will
mostly follow the following format:

\begin{enumerate}
\item We explain what we want to achieve in plain
language.

\item We present a cshell script that contains
all commands used to generate the illustration.

\item We explain the rationale behind the commands.

\item We present the illustration, 50\% reduced in size, and without
the timestamp (\Opt{U}).
\end{enumerate}

A detailed discussion of each command is not given;
we refer you to the manual pages for command line
syntax, etc.  We encourage you to run these scripts for yourself.
See Appendix D if you would like an electronic version
of all the shell-scripts (both \progname{csh} and \progname{bash} scripts
are available; only the \progname{csh}-scripts are discussed here) and support
data used below.  Note that all examples explicitly specifies the
measurement units, so although we use inches you should be able
to run these scripts and get the same plots even if you have cm
as the default measure unit.  The examples are all written to be ``quiet'',
that is no information is echoed to the screen.  Thus,
these scripts are well suited for background execution.
Note that we also end each script by cleaning up after
ourselves. Because \progname{awk} is broken as designed on some
systems, and \progname{nawk} is not available on others we refer
to \progname{\$AWK} in the scripts below; the \progname{do\_examples}
scripts will set this when running all examples. 

\section{The making of contour maps}

\index{Example!contour maps|(}

We want to create two contour maps of the low order geoid
using the Hammer equal area projection.  Our gridded data
file is called \filename{osu91a1f\_16.grd} and contains a global 1\DS\ 
by 1\DS\ gridded geoid (we will see how to make gridded
files later).  We would like to show one map centered on
Greenwich and one centered on the dateline.  Positive contours
should be drawn with a solid pen and negative contours with
a dashed pen.  Annotations should occur for every 50 m contour
level, and both contour maps should show the continents in
light gray in the background.  Finally, we want a rectangular
frame surrounding the two maps.  This is how it is done:

\input{scripts/GMT_example_1} 

The first command draws a box surrounding the maps.  This is
followed by two sequences of
\GMTprog{pscoast}, \GMTprogi{grdcontour}, \GMTprogi{grdcontour}.
They differ in that the first is centered on Greenwich; the
second on the dateline.  We use the limit option (\Opt{L})
in \GMTprog{grdcontour} to select negative contours only and plot
those with a dashed pen, then positive contours only and draw
with a solid pen [Default].  The \Opt{T} option causes tickmarks
pointing in the downhill direction to be drawn on the innermost,
closed contours.  For the upper panel we also added - and + to
the local lows and highs.  You can find this illustration as
Figure~\ref{fig:GMT_example_01}.

\GMTfig[ht]{GMT_example_01}{Contour maps of gridded data}

\index{Example!contour maps|)}

\section{Image presentations}
\index{Example!image presentations|(}

As our second example we will demonstrate how to make color
images from gridded data sets (again, we will deferr the
actual making of gridded files to later examples).  We will
use the supplemental program \GMTprog{grdraster} to extract 2-D
grdfiles of bathymetry and Geosat geoid heights and put the
two images on the same page.  The region of interest is the
Hawaiian islands, and due to the oblique trend of the island
chain we prefer to rotate our geographical data sets using
an oblique Mercator projection defined by the hotspot pole
at (68\DS W, 69\DS N).  We choose the point (190\DS ,
25.5\DS ) to be the center of our projection (e.g., the
local origin), and we want to image a rectangular region
defined by the longitudes and latitudes of the lower left
and upper right corner of region.  In our case we choose
(160\DS , 20\DS ) and (220\DS , 30\DS ) as the
corners.  We use \GMTprog{grdimage} to make the illustration:

\input{scripts/GMT_example_2} 

The first step extracts the 2-D data sets from the local
data base using \GMTprog{grdraster}, which is a supplemental
utility program (see Appendix A) that may be adapted to
reflect the nature of your data base format.  It
automatically figures out the required extent of the region
given the two corners points and the projection.  The extreme
meridians and parallels enclosing the oblique region is
\Opt{R}159:50/220:10/3:10/47:35.  This is
the area extracted by \GMTprog{grdraster}.  For your convenience
we have commented out those lines and provided the two
extracted files so you do not need \GMTprog{grdraster} to try
this example.  By using the embedded grdfile format
mechanism we saved the topography using kilometers as the
data unit.  We now have two grdfiles with bathymetry and
geoid heights, respectively.  We use \GMTprog{makecpt} to generate
a linear color palette file \filename{geoid.cpt} for the geoid and use
\GMTprog{grd2cpt} to get a histogram-equalized cpt file \filename{topo.cpt}
for the topography data.  To emphasize the structures in
the data we calculate the slopes in the north-south direction
using \GMTprog{grdgradient}; these will be used to modulate the
color image.  Next we run \GMTprog{grdimage} to create a
color-code image of the Geosat geoid heights, and draw a
color scale to the right of the image with \GMTprog{psscale}.
We also annotate the color scales with \GMTprog{psscale}.
Similarly, we run \GMTprog{grdimage} but specify \Opt{Y}4.5
to plot above the previous image.  Adding scale and label
the two plots a) and b) completes the illustration
(Figure~\ref{fig:GMT_example_02}).
\index{Example!image presentations|)}
\GMTfig[ht]{GMT_example_02}{Color images from gridded data}

\section{Spectral estimation and xy-plots}
\index{Example!Spectral estimation|(}
\index{Example!xy plots|(}

In this example we will show how to use the \GMT\ programs
\GMTprog{fitcircle}, \GMTprog{project}, \GMTprog{sample1d},
\GMTprog{spectrum1d}, \GMTprog{psxy}, and \GMTprog{pstext}.
Suppose you have (lon, lat, gravity) along a satellite track
in a file called \filename{sat.xyg}, and (lon, lat, gravity)
along a ship track in a file called \filename{ship.xyg}.
You want to make a cross-spectral analysis of these data.
First, you will have to get the two data sets into equidistantly
sampled time-series form.  To do this, it will be convenient to
project these along the great circle that best fits the sat track.
We must use \GMTprog{fitcircle} to find this great circle and choose
the L$_2$  estimates of best pole.  We project the data using
\GMTprog{project} to find out what their ranges are in the projected
coordinate.  The \GMTprog{minmax} utility will report the minimum and
maximum values for multi-column ASCII tables.  Use this information
to select the range of the projected distance coordinate they have
in common.  The script prompts you for that information after
reporting the values.  We decide to make a file of equidistant
sampling points spaced 1 km apart from -1167 to +1169, and use
the \UNIX\ utility \progname{\$AWK} to accomplish this step.  We can then
resample the projected data, and carry out the cross-spectral
calculations, assuming that the ship is the input and the satellite
is the output data.  There are several intermediate steps that
produce helpful plots showing the effect of the various processing
steps (\filename{example\_3[a--f].ps}), while the final plot
\filename{example\_03.ps} shows
the ship and sat power in one diagram and the coherency on another
diagram, both on the same page.  Note the extended use of
\GMTprog{pstext} and \GMTprog{psxy} to put labels and legends directly on
the plots.  For that purpose we often use \Opt{Jx}1i and specify
positions in inches directly.  Thus, the complete automated
script reads:

\input{scripts/GMT_example_3} 

The final illustration (Figure~\ref{fig:GMT_example_03}) shows that the ship gravity
anomalies have more power than altimetry derived gravity for
short wavelengths and that the coherency between the two
signals improves dramatically for wavelengths $>$ 20 km.
\GMTfig[ht]{GMT_example_03}{Spectral estimation and $x/y$-plots}

\index{Example!Spectral estimation|)}
\index{Example!xy plots|)}

\section{A 3-D perspective mesh plot}
\index{Example!3-D mesh plot|(}

This example will illustrate how to make a fairly complicated
composite figure.  We need a subset of the ETOPO5 bathymetry�\footnote{
These data are available on CD-ROM from NGDC (www.ngdc.noaa.gov).}
and Geosat geoid data sets which we will extract from the local
data bases using \GMTprog{grdraster}.  We would like to show a
2-layer perspective plot where layer one shows a contour map
of the marine geoid with the location of the Hawaiian islands
superposed, and a second layer showing the 3-D mesh plot of
the topography.  We also add an arrow pointing north and some
text.  This is how to do it:

\input{scripts/GMT_example_4} 

The purpose of the color palette file \filename{zero.cpt} is to have
the positive topography mesh painted light gray (the remainder
is white).  Figure~\ref{fig:GMT_example_04} shows the complete illustration.
\GMTfig[ht]{GMT_example_04}{3-D perspective mesh plot}

A color version of this figure was used in our first article in EOS
Trans. AGU  (Oct. 8th, 1991).  It was created along similar
lines, but instead of a mesh plot we chose a color-coded surface
with artificial illumination from a light-source due north.
We choose to use the \Opt{Qi} option in \GMTprog{grdview} to
achieve a high degree of smoothness.  Here, we select 100 dpi
since that will be the resolution of our final raster
(The EOS raster was 300 dpi). We used \GMTprog{grdgradient} to
provide the intensity files.  The following script creates
the color \PS\ file.  Note that the size of the
resulting output file is directly dependent on the square of
the dpi chosen for the scanline conversion.  A higher value
for dpi in \Opt{Qi} would have resulted in a much larger
output file.  The cpt files were taken from Example 2.

\input{scripts/GMT_example_4c} 

\index{Example!3-D mesh plot|)}

\section{A 3-D illuminated surface in black and white}
\index{Example!3-D illuminated surface|(}

Instead of a mesh plot we may choose to show 3-D surfaces using
artificial illumination.  For this example we will use
\GMTprog{grdmath} to make a grdfile that contains the surface given
by the function $z(x, y) = \cos (2\pi r/8)\cdot e^{-r/10}$, where
$r^2 = (x^2 + y^2)$.  The illumination is obtained by passing
two grdfiles to \GMTprog{grdview}: One with the {\it z}-values
(the surface) and another with intensity values (which should
be in the $\pm$1 range).  We use \GMTprog{grdgradient} to compute
the horizontal gradients in the direction of the artificial
light source.  The \filename{gray.cpt} file only has one line that states
that all {\it z} values should have the gray level 128.  Thus,
variations in shade are entirely due to variations in gradients,
or illuminations.  We choose to illuminate from the SW and view
the surface from SE:

\input{scripts/GMT_example_5} 

The variations in intensity could be made more dramatic by
using \GMTprog{grdmath} to scale the intensity file before
running \GMTprog{grdview}.  For very rough data sets one may
improve the smoothness of the intensities by passing the
output of \GMTprog{grdgradient} to \GMTprog{grdhisteq}.  The
shell-script above will result in a plot like the one in
Figure~\ref{fig:GMT_example_05}.
\GMTfig[ht]{GMT_example_05}{3-D illuminated surface}

\index{Example!3-D illuminated surface|)}

\section{Plotting of histograms}
\index{Example!histograms|(}

\GMT\ provides two tools to render histograms: \GMTprog{pshistogram}
and \GMTprog{psrose}.  The former takes care of regular histograms
whereas the latter deals with polar histograms (rose diagrams,
sector diagrams, and windrose diagrams).  We will show an
example that involves both programs.  The file \filename{fractures.yx}
contains a compilation of fracture lengths and directions as
digitized from geological maps.  The file \filename{v3206.t} contains
all the bathymetry measurements from {\it Vema} cruise 3206.
Our complete figure (Figure~\ref{fig:GMT_example_06}) was made running
this script:

\input{scripts/GMT_example_6} 
\GMTfig[ht]{GMT_example_06}{Two kinds of histograms}

\index{Example!histograms|)}

\section{A simple location map}
\index{Example!location map|(}

Many scientific papers start out by showing a location map
of the region of interest. This map will typically also
contain certain features and labels.  This example will
present a location map for the equatorial Atlantic ocean,
where fracture zones and mid-ocean ridge segments have been
plotted.  We also would like to plot earthquake locations
and available isochrons.  We have obtained one file,
\filename{quakes.xym}, which contains the position and magnitude of
available earthquakes in the region.  We choose to use
magnitude/100 for the symbol-size in inches.  The digital
fracture zone traces (\filename{fz.xy}) and isochrons (0 isochron as
\filename{ridge.xy}, the rest as \filename{isochrons.xy}) were digitized from
available maps�\footnote{These data are available on CD-ROM from NGDC (www.ngdc.noaa.gov).}.  We create the final location map
(Figure~\ref{fig:GMT_example_07}) with the following script:

\input{scripts/GMT_example_7} 

The same figure could equally well be made in color, which
could be rasterized and made into a slide for a meeting
presentation.  The script is similar to the one outlined
above, except we would choose a color for land and oceans,
and select colored symbols and pens rather than black and white.
\GMTfig[ht]{GMT_example_07}{A typical location map}

\index{Example!location map|)}

\section{A 3-D histogram}
\index{Example!3-D histogram|(}

The program \GMTprog{psxyz} allows us to plot three-dimensional
symbols, including columnar plots.  As a simple demonstration,
we will convert a gridded netCDF of bathymetry into an
ASCII $xyz$ table and use the height information to draw a
2-D histogram in a 3-D perspective view.  Our gridded
bathymetry file is called \filename{topo.grd} and covers the region
from 0 to 5 \DS E and 0 to 5 \DS N.  Depth ranges from
-5000 meter to sea-level.  We produce the illustration by
running this command:

\input{scripts/GMT_example_8} 

The output can be viewed in Figure~\ref{fig:GMT_example_08}.
\GMTfig[ht]{GMT_example_08}{A 3-D histogram}

\index{Example!3-D histogram|)}

\section{Plotting time-series along tracks}
\index{Example!wiggles|(}

A common application in many scientific disciplines involves
plotting one or several time-series as as ``wiggles'' along
tracks.  Marine geophysicists often display magnetic anomalies
in this manner, and seismologists use the technique when
plotting individual seismic traces.  In our example we will
show how a set of Geosat sea surface slope profiles from the
south Pacific can be plotted as ``wiggles'' using the
\GMTprog{pswiggle} program. We will embellish the plot with track
numbers, the location of the Pacific-Antarctic Ridge, recognized
fracture zones in the area, and a ``wiggle'' scale.  The
Geosat tracks are stored in the files \filename{*.xys}, the ridge in
\filename{ridge.xy}, and all the fracture zones are stored in the multiple
segment file \filename{fz.xy}.  We extract the profile id (which is the
first part of the file name for each profile) and the last point
in each of the track files to construct an input file for
\GMTprog{pstext} that will label each profile with the track
number.  We know the profiles trend approximately N40\DS E
so we want the labels to have that same orientation (i.e., the
angle with the baseline must be 50\DS ).  We do this by
extracting the last record from each track, paste this file
with the \filename{tracks.lis} file, and use \progname{\$AWK} to create the
format needed for \GMTprog{pstext}.  Note we offset the positions
by -0.05 inch with \Opt{D} in order to have a small gap
between the profile and the label:

\input{scripts/GMT_example_9} 

The output shows the sea-surface slopes along 42 descending
Geosat tracks in the Eltanin and Udintsev fracture zone
region in a Mercator projection (Figure~\ref{fig:GMT_example_09}).
\GMTfig[ht]{GMT_example_09}{Time-series as ``wiggles'' along a track}

\index{Example!wiggles|)}

\section{A geographical bar graph plot}
\index{Example!bar graph|(}

Our next and perhaps silliest example presents a
three-dimensional bargraph plot showing the geographic
distribution of the membership in the American Geophysical
Union (AGU).  The input data was taken from the 1991 AGU
member directory and added up to give total members per
continent.  We decide to plot a 3-D column centered on
each continent with a height that is proportional to the
logarithm of the membership.  A log$_{10}$-scale is
used since the memberships vary by almost 3 orders of
magnitude.  We choose a plain linear projection for the
basemap and add the columns and text on top. Our script reads:

\input{scripts/GMT_example_10}

with the result presented in Figure~\ref{fig:GMT_example_10}.

\GMTfig[ht]{GMT_example_10}{Geographical bar graph}

\index{Example!bar graph|)}

\section{Making a 3-D RGB color cube}
\index{Example!3-D RGB color cube|(}

In this example we generate a series of 6 color images,
arranged in the shape of a cross, that can be cut out
and assembled into a 3-D color cube.  The six faces of
the cube represent the outside of the R-G-B color space.
On each face one of the color components is fixed at either
0 or 255 and the other two components vary smoothly across
the face from 0 to 255.  The cube is configured as a
right-handed coordinate system with {\it x-y-z} mapping
R-G-B.  Hence, the 8 corners of the cube represent the
primaries red, green, and blue, plus the secondaries cyan,
magenta and yellow, plus black and white.

The method for generating the 6 color faces utilizes
\progname{\$AWK} in two steps.  First, a {\it z}-grid is composed
which is 256 by 256 with {\it z}-values increasing in a planar
fashion from 0 to 65535.  This {\it z}-grid is common to all
six faces.  The color variations are generated by creating a
different color palette for each face using the supplied
\progname{\$AWK} script \filename{rgb\_cube.awk}.  This script generates a
``cpt'' file appropriate for each face using arguments for
each of the three color components.  The arguments specify
if that component ($r,g,b$) is to be held fixed at 0 or 255,
is to vary in {\it x}, or is to vary in {\it y}.  If the
color is to increase in {\it x} or {\it y}, a lower case
{\it x} or {\it y} is specified; if the color is to decrease
in {\it x} or {\it y}, an upper case {\it X} or {\it Y} is
used. Here is the shell script and accompanying \progname{\$AWK}
script to generate the RGB cube:

\input{scripts/GMT_example_11} 

The \progname{\$AWK} script \filename{rgb\_cube.awk} is as follows:

\input{scripts/rgb_cube_awk}

The cube can be viewed in Figure~\ref{fig:GMT_example_11}

\GMTfig[ht]{GMT_example_11}{The color cube}

\index{Example!3-D RGB color cube|)}

\section{Optimal triangulation of data}
\index{Example!triangulation|(}

Our next example (Figure~\ref{fig:GMT_example_12})
operates on a data set of topographic
readings non-uniformly distributed in the plane (Table
5.11 in Davis: {\it Statistics and Data Analysis in Geology},
J. Wiley).  We use \GMTprog{triangulate} to perform the optimal
Delaunay triangulation, then use the output to draw the
resulting network.  We label the node numbers as well as
the node values, and call \GMTprog{pscontour} to make a contour
map and image directly from the raw data.  Thus, in this
example we do not actually make gridded files but still
are able to contour and image the data.  We use a color
palette table \filename{topo.cpt} (supplied with the script data
separately).  The script becomes:

\input{scripts/GMT_example_12} 
\GMTfig[ht]{GMT_example_12}{Optimal triangulation of data}

\index{Example!triangulation|)}

\section{Plotting of vector fields}
\index{Example!vector fields|(}

In many areas, such as fluid dynamics and elasticity,
it is desirable to plot vector fields of various kinds.
\GMT\ provides a way to illustrate 2-component vector fields
using the \GMTprog{grdvector} utility.  The two components of
the field (Cartesian or polar components) are stored in
separate grdfiles.  In this example we use \GMTprog{grdmath}
to generate a surface $z(x, y) = x \cdot \exp(-x^2 -y^2)$
and to calculate $\nabla z$ by
returning the {\it x}- and {\it y}-derivatives separately.
We superpose the gradient vector field and the surface
{\it z} and also plot the components of the gradient
in separate windows:

\input{scripts/GMT_example_13} 

A \GMTprog{pstext} call to place a header finishes the plot
(Figure~\ref{fig:GMT_example_13}.

\GMTfig[ht]{GMT_example_13}{Display of vector fields in \gmt}

\index{Example!vector fields|)}

\section{Gridding of data and trend surfaces}
\index{Example!gridding and trend surfaces|(}

This example shows how one goes from randomly spaced data
points to an evenly sampled surface.  First we plot the
distribution and values of our raw data set (table 5.11
from example 12).  We choose an equidistant grid and run
\GMTprog{blockmean} which preprocesses the data to avoid aliasing.
The dashed lines indicate the logical blocks used by
\GMTprog{blockmean}; all points inside a given bin will be averaged.
The logical blocks are drawn from a temporary file we make on
the fly within the shell script.  The processed data is then
gridded with the \GMTprog{surface} program and contoured every 25
units.  A most important point here is that \GMTprog{blockmean},
\GMTprog{blockmedian}, or \GMTprog{blockmode} should always be run
prior to running \GMTprog{surface}, and both of these steps must use the same
grid interval.  We use \GMTprog{grdtrend} to fit a bicubic trend
surface to the gridded data, contour it as well, and sample
both gridded files along a diagonal transect using \GMTprog{grdtrack}.
The bottom panel compares the gridded (solid line) and bicubic
trend (dashed line) along the transect using \GMTprog{psxy}
(Figure~\ref{fig:GMT_example_14}):

\input{scripts/GMT_example_14} 
\GMTfig[ht]{GMT_example_14}{Gridding of data and trend surfaces}

\index{Example!gridding and trend surfaces|)}

\section{Gridding, contouring, and masking of unconstrained areas}
\index{Example!gridding, contouring, and masking|(}

This example (Figure~\ref{fig:GMT_example_15}) demonstrates
some off the different ways one
can use to grid data in \GMT, and how to deal with unconstrained
areas.  We first convert a large ASCII file to binary with
\GMTprog{gmtconvert} since the binary file will read and process
much faster.  Our lower left plot illustrates the results of
gridding using a nearest neighbor technique (\GMTprog{nearneighbor})
which is a local method: No output is given where there are no data.
Next (lower right), we use a minimum curvature technique
(\GMTprog{surface}) which is a global method.  Hence, the contours
cover the entire map allthough the data are only available for
portions of the area (indicated by the gray areas plotted using
\GMTprog{psmask}).  The top left scenario illustrates how we can
create a clip path (using \GMTprog{psmask}) based on the data coverage
to eliminate contours outside the constrained area.
Finally (top right) we simply employ \GMTprog{pscoast} to overlay
gray landmasses to cover up the unwanted contours, and end by
plotting a star at the deepest point on the map with \GMTprog{psxy}.
This point was extracted from the gridded files using \GMTprog{grdinfo}.

\input{scripts/GMT_example_15} 
\GMTfig[ht]{GMT_example_15}{Gridding, contouring, and masking of data}

\index{Example!gridding, contouring, and masking|)}

\section{Gridding of data, continued}
\index{Example!gridding|(}

\GMTprog{pscontour} (for contouring) and \GMTprog{triangulate}
(for gridding) use the simplest method of interpolating
data:  a Delaunay triangulation (see Example 12) which
forms $z(x, y)$ as a union of planar triangular facets.
One advantage of this method is that it will not extrapolate
$z(x, y)$ beyond the convex hull of the input ({\it x, y})
data.  Another is that it will not estimate a {\it z} value
above or below the local bounds on any triangle.
A disadvantage is that the $z(x, y)$ surface is not
differentiable, but has sharp kinks at triangle edges and
thus also along contours.  This may not look physically
reasonable, but it can be filtered later (last panel below).
\GMTprog{surface} can be used to generate a higher-order
(smooth and differentiable) interpolation of $z(x, y)$ onto
a grid, after which the grid may be illustrated (\GMTprog{grdcontour},
\GMTprog{grdimage}, \GMTprog{grdview}).  \GMTprog{surface} will interpolate
to all ({\it x, y}) points in a rectangular region, and thus
will extrapolate beyond the convex hull of the data.  However,
this can be masked out in various ways (see Example 15).

A more serious objection is that \GMTprog{surface} may estimate
{\it z} values outside the local range of the data (note area
near {\it x} = 0.8, {\it y} = 5.3).  This commonly happens when
the default tension value of zero is used to create a ``minimum
curvature'' (most smooth) interpolant.  \GMTprog{surface} can be
used with non-zero tension to partially  overcome this problem.
The limiting value $tension = 1$ should approximate the triangulation,
while a value between 0 and 1 may yield a good compromise between
the above two cases.  A value of 0.5 is shown here
(Figure~\ref{fig:GMT_example_16}).  A side
effect of the tension is that it tends to make the contours turn
near the edges of the domain so that they approach the edge from
a perpendicular direction.  A solution is to use \GMTprog{surface}
in a larger area and then use \GMTprog{grdcut} to cut out the desired
smaller area.  Another way to achieve a compromise is to
interpolate the data to a grid and then filter the grid using
\GMTprog{grdfft} or \GMTprog{grdfilter}.  The latter can handle grids
containing ``NaN'' values and it can do  median and mode filters
as well as convolutions.  Shown here is \GMTprog{triangulate} followed
by \GMTprog{grdfilter}.  Note that the filter has done some
extrapolation beyond the convex hull of the original {\it x, y}
values.  The ``best'' smooth approximation of $z(x, y)$ depends
on the errors in the data and the physical laws obeyed by {\it z}.
\GMT\ cannot always do the ``best'' thing but it offers great
flexibility through its combinations of tools.  We illustrate all
four solutions using a cpt file that contains color fills,
patterns, and a ``skip slice'' request for $700 < z < 725$.

\input{scripts/GMT_example_16} 
\GMTfig[ht]{GMT_example_16}{More ways to grid data}

\index{Example!gridding|)}

\section{Images clipped by coastlines}
\index{Example!image clipping|(}

This example demonstrates how \GMTprog{pscoast} can be used
to set up clippaths based on coastlines.  This approach
is well suited when different gridded data sets are to be
merged on a plot using different color palette files.
Merging the files themselves may not be doable since they
may represent different data sets, as we show in this example.
Here, we lay down a color map of the geoid field near India
with \GMTprog{grdimage}, use \GMTprog{pscoast} to set up land
clippaths, and then overlay topography from the ETOPO5 data
set with another call to \GMTprog{grdimage}.  We finally undo
the clippath with a second call to \GMTprog{pscoast} with the
option \Opt{Q} (Figure~\ref{fig:GMT_example_17}): 

\input{scripts/GMT_example_17} 
\GMTfig[ht]{GMT_example_17}{Clipping of images using coastlines}

\index{Example!image clipping|)}

\section{Volumes and Spatial Selections}
\index{Example!spatial selections|(}

To demonstrate potential usage of the new programs
\GMTprog{grdvolume} and \GMTprog{gmtselect} we extract a subset
of the Sandwell \& Smith altimetric gravity field\footnote{
See http://topex.ucsd.edu/marine\_grav/mar\_grav.html.}�
for the northern Pacific and decide to isolate all seamounts that
(1) exceed 50 mGal in amplitude and (2) are within 200 km
of the Pratt seamount.  We do this by dumping the 50 mGal
contours to disk, then making a simple \progname{\$AWK} script
\filename{center.awk} that returns the mean location of the points
making up each closed polygon, and then pass these locations
to \GMTprog{gmtselect} which retains only the points within 200
km of Pratt.  We then mask out all the data outside this
radius and use \GMTprog{grdvolume} to determine the combined
area and volumes of the chosen seamounts.

\input{scripts/GMT_example_18}

Our illustration is presented in Figure~\ref{fig:GMT_example_18}.

\GMTfig[ht]{GMT_example_18}{Volumes and geo-spatial selections}

\index{Example!spatial selections|)}

\section{Color patterns on maps}
\index{Example!color patterns|(}

\GMT\ 3.1 introduced color patterns and this examples give
a few cases of how to use this new feature.  We make a phony
poster that advertises an international conference on \GMT\
in Honolulu.  We use \GMTprog{grdmath}, \GMTprog{makecpt}, and
\GMTprog{grdimage} to draw pleasing color backgrounds on maps,
and overlay \GMTprog{pscoast} clippaths to have the patterns
change at the coastlines.  The middle panel demonstrates a
simple \GMTprog{pscoast} call where the built-in pattern \# 86
is drawn at 100 dpi but with the black and white pixels
replaced with color combinations.  The final panel repeats
the top panel except that the land and sea images have changed places
(Figure~\ref{fig:GMT_example_19}).

\input{scripts/GMT_example_19} 
\GMTfig[ht]{GMT_example_19}{Using color patterns in illustrations}

\index{Example!color patterns|)}

\section{Custom map symbols}
\index{Example!custom map symbols|(}

One is often required to make special maps that shows the
distribution of certain features but one would prefer to
use a custom symbol instead of the built-in circles,
squares, triangles, etc. in the \GMT\ plotting programs
\GMTprog{psxy} and \GMTprog{psxyz}.  Here we demonstrate one
approach that allows for a fair bit of flexibility in
designing ones own symbols.  The following recipe is used
when designing a new symbol. (1) Use \GMTprog{psbasemap} (or
engineering paper!) to set up an empty grid that goes from
-0.5 to +0.5 in both {\it x} and {\it y}.  Use ruler and
compass to draw your new symbol using straight lines,
arcs of circles, and stand-alone geometrical objects (see \GMTprog{psxy} man page for
a full deccription of symbol design).  This is how your symbol will look when
a size of 1 inch is chosen.  Figure~\ref{fig:GMT_volcano} illustrates a
new symbol we will call volcano.

\GMTfig[ht]{GMT_volcano}{Making a new volcano symbol for \gmt}

(2) After designing the symbol we will encode it using a
simple set of rules.  In our case we describe our volcano
using these 3 freeform polygon generators:

\begin{tabbing} 

$x_0$ $y_0$ $r$ {\bf C} [ \Opt{G}{\it fill} ] [ \Opt{W}{\it pen} ] \=Draw \kill

$x_0$ $y_0$ {\bf M} [ \Opt{G}{\it fill} ] [ \Opt{W}{\it pen} ] \> Start
new element at $x_0$, $y_0$ \\ 

$x_1$ $y_1$ {\bf D} \> Draw straight line from current point to $x_1$, $y_1$ \\ 

$x_0$ $y_0$ $r$ $\alpha_1$ $\alpha_2$ {\bf A} \> Draw
arc segment of radius $r$ from angle $\alpha_1$ to $\alpha_2$

\end{tabbing} 

We also add a few stand-alone circles (for other symbols, see \GMTprog{psxy} man page):

\begin{tabbing} 
$x_0$ $y_0$ $r$ {\bf C} [ \Opt{G}{\it fill} ] [ \Opt{W}{\it pen} ] \=Draw \kill
$x_0$ $y_0$ $r$ {\bf C} [ \Opt{G}{\it fill} ] [ \Opt{W}{\it pen} ] \> Draw
single circle of radius $r$ around $x_0$, $y_0$ \\
\end{tabbing} 

The optional \Opt{G} and \Opt{W} can be used to hardwire
the color fill and pen for segments.  By default the segments
are painted based on the values of the command line settings.

Manually applying these rules to our symbol results in a
definition file \filename{volcano.def}:

\input{scripts/volcano_def} 

The values refer to positions and dimensions illustrated
in Figure~\ref{fig:GMT_volcano} above.  (3) Given a proper
definition file we may now use it with \GMTprog{psxy} or \GMTprog{psxyz}.

We are now ready to give it a try.  Based on the hotspot
locations in the file \filename{hotspots.d} (with a 3rd column
giving the desired symbol sizes in inches) we lay down a
world map and overlay red volcano symbols using our custom-built
volcano symbol and \GMTprog{psxy}.
Without further discussion we also make a definition for a multi-
colored bulls-eye symbol:

\input{scripts/bullseye_def} 

Here is our final map script:

\input{scripts/GMT_example_20}

which produces the plot in Figure~\ref{fig:GMT_example_20}.

\GMTfig[ht]{GMT_example_20}{Using custom symbols in \gmt}

Given these guidelines you can easily make your own symbols.
Symbols with more than one color can be obtained by making
several symbol components.  E.g., to have yellow smoke coming
out of red volcanoes we would make two symbols: one with just
the cone and caldera and the other with the bubbles.  These
would be plotted consecutively using the desired colors.
Alternatively, like in \filename{bullseye.def}, we may
specify colors directly for the various segments.  Note that
the custom symbols, unlike the built-in symbols in \GMT, can
be used with the built-in patterns (Appendix E).  Other
approaches are also possible, of course.
\index{Example!custom map symbols|)}
