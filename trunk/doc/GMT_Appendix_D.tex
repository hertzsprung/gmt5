%------------------------------------------
%	$Id: GMT_Appendix_D.tex,v 1.14 2006-02-28 01:32:04 pwessel Exp $
%
%	The GMT Documentation Project
%	Copyright 2000-2006.
%	Paul Wessel and Walter H. F. Smith
%------------------------------------------
%
\chapter{Availability of \gmt\ and related code}
\label{app:D}
\index{GMT@\GMT!obtaining}
\thispagestyle{headings}

All the source code, support data, \PS\, PDF,
and HTML versions of all documentation (including \UNIX\
manual pages) can be obtained by anonymous
ftp from several mirror sites.  We also maintain a \GMT\
page on the World Wide Web (http://gmt.soest.hawaii.edu);
See this page for installation directions 
which allow for a simplified, automatic install procedure
(for a CD-R solution, see \htmladdnormallink{http://www.geoware-online.com}{http://www.geoware-online.com}.)

The \GMT\ tar archives are available both in \progname{gzip}
and \progname{bzip2} format.  If neither of these utilities
are installed on your system, you should know that the former
program is available from GNU\footnote{www.gnu.org} while the
latter can be obtained from RedHat\footnote{http://sources.redhat.com/bzip2/index.html}.
\progname{bzip2} compresses much better than \progname{gzip}:
for example, the full resolution coastline database is only
$\sim$29 Mb in \progname{bzip2} format compared to a hefty
$\sim$44 Mb in \progname{gzip}.  These files have the .bz2 suffix. \\

The GMT archives are as follows:

\begin{description}

\item[GMT\_progs.tar.\{gz,bz2\}] Contains all \GMT\ source code,
cpt files, and \PS\ patterns.

\item[GMT\_share.tar.\{gz,bz2\}] Contains the intermediate,
low, and crude resolutions of the coastline database.  Required
with GMT\_progs.tar for minimal setup needed to run \GMT.

\item[GMT\_man.tar.\{gz,bz2\}] Contains all the \UNIX\ manual pages.

\item[GMT\_ps.tar.\{gz,bz2\}] Contains PostScript versions all \GMT\ documentation
(man pages, Cookbook and Technical Reference, and the tutorial).

\item[GMT\_pdf.tar.\{gz,bz2\}] Contains PDF versions all \GMT\ documentation
(man pages, Cookbook and Technical Reference, and the tutorial).

\item[GMT\_web.tar.\{gz,bz2\}] Contains HTML versions all \GMT\ documentation
(man pages, Cookbook and Technical Reference, and the tutorial).

\item[GMT\_tut.tar.\{gz,bz2\}] Contains the data files used in the tutorial.

\item[GMT\_full.tar.\{gz,bz2\}] Contains the optional
full-resolution coastline database.

\item[GMT\_high.tar.\{gz,bz2\}] Contains the optional
high-resolution coastline database.

\item[GMT\_scripts.tar.\{gz,bz2\}] Contains all the shell scripts
and support data used in the Cookbook section.

\item[GMT\_suppl.tar.\{gz,bz2\}] Contains several programs
written by us and \GMT\ users elsewhere. (See Appendix~\ref{app:A} for more
details).

\end{description}

\index{GMT@\GMT!binaries for Win32}

All of the above archives are also available as Windows ZIP archives, e.g., {\bf GMT\_progs.zip}.
For Windows users who do not want to compile themselves, there are two zip files with Win32 executables:

\begin{description}

\item[GMT\_exe.zip] ZIP archive with all main \GMT\ executables.

\item[GMT\_suppl\_exe.zip] ZIP archive with all supplemental \GMT\ executables.

\end{description}

\index{netCDF, obtaining}

The netCDF library that makes up the backbone of the grdfile
I/O operations can be obtained from  Unidata.  A compressed
tar file can be accessed (in binary mode) from the file
\filename{netcdf.tar.Z} in the anonymous FTP directory of
\underline{unidata.ucar.edu}.
The software distribution includes a \PS\ file of
the netCDF User's Guide, and there is also online documentation
from their web site.  [\htmladdnormallink{netcdfgroup@unidata.ucar.edu}{mailto:netcdfgroup@unidata.ucar.edu}
is available for discussion of the netCDF interface and
announcements about netCDF bugs, fixes, and enhancements.  To
subscribe, send a request to
\htmladdnormallink{netcdfgroup-adm@unidata.ucar.edu}{mailto:netcdfgroup-adm@unidata.ucar.edu}].
Precompiled libraries for WIN32 are also available\footnote{Required
with \filename{GMT\_exe.zip}}.
