%-----------------------------------------
%	$Id: GMT_Appendix_A.tex,v 1.38 2009-05-19 20:20:53 guru Exp $
%
%	The GMT Documentation Project
%	Copyright 2000-2009.
%	Paul Wessel and Walter H. F. Smith
%-----------------------------------------
%

\chapter{\gmt\ supplemental packages}
\label{app:A}
\index{GMT@\GMT!supplemental packages}
\thispagestyle{headings}

These packages are for the most part written
and supported by us, but there are some exceptions.
 They provide extensions of \GMT\
that are needed for particular rather than general
applications.  The software is provided in a separate,
supplemental archive (GMT\_suppl.tar.gz (or .bz2); see
Appendix~\ref{app:D}).  Questions or bug reports for this
software should be addressed to the person(s) listed in
the \filename{README} file associated with the particular program.
It is not guaranteed that these programs are fully ANSI-C,
Y2K, or POSIX compliant, or that they necessarily will
install smoothly on all platforms, but most do.  Note that
the data sets some of these programs work on are not distributed
with these packages; they must be obtained separately.
The contents of the supplemental archive may change
without notice; at this writing it contains these directories:

\section{dbase: gridded data extractor}

This package contains \GMTprog{grdraster} which you can use to extract
data from global gridded data sets such as those available from NGDC.
We have used it to prepare some of the grids in the examples (Chapter~\ref{ch:6}).
You can also customize it to read your own data sets.  The package is
maintained by the \GMT\ developers.

\section{gshhs: GSHHS data extractor}

This package contains \GMTprog{gshhs} which you can use to extract
shoreline polygons from the Global Self-consistent Hierarchical High-resolution
Shorelines (GSHHS) available separately from \htmladdnormallinkfoot{NGDC}{http://www.ngdc.noaa.gov/mgg/shorelines/gshhs.html}
or the \htmladdnormallinkfoot{GSHHS home page}{http://www.soest.hawaii.edu/wessel/gshhs/gshhs.html}
(GSHHS is the polygon data base from which
the \GMT\ coastlines derive).  It also contains \GMTprog{gshhs\_dp}
for cleverly decimating a shoreline, and \GMTprog{gshhstograss} to convert shoreline
segments to the GRASS database format; the latter program is maintained by
Simon Cox\footnote{Simon.Cox@csiro.au}.
The package is maintained by Paul Wessel.

\section{imgsrc: gridded altimetry extractor}

This package consists of the program \GMTprog{img2mercgrd} to
extract subsets of the global gravity and predicted topography
solutions derived from satellite altimetry\footnote{For data bases,
see http://topex.ucsd.edu/marine\_grav/mar\_grav.html.}.  The package
is maintained by Walter Smith\footnote{walter@raptor.grdl.noaa.gov}.

\section{meca: seismology and geodesy symbols}

This package contains the programs \GMTprog{pscoupe}, \GMTprog{psmeca},
\GMTprog{pspolar}, and \GMTprog{psvelo} which are used
by seismologists and geodesists for plotting focal mechanisms (including
cross-sections and polarities), error ellipses, velocity arrows, rotational
wedges, and more.  The package is maintained by
Kurt Feigl\footnote{Kurt.Feigl@cnes.fr} and
Genevieve Patau\footnote{patau@ipgp.jussieu.fr}.

\section{mex: Matlab/Octave--\gmt\ interface}

Here you will find the mex files \GMTprog{grdinfo}, \GMTprog{grdread},
and \GMTprog{grdwrite}, which can be used in Matlab or Octave to read and write
grid files.  The package originated with David Sandwell, UCSD,
and was subsequently modified by Paul Wessel and Phil Sharfstein, UCSB.
It is now maintained by Paul Wessel.

\section{mgd77: MGD77 extractor and plotting tools}

This package currently holds the programs \GMTprog{mgd77convert}, \GMTprog{mgd77info}, \GMTprog{mgd77list},
\GMTprog{mgd77magref}, \GMTprog{mgd77manage}, \GMTprog{mgd77path}, \GMTprog{mgd77sniffer}, and \GMTprog{mgd77track} which can be
used to extract information or data values from or plot marine geophysical
data files in the ASCII MGD77 or netCDF MGD77+ formats\footnote{The ASCII MGD77 data are available on CD-ROM from NGDC
(www.ngdc.noaa.gov).}).  We expect this package eventually to replace the {\bf mgg} package.
The package is maintained by Paul Wessel.

\section{mgg: GMT-MGD77 extractor and plotting tools}

This package holds the legacy programs \GMTprog{binlegs}, \GMTprog{dat2gmt},
\GMTprog{gmt2dat}, \GMTprog{gmtinfo}, \GMTprog{gmtlegs}, \GMTprog{gmtlist},
\GMTprog{gmtpath}, \GMTprog{gmttrack}, and \GMTprog{mgd77togmt}, which can be
used to maintain, access, extract data from, and plot marine geophysical
data files converted from the MGD77 format to the .gmt format\footnote{These data are available on CD-ROM from NGDC
(www.ngdc.noaa.gov).}). The package is maintained by the \GMT\ developers.

\section{misc: Miscellaneous tools}

At the moment, this package contains the programs \GMTprog{psmegaplot}
which you can use to make large posters using a simple laserwriter,
\GMTprog{makepattern} which generates raster patterns from \GMT\ 3.0
icon files, \GMTprog{gmt2kml} which converts \GMT\ table data to Google
Earth's KML format, \GMTprog{gmtdigitize} which provides a GMT interface
to a digitizing tablet via a serial port, \GMTprog{gmtstitch} which can
be used to assemble pieces digitized lines into complete lines or
polygons, \GMTprog{gmtdp} which performs line reduction using the Douglas-Peucker algorithm,
and \GMTprog{nc2xy} which can extract data from column-oriented netCDF files.
The package is maintained by Paul Wessel.
The increasingly popular utility \GMTprog{ps2raster}, which simplifies the rasterization of \GMT
\PS\ to raster formats (see Appendix~\ref{app:C}), was moved to the general tools starting with \GMT\ 4.2.0.

\section{segyprogs: plotting SEGY seismic data}

This package contains programs to plot SEGY seismic data files using
the \GMT\ mapping transformations and postscript library. \GMTprog{pssegy} generates
a 2-D plot (x:location and y:time/depth) while \GMTprog{pssegyz} generates a
3-D plot (x and y: location coordinates, z: time/depth). Locations may be
read from predefined or arbitrary portions of each trace header.  Finally,
\GMTprog{segy2grd} can convert SEGY data to a \GMT\ grid file.
The package is maintained by Tim Henstock\footnote{Timothy.J.Henstock@soc.soton.ac.uk}.

\section{sph: spherical triangulation and gridding}

This package contains the main programs \GMTprog{sphtriangulate},
which you can use to generate data for Delaunay or Voronoi diagrams, 
\GMTprog{sphdistance} which calculates distances from lines to grid
nodes using Voronoi decomposition of the data, and
\GMTprog{sphinterpolate} which performs gridding under tension on
a sphere.  These programs passes the heavy work onto the two Fortran-77
packages SSRFPACK and STRIPACK by Robert Renka; here they have been
translated to C with assistance from \progname{f2c}.
The package is maintained by Paul Wessel.

\section{spotter: backtracking and hotspotting}

This package contains the plate tectonic programs \GMTprog{backtracker},
which you can use to move geologic markers forward or backward in time, 
\GMTprog{grdrotater} which rotates entire grids using a finite rotation,
\GMTprog{hotspotter} which generates CVA grids based on seamount locations
and a set of absolute plate motion stage poles (\GMTprog{grdspotter} does the
same using a bathymetry grid instead of seamount locations), \GMTprog{originator},
which associates seamounts with the most likely hotspot origins, and \GMTprog{rotconverter}
which does various operations involving finite rotations on a sphere.  The package
is maintained by Paul Wessel.

\section{x2sys: track crossover error estimation}

This package contains the tools \GMTprog{x2sys\_datalist},
which allows you to extract data from almost any binary or ASCII
data file, and \GMTprog{x2sys\_cross} which determines crossover
locations and errors generated by one or several geospatial tracks.
Newly added are the tools \GMTprog{x2sys\_init},  \GMTprog{x2sys\_binlist},
\GMTprog{x2sys\_get}, \GMTprog{x2sys\_list}, \GMTprog{x2sys\_put},
\GMTprog{x2sys\_report} and \GMTprog{x2sys\_solve} which extends the track-management system
employed by the mgg supplement to generic track data of any format.
This package represents a new generation of tools intended to replace the old
``X\_SYSTEM'' crossover tools (below).  The package is maintained
by Paul Wessel.

\section{x\_system: track crossover error estimation}

This package contains the tools \GMTprog{x\_edit},
\GMTprog{x\_init}, \GMTprog{x\_list}, \GMTprog{x\_over}, \GMTprog{x\_remove},
\GMTprog{x\_report}, \GMTprog{x\_setup}, \GMTprog{x\_solve\_dc\_drift}, and
\GMTprog{x\_update}.
Collectively, they make up the old ``XSYSTEM'' crossover tools.  This
package with remain in the \GMT\ supplemental archive until \progname{x2sys} is complete.
The package is maintained by Paul Wessel.

\section{xgrid: visual editor for grid files}

The package contains an X11 editor (\GMTprog{xgridedit}) for visual
editing of grid files.  It was originally developed by Hugh Fisher, CRES,
in March 1992 but is now maintained by Lloyd Parkes\footnote{lloyd@must-have-coffee.gen.nz}.
