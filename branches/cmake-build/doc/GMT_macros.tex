\usepackage{ifpdf}
\usepackage{graphicx}
\usepackage{makeidx}
\usepackage{a4wide}
\usepackage{float}
\usepackage{GMT}
\usepackage{times}
\usepackage{color}
\usepackage{mathptmx}
\usepackage{verbatim}
\usepackage{moreverb} % do not ignore tabs in verbatiminput
\let\verbatiminput=\verbatimtabinput
\def\verbatimtabsize{2\relax}
\usepackage{tocbibind}
\usepackage{hyperref}

\usepackage{textcomp}
%\usepackage[utf8]{inputenc}
\usepackage[T1]{fontenc}
\usepackage[english]{babel}
\usepackage{microtype}

\input{GMT_version}

\newcommand{\GMTSITE}{gmt.soest.hawaii.edu}
\newcommand{\gmt}{\href{http://\GMTSITE}{\textbf{GMT}}}

\graphicspath{{fig/}{ppt/}}
\ifpdf
  % Here for creating PDF output with pdfLaTeX
  \pdfcompresslevel=9
  \DeclareGraphicsExtensions{.pdf}
  \hypersetup{%
    pdfauthor={Paul Wessel, Walter H. F. Smith},
    pdftitle={The Generic Mapping Tools Version \GMTDOCVERSION---\GMTTITLE},
    pdfsubject={GMT: Generic Mapping Tools},
    pdfkeywords={GMT, projections, mapping},
    pdfcreator={pdfLaTeX},
    bookmarksopen=true,
    bookmarksnumbered=true,
    hypertexnames=true,
    breaklinks=true
  }%
  \newcommand{\GMTprogi}[1]{\href{run:../man/#1.html}{\textsfbf{#1}}}
  \newcommand{\GMT}{\textit{GMT}}
\fi

\pagecolor{white}

% GMTfig will insert an image file, add a label, and set a caption
\newcommand{\GMTfig}[3][tbp]{\begin{figure}[#1]\centering\includegraphics{#2}\caption{#3}\label{fig:#2}\end{figure}}
% Same for examples (scales by 50% by default)
\newcommand{\GMTexample}[3][scale=0.5]{\begin{figure}[hbtp]\centering\includegraphics[#1]{example_#2}\caption{#3}\label{fig:GMT_example_#2}\end{figure}}
\newcommand{\GMTanimation}[3][scale=0.5]{\begin{figure}[hbtp]\centering\includegraphics[#1]{anim_#2}\caption{#3}\label{fig:GMT_animation_#2}\end{figure}}

\newcommand{\PS}{\textit{PostScript}}
\newcommand{\UNIX}{\textit{UNIX}}
\newcommand{\id}[1]{#1\index{#1}}

\newcommand{\textsfbf}[1]{{\sffamily\bfseries #1\/}}
\newcommand{\textslbf}[1]{{\slshape\bfseries #1\/}}

\newcommand{\GMTprog}[1]{\GMTprogi{#1}\index{#1@\textsfbf{#1}}}

\newcommand{\script}[1]{%
   \begingroup
   \scriptsize\vspace{0.25\baselineskip}\noindent\hrulefill\vspace{-0.75\baselineskip}%
   \verbatiminput{scripts/#1.txt}%
   \vspace{-1.25\baselineskip}\noindent\hrulefill\vspace{0.25\baselineskip}%
   \endgroup
}

\newcommand{\GMTfunc}[1]{\texttt{#1}}
\newcommand{\filename}[1]{\underline{#1}}

\setcounter{topnumber}{2}
\renewcommand{\topfraction}{.8}
\setcounter{bottomnumber}{1}
\renewcommand{\bottomfraction}{.7}
\setcounter{totalnumber}{3}
\renewcommand{\textfraction}{.2}
\renewcommand{\floatpagefraction}{.7}

\sloppy

\newcommand{\DS}{\textdegree}
\newcommand{\PM}{\textpm}
\newcommand{\progname}[1]{\textslbf{#1}\index{#1@\textslbf{#1}}}
\newcommand{\Opt}[2][]{\texttt{\textbf{-#2}#1}}

% Restrict the indentation of lists
\setlength\leftmargin\parindent
\setlength\leftmargini\parindent
\setlength\leftmarginii\parindent
\setlength\leftmarginiii\parindent
\setlength\leftmarginiv\parindent
\setlength\leftmarginv\parindent
\setlength\leftmarginvi\parindent
