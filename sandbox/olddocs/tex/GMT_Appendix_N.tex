%------------------------------------------
%	$Id$
%
%	The GMT Documentation Project
%	Copyright (c) 2000-2013.
%	P. Wessel, W. H. F. Smith, R. Scharroo, J. Luis and F. Wobbe
%------------------------------------------
%
\chapter{Annotation of Contours and ``Quoted Lines''}
\label{app:O}

The \GMT\ programs \GMTprog{grdcontour} (for data given as 2-dimensional grids)
and \GMTprog{pscontour} (for \emph{x,y,z} tables) allow
for contouring of data sets, while \GMTprog{psxy} and \GMTprog{psxyz}
can plot lines based on \emph{x,y}- and \emph{x,y,z}-tables, respectively.
In both cases it may be necessary to attach labels to
these lines.  Clever or optimal placements of labels is a very difficult
topic, and \GMT\ provides several algorithms for this placement as well
as complete freedom in specifying the attributes of the labels.
Because of the richness of these choices we present this Appendix which
summarizes the various options and gives several examples of their use.

\section{Label Placement}

While the previous \GMT\ versions 1--3 allowed for a single algorithm that
determined where labels would be placed, \GMT\ 4 allows for five different
algorithms.  Furthermore, a new ``symbol'' option (\Opt{Sq} for ``quoted line'')
has been added to \GMTprog{psxy} and \GMTprog{psxyz} and hence the new label
placement mechanisms apply to those programs as well.  The contouring programs
expect the algorithm to be specified as arguments to \Opt{G} while the line plotting
programs expect the same arguments to follow \Opt{Sq}.  The information appended
to these options is the same in both cases and is of the form [\textbf{code}]\emph{info}.
The five algorithms correspond to the five codes below (some codes will appear in both
upper and lower case; they share the same algorithm but differ in some other ways).
In what follows, the phrase ``line segment'' is taken to mean either a contour or a line
to be labeled.  The codes are:
\begin{description}
\item [d:] Full syntax is \textbf{d}\emph{dist}[\textbf{c$|$i$|$p}][/\emph{frac}].
Place labels according to the distance measured along the projected line on the map.  Append the
unit you want to measure distances in [Default is taken from \GMTdef{PROJ\_LENGTH\_UNIT}].
Starting at the beginning of a line, place labels every \emph{dist} increment of
distance along the line.  To ensure that closed lines whose total length is less
than \emph{dist} get annotated, we may append \emph{frac} which will place the first
label at the distance $d =$ \emph{dist} $\times$ \emph{frac} from the start of a closed
line (and every \emph{dist} thereafter).  If not given, \emph{frac} defaults to 0.25.
\item [D:] Full syntax is \textbf{D}\emph{dist}[\textbf{d$|$m$|$s$|$e$|$f$|$k$|$M$|$n}][/\emph{frac}].
This option is similar to \textbf{d} except the original data must be referred to geographic
coordinates (and a map projection must have been chosen) and actual Earth\footnote{or whatever
planet we are dealing with.} surface distances along the lines are considered.  Append the
unit you want to measure distances in; choose among arc \textbf{d}egree, \textbf{m}inute, and \textbf{s}econd,
or m\textbf{e}ter [Default], \textbf{f}eet, \textbf{k}ilometer, statute \textbf{M}iles, or \textbf{n}autical miles.  Other aspects are similar to code \textbf{d}.
\item [f:] Full syntax is \textbf{f}\emph{fix.txt}[/\emph{slop}[\textbf{c$|$i$|$p}]].  Here, an ASCII file \emph{fix.txt} is
given which must contain records whose first two columns hold the coordinates of points along
the lines at which locations the labels should be placed.  Labels will only be placed if the coordinates
match the line coordinates to within a distance of \emph{slop} (append unit or we use \GMTdef{PROJ\_LENGTH\_UNIT}).
The default \emph{slop} is zero, meaning only exact coordinate matches will do.
\item [l:] Full syntax is \textbf{l}\emph{line1}[,\emph{line2}[, ...]].  One or more straight line segments
are specified separated by commas, and labels will be placed at the intersections between these lines and
our line segments.  Each \emph{line} specification implies a \emph{start} and \emph{stop} point, each
corresponding to a coordinate pair.  These pairs can be regular coordinate pairs (i.e., longitude/latitude separated by a slash), or they
can be two-character codes that refer to predetermined points relative to the map region.  These codes
are taken from the \GMTprog{pstext} justification keys [\textbf{L$|$C$|$R}][\textbf{B$|$M$|$T}] so that the first
character determines the $x$-coordinate and the second determines the $y$-coordinate.  In \GMTprog{grdcontour},
you can also use the two codes \textbf{Z+} and \textbf{Z-} as shorthands for the location of the grid's global
maximum and minimum, respectively.  For example, the \emph{line} \textbf{LT}/\textbf{RB} is a diagonal from the
upper left to the lower right map corner, while \textbf{Z-}/135W/15S is a line from the grid minimum to the point
(135\DS W, 15\DS S).
\item [L:] Same as \textbf{l} except we will treat the lines given as great circle start/stop coordinates and fill in
the points between before looking for intersections.
\item [n:] Full syntax is \textbf{n}\emph{number}[/\emph{minlength}[\textbf{c$|$i$|$p}]].  Place
\emph{number} of labels along each line regardless of total line length.  The line is divided into \emph{number}
segments and the labels are placed at the centers of these segments.  Optionally, you may give a \emph{minlength}
distance to ensure that no labels are placed closer than this distance to its neighbors.
\item [N:] Full syntax is \textbf{N}\emph{number}[/\emph{minlength}[\textbf{c$|$i$|$p}]].  Similar to
code \textbf{n} but here labels are placed at the ends of each segment (for \emph{number} $\geq 2$).  A special
case arises for \emph{number} $= 1$ when a single label will be placed according to the sign of \emph{number}:
$-1$ places one label justified at the start of the line, while $+1$ places one label justified at the end of the line.
\item [x:] Full syntax is \textbf{x}\emph{cross.d}.  Here, an ASCII file \emph{cross.d} is
a multi-segment file whose lines will intersect our segment lines; labels will be placed at these intersections.
\item [X:] Same as \textbf{x} except we treat the lines given as great circle start/stop coordinates and fill in the
points between before looking for intersections.
\end{description}
Only one algorithm can be specified at any given time.

\section{Label Attributes}
Determining where to place labels is half the battle.  The other half is to specify exactly
what are the attributes of the labels.  It turns out that there are quite a few possible
attributes that we may want to control, hence understanding how to specify these attributes
becomes important.  In the contouring programs, one or more attributes may be appended to the
\Opt{A} option using the format +\emph{code}[\emph{args}] for each attribute, whereas for the line
plotting programs these attributes are appended to the \Opt{Sq} option following a colon (:)
that separates the label codes from the placement algorithm.  Several of the attributes do not
apply to contours so we start off with listing those that apply universally.  These codes are:
\begin{description}
\item [+a:]  Controls the angle of the label relative to the angle of the line.  Append \textbf{n}
for normal to the line, give a fixed \emph{angle} measured counter-clockwise relative to the
horizontal. or append \textbf{p} for parallel to the line [Default].  If using \GMTprog{grdcontour}
the latter option you may further append \textbf{u} or \textbf{d} to get annotations whose upper edge
always face the next higher or lower contour line.
\item [+c:] Surrounding each label is an imaginary label ``textbox'' which defines a region in
which no segment lines should be visible.  The initial box provides an exact fit to the enclosed
text but clearance may be extended in both the horizontal and vertical directions (relative to the label
baseline) by the given amounts.  If these should be different amounts please separate them by
a slash; otherwise the single value applies to both directions.  Append the distance units of
your choice (\textbf{c$|$i$|$m$|$p}), or give \% to indicate that the clearance should be this
fixed percentage of the label font size in use.  The default is 15\%.
\item [+d:] Debug mode.  This is useful when testing contour placement as it will draw the normally
invisible helper lines and points in the label placement algorithms above.
\item [+d:] Delayed mode, to delay the plotting of the text as text clipping is set instead.
\item [+f:] Specifies the desired label font, including size or color.  See \GMTprog{pstext} for font names or numbers.  The
default font is given by \GMTdef{FONT\_ANNOT\_PRIMARY}.
\item [+g:] Selects opaque rather than the default transparent text boxes.  You may optionally append
the color you want to fill the label boxes; the default is the same as \GMTdef{PS\_PAGE\_COLOR}.
\item [+j:] Selects the justification of the label relative to the placement points determined
above.  Normally this is center/mid justified (\textbf{CM} in \GMTprog{pstext} justification parlance) and
this is indeed the default setting.  Override by using this option and append another justification
key code from [\textbf{L$|$C$|$R}][\textbf{B$|$M$|$T}].  Note for curved text (\textbf{+v}) only vertical
justification will be affected.
\item [+o:] Request a rounded, rectangular label box shape; the default is rectangular.  This is only
manifested if the box is filled or outlined, neither of which is implied by this option alone (see \textbf{+g}
and \textbf{+p}).  As this option only applies to straight text, it is ignored if \textbf{+v} is given.
\item [+p:] Selects the drawing of the label box outline; append your preferred \emph{pen} unless you
want the default \GMT\ pen [0.25p,black].
\item [+r:] Do not place labels at points along the line whose local radius of curvature falls below
the given threshold value.  Append the radius unit of your choice (\textbf{c$|$i$|$p}) [Default is 0].
\item [+u:] Append the chosen \emph{unit} to the label.  Normally a space will separate the label
and the unit.  If you want to close this gap, append a \emph{unit} that begins with a hyphen (--).
If you are contouring with \GMTprog{grdcontour} and you specify this option without appending a unit, the unit will be taken from the
$z$-unit attribute of the grid header.
\item [+v:]  Place curved labels that follow the wiggles of the line segments.  This is especially
useful if the labels are long relative to the length-scale of the wiggles.  The default places labels
on an invisible straight line at the angle determined.
\item [+w:] The angle of the line at the point of straight label placement is calculated by a least-squares
fit to the \emph{width} closest points.  If not specified, \emph{width} defaults to 10.
\item [+=:]  Similar in most regards to \textbf{+u} but applies instead to a label \emph{prefix} which
you must append.
\end{description}
For contours, the label will be the value of the contour (possibly modified by \textbf{+u} or \textbf{+=}).
However, for quoted lines other options apply:
\begin{description}
\item [+l:] Append a fixed \emph{label} that will be placed at all label locations.  If the label contains
spaces you must place it inside matching quotes.
\item [+L:] Append a code \emph{flag} that will determine the label.  Available codes are:
\begin{description}
\item [+Lh:]  Take the label from the current multi-segment header (hence it is assumed that the
input line segments are given in the multi-segment file format; if not we pick the single label
from the file's header record).  We first scan the header for an embedded
\Opt{L}\emph{label} option; if none is found we instead use the first word following the segment marker [$>$].
\item [+Ld:] Take the Cartesian plot distances along the line as the label; append \textbf{c$|$i$|$p}
as the unit [Default is \GMTdef{PROJ\_LENGTH\_UNIT}].  The label will be formatted according to the \GMTdef{FORMAT\_FLOAT\_OUT}
string, \emph{unless} label placement was determined from map distances along the segment lines, in which
case we determine the appropriate format from the distance value itself. 
\item [+LD:]  Calculate actual Earth surface distances and use the distance at the label placement point as
the label; append \textbf{d$|$e$|$f$|$k$|$m$|$M$|$n$|$s} to specify the unit [If not given we default to \textbf{d}egrees,
\emph{unless} label placement was determined from map distances along the segment lines, in which case we
use the same unit specified for that algorithm].  Requires a map projection to be used.
\item [+Lf:]  Use all text after the 2nd column in the fixed label location file \emph{fix.txt} as labels.
This choice obviously requires the fixed label location algorithm (code \textbf{f}) to be in effect.
\item [+Ln:]  Use the running number of the current multi-segment as label.
\item [+LN:]  Use a slash-separated combination of the current file number and the current multi-segment number as label.
\item [+Lx:]  As \textbf{h} but use the multi-segment headers in the \emph{cross.d} file instead.
This choice obviously requires the crossing segments location algorithm (code \textbf{x$|$X}) to be in effect.
\end{description}
\end{description}

\section{Examples of Contour Label Placement}

We will demonstrate the use of these options with a few simple examples.
First, we will contour a subset of the global geoid data used in \GMT\ Example 01;
the region selected encompasses the world's strongest ``geoid dipole'': the Indian Low
and the New Guinea High.

\subsection{Equidistant labels}

Our first example uses the default placement algorithm.  Because of the size
of the map we request contour labels every 1.5 inches along the lines:

\script{GMT_App_O_1}

\noindent
As seen in Figure~\ref{fig:GMT_App_O_1}, the contours are placed rather arbitrary.
The string of contours for $-40$ to $60$ align well but that is a fortuitous 
consequence of reaching the 1.5 inch distance from the start at the bottom of the map.

\GMTfig[H]{GMT_App_O_1}{Equidistant contour label placement with \Opt{Gd}, the only algorithm
available in previous \gmt\ versions.}

\subsection{Fixed number of labels}

We now exercise the option for specifying exactly how many labels each contour line
should have:

\script{GMT_App_O_2}

\noindent
By selecting only one label per contour and requiring that labels only be placed on
contour lines whose length exceed 1 inch, we achieve the effect shown in Figure~\ref{fig:GMT_App_O_2}.

\GMTfig[H]{GMT_App_O_2}{Placing one label per contour that exceed 1 inch in length,
centered on the segment with \Opt{Gn}.}

\subsection{Prescribed label placements}

Here, we specify four points where we would like contour labels to be placed.  Our points
are not exactly on the contour lines so we give a nonzero ``slop'' to be used in the
distance calculations: The point on the contour closest to our fixed points and within
the given maximum distance will host the label.

\script{GMT_App_O_3}

\noindent
The angle of the label is evaluated from the contour line geometry, and the final result
is shown in Figure~\ref{fig:GMT_App_O_3}.
\GMTfig[H]{GMT_App_O_3}{Four labels are positioned on the points along the contours that
are closest to the locations given in the file \protect\filename{fix.txt} in the \protect\Opt{Gf} option.}
To aid in understanding the algorithm we chose to specify ``debug'' mode (\textbf{+d}) which placed a
small circle at each of the fixed points.

\subsection{Label placement at simple line intersections}

Often, it will suffice to place contours at the imaginary intersections between the
contour lines and a well-placed straight line segment.  The \Opt{Gl} or \Opt{GL}
algorithms work well in those cases:

\script{GMT_App_O_4}

\noindent
The obvious choice in this example is to specify a great circle between the high and
the low, thus placing all labels between these extrema.

\GMTfig[H]{GMT_App_O_4}{Labels are placed at the intersections between contours and the
great circle specified in the \protect\Opt{GL} option.}
\noindent
The thin debug line in Figure~\ref{fig:GMT_App_O_4} shows the great circle and the
intersections where labels are plotted.  Note that any number of such lines could be specified;
here we are content with just one.

\subsection{Label placement at general line intersections}

If (1) the number of intersecting straight line segments needed to pick the desired label
positions becomes too large to be given conveniently on the command line, or (2) we have
another data set or lines whose intersections we wish to use, the general crossing
algorithm makes more sense:

\script{GMT_App_O_5}

\GMTfig[H]{GMT_App_O_5}{Labels are placed at the intersections between contours and the
multi-segment lines specified in the \protect\Opt{GX} option.}

\noindent
In this case, we have created three strands of lines whose intersections with the contours
define the label placements, presented in Figure~\ref{fig:GMT_App_O_5}.

\section{Examples of Label Attributes}

We will now demonstrate some of the ways to play with the label attributes.  To do so we
will use \GMTprog{psxy} on a great-circle line connecting the geoid extrema, along which
we have sampled the ETOPO5 relief data set.  The file \filename{transect.d} thus contains
\emph{lon, lat, dist, geoid, relief}, with distances in km.

\subsection{Label placement by along-track distances, 1}

This example will change the orientation of labels from along-track to across-track, and
surrounds the labels with an opaque, outlined text box so that the label is more readable.  We choose
the place the labels every 1000 km along the line and use that distance as the label.  The
labels are placed normal to the line:

\script{GMT_App_O_6}

\GMTfig[H]{GMT_App_O_6}{Labels attributes are controlled with the arguments to the \protect\Opt{Sq} option.}

\noindent
The composite illustration in Figure~\ref{fig:GMT_App_O_6} shows the new effects.  Note that
the line connecting the extrema does not end exactly at the `-' and `+' symbols.  This is
because the placements of those symbols are based on the mean coordinates of the contour and
not the locations of the (local or global) extrema.

\subsection{Label placement by along-track distances, 2}

A small variation on this theme is to place the labels parallel to the line, use
spherical degrees for placement, append the degree symbol as a unit for the
labels, choose a rounded rectangular text box, and inverse-video the label:

\script{GMT_App_O_7}

\noindent
The output is presented as Figure~\ref{fig:GMT_App_O_7}.

\GMTfig[H]{GMT_App_O_7}{Another label attribute example.}

\subsection{Using a different data set for labels}

In the next example we will use the bathymetry values along the transect as
our label, with placement determined by the distance along track.  We choose
to place labels every 1500 km.  To do this we need to pull out those records
whose distances are multiples of 1500 km and create a ``fixed points'' file
that can be used to place labels and specify the labels.  This is done with
\progname{awk}.

\script{GMT_App_O_8}

\noindent
The output is presented as Figure~\ref{fig:GMT_App_O_8}.

\GMTfig[H]{GMT_App_O_8}{Labels based on another data set (here bathymetry) while
the placement is based on distances.}

\section{Putting it all together}

Finally, we will make a more complex composite illustration that uses several of the
label placement and label attribute settings discussed in the previous sections.  We
make a map showing the tsunami travel times (in hours) from a hypothetical catastrophic
landslide in the Canary Islands\footnote{Travel times were calculated using Geoware's
travel time calculator, \progname{ttt};
see \htmladdnormallink{(http://www.geoware-online.com)}{(http://www.geoware-online.com)}}.
We lay down a color map based on the travel times
and the shape of the seafloor, and travel time contours with curved labels as well
as a few quoted lines.  The final script is

\script{GMT_App_O_9}

\noindent
with the complete illustration presented as Figure~\ref{fig:GMT_App_O_9}.
\GMTfig[H]{GMT_App_O_9}{Tsunami travel times from the Canary Islands to places
in the Atlantic, in particular New York.  Should a catastrophic landslide occur
it is possible that New York will experience a large tsunami about 8 hours after
the event.}
