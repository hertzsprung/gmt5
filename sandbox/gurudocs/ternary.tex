%	$Id$
% Internal documentation for adding a ternary diagram mode
% Paul Wessel, April 21, 2015.
\documentclass[12pt,letterpaper,margin=0.5in]{report}
\usepackage{times}
\usepackage{graphicx}
\usepackage{breqn}
\usepackage[margin=0.5in]{geometry}
\usepackage{lscape}
\textheight = 9 in
\topmargin = -1 in
\begin{document}

\section*{ADDING TERNARY DIAGRAMS TO GMT}

NSF/Geoinformatics has funded us to add ternary plots to GMT.
These may contain symbols, lines, contours, images, and text and are
in all regards just another kind of projection, except input data are
typically triplets $(a,b,c)$ [with $c = S - a -b$ for some sum $S$] that must first be converted to $(x,y)$ before
regular GMT plotting, gridding, and other operations can commence.  I suggest we implement ternary plots
via a low-level i/o function that converts $(a,b,c)$ to $(x,y)$ during read when a ternary flag is set.
This would be similar to two other low-level checks we currently perform, such as adjusting
longitude periodicities or inverse-projecting incoming projected coordinates (e.g., convert UTM meters to lon/lat).
The GMT modules only see the $(x,y)$ coordinates.
Because additional columns may be present there will be some reshuffling of column arguments.

The standard {\bf -R} option is flexible but not easily
used to deal with ternary specifications.  We could simply extend it to handle
a ternary mode, e.g., {\bf -Rt}.  Since {\bf -R} is required we let {\bf -Rt} trigger the
ternary flag.
With those changes, setting up a ternary projection is not actually needed as we proceed with
{\bf -Jx}{\it scale} or {\bf -JX}{\it width}; the ternary flag will let us set a triangular
clip path under-the-hood. 

It would therefore seem we only have these challenges:
\begin{itemize}
	\item Specification of annotation, tick, and label information for \emph{three} instead of two axes in the plane.
	This will need a specialized {\bf -B} mode.  Note there may be
	four axes if $z$ is involved for a 3-D plot.  Alternatively, we use the {\bf -Bx}, {\bf -By}, {\bf -Bz} for the three
	planar axes and add {\bf -Bt} for the 3rd dimension.  Or we introduce three new modes {\bf -Ba}, {\bf -Bb}, {\bf -Bc}.
	The ternary flag lets us call the ternary-aware basemap sub-function.
	\item The standard GMT plot setup  automatically issues a triangular clip path (unless {\bf -N} is used) as is true
	for other projections.  The clip path is terminated at the end of a module.
	\item For gridding $(a,b,c)$ data the $(x,y)$ ends up with a Cartesian {\bf -R} setting with regions of no data (outside triangle
	but inside rectangle).  We could automatically apply a NaN-mask to those undefined regions (or not, since clipping will mask them).
	\item When using {\bf -Rt} with grids it is understood that grids are given in $(x,y)$ coordinates that were converted
	from $(a,b,c)$ [or there could be a flag in the grid header] and this would allow the triangular clipping to be set when needed.
\end{itemize}

Action items:
\begin{itemize}
\item Add the mode {\bf -Rt}{\it a0/a1/b0/b1/c0/c1[/z0/z1]} for ``ternary region specification''.
\item Add the option {\bf -Bt}{\it args} for axis specification of the vertical frame in 3-D ternary plots (or one of the other schemes).
\item With ternary flag is true, map $(a,b,c)$ to $(x,y)$ during low-level input.
\end{itemize}
Per Wikipedia, given  $(a,b,c)$ coordinates we compute $(x,y)$ Cartesian coordinates via
\begin{equation}
	x = \frac{1}{2}\frac{2b+c}{a+b+c}, \quad \quad y = \frac{\sqrt{3}}{2}\frac{c}{a+b+c}.
\end{equation}
where it is implied that $a = 100$\% is the lower left vertex at $(0,0)$ and $b = 100$\% is the lower right vertex at $(1,0)$ and
$c = 100$\% is at $(\frac{1}{2}, \frac{\sqrt{3}}{2})$.  These non-dimensional distances are then scaled by the map projection
to the desired lengths.  Grids created from ternary data would go from 0 to 1 in $x$ and 0 to $\frac{\sqrt{3}}{2}$ in $y$. This
means we cannot have an exact integer node count in $y$ but must exceed max y by a tiny bit.
\end{document}
