%------------------------------------------
%	$Id: GMT_Chapter_5.tex,v 1.7 2002-01-17 22:57:17 pwessel Exp $
%
%	The GMT Documentation Project
%	Copyright 2000-2002.
%	Paul Wessel and Walter H. F. Smith
%------------------------------------------
%
\chapter{\gmt\ Coordinate Transformations}
\thispagestyle{headings}

\GMT\ programs read real-world coordinates and convert them to positions on a plot.
This is achieved by selecting one of several coordinate transformations or projections.
We distinguish between three sets of such conversions:

\begin{itemize}
\item Cartesian coordinate transformations
\item Polar coordinate transformations
\item Map coordinate transformations
\end{itemize}

The next chapter will be dedicated to \GMT\ map projections in its entirety.  Meanwhile, the present chapter
will summarize the properties of the Cartesian and Polar coordinate transformations available in \GMT, list
which parameters define them, and demonstrate how they are used to create simple plot axes.  We will mostly
be using \GMTprog{psbasemap} (and occasionally \GMTprog{psxy}) to demostrate the various transformations.
Our illustrations may differ from those you reproduce with the same commands because of different settings
in our \filename{.gmtdefaults} file.)  Finally, note that while we will specify dimensions in inches (by
appending {\bf i}), you may want to use cm ({\bf c}), meters ({\bf m}), or points ({\bf p}) as unit instead
(see the \GMTprog{gmtdefaults} man page). 

%------------------------------------------
%	$Id: GMT_Chapter_5.tex,v 1.13 2005-12-17 05:59:21 pwessel Exp $
%
%	The GMT Documentation Project
%	Copyright 2000-2006.
%	Paul Wessel and Walter H. F. Smith
%------------------------------------------
%
\chapter{\gmt\ Coordinate Transformations}
\label{ch:5}
\thispagestyle{headings}

\GMT\ programs read real-world coordinates and convert them to positions on a plot.
This is achieved by selecting one of several coordinate transformations or projections.
We distinguish between three sets of such conversions:

\begin{itemize}
\item Cartesian coordinate transformations
\item Polar coordinate transformations
\item Map coordinate transformations
\end{itemize}

The next chapter will be dedicated to \GMT\ map projections in its entirety.  Meanwhile, the present chapter
will summarize the properties of the Cartesian and Polar coordinate transformations available in \GMT, list
which parameters define them, and demonstrate how they are used to create simple plot axes.  We will mostly
be using \GMTprog{psbasemap} (and occasionally \GMTprog{psxy}) to demostrate the various transformations.
Our illustrations may differ from those you reproduce with the same commands because of different settings
in our \filename{.gmtdefaults4} file.)  Finally, note that while we will specify dimensions in inches (by
appending {\bf i}), you may want to use cm ({\bf c}), meters ({\bf m}), or points ({\bf p}) as unit instead
(see the \GMTprog{gmtdefaults} man page). 

%------------------------------------------
%	$Id: GMT_Chapter_5.tex,v 1.13 2005-12-17 05:59:21 pwessel Exp $
%
%	The GMT Documentation Project
%	Copyright 2000-2006.
%	Paul Wessel and Walter H. F. Smith
%------------------------------------------
%
\chapter{\gmt\ Coordinate Transformations}
\label{ch:5}
\thispagestyle{headings}

\GMT\ programs read real-world coordinates and convert them to positions on a plot.
This is achieved by selecting one of several coordinate transformations or projections.
We distinguish between three sets of such conversions:

\begin{itemize}
\item Cartesian coordinate transformations
\item Polar coordinate transformations
\item Map coordinate transformations
\end{itemize}

The next chapter will be dedicated to \GMT\ map projections in its entirety.  Meanwhile, the present chapter
will summarize the properties of the Cartesian and Polar coordinate transformations available in \GMT, list
which parameters define them, and demonstrate how they are used to create simple plot axes.  We will mostly
be using \GMTprog{psbasemap} (and occasionally \GMTprog{psxy}) to demostrate the various transformations.
Our illustrations may differ from those you reproduce with the same commands because of different settings
in our \filename{.gmtdefaults4} file.)  Finally, note that while we will specify dimensions in inches (by
appending {\bf i}), you may want to use cm ({\bf c}), meters ({\bf m}), or points ({\bf p}) as unit instead
(see the \GMTprog{gmtdefaults} man page). 

%------------------------------------------
%	$Id: GMT_Chapter_5.tex,v 1.13 2005-12-17 05:59:21 pwessel Exp $
%
%	The GMT Documentation Project
%	Copyright 2000-2006.
%	Paul Wessel and Walter H. F. Smith
%------------------------------------------
%
\chapter{\gmt\ Coordinate Transformations}
\label{ch:5}
\thispagestyle{headings}

\GMT\ programs read real-world coordinates and convert them to positions on a plot.
This is achieved by selecting one of several coordinate transformations or projections.
We distinguish between three sets of such conversions:

\begin{itemize}
\item Cartesian coordinate transformations
\item Polar coordinate transformations
\item Map coordinate transformations
\end{itemize}

The next chapter will be dedicated to \GMT\ map projections in its entirety.  Meanwhile, the present chapter
will summarize the properties of the Cartesian and Polar coordinate transformations available in \GMT, list
which parameters define them, and demonstrate how they are used to create simple plot axes.  We will mostly
be using \GMTprog{psbasemap} (and occasionally \GMTprog{psxy}) to demostrate the various transformations.
Our illustrations may differ from those you reproduce with the same commands because of different settings
in our \filename{.gmtdefaults4} file.)  Finally, note that while we will specify dimensions in inches (by
appending {\bf i}), you may want to use cm ({\bf c}), meters ({\bf m}), or points ({\bf p}) as unit instead
(see the \GMTprog{gmtdefaults} man page). 

\input{GMT_Chapter_5.1}
\input{GMT_Chapter_5.2}

%------------------------------------------
%	$Id: GMT_Chapter_5.tex,v 1.13 2005-12-17 05:59:21 pwessel Exp $
%
%	The GMT Documentation Project
%	Copyright 2000-2006.
%	Paul Wessel and Walter H. F. Smith
%------------------------------------------
%
\chapter{\gmt\ Coordinate Transformations}
\label{ch:5}
\thispagestyle{headings}

\GMT\ programs read real-world coordinates and convert them to positions on a plot.
This is achieved by selecting one of several coordinate transformations or projections.
We distinguish between three sets of such conversions:

\begin{itemize}
\item Cartesian coordinate transformations
\item Polar coordinate transformations
\item Map coordinate transformations
\end{itemize}

The next chapter will be dedicated to \GMT\ map projections in its entirety.  Meanwhile, the present chapter
will summarize the properties of the Cartesian and Polar coordinate transformations available in \GMT, list
which parameters define them, and demonstrate how they are used to create simple plot axes.  We will mostly
be using \GMTprog{psbasemap} (and occasionally \GMTprog{psxy}) to demostrate the various transformations.
Our illustrations may differ from those you reproduce with the same commands because of different settings
in our \filename{.gmtdefaults4} file.)  Finally, note that while we will specify dimensions in inches (by
appending {\bf i}), you may want to use cm ({\bf c}), meters ({\bf m}), or points ({\bf p}) as unit instead
(see the \GMTprog{gmtdefaults} man page). 

\input{GMT_Chapter_5.1}
\input{GMT_Chapter_5.2}


%------------------------------------------
%	$Id: GMT_Chapter_5.tex,v 1.13 2005-12-17 05:59:21 pwessel Exp $
%
%	The GMT Documentation Project
%	Copyright 2000-2006.
%	Paul Wessel and Walter H. F. Smith
%------------------------------------------
%
\chapter{\gmt\ Coordinate Transformations}
\label{ch:5}
\thispagestyle{headings}

\GMT\ programs read real-world coordinates and convert them to positions on a plot.
This is achieved by selecting one of several coordinate transformations or projections.
We distinguish between three sets of such conversions:

\begin{itemize}
\item Cartesian coordinate transformations
\item Polar coordinate transformations
\item Map coordinate transformations
\end{itemize}

The next chapter will be dedicated to \GMT\ map projections in its entirety.  Meanwhile, the present chapter
will summarize the properties of the Cartesian and Polar coordinate transformations available in \GMT, list
which parameters define them, and demonstrate how they are used to create simple plot axes.  We will mostly
be using \GMTprog{psbasemap} (and occasionally \GMTprog{psxy}) to demostrate the various transformations.
Our illustrations may differ from those you reproduce with the same commands because of different settings
in our \filename{.gmtdefaults4} file.)  Finally, note that while we will specify dimensions in inches (by
appending {\bf i}), you may want to use cm ({\bf c}), meters ({\bf m}), or points ({\bf p}) as unit instead
(see the \GMTprog{gmtdefaults} man page). 

%------------------------------------------
%	$Id: GMT_Chapter_5.tex,v 1.13 2005-12-17 05:59:21 pwessel Exp $
%
%	The GMT Documentation Project
%	Copyright 2000-2006.
%	Paul Wessel and Walter H. F. Smith
%------------------------------------------
%
\chapter{\gmt\ Coordinate Transformations}
\label{ch:5}
\thispagestyle{headings}

\GMT\ programs read real-world coordinates and convert them to positions on a plot.
This is achieved by selecting one of several coordinate transformations or projections.
We distinguish between three sets of such conversions:

\begin{itemize}
\item Cartesian coordinate transformations
\item Polar coordinate transformations
\item Map coordinate transformations
\end{itemize}

The next chapter will be dedicated to \GMT\ map projections in its entirety.  Meanwhile, the present chapter
will summarize the properties of the Cartesian and Polar coordinate transformations available in \GMT, list
which parameters define them, and demonstrate how they are used to create simple plot axes.  We will mostly
be using \GMTprog{psbasemap} (and occasionally \GMTprog{psxy}) to demostrate the various transformations.
Our illustrations may differ from those you reproduce with the same commands because of different settings
in our \filename{.gmtdefaults4} file.)  Finally, note that while we will specify dimensions in inches (by
appending {\bf i}), you may want to use cm ({\bf c}), meters ({\bf m}), or points ({\bf p}) as unit instead
(see the \GMTprog{gmtdefaults} man page). 

\input{GMT_Chapter_5.1}
\input{GMT_Chapter_5.2}

%------------------------------------------
%	$Id: GMT_Chapter_5.tex,v 1.13 2005-12-17 05:59:21 pwessel Exp $
%
%	The GMT Documentation Project
%	Copyright 2000-2006.
%	Paul Wessel and Walter H. F. Smith
%------------------------------------------
%
\chapter{\gmt\ Coordinate Transformations}
\label{ch:5}
\thispagestyle{headings}

\GMT\ programs read real-world coordinates and convert them to positions on a plot.
This is achieved by selecting one of several coordinate transformations or projections.
We distinguish between three sets of such conversions:

\begin{itemize}
\item Cartesian coordinate transformations
\item Polar coordinate transformations
\item Map coordinate transformations
\end{itemize}

The next chapter will be dedicated to \GMT\ map projections in its entirety.  Meanwhile, the present chapter
will summarize the properties of the Cartesian and Polar coordinate transformations available in \GMT, list
which parameters define them, and demonstrate how they are used to create simple plot axes.  We will mostly
be using \GMTprog{psbasemap} (and occasionally \GMTprog{psxy}) to demostrate the various transformations.
Our illustrations may differ from those you reproduce with the same commands because of different settings
in our \filename{.gmtdefaults4} file.)  Finally, note that while we will specify dimensions in inches (by
appending {\bf i}), you may want to use cm ({\bf c}), meters ({\bf m}), or points ({\bf p}) as unit instead
(see the \GMTprog{gmtdefaults} man page). 

\input{GMT_Chapter_5.1}
\input{GMT_Chapter_5.2}



%------------------------------------------
%	$Id: GMT_Chapter_5.tex,v 1.13 2005-12-17 05:59:21 pwessel Exp $
%
%	The GMT Documentation Project
%	Copyright 2000-2006.
%	Paul Wessel and Walter H. F. Smith
%------------------------------------------
%
\chapter{\gmt\ Coordinate Transformations}
\label{ch:5}
\thispagestyle{headings}

\GMT\ programs read real-world coordinates and convert them to positions on a plot.
This is achieved by selecting one of several coordinate transformations or projections.
We distinguish between three sets of such conversions:

\begin{itemize}
\item Cartesian coordinate transformations
\item Polar coordinate transformations
\item Map coordinate transformations
\end{itemize}

The next chapter will be dedicated to \GMT\ map projections in its entirety.  Meanwhile, the present chapter
will summarize the properties of the Cartesian and Polar coordinate transformations available in \GMT, list
which parameters define them, and demonstrate how they are used to create simple plot axes.  We will mostly
be using \GMTprog{psbasemap} (and occasionally \GMTprog{psxy}) to demostrate the various transformations.
Our illustrations may differ from those you reproduce with the same commands because of different settings
in our \filename{.gmtdefaults4} file.)  Finally, note that while we will specify dimensions in inches (by
appending {\bf i}), you may want to use cm ({\bf c}), meters ({\bf m}), or points ({\bf p}) as unit instead
(see the \GMTprog{gmtdefaults} man page). 

%------------------------------------------
%	$Id: GMT_Chapter_5.tex,v 1.13 2005-12-17 05:59:21 pwessel Exp $
%
%	The GMT Documentation Project
%	Copyright 2000-2006.
%	Paul Wessel and Walter H. F. Smith
%------------------------------------------
%
\chapter{\gmt\ Coordinate Transformations}
\label{ch:5}
\thispagestyle{headings}

\GMT\ programs read real-world coordinates and convert them to positions on a plot.
This is achieved by selecting one of several coordinate transformations or projections.
We distinguish between three sets of such conversions:

\begin{itemize}
\item Cartesian coordinate transformations
\item Polar coordinate transformations
\item Map coordinate transformations
\end{itemize}

The next chapter will be dedicated to \GMT\ map projections in its entirety.  Meanwhile, the present chapter
will summarize the properties of the Cartesian and Polar coordinate transformations available in \GMT, list
which parameters define them, and demonstrate how they are used to create simple plot axes.  We will mostly
be using \GMTprog{psbasemap} (and occasionally \GMTprog{psxy}) to demostrate the various transformations.
Our illustrations may differ from those you reproduce with the same commands because of different settings
in our \filename{.gmtdefaults4} file.)  Finally, note that while we will specify dimensions in inches (by
appending {\bf i}), you may want to use cm ({\bf c}), meters ({\bf m}), or points ({\bf p}) as unit instead
(see the \GMTprog{gmtdefaults} man page). 

%------------------------------------------
%	$Id: GMT_Chapter_5.tex,v 1.13 2005-12-17 05:59:21 pwessel Exp $
%
%	The GMT Documentation Project
%	Copyright 2000-2006.
%	Paul Wessel and Walter H. F. Smith
%------------------------------------------
%
\chapter{\gmt\ Coordinate Transformations}
\label{ch:5}
\thispagestyle{headings}

\GMT\ programs read real-world coordinates and convert them to positions on a plot.
This is achieved by selecting one of several coordinate transformations or projections.
We distinguish between three sets of such conversions:

\begin{itemize}
\item Cartesian coordinate transformations
\item Polar coordinate transformations
\item Map coordinate transformations
\end{itemize}

The next chapter will be dedicated to \GMT\ map projections in its entirety.  Meanwhile, the present chapter
will summarize the properties of the Cartesian and Polar coordinate transformations available in \GMT, list
which parameters define them, and demonstrate how they are used to create simple plot axes.  We will mostly
be using \GMTprog{psbasemap} (and occasionally \GMTprog{psxy}) to demostrate the various transformations.
Our illustrations may differ from those you reproduce with the same commands because of different settings
in our \filename{.gmtdefaults4} file.)  Finally, note that while we will specify dimensions in inches (by
appending {\bf i}), you may want to use cm ({\bf c}), meters ({\bf m}), or points ({\bf p}) as unit instead
(see the \GMTprog{gmtdefaults} man page). 

\input{GMT_Chapter_5.1}
\input{GMT_Chapter_5.2}

%------------------------------------------
%	$Id: GMT_Chapter_5.tex,v 1.13 2005-12-17 05:59:21 pwessel Exp $
%
%	The GMT Documentation Project
%	Copyright 2000-2006.
%	Paul Wessel and Walter H. F. Smith
%------------------------------------------
%
\chapter{\gmt\ Coordinate Transformations}
\label{ch:5}
\thispagestyle{headings}

\GMT\ programs read real-world coordinates and convert them to positions on a plot.
This is achieved by selecting one of several coordinate transformations or projections.
We distinguish between three sets of such conversions:

\begin{itemize}
\item Cartesian coordinate transformations
\item Polar coordinate transformations
\item Map coordinate transformations
\end{itemize}

The next chapter will be dedicated to \GMT\ map projections in its entirety.  Meanwhile, the present chapter
will summarize the properties of the Cartesian and Polar coordinate transformations available in \GMT, list
which parameters define them, and demonstrate how they are used to create simple plot axes.  We will mostly
be using \GMTprog{psbasemap} (and occasionally \GMTprog{psxy}) to demostrate the various transformations.
Our illustrations may differ from those you reproduce with the same commands because of different settings
in our \filename{.gmtdefaults4} file.)  Finally, note that while we will specify dimensions in inches (by
appending {\bf i}), you may want to use cm ({\bf c}), meters ({\bf m}), or points ({\bf p}) as unit instead
(see the \GMTprog{gmtdefaults} man page). 

\input{GMT_Chapter_5.1}
\input{GMT_Chapter_5.2}


%------------------------------------------
%	$Id: GMT_Chapter_5.tex,v 1.13 2005-12-17 05:59:21 pwessel Exp $
%
%	The GMT Documentation Project
%	Copyright 2000-2006.
%	Paul Wessel and Walter H. F. Smith
%------------------------------------------
%
\chapter{\gmt\ Coordinate Transformations}
\label{ch:5}
\thispagestyle{headings}

\GMT\ programs read real-world coordinates and convert them to positions on a plot.
This is achieved by selecting one of several coordinate transformations or projections.
We distinguish between three sets of such conversions:

\begin{itemize}
\item Cartesian coordinate transformations
\item Polar coordinate transformations
\item Map coordinate transformations
\end{itemize}

The next chapter will be dedicated to \GMT\ map projections in its entirety.  Meanwhile, the present chapter
will summarize the properties of the Cartesian and Polar coordinate transformations available in \GMT, list
which parameters define them, and demonstrate how they are used to create simple plot axes.  We will mostly
be using \GMTprog{psbasemap} (and occasionally \GMTprog{psxy}) to demostrate the various transformations.
Our illustrations may differ from those you reproduce with the same commands because of different settings
in our \filename{.gmtdefaults4} file.)  Finally, note that while we will specify dimensions in inches (by
appending {\bf i}), you may want to use cm ({\bf c}), meters ({\bf m}), or points ({\bf p}) as unit instead
(see the \GMTprog{gmtdefaults} man page). 

%------------------------------------------
%	$Id: GMT_Chapter_5.tex,v 1.13 2005-12-17 05:59:21 pwessel Exp $
%
%	The GMT Documentation Project
%	Copyright 2000-2006.
%	Paul Wessel and Walter H. F. Smith
%------------------------------------------
%
\chapter{\gmt\ Coordinate Transformations}
\label{ch:5}
\thispagestyle{headings}

\GMT\ programs read real-world coordinates and convert them to positions on a plot.
This is achieved by selecting one of several coordinate transformations or projections.
We distinguish between three sets of such conversions:

\begin{itemize}
\item Cartesian coordinate transformations
\item Polar coordinate transformations
\item Map coordinate transformations
\end{itemize}

The next chapter will be dedicated to \GMT\ map projections in its entirety.  Meanwhile, the present chapter
will summarize the properties of the Cartesian and Polar coordinate transformations available in \GMT, list
which parameters define them, and demonstrate how they are used to create simple plot axes.  We will mostly
be using \GMTprog{psbasemap} (and occasionally \GMTprog{psxy}) to demostrate the various transformations.
Our illustrations may differ from those you reproduce with the same commands because of different settings
in our \filename{.gmtdefaults4} file.)  Finally, note that while we will specify dimensions in inches (by
appending {\bf i}), you may want to use cm ({\bf c}), meters ({\bf m}), or points ({\bf p}) as unit instead
(see the \GMTprog{gmtdefaults} man page). 

\input{GMT_Chapter_5.1}
\input{GMT_Chapter_5.2}

%------------------------------------------
%	$Id: GMT_Chapter_5.tex,v 1.13 2005-12-17 05:59:21 pwessel Exp $
%
%	The GMT Documentation Project
%	Copyright 2000-2006.
%	Paul Wessel and Walter H. F. Smith
%------------------------------------------
%
\chapter{\gmt\ Coordinate Transformations}
\label{ch:5}
\thispagestyle{headings}

\GMT\ programs read real-world coordinates and convert them to positions on a plot.
This is achieved by selecting one of several coordinate transformations or projections.
We distinguish between three sets of such conversions:

\begin{itemize}
\item Cartesian coordinate transformations
\item Polar coordinate transformations
\item Map coordinate transformations
\end{itemize}

The next chapter will be dedicated to \GMT\ map projections in its entirety.  Meanwhile, the present chapter
will summarize the properties of the Cartesian and Polar coordinate transformations available in \GMT, list
which parameters define them, and demonstrate how they are used to create simple plot axes.  We will mostly
be using \GMTprog{psbasemap} (and occasionally \GMTprog{psxy}) to demostrate the various transformations.
Our illustrations may differ from those you reproduce with the same commands because of different settings
in our \filename{.gmtdefaults4} file.)  Finally, note that while we will specify dimensions in inches (by
appending {\bf i}), you may want to use cm ({\bf c}), meters ({\bf m}), or points ({\bf p}) as unit instead
(see the \GMTprog{gmtdefaults} man page). 

\input{GMT_Chapter_5.1}
\input{GMT_Chapter_5.2}



