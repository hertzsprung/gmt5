\usepackage{ifpdf}
\usepackage{graphicx}
\usepackage{makeidx}
\usepackage{a4wide}
\usepackage{float}
\usepackage{GMT}
\usepackage{times}
\usepackage{color}
\usepackage{mathptmx}
\usepackage{verbatim}
\usepackage{tocbibind}
\usepackage{html}% Implies \usepackage{hyperref}

\input{GMT_version}

\ifpdf
   % Here for creating PDF output with pdfLaTeX
   \pdfcompresslevel=9
   \DeclareGraphicsExtensions{.pdf}
   \hypersetup{%
      pdfauthor={Paul Wessel, Walter H. F. Smith},
      pdftitle={The Generic Mapping Tools Version \GMTDOCVERSION---\GMTTITLE},
      pdfsubject={GMT: Generic Mapping Tools},
      pdfkeywords={GMT, projections, mapping},
      pdfcreator={pdfLaTeX},
      bookmarksopen=true,
      bookmarksnumbered=true,
      hypertexnames=true,
      breaklinks=true,
      %pdfstartview={FitH},
      %linkbordercolor={1 1 0},
      %urlbordercolor={1 0 0},
   }%
   \newcommand{\GMT}{\textit{GMT}}
   \newcommand{\gmt}{\href{http://\GMTSITE}{\textbf{GMT}}}
   \newcommand{\GMTprogi}[1]{\htmladdnormallink{\textsfbf{#1}}{../html/#1.html}}
\else
   % Here for creating PS or HTML output with LaTeX
   \DeclareGraphicsExtensions{.eps}
   \newcommand{\GMT}{\htmladdnormallink{\includegraphics{fig/GMT_glyph10.eps}}{http://\GMTSITE}}
   \newcommand{\gmt}{\htmladdnormallink{GMT}{http://\GMTSITE}}
   \newcommand{\GMTprogi}[1]{\htmladdnormallink{\textsfbf{#1}}{../#1.html}}
\fi

\pagecolor{white}

% GMTfig will insert an eps file, add a label, and set a caption
\newcommand{\GMTfig}[3][tbp]{\begin{figure}[#1]\centering\includegraphics{scripts/#2}\caption{#3}\label{fig:#2}\end{figure}}
% Same for examples (scales by 50% by default)
\newcommand{\GMTexample}[3][scale=0.5]{\begin{figure}[hbtp]\centering\includegraphics[#1]{scripts/example_#2}\caption{#3}\label{fig:GMT_example_#2}\end{figure}}

\newcommand{\PS}{\textit{PostScript}}
\newcommand{\UNIX}{\textit{UNIX}}
\newcommand{\id}[1]{#1\index{#1}}

\newcommand{\textsfbf}[1]{{\sffamily\bfseries #1\/}}
\newcommand{\textslbf}[1]{{\slshape\bfseries #1\/}}

\newcommand{\GMTprog}[1]{\GMTprogi{#1}\index{#1@\textsfbf{#1}}}

\newcommand{\script}[1]{%
   \begingroup
   \scriptsize\vspace{0.25\baselineskip}\noindent\hrulefill\vspace{-0.75\baselineskip}%
   \verbatiminput{scripts/#1.tex}%
   \vspace{-1.25\baselineskip}\noindent\hrulefill\vspace{0.25\baselineskip}%
   \endgroup
}

\newcommand{\GMTfunc}[1]{\texttt{#1}}
\newcommand{\filename}[1]{\underline{#1}}

\setcounter{topnumber}{2}
\renewcommand{\topfraction}{.8}
\setcounter{bottomnumber}{1}
\renewcommand{\bottomfraction}{.7}
\setcounter{totalnumber}{3}
\renewcommand{\textfraction}{.2}
\renewcommand{\floatpagefraction}{.7}

\sloppy

\newcommand{\DS}{$^{\circ}$}
\newcommand{\PM}{$\pm$}
\newcommand{\progname}[1]{\textslbf{#1}\index{#1@\textslbf{#1}}}
\newcommand{\Opt}[1]{\textbf{--#1}}

% Now overrule a few things for running LaTeX2HTML.
% Note that we need \gdef because they are inside an environment and would otherwise be local to
% that.
\begin{htmlonly}
   \gdef\GMT{\htmladdnormallink{\textbf{GMT}}{http://\GMTSITE}}
   \gdef\gmt{\GMT}
   \gdef\script#1{\htmlrule\verbatiminput{scripts/#1.tex}\htmlrule}
   \gdef\DS{�}
   \gdef\Opt#1{\textbf{-#1}}
   \gdef\PM{�}
   \gdef\progname#1{\textit{#1}\index{#1@\textit{#1}}}
\end{htmlonly}

% Restrict the indentation of lists
\setlength\leftmargin\parindent
\setlength\leftmargini\parindent
\setlength\leftmarginii\parindent
\setlength\leftmarginiii\parindent
\setlength\leftmarginiv\parindent
\setlength\leftmarginv\parindent
\setlength\leftmarginvi\parindent
