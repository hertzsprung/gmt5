%	$Id$
% Internal documentation for adding a ternary diagram mode
% Paul Wessel, April 21, 2015.
\documentclass[12pt,letterpaper,margin=0.5in]{report}
\usepackage{times}
\usepackage{graphicx}
\usepackage{breqn}
\usepackage[margin=0.5in]{geometry}
\usepackage{lscape}
\textheight = 9 in
\topmargin = -1 in
\begin{document}

\section*{ADDING TERNARY DIAGRAMS TO GMT}

Given that ternary diagrams may contain symbols, lines, contours, images, and text it may
be simplest to implement ternary plots via a projection that converts (a,b,c) to (x,y).
However, most GMT tools expect (x,y) for positioning and not (a,b,c).  Thus, it would require
much work to allow modules to read (a,b,c) and internally determine (x,y) which is what
the rest of the module may need.  Also, the standard -R option is Cartesian and not easily
changeable to deal with ternary specification.  One possibility
is to have a single new module ternary that may (depending on options)
\begin{itemize}
	\item Create a ternary basemap \emph{PostScript} overlay.  This is used to present the standard triangle with labels and annotations.
	  It may need a specialized ``-B'' option to set 3 different axes (4 if 3-D).
	\item Initiate a triangular clip path.  This is needed so that the plotting that follow (of x,y data using standard
		linear projection overlays) is clipped to the triangular domain.
	\item Terminate the triangular clip path when done.
	\item Project (a,b,c) to (x,y) so that other GMT modules can plot or processed data further. For instance, we may wish to pipe
		the output from this module through psxy to plot points.
\end{itemize}
If we did it this way then only one new module would be needed and it would uniquely handle the special input type
and projection needed.  Here, perhaps something other than the typical -R may be needed and the same with the size
of the plot.  Other modules would use -JXwidth to overlay material.  Assuming we don't have crazy stuff like geographical coordinates
and time along these axes we could do
\begin{itemize}
\item {\bf -D}{\it a0/a1/b0/b1/c0/c1[/z0/z1]} to specify domain.
\item {\bf -T}{\it width} to specify base width.
\item {\bf -Aaa}{\it int}{\bf f}{\it int}{\bf g}{\it int}, {\bf -Aba}{\it int}{\bf f}{\it int}{\bf g}{\it int},
{\bf -Aca}{\it int}{\bf f}{\it int}{\bf g}{\it int}, {\bf -Aza}{\it int}{\bf f}{\it int}{\bf g}{\it int} for annotation intervals etc.
Probably add {\bf +l}``label'' to each of these to label the corner where a,b,z is 100\%.
\end{itemize}
\end{document}
