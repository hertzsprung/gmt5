%------------------------------------------
%	$Id$
%
%	The GMT Documentation Project
%	Copyright (c) 2000-2012.
%	P. Wessel, W. H. F. Smith, R. Scharroo, and J. Luis
%------------------------------------------
%
\chapter{\PS\ fonts used by \gmt}
\label{app:G}
\index{Font!standard}
\thispagestyle{headings}

\GMT\ uses the standard 35 fonts that come with most \PS\
laserwriters.  If your printer does not support some of these
fonts, it will automatically substitute the default font (which is
usually Courier).  The following is
a list of the \GMT\ fonts: \\

\GMTfig[h]{GMT_App_G}{The standard 35 \PS\ fonts recognized by \gmt.}

For the special fonts Symbol (12) and ZapfDingbats (34), see the
octal charts in Appendix~\ref{app:F}.  When specifying fonts in
\GMT, you can either give the entire font name \emph{or} just the
font number listed in this table.  To change the fonts used in
plotting basemap frames, see the man page for
\GMTprog{gmt.conf}.  For direct plotting of text-strings, see
the man page for \GMTprog{pstext}.


\section{Using non-default fonts with \GMT}
\label{sec:non-default-fonts}

To add additional fonts that you may have purchased or that are
available freely in the internet or at your institution, see the
instructions in the \filename{CUSTOM\_font\_info.d} under the
\filename{share/pslib} directory and continue reading.
%
\GMT\ does not read or process any font files and thus does not
know anything about installed fonts and their metrics. In order to
use extra fonts in \GMT\ you need to specify the \PS\ name of the
relevant fonts in the file \filename{CUSTOM\_font\_info.d}. You
can either edit the existing file distributed with \GMT\ to make
the changes global or you can create a new file in the current
working directory, e.g.:
%
\begin{verbatim}
LinBiolinumO      0.700    0
LinLibertineOB    0.700    0
\end{verbatim}
%
The format is a space delimited list of the \PS\ font name, the
font height-point size-ratio, and a boolean variable that tells
\GMT\ to re-encode the font (if set to zero). The latter has to be
set to zero as additional fonts will most likely not come in
standard \PS\ encoding. \GMT\ determines how tall typical
annotations might be from the font size ratio so that the vertical
position of labels and titles can be adjusted to a more uniform
typesetting. Now, you can set the \GMT\ font parameters to your
non-standard fonts:
%
\begin{verbatim}
gmtset FONT LinBiolinumO \
 FONT_TITLE 28p,LinLibertineOB \
 PS_CHAR_ENCODING ISO-8859-1 \
 MAP_DEGREE_SYMBOL degree
\end{verbatim}
%
After setting the encoding and the degree symbol, the
configuration part for \GMT\ is finished and you can proceed to
create \GMT-maps as usual. An example script is discussed in
Section~\ref{sec:non-default-fonts-example}.


\subsection{Embedding fonts in PostScript and PDF}

If you have Type 1 fonts in PFA (Printer Font ASCII) format you
can embed them directly by copying them at the very top of your
\PS-file, before even the \%!PS header comment. PFB (Printer Font
Binary), TrueType or OpenType fonts cannot be embedded in \PS\
directly and therefore have to be converted to PFA first.

However, you most likely will have to tell \progname{Ghostscript}
where to find your custom fonts in order to convert your
\GMT-\PS-plot to PDF or an image with \GMTprog{ps2raster}. When
you have used the correct \PS-names of the fonts in
\filename{CUSTOM\_font\_info.d} you only need to point the
\verb#GS_FONTPATH# environment variable to the directory where the
font files can be found and invoke \GMTprog{ps2raster} in the
usual way. Likewise you can specify \progname{Ghostscript}'s
\verb#-sFONTPATH# option on the command line with
\Opt[-sFONTPATH=/path/to/fontdir]{-C}.  \progname{Ghostscript},
which is invoked by \GMTprog{ps2raster}, does not depend on file
names. It will automatically find the relevant font files by their
\PS-names and embed and subset them in PDF-files. This is quite
convenient as the document can be displayed and printed even on
other computers when the font is not available locally. There is
no need to convert your fonts as \progname{Ghostscript} can handle
all Type 1, TrueType and OpenType fonts. Note also, that you do
not need to edit \progname{Ghostscript}'s Fontmap.GS.

If you do not want or cannot embed the fonts you can convert them
to outlines (shapes with fills) with \progname{Ghostscript} in the
following way:
\begin{verbatim}
gs -q -dNOCACHE -dSAFER -dNOPAUSE -dBATCH -dNOPLATFONTS \
  -sDEVICE=pswrite -sFONTPATH="/path/to/fontdir" \
  -sOutputFile=mapWithOutlinedFonts.ps map.ps
\end{verbatim}
Note, that this only works with the \emph{pswrite} device. If you
need outlined fonts in PDF, create the PDF from the converted
\PS-file.  Also, \GMTprog{ps2raster} cannot correctly crop
\progname{Ghostscript} converted \PS-files anymore. Use Heiko
Oberdiek's
\htmladdnormallinkfoot{\progname{pdfcrop}}{http://code.google.com/p/pdfcrop2/}
% \GMTprog{pdfcrop}
instead or crop with \GMTprog{ps2raster} \Opt{A} \Opt{Te} before
(See Example~\ref{sec:non-default-fonts-example}).


\subsection{Character encoding}

Since \PS\ itself does not support Unicode fonts, \progname{Ghostscript} will
re-encode the fonts on the fly. You have to make sure to set the
correct \verb#PS_CHAR_ENCODING# with gmtset and save your script
file with the same character encoding. Alternatively, you can
substitute all non ASCII characters with their corresponding octal
codes, e.g., \textbackslash 265 instead of \textmu. Note, that
\PS\ fonts support only a small range of glyphs and you may have
to switch the \verb#PS_CHAR_ENCODING# within your script.
