%------------------------------------------
%	$Id$
%
%	The GMT Documentation Project
%	Copyright (c) 2000-2011.
%	P. Wessel, W. H. F. Smith, R. Scharroo, and J. Luis
%------------------------------------------
%
\chapter{Custom Plot Symbols}
\label{app:N}
\index{symbols!custom}
\thispagestyle{headings}

\section{Background}
\index{Custom Symbols}
\index{Symbols, custom}

The \GMT\ tools \GMTprog{psxy.c} and \GMTprog{psxyz.c} are capable of using
custom symbols as alternatives to the built-in, standard geometrical shapes
like circles, triangles, and many others.  One the command line, custom
symbols are selected via the \Opt{Sk}\emph{symbolname}[.\emph{size}] symbol
selection, where \emph{symbolname} refers to a special symbol definition
file called \filename{symbolname.def} that must be available via the standard
\GMT\ user paths.  Several custom symbols comes pre-configured with \GMT\
(see Figure~\ref{fig:GMT_App_N_1})

\GMTfig[h]{GMT_App_N_1}{Custom plot symbols supported by \gmt. Note that we
only show the symbol outline and not any fill.  These are all single-parameter symbols.
Be aware that some symbols may have a hardwired fill or no-fill component,
while others duplicate what is already available as standard built-in symbols.}

You may find it convenient to examine some of these and use them as a starting
point for your own design; they can be found in \GMT's share/custom directory.

% The following file will contain more \GMTfig commands when needed.
% \input{GMT_Appendix_N_inc.tex}
\section{The macro language}

To make your own custom plot symbol, you will need to design your own *.def files.
This section defines the language used to build custom symbols.  You can place these
definition files in your current directory or your .gmt user directory.  When designing
the symbol, you are doing so in a relative coordinate system centered on (0,0).  This point
will be mapped to the actual location specified by your data coordinates.  Furthermore,
your symbol should be constructed within the domain ${-\frac{1}{2},+\frac{1}{2},-\frac{1}{2},+\frac{1}{2}}$,
resulting in a 1 by 1 relative canvas.  This 1 x 1 box will be scaled by your actual
symbol size when plotted.

\subsection{Comment lines}

Your definition file may have any number of comment lines, defined to begin with
the character \#.  These are skipped by \GMT\ but provides a mechanism for you to
clarify what your symbol does.

\subsection{Symbol variables}

Simple symbols, such as circles and triangles, only take a single parameter: the
symbol size, which is either given on the command line (via \Opt{Sk}) or as part
of the input data.  However, more complicated symbols, such as the ellipse or vector symbols,
may require more parameters.  If your custom symbol requires more than the single size parameter
you must include the line
\begin{quote}
	\textbf{N}: \emph{n\_extra\_parameters} [\emph{types}]
\end{quote}
before any other macro commands.  It is an optional statement in that \emph{n\_extra\_parameters} will
default to 0 unless explicitly set.  By default the extra parameters are considered to be quantities
that should be passed directly to the symbol machinery.  However, you can use the \emph{types} argument
to specify different types of parameters.  The available types are
\begin{description}
	\item [a]: Geographic angle, to be converted to map angle given the current map projection.
	\item [l]: Length, i.e., an additional length scale (in cm, inch, or point as per {\bf PROJ\_LENGTH\_UNIT}) in addition to the given symbol size.
	\item [o]: Other, i.e., a quantity to be passed to the custom symbol as is.
\end{description}
To use the extra parameters in your macro you address them as \$1, \$2, etc.

\subsection{Macro commands}

The custom symbol language contains commands to rotate the relative coordinate system,
draw free-form polygons and lines, change the current fill and/or pen, and include basic
geometric symbols as part of the overall design (e.g., circles, triangles, etc.).
The available commands are listed in
Table~\ref{tbl:custsymb}.
\begin{table}[H]
\small
\centering
\begin{tabular}{|l|c|l|l|} \hline
Name 		& 	Code		&	Purpose				&	Arguments \\ \hline
rotate		&	\textbf{R}	& Rotate the coordinate system		& $\alpha$ \\ \hline
moveto		&	\textbf{M}	& Set a new anchor point 		& $x_0, y_0$  \\ \hline
drawto		&	\textbf{D}	& Draw line from previous point 	& $x, y$  \\ \hline
arc		&	\textbf{A}	& Append circular arc to existing path 	& $x_c, y_c, r, \alpha_1, \alpha_2$  \\ \hline
stroke		&	\textbf{S}	& Stroke existing path only 		&  \\ \hline \hline
texture		&	\textbf{T}	& Change current pen and fill		&  \\ \hline \hline
star		&	\textbf{a}	& Plot a star 				& $x, y$,\emph{size}  \\ \hline
circle		&	\textbf{c}	& Plot a circle 			& $x, y$,\emph{size}  \\ \hline
diamond		&	\textbf{d}	& Plot a diamond 			& $x, y$,\emph{size}  \\ \hline
ellipse		&	\textbf{e}	& Plot a ellipse 			& $x, y, \alpha$,\emph{major},\emph{minor}  \\ \hline
octagon		&	\textbf{g}	& Plot an octagon 			& $x, y$,\emph{size}  \\ \hline
hexagon		&	\textbf{h}	& Plot a hexagon 			& $x, y$,\emph{size}  \\ \hline
invtriangle	&	\textbf{i}	& Plot an inverted triangle		& $x, y$,\emph{size}  \\ \hline
letter		&	\textbf{l}	& Plot a letter 			& $x, y$,\emph{size}, \emph{string}  \\ \hline
marc		&	\textbf{m}	& Plot a math arc 			& $x, y, r, \alpha_1, \alpha_2$  \\ \hline
pentagon	&	\textbf{n}	& Plot a pentagon 			& $x, y$,\emph{size}  \\ \hline
plus		&	\textbf{+}	& Plot a plus sign 			& $x, y$,\emph{size}  \\ \hline
rect		&	\textbf{r}	& Plot a rectangle 			& $x, y$, \emph{width}, \emph{height}  \\ \hline
square		&	\textbf{s}	& Plot a square 			& $x, y$,\emph{size}  \\ \hline
triangle	&	\textbf{t}	& Plot a triangle 			& $x, y$,\emph{size}  \\ \hline
wedge		&	\textbf{w}	& Plot a wedge 				& $x, y, r, \alpha_1, \alpha_2$  \\ \hline
cross		&	\textbf{x}	& Plot a cross 				& $x, y$,\emph{size}  \\ \hline
x-dash		&	\textbf{-}	& Plot a x-dash 			& $x, y$,\emph{size}  \\ \hline
y-dash		&	\textbf{y}	& Plot a y-dash 			& $x, y$,\emph{size}  \\ \hline
\end{tabular}
\label{tbl:custsymb}
\caption{List of macro codes used to define custom symbols.  Note that arguments are given first with the symbol code at the end.
All angles are expected in degrees. The \textbf{S} macro is only used if you just want to draw a line and not close/fill it as
a polygon [Default]; place it on its own line after your last path construction operator.}
\end{table} 
For \textbf{M}, \textbf{T}, and all the lower-case symbol codes you may optionally append specific pens (with \Opt{W}\emph{pen})
and fills (with \Opt{G}\emph{pen}).  These settings will override the pens and fills you may have specified
on the command line.  Passing \Opt{G}- or \Opt{W}- means no fill or outline, respectively.

\subsection{Conditional statements}

There are two types of conditional statements in the macro language: A simple condition
preceding a single command, or a more elaborate if--then--elseif--else construct.  In any test you
may use one (and only one) of many logical operators, as listed in Table~\ref{tbl:custop}.
\begin{table}[H]
\small
\centering
\begin{tabular}{|c|l|} \hline
Operator 	& 	Purpose					\\ \hline
$<$ 		& 	Is \emph{var} less than \emph{constant}?		\\ \hline
$<$= 		& 	Is \emph{var} less than or equal to \emph{constant}?		\\ \hline
== 		& 	Is \emph{var} equal to \emph{constant}?		\\ \hline
!= 		& 	Is \emph{var} not equal to \emph{constant}?		\\ \hline
$>$= 		& 	Is \emph{var} greater than or equal to \emph{constant}?		\\ \hline
$>$ 		& 	Is \emph{var} greater than \emph{constant}?		\\ \hline
\% 		& 	Does \emph{var} have a remainder with \emph{constant}?		\\ \hline
!\% 		& 	Is \emph{var} an exact multiple of \emph{constant}?		\\ \hline
$<>$ 		& 	Is \emph{var} within the exclusive range of \emph{constant}?	\\ \hline
[] 		& 	Is \emph{var} within the inclusive range of \emph{constant}?	\\ \hline
$<$] 		& 	Is \emph{var} within the in/ex-clusive range of \emph{constant}?	\\ \hline
[$>$		& 	Is \emph{var} within the ex/in-clusive range of \emph{constant}?	\\ \hline
\end{tabular}
\label{tbl:custop}
\caption{List of operators to use for logical tests.  The equality and inequality operators
may be compared to NaN.  Note that the last four range operators
expect \emph{constant} to contain two colon-separated values (e.g., 0:9.5).}
\end{table} 

\subsubsection{Simple conditional test}
The simple if-test uses a one-line format, defined as
\begin{quote}
	\textbf{if} \emph{var OP constant} \textbf{then}  \emph{command}
\end{quote}
where \emph{var} must be one of the symbol parameters, specified as \$1, \$2, \$3, etc.
You must document what these additional parameters control.  For example, to plot a
small cyan circle at (0.2, 0.3) with diameter 0.4 only if \$2 exceeds 45 you would write
\begin{verbatim}
	if $2 > 45 then 0.2 0.3 0.4 c -Gcyan
\end{verbatim}
Note that this form of the conditional test has no mechanism for an \textbf{else} branch, but this can be accomplished
by repeating the test but reversing the logic for the second copy, e.g.,
\begin{verbatim}
	if $1 > 10 then 0 0 0.5 c -Gred
	if $1 <= 10 then 0 0 0.5 c -Gblue
\end{verbatim}
or you may instead consider the complete conditional construct below.
\subsubsection{Complete conditional test}

The complete conditional test uses a multi-line format, such as
\begin{tabbing}
	\textbf{if} \emph{var} \= \emph{OP constant} \textbf{then}  \{ \\
	\> $<$one or more lines with commands$>$ \\
	\} \textbf{elseif} \emph{var OP constant} \textbf{then}  \{ \\
	\> $<$one or more lines with commands$>$ \\
	\} \textbf{else} \{ \\
	\> $<$one or more lines with commands$>$ \\
	\}
\end{tabbing}
The \textbf{elseif} (one or more) and \textbf{else} branches are optional.  Note that the syntax is strictly enforced,
meaning the opening brace must appear after \textbf{then} with nothing following it, and the closing brace
must appear by itself with no other text, and that the \textbf{elseif} and \textbf{else} statements must have both closing
and opening braces on the same line (and nothing else).  You may nest tests as well (up to 10 levels deep), e.g.,
\begin{verbatim}
if $1 > 45 then {
        if $2 [> 0:10 then 0 0 0.5 c -Gred
} elseif $1 < 15 then {
        if $2 [> 0:10 then 0 0 0.5 c -Ggreen
} else {
        if $2 [> 10:20 then {
                0 0 M -W1p,blue
                0.3 0.3 D
                S
                0.3 0.3 0.3 c -Gcyan
        }
}
\end{verbatim}

%\subsection{PostScript symbols}
% [This may or may not be implemented in the future ]
%For very complicated symbols it may in fact be simpler to design the entire symbol in
%the \PS\ language.  In that case, your definition file may only contain the macro code
%\textbf{N} (if needed), and the rest of the file should contain the \PS\ code for the symbol,
%starting with the naming \texttt{/Sk\_}\emph{symbolname} and ending with the line \texttt{def}.  Your
%\PS\ code will be passed the symbol size and all \emph{n\_extra\_parameters} you have specified (if any).
%\GMT\ will automatically encapsulate the symbol within a \emph{gsave--grestore} pair, thus limiting
%any changes to the \PS\ state you may cause to the symbol execution only.
