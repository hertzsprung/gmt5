%------------------------------------------
%	$Id: GMT_Appendix_P.tex,v 1.5 2005-07-12 04:13:24 pwessel Exp $
%
%	The GMT Documentation Project
%	Copyright 2000-2005.
%	Paul Wessel and Walter H. F. Smith
%------------------------------------------
%

\chapter{Using both \gmt\ 3 and 4}
\index{\GMT, using 3 and 4}
\thispagestyle{headings}

We encourage all \GMT\ users to start using version 4 immediately; it has been tested extensively by
the \GMT\ team and has benefitted from bug reports for the 3.4.x versions.  Users who still worry about the
new version breaking things may install both \GMT\ 3.4.5 and 4 and use our utility \progname{gmtswitch}
to select their current version should the need to switch arises.  You will find \progname{gmtswitch}
in the top-level \GMT 4 directory; install as explained below.

Because \GMT\ 4 is backwards compatible with the 3.4.x series yet maintains its parameters
and history in separate hidden files (e.g., \filename{.gmtdefaults4} versus \filename{.gmtdefaults})
it is possible to install and use both versions on the same workstation.  To simplify such
setups we supply the utility \progname{gmtswitch} which simplifies switching back and forth
between any number of installed \GMT\ 3-versions and \GMT\ 4.  Place the \progname{gmtswitch} Bourne shell script in your
general executable path (not in one of the \GMT\ bin directories) and run it after you have
finished installing all \GMT\ versions of interest.  The first time you run \progname{gmtswitch}
it will try to find all the available versions installed on your file system.  The versions
found will be listed in the file \filename{.gmtversions} in your home directory; each line
is the full path to a \GMT\ root directory (e.g., /usr/local/GMT3.4.2).  You may
manually add or remove entries there at any time.  You are then instructed to make two
changes to your environment (the details are shell-dependent but explained by \progname{gmtswitch}):
\begin{enumerate}
\item Define the environmental variable {\bf GMTHOME} to point to {\bf \$HOME}/this\_gmt,
where {\bf \$HOME} is the full path to your home directory.
Here, \filename{this\_gmt} is a symbolic link that will be created and maintained by \progname{gmtswitch}
to point to a directory with one of the installed versions. 
\item Make sure {\bf \$GMTHOME}/bin is in your executable {\bf PATH}.
\end{enumerate}
Make those edits, logout, and log and back in again.  The next time you run \progname{gmtswitch}
you will be able to switch between versions.  Typing \progname{gmtswitch} with no argument will list the
available versions in a numerical menu and prompt you to choose one, whereas \progname{gmtswitch} {\it version}
will immediately switch to that version ({\it version} must be a piece of unique text making
up the full path to a verson, e.g., 3.4.2).  If you use \progname{tcsh} or \progname{csh} you may have to type
``rehash'' to initiate the path changes.
