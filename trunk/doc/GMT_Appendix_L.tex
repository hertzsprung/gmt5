%------------------------------------------
%	$Id: GMT_Appendix_L.tex,v 1.5 2003-01-14 02:38:03 pwessel Exp $
%
%	The GMT Documentation Project
%	Copyright 2000-2003.
%	Paul Wessel and Walter H. F. Smith
%------------------------------------------
%
\chapter{\gmt\ on non-\UNIX\ platforms}
\thispagestyle{headings}

\section{Introduction}
\index{GMT@\GMT!on non-\UNIX\ platforms}
\index{GMT@\GMT!Macs running MkLinux}
\index{GMT@\GMT!Macs running MachTen}
\index{GMT@\GMT!PCs running Interix}
\index{GMT@\GMT!PCs running Linux}

While \GMT\ can be ported to non-\UNIX\ systems such as
Windows and MacOS, it is also true that one of the
strengths of \GMT\ lies its symbiotic relationship with
\UNIX.  We therefore recommend that \GMT\ be installed in
a POSIX-compliant \UNIX\ PC environment such as Linux (PC)
or MkLinux (Mac).  There are also commercial products
for PCs (e.g., Interix [formerly OpenNT]\footnote{www.interix.com})
and Macs (e.g., MachTen\footnote{www.tenon.com}) that will
provide a POSIX environment
without rebooting into \UNIX.  Installation of \GMT\ under
Interix or Machten is no different than under other POSIX
\UNIX systems.

However, if you own a PC and need a public domain,
no-cost solution other than Linux you have a few
additional options.  At the time of this writing they
include

\begin{enumerate}

\item Install \GMT\ under Cygwin (A GNU port to Windows). 

\item Install \GMT\ under DJGPP (another GNU port to Windows/DOS).

\item Install \GMT\ directly using Microsoft C/C++ or other
compilers.
\index{GMT@\GMT!compile with Microsoft C/C++}

\end{enumerate}

Unlike the first two, the latter will not provide you with any
\UNIX\ tools so you will be limited to what you can do with
DOS batch files.

\section{Cygwin and \gmt}
\index{GMT@\GMT!under Cygwin|(}
\index{Cygwin|(}

Because \GMT\ works best in conjugation with \UNIX\ tools we
suggest you install \GMT\ using the Cygwin product from
Cygnus (now assimilated by Redhat, Inc.).  This free version works on any Windows version
and it comes with both the Bourne Again shell \progname{bash} and the \progname{tcsh}.
You also have access to most standard GNU development tools such
as compilers and text processing tools (\progname{awk},
\progname{grep}, \progname{sed}, etc.).

Follow the instructions on the Cygwin page\footnote{sources.redhat.com/cygwin} on how
to install the package; note you must explicitly add all the development tool
packages (e.g., \progname{gcc} etc) as the basic installation does not include them by default.
Once you are up and running under Cygwin, you may install \GMT\ 
the same way you do under any other \UNIX\ platform by either
running the automated install via \progname{install\_gmt} or manually
running configure first, then type make all.
For details see the general README file.

\index{GMT@\GMT!under Cygwin|)}
\index{Cygwin|)}

\section{DJGPP and \gmt}
\index{GMT@\GMT!under DJGPP|(}
\index{DJGPP|(}

DJGPP\footnote{See www.gnu.org for details.} is similar to Cygwin
in that it provides precompiled \UNIX\ tools for DOS/WIN32,
including the \progname{bash} shell.  At the time of this writing we
have not been successful in compiling netCDF in this
environment.  This is fully due to our limited understanding
of the innards of the netCDF installation whose configure
script did not work for us.  As soon as this problem is
overcome we expect a smooth install similar to that of Cygwin.
\index{GMT@\GMT!under DJGPP|)}
\index{DJGPP|)}

\section{WIN32 and \gmt}
\index{GMT@\GMT!under Win32|(}
\index{Win32 and \GMT|(}

\GMT\ will compile and install using the Microsoft Visual C/C++
compiler.  We expect other WIN32 C compilers to give similar
results.  Since \progname{configure} cannot be run you
must manually rename \filename{gmt\_notposix.h.in} to
\filename{gmt\_notposix.h}.  The netCDF home page gives full
information on how to compile and install netCDF; precompiled
libraries are also available.  At present we simply have a lame \filename{gmtinstall.bat}
file that compiles the entire \GMT\
package, and \filename{gmtsuppl.bat} which compiles most of the
supplemental programs.  If you just need to run \GMT\ and do not want to mess with compilations,
get the precompiled binaries from the \GMT\ ftp sites.

\index{GMT@\GMT!under Win32|)}
\index{Win32 and \GMT|)}

\section{OS/2 and \gmt}
\index{GMT@\GMT!under O/S2}
\index{O/S2 and \GMT}

\GMT\ has been ported to OS/2 by \htmladdnormallinkfoot{Allen Cogbill}{mailto:ahc@lanl.gov},
Los Alamos National Laboratory.
One must have \htmladdnormallinkfoot{EMX}{ftp://ftp.geophysics.lanl.gov/pub/EES3/pub/gmt/emxrt.zip}
installed in order to use the executables.  All features 
that are present in the \UNIX\ version of \GMT\ are available in the OS/2 version.
All executables may be obtained using links in the following
\htmladdnormallinkfoot{document}{ftp://ees.lanl.gov/pub/EES3/pub/gmt/gmt4os2.html},
which provides more detail on the port.

\section{MacOS and \gmt}
\index{GMT@\GMT!under MacOS}
\index{MacOS and \GMT}

\GMT\ has not been ported to the classical Macintosh platform
(i.e. MacOS 9.x or earlier).  For that OS your only option is MachTen.
However, \GMT\ will install directly under MacOS X.
