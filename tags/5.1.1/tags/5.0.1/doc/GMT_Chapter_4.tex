%------------------------------------------
%	$Id$
%
%	The GMT Documentation Project
%	Copyright (c) 2000-2012.
%	P. Wessel, W. H. F. Smith, R. Scharroo, and J. Luis
%------------------------------------------
%
\chapter{General features}
\label{ch:4}
\thispagestyle{headings}

This section explains features common to all the programs
in \GMT\ and summarizes the philosophy behind the system.  Some
of the features described here may make more sense once you reach
the cook-book section where we present actual examples of their use. 

\section{\gmt\ units}
\index{GMT@\GMT!units|(}
\index{Dimensions|(}
\index{Units|(}

While \GMT\ has default units for both actual Earth distances and
plot lengths (dimensions) of maps, it is recommended that you specifically indicate
the units of your arguments by appending the unit character, as discussed below.
This will aid you in debugging, let others understand your scripts, and remove
any uncertainty as to what unit you thought you wanted.

\subsection{Distance units}

For Cartesian data and scaling the data units do not normally matter (they could be
kg or Lumens for all we know) and are never entered.  Geographic data are different
as distances can be specified in a variety of ways.  \GMT\ programs that accept actual
Earth length scales like search radii or distances can therefore
handle a variety of units.  These choices are listed in Table~\ref{tbl:distunits};
simply append the desired unit to the distance value you supply.  A value without
a unit suffix will be consider to be in meters.  For example, a distance of
30 nautical miles should be given as 30\textbf{n}.
\begin{table}[H]
\centering
\index{Distance units}%
\index{Units, distance}%
\begin{tabular}{|l|l||l|l|} \hline
\multicolumn{1}{|c|}{\emph{Suffix}} & \multicolumn{1}{c|}{\emph{Distance unit}} & \multicolumn{1}{|c|}{\emph{Suffix}} & \multicolumn{1}{c|}{\emph{Distance unit}} \\ \hline
\textbf{d}	&	Degrees of arc		& \textbf{m}	&	Minutes of arc  \\ \hline
\textbf{e}	&	Meters [Default]	& \textbf{M}	&	Statute miles   \\ \hline
\textbf{f}	&	Feet 			& \textbf{n}	&	Nautical miles  \\ \hline
\textbf{k}	&	Kilometers		& \textbf{s}	&	Seconds of arc  \\ \hline
\end{tabular}
\caption{Distance units recognized by \gmt\ command line switches.}
\label{tbl:distunits}
\end{table} 

\subsection{Distance calculations}

The calculation of distances on Earth (or other planetary bodies) depends on the
ellipsoidal parameters of the body and the method of computation.  \GMT\ offers
three alternatives that trade off accuracy and computation time.

\subsubsection{Flat Earth distances}
Quick, but approximate ``Flat Earth'' calculations make a first-order correction
for the spherical nature of a planetary body by computing the distance between
two points A and B as
\begin{equation}
	d_f = R \sqrt{(\theta_A - \theta_B)^2 + \frac{\theta_A + \theta_B}{2}(\lambda_A - \lambda_B)^2},
	\label{eq:flatearth}
\end{equation}
where $R$ is the mean (or spherical) radius of the planet, $\theta$ is latitude, and the difference in longitudes, $\Delta \lambda$,
is adjusted for any jumps that might occur across Greenwich or the Dateline.  This approach is suitable
when the points you use to compute $d_f$ do not greatly differ in latitude and computation
speed is paramount. You can specify this mode of computation by using the \textbf{-} prefix to
the specified distance (or to the unit itself in cases where no distance is required and only a unit is expected).
For instance, a search radius of 50 statute miles using this mode of computation might be specified via \Opt{S-}50\textbf{M}.

\subsubsection{Great circle distances}
This is the default distance calculation, which will also approximate the planetary body by a sphere of mean
radius $R$. However, we compute an exact distance between two points A and B on such a sphere via
the Haversine equation
\begin{equation}
	d_g = 2R \sin^{-1}  {\sqrt{\sin^2\frac{\theta_A - \theta_B}{2} + \cos \theta_A \cos \theta_B \sin^2 \frac{\lambda_A - \lambda_B}{2}} },
	\label{eq:greatcircle}
\end{equation}
This approach is suitable for most situations unless exact calculations for an ellipsoid
is required (typically for a limited surface area).  For instance, a search radius of 5000 feet using this
mode of computation would be specified as \Opt{S}5000\textbf{f}.

\subsubsection{Geodesic distances}
For the most accurate calculations we use a full ellipsoidal formulation.  Currently,
we are using Rudoe's formula.  You select this mode of computation by using the \textbf{+} prefix to
the specified distance (or to the unit itself in cases where no distance is required).
For instance, a search radius of 20 km using this mode of computation would be set by \Opt{S+}20\textbf{k}.

\subsection{Length units}

\GMT\ programs can accept dimensional quantities and plot lengths
in \textbf{c}m, \textbf{i}nch,
or \textbf{p}oint (1/72 of an inch)\footnote{\PS\ definition.
In the typesetting industry a slightly different definition of point
(1/72.27 inch) is used.}.  There are two ways to ensure that \GMT\ understands
which unit you intend to use:

\begin{enumerate}
\item Append the desired unit to the dimension you supply.  This
way is explicit and clearly communicates what you intend, e.g.,
\Opt{X}4\textbf{c} means the length being passed to the \Opt{X} switch is 4 cm.

\item Set the parameter \textbf{PROJ\_LENGTH\_UNIT} to the desired unit.  Then, all
dimensions without explicit unit will be interpreted accordingly.

\end{enumerate}
The latter method is less secure as other users may have a different unit
set and your script may not work as intended.  We therefore recommend
you always supply the desired unit explicitly.
\index{GMT@\GMT!units|)}
\index{Dimensions|)}
\index{Units|)}

\section{\gmt\ defaults}
\label{sec:gmt.conf}
\subsection{Overview and the \protect\filename{gmt.conf}\ file}

\index{GMT@\GMT!defaults|(}
\index{Default settings|(}

\GMTfig[h]{GMT_Defaults_1a}{Some \gmt\ parameters that affect plot appearance.}

There are about 100 parameters which can be adjusted individually
to modify the appearance of plots or affect the manipulation of data.
When a program is run, it initializes all parameters to the \GMT\
defaults\footnote{Choose between SI and US default units by modifying
\filename{gmt.conf} in the \GMT\ share directory.},
then tries to open the file \filename{gmt.conf}
in the current directory\footnote{To remain backwards compatible with \GMT\ 4.x
we will also look for \filename{.gmtdefaults4} but only if \filename{gmt.conf}
cannot be found.}.  If not found, it will look for that file in a sub-directory
\filename{\~/.gmt} of your home directory, and finally in your home directory itself.  If successful, the program will read the
contents and set the default values to those provided in the file.
By editing this file you can affect features such as pen thicknesses
used for maps, fonts and font sizes used for annotations and labels,
color of the pens, dots-per-inch resolution of the hardcopy device,
what type of spline interpolant to use, and many other choices
(A complete list of all the parameters and their default values can
be found in the \GMTprog{gmt.conf} manual pages).  Figures
\ref{fig:GMT_Defaults_1a}, \ref{fig:GMT_Defaults_1b},
and \ref{fig:GMT_Defaults_1c} show the parameters that affect plots). You may create
your own \filename{gmt.conf} files by running \GMTprog{gmtdefaults}
and then modify those parameters you want to change.  If you want
to use the parameter settings in another file you can do so by
specifying \texttt{+<defaultfile>} on the command line.
This makes it easy to maintain several distinct parameter settings,
corresponding perhaps to the unique styles required by different
journals or simply reflecting font changes necessary to make
readable overheads and slides.  Note that any arguments given on
the command line (see below) will take precedent over the default
values.  E.g., if your \filename{gmt.conf} file has \emph{x}
offset = 1\textbf{i} as default, the \Opt{X}1.5\textbf{i} option will override the
default and set the offset to 1.5 inches. 

There are at least two good reasons why the \GMT\ default options
are placed in a separate parameter file:

\begin{enumerate}

\item It would not be practical to allow for command-line syntax
covering so many options, many of which are rarely or never
changed (such as the ellipsoid used for map projections).

\item It is convenient to keep separate \filename{gmt.conf}
files for specific projects, so that one may achieve a special
effect simply by running \GMT\ commands in the directory whose
\filename{gmt.conf} file has the desired settings.  For example,
when making final illustrations for a journal article one must often
standardize on font sizes and font types, etc.  Keeping all those
settings in a separate \filename{gmt.conf} file simplifies this
process and will allow you to generate those illustrations with the same settings
later on.  Likewise, \GMT\ scripts that make figures for PowerPoint
presentations often use a different color scheme and font size than
output intended for laser printers.  Organizing these various scenarios
into separate \filename{gmt.conf} files will minimize headaches
associated with micro-editing of illustrations.

\GMTfig[h]{GMT_Defaults_1b}{More \gmt\ parameters that affect plot appearance.}

\end{enumerate}

\subsection{Changing \gmt\ defaults}

As mentioned, \GMT\ programs will attempt to open a file named
\filename{gmt.conf}.  At times it may be desirable to override
that default.  There are several ways in which this can be accomplished.
\begin{enumerate}
\item One method is to start each script by saving a
copy of the current \filename{gmt.conf}, then copying the desired
\filename{gmt.conf} file to the current directory, and finally
reverting the changes at the end of the script.  Possible side effects
include premature ending of the script due to user error or bugs which
means the final resetting does not take place (unless you write your
script very carefully.)
\item To permanently change some of the \GMT\ parameters on the fly
inside a script the \GMTprog{gmtset} utility can be used.  E.g., to
change the primary annotation font to 12 point Times-Bold in red we run \\

\texttt{gmtset FONT\_ANNOT\_PRIMARY 12p,Times-Bold,red} \\

These changes will remain in effect until they are overridden.
\item If all you want to achieve is to change a few parameters during
the execution of a single command but otherwise leave the environment intact, consider
passing the parameter changes on the command line via the {--}{--}\emph{PAR=value}
mechanism.  For instance, to temporarily set the output format for floating
points to have lots of decimals, say, for map projection coordinate output,
append {--}{--}\textbf{FORMAT\_FLOAT\_OUT}=\%.12lg to the command in question.
\item Finally, \GMT\ provides to possibility to override the
settings only  during the running of a single script, reverting to the original settings
after the script is run, as if the script was run in ``isolation''. The isolation mode
is discussed in Section~\ref{sec:isolationmode}.
\end{enumerate}
In addition to those parameters
that directly affect the plot there are numerous parameters than
modify units, scales, etc.  For a complete listing, see the
\GMTprog{gmt.conf} man pages.  We suggest that you go through
all the available parameters at least once so that you know what is
available to change via one of the described mechanisms.

\GMTfig[h]{GMT_Defaults_1c}{Even more \gmt\ parameters that affect plot appearance.}

\index{GMT@\GMT!defaults|)}
\index{Default settings|)}

\section{Command line arguments} 
\index{Command line!arguments}%
\index{Arguments, command line}%

Each program requires certain arguments specific to its operation.
These are explained in the manual pages and in the usage messages.
Most programs are ``case-sensitive''; almost all options must start
with an upper-case letter.  We have tried to choose letters of the
alphabet which stand for the argument so that they will be easy to
remember.  Each argument specification begins with a hyphen
(except input file names; see below), followed by a letter, and
sometimes a number or character string immediately after the letter.
\emph{Do not} space between the hyphen, letter, and number or string.
\emph{Do} space between options.  Example:

\vspace{\baselineskip} 

\texttt{pscoast -R0/20/0/20 -Ggray -JM6i -Wthin -B5 -V -P $>$ map.ps}

\section{Standardized command line options} 
\index{Standardized command line options}%
\index{Command line!standardized options|(}%
\label{sec:stopt}

Most of the programs take many of the same arguments like those
related to setting the data region, the map projection, etc.
The 24 switches in Table~\ref{tbl:switches} have the same meaning
in all the programs (although some programs may not use all of them).
These options will be described here as well as in the manual pages,
as is vital that you understand how to use these options.  We will present
these options in order of importance (some are use a lot more than others).

\begin{table}
\centering
\index{Standardized command line options}%
\index{\Opt{B} (set annotations and ticks)}%
\index{\Opt{J} (set map projection)}%
\index{\Opt{K} (continue plot)}%
\index{\Opt{O} (overlay plot)}%
\index{\Opt{P} (portrait orientation)}%
\index{\Opt{R} (set region)}%
\index{\Opt{U} (plot timestamp)}%
\index{\Opt{V} (verbose mode)}%
\index{\Opt{X} (shift plot in $x$)}%
\index{\Opt{Y} (shift plot in $y$)}%
\index{\Opt{a} (aspatial data)}%
\index{\Opt{b} (binary i/o)}%
\index{\Opt{c} (set \# of copies)}%
\index{\Opt{f} (formatting of i/o)}%
\index{\Opt{g} (detect data gaps)}%
\index{\Opt{h} (header records)}%
\index{\Opt{i} (select input columns)}%
\index{\Opt{n} (control grid interpolation settings)}%
\index{\Opt{o} (select output columns)}%
\index{\Opt{p} (set perspective view)}%
\index{\Opt{r} (set pixel registration)}%
\index{\Opt{s} (NaN-record treatment)}%
\index{\Opt{t} (layer transparency)}%
\index{\Opt{:} (input and/or output is $y,x$, not $x,y$)}%

\begin{tabular}{|l|l|} \hline
\multicolumn{1}{|c|}{\emph{Option}}	&	\multicolumn{1}{c|}{\emph{Meaning}} \\ \hline
\Opt{B}	&	Define tickmarks, annotations, and labels for basemaps and axes  \\ \hline
\Opt{J}	&	Select a map projection or coordinate transformation  \\ \hline
\Opt{K}	&	Allow more plot code to be appended to this plot later \\ \hline
\Opt{O}	&	Allow this plot code to be appended to an existing plot \\ \hline
\Opt{P}	&	Select Portrait plot orientation [Default is landscape] \\ \hline
\Opt{R}	&	Define the extent of the map/plot region \\ \hline
\Opt{U}	&	Plot a time-stamp, by default in the lower left corner of page  \\ \hline
\Opt{V}	&	Select verbose operation; reporting on progress  \\ \hline
\Opt{X}	&	Set the \emph{x}-coordinate for the plot origin on the page  \\ \hline
\Opt{Y}	&	Set the \emph{y}-coordinate for the plot origin on the page  \\ \hline
\Opt{a}	&	Associate aspatial data from OGR/GMT files with data columns  \\ \hline
\Opt{b}	&	Select binary input and/or output  \\ \hline
\Opt{c}	&	Specify the number of plot copies  \\ \hline
\Opt{f}	&	Specify the data format on a per column basis  \\ \hline
\Opt{g}	&	Identify data gaps based on supplied criteria  \\ \hline
\Opt{h}	&	Specify that input/output tables have header record(s)  \\ \hline
\Opt{i}	&	Specify which input columns to read  \\ \hline
\Opt{n}	&	Specify grid interpolation settings  \\ \hline
\Opt{o}	&	Specify which output columns to write  \\ \hline
\Opt{p}	&	Control perspective views for plots  \\ \hline
\Opt{r}	&	Set the grid registration to pixel [Default is gridline]  \\ \hline
\Opt{s}	&	Control output of records containing one or more NaNs  \\ \hline
\Opt{t}	&	Change layer PDF transparency  \\ \hline
\Opt{:}	&	Assume input geographic data are (\emph{lat,lon}) and not (\emph{lon,lat})  \\ \hline
\end{tabular}
\caption{The 24 standardized \gmt\ command line switches.}
\label{tbl:switches}
\end{table} 

\subsection{Data domain or map region: The \Opt{R} option}
\index{\Opt{R} (set region)}
\index{Region, specifying}
\label{sec:R}
\GMTfig[h]{GMT_-R}{The plot region can be specified in two different ways. (a) Extreme values
for each dimension, or (b) coordinates of lower left and upper right corners.}

The \Opt{R} option defines the map region or data domain of interest.  It may be specified
in one of three ways (Figure~\ref{fig:GMT_-R}):
\begin{enumerate}
\item \Opt{R}\emph{xmin}/\emph{xmax}/\emph{ymin}/\emph{ymax}.  This is the standard way to specify
Cartesian data domains and geographical regions when using map projections where meridians and
parallels are rectilinear.
\item \Opt{R}\emph{xlleft}/\emph{ylleft}/\emph{xuright}/\emph{yuright}\textbf{r}.
This form is used with map projections that are oblique, making meridians and parallels poor
choices for map boundaries.  Here, we instead specify the lower left corner and upper right
corner geographic coordinates, followed by the suffix \textbf{r}.
\item \Opt{R}\emph{gridfile}.  This will copy the domain settings found for the grid in specified
file.  Note that depending on the nature of the calling program, this mechanism will also set grid spacing
and possibly the grid registration (see Section~\ref{sec:grid_registration}).
\end{enumerate}
For rectilinear projections the first two forms give identical results.  Depending on the selected map
projection (or the kind of expected input data), the boundary coordinates may take on three different
formats:

\begin{description}
\item [Geographic coordinates:]  These are longitudes and latitudes and may be given in decimal degrees (e.g., -123.45417)
or in the
[\PM]\emph{ddd}[:\emph{mm}[:\emph{ss}[\emph{.xxx}]]][\textbf{W}$|$\textbf{E}$|$\textbf{S}$|$\textbf{N}]
format (e.g., 123:27:15W).  Note that \Opt{Rg} and \Opt{Rd} are shorthands for ``global domain''
\Opt{R}\emph{0}/\emph{360}/\emph{-90}/\emph{90}
and  \Opt{R}\emph{-180}/\emph{180}/\emph{-90}/\emph{90}, respectively.

When used in conjunction with the Cartesian Linear Transformation (\Opt{Jx} or \Opt{JX}) ---which can be used
to map floating point data, geographical coordinates, as well as time coordinates--- it is prudent to indicate
that you are using geographical coordinates in one of the following ways:
\begin{itemize}
\item Use \Opt{Rg} or \Opt{Rd} to indicate the global domain.
\item Use \Opt{Rg}\emph{xmin}/\emph{xmax}/\emph{ymin}/\emph{ymax} to indicate a limited geographic domain.
\item Add \textbf{W}, \textbf{E}, \textbf{S}, or \textbf{N} to the coordinate limits or add the generic \textbf{D} or
\textbf{G}. Example: \Opt{R}\emph{0}/\emph{360G}/\emph{-90}/\emph{90N}.
\end{itemize}
Alternatively, you may indicate geographical coordinates by supplying \Opt{fg}; see Section \ref{sec:fg_option}.
\item [Calendar time coordinates:]  These are absolute time coordinates referring to a Gregorian or ISO calendar.
The general format is [\emph{date}]\textbf{T}[\emph{clock}], where \emph{date} must be in the
\emph{yyyy}[\emph{-mm}[\emph{-dd}]] (year, month, day-of-month)
or \emph{yyyy}[\emph{-jjj}] (year and day-of-year) for Gregorian calendars and
\emph{yyyy}[\emph{-}\textbf{W}\emph{ww}[\emph{-d}]] (year, week, and
day-of-week) for the ISO calendar.  If no \emph{date} is given we assume the present day.  Following the
[optional] \emph{date} string we require the \textbf{T} flag.

The optional \emph{clock} string is a 24-hour clock in \emph{hh}[\emph{:mm}[\emph{:ss}[\emph{.xxx}]]] format.
If no \emph{clock} is given
it implies 00:00:00, i.e., the start of the specified day.
Note that not all of the specified entities need be present in the data.  All calendar date-clock strings are internally represented as double precision seconds since
proleptic Gregorian date Monday January 1 00:00:00 0001.  Proleptic means we assume that the modern calendar
can be extrapolated forward and backward; a year zero is used, and Gregory's reforms\footnote{The Gregorian Calendar
is a revision of the Julian Calendar which was instituted in a papal bull by Pope Gregory XIII in 1582. The reason for the calendar
change was to correct for drift in the dates of significant religious observations (primarily Easter) and to prevent further drift
in the dates. The important effects of the change were (a) Drop 10 days from October 1582 to realign the Vernal Equinox with 21 March,
(b) change leap year selection so that not all years ending in ``00'' are leap years, and (c) change the beginning of the year to
1 January from 25 March.  Adoption of the new calendar was essentially immediate within Catholic countries. In the Protestant countries,
where papal authority was neither recognized not appreciated, adoption came more slowly. 
England finally adopted the new calendar in 1752, with eleven days removed from September. The additional day came because the old and
new calendars disagreed on whether 1700 was a leap year, so the Julian calendar had to be adjusted by one more day.} are extrapolated
backward.  Note that this is not historical.

\item [Relative time coordinates:]  These are coordinates which count seconds, hours, days or years relative to a
given epoch. A combination of the parameters \textbf{TIME\_EPOCH} and
\textbf{TIME\_UNIT} define the epoch and time unit. The parameter \textbf{TIME\_SYSTEM} provides a few shorthands for common combinations
of epoch and unit, like \textbf{j2000} for days since noon of 1 Jan 2000.
Denote relative time coordinates by appending the optional lower case
\textbf{t} after the value.  When it is otherwise apparent that the coordinate is relative time (for example by using
the \Opt{f} switch), the \textbf{t} can be omitted.

\item [Other coordinates:]  These are simply any coordinates that are not related to geographic or calendar time or relative
time and are
expected to be simple floating point values such as [\PM]\emph{xxx.xxx}[E$|$e$|$D$|$d[\PM]xx], i.e., regular or exponential
notations, with the enhancement to understand FORTRAN double precision output which may use D instead of E for exponents.
These values are simply converted as they are to internal representation.\footnote{While
UTM coordinates clearly refer to points on the Earth, in this context they are considered ``other''.  Thus, when we
refer to ``geographical'' coordinates herein we imply longitude, latitude.}
\end{description}

\subsection{Coordinate transformations and map projections: The \Opt{J} option}
\index{\Opt{J} (set map projection)}
\index{Map projections}

This option selects the coordinate transformation or map projection.  The general format is
\begin{itemize}
\item \Opt{J}$\delta$[\emph{parameters}/]\emph{scale}.  Here, $\delta$ is a \emph{lower-case}
letter of the alphabet that selects a particular map projection, the \emph{parameters}
is zero or more slash-delimited projection parameter, and \emph{scale} is map scale given in
distance units per degree or as 1:xxxxx.
\item \Opt{J}$\Delta$[\emph{parameters}/]\emph{width}.  Here, $\Delta$ is an \emph{upper-case}
letter of the alphabet that selects a particular map projection, the \emph{parameters}
is zero or more slash-delimited projection parameter, and \emph{width} is map width (map
height is automatically computed from the implied map scale and region).
\end{itemize}
Since \GMT\ version 4.3.0, there is an alternative way to specify the projections: use the same abbreviation as in the mapping package \progname{Proj4}. The options thus either look like:
\begin{itemize}
\item \Opt{J}\emph{abbrev}/[\emph{parameters}/]\emph{scale}.  Here, \textbf{abbrev} is a \emph{lower-case}
abbreviation that selects a particular map projection, the \emph{parameters}
is zero or more slash-delimited projection parameter, and \emph{scale} is map scale given in
distance units per degree or as 1:xxxxx.
\item \Opt{J}\emph{Abbrev}/[\emph{parameters}/]\emph{width}.  Here, \textbf{Abbrev} is an \emph{capitalized} abbreviation that selects a particular map projection, the \emph{parameters}
is zero or more slash-delimited projection parameter, and \emph{width} is map width (map
height is automatically computed from the implied map scale and region).
\end{itemize}

\GMTfig[h]{GMT_-J}{The 30+ map projections and coordinate transformations available in \gmt.}

The projections available in \GMT\ are presented in Figure~\ref{fig:GMT_-J}.
For details on all \GMT\ projections and the required parameters, see the \GMTprog{psbasemap} man page.
We will also show examples of every projection in the next Chapters, and a quick
summary of projection syntax was given in Chapter~\ref{ch:3}.

\subsection{Map frame and axes annotations: The \Opt{B} option}
\index{\Opt{B} (set annotations and ticks)}
\index{Tickmarks}
\index{Annotations}
\index{Gridlines}
\index{Frame}
\index{Basemap}
\label{sec:timeaxis}
This is by far the most complicated option in \GMT, but most examples
of its usage are actually quite simple.
Given as \Opt{B}[\textbf{p}$|$\textbf{s}]\emph{xinfo}[/\emph{yinfo}[/\emph{zinfo}]][:."title
string":][\textbf{W}$|$\textbf{w}][\textbf{E}$|$\textbf{e}][\textbf{S}$|$\textbf{s}][\textbf{N}$|$\textbf{n}][\textbf{Z}$|$\textbf{z}[\textbf{+}]][\textbf{+g}\emph{fill}],
this switch specifies map boundaries (or plot axes) to be plotted by using the
selected information. The optional flag following \Opt{B} selects \textbf{p}(rimary) [Default] or \textbf{s}(econdary)
axes information (mostly used for time axes annotations; see examples below).
The components \emph{xinfo}, \emph{yinfo} and \emph{zinfo} are of the form \\

\par \emph{info}[:"axis label":][:="prefix":][:,"unit label":] \\

\noindent
where \emph{info} is one or more concatenated substrings of the form
[\textbf{t}]\emph{stride}[\emph{phase}][\textbf{u}].  The \textbf{t} flag sets the axis item of interest; the
available items are listed in Table~\ref{tbl:inttype}.

By default, all 4 map boundaries (or plot axes) are plotted (denoted \textbf{W}, \textbf{E}, \textbf{S},
\textbf{N}).  To change this selection, append the codes for those you want
(e.g., \textbf{WSn}).  In this example, the lower case \textbf{n} denotes to draw the axis and (major and minor) tick marks on the ``northern'' (top) edge of the plot. The upper case \textbf{WS} will annotate the ``western'' and ''southern'' axes with numerals and plot the optional axis label in addition to
draw axis/tick-marks.  The title, if given, will appear centered above the plot.  Unit label or prefix may start with a
leading -- to suppress the space between it and the annotation.  Normally, equidistant annotations
occur at multiples of \emph{stride}; you can phase-shift this by appending \emph{phase}, which can be a
positive or negative number.  Finally, note you may paint the canvas by appending the \textbf{+g}\emph{fill} modifier.
\begin{table}[H]
\centering
\begin{tabular}{|c|l|} \hline
\emph{Flag}	& \emph{Description} \\ \hline
\textbf{a}	&	Annotation and major tick spacing \\ \hline
\textbf{f}	&	Minor tick spacing \\ \hline
\textbf{g}	&	Grid line spacing \\ \hline
\end{tabular}
\caption{Interval type codes.}
\label{tbl:inttype}
\end{table}
\noindent
Note that the appearance of certain time annotations (month-, week-, and day-names) may be affected
by the \textbf{TIME\_LANGUAGE}, \textbf{FORMAT\_TIME\_PRIMARY\_MAP}, and \textbf{FORMAT\_TIME\_SECONDARY\_MAP} settings.

For automated plots the region may not always be the same and thus it can be difficult to determine the appropriate \emph{stride} in advance. Here \GMT{} provides the opportunity to autoselect the spacing between the major and minor ticks and the grid lines, by not specifying the \emph{stride} value. For example, \Opt{Bafg} will select all three spacings automatically for both axes. In case of longitude-latitude plots, this will keep the spacing the same on both axes. You can also use \Opt{Bafg/afg} to autoselect them separately.

In the case of automatic spacing, when the \emph{stride} argument is omitted after \textbf{g}, the grid line spacing is chosen the same as the minor tick spacing; unless \textbf{g} is used in consort with \textbf{a}, then the grid lines are spaced the same as the annotations.

The unit flag \textbf{u} can take on one of 18 codes; these are listed in  Table~\ref{tbl:units}.
Almost all of these units are time-axis specific.  However, the \textbf{m} and \textbf{s} units will be
interpreted as arc minutes and arc seconds, respectively, when a map projection is in effect.

\begin{table}[h]
\centering
\begin{tabular}{|c|l|l|} \hline
\emph{Flag}	& \emph{Unit} & \emph{Description} \\ \hline
\textbf{Y}	&	year		& Plot using all 4 digits \\ \hline
\textbf{y}	&	year		& Plot using last 2 digits \\ \hline
\textbf{O}	&	month		& Format annotation using \textbf{FORMAT\_DATE\_MAP} \\ \hline
\textbf{o}	&	month		& Plot as 2-digit integer (1--12) \\ \hline
\textbf{U}	&	ISO week	& Format annotation using \textbf{FORMAT\_DATE\_MAP} \\ \hline
\textbf{u}	&	ISO week	& Plot as 2-digit integer (1--53) \\ \hline
\textbf{r}	&	Gregorian week	& 7-day stride from start of week (see \textbf{TIME\_WEEK\_START}) \\ \hline
\textbf{K}	&	ISO weekday	& Plot name of weekday in selected language \\ \hline
\textbf{k}	&	weekday		& Plot number of day in the week (1-7)  (see \textbf{TIME\_WEEK\_START})\\ \hline
\textbf{D}	&	date		& Format annotation using \textbf{FORMAT\_DATE\_MAP} \\ \hline
\textbf{d}	&	day		& Plot day of month (1--31) or day of year (1--366) \\
		&			& (see \bf{FORMAT\_DATE\_MAP} \\ \hline
\textbf{R}	&	day		& Same as \textbf{d}; annotations aligned with week (see \textbf{TIME\_WEEK\_START})\\ \hline
\textbf{H}	&	hour		& Format annotation using \textbf{FORMAT\_CLOCK\_MAP} \\ \hline
\textbf{h}	&	hour		& Plot as 2-digit integer (0--24) \\ \hline
\textbf{M}	&	minute		& Format annotation using \textbf{FORMAT\_CLOCK\_MAP} \\ \hline
\textbf{m}	&	minute		& Plot as 2-digit integer (0--60) \\ \hline
\textbf{S}	&	seconds		& Format annotation using \textbf{FORMAT\_CLOCK\_MAP} \\ \hline
\textbf{s}	&	seconds		& Plot as 2-digit integer (0--60) \\ \hline
\end{tabular}
\caption{Interval unit codes.}
\label{tbl:units}
\end{table}

There may be two levels of annotations.  Here, ``primary'' refers to the annotation
that is closest to the axis (this is the primary annotation), while ``secondary'' refers to the secondary
annotation that is plotted further from the axis.  The examples below
will clarify what is meant.  Note that the terms ``primary'' and ``secondary'' do not reflect any hierarchical
order of units: The ``primary'' annotation interval is usually smaller (e.g., days) while the
``secondary'' annotation interval typically is larger (e.g., months).

\subsubsection{Geographic basemaps}

Geographic basemaps may differ from regular plot axis in that some projections support a
``fancy'' form of axis and is selected by the \textbf{MAP\_FRAME\_TYPE} setting.  The annotations
will be formatted according to the \textbf{FORMAT\_GEO\_MAP} template and \textbf{MAP\_DEGREE\_SYMBOL}
setting.  A simple example of part of a basemap is shown in Figure~\ref{fig:GMT_-B_geo_1}.

\GMTfig[h]{GMT_-B_geo_1}{Geographic map border using separate selections for annotation,
frame, and grid intervals.  Formatting of the annotation is controlled by
the parameter \textbf{FORMAT\_GEO\_MAP} in your \protect\filename{gmt.conf}\ file.}

The machinery for primary and secondary annotations introduced for time-series axes can
also be utilized for geographic basemaps.  This may be used to separate
degree annotations from minutes- and seconds-annotations.  For a more complicated basemap
example using several sets of intervals, including different intervals and pen attributes
for grid lines and grid crosses, see Figure~\ref{fig:GMT_-B_geo_2}.

\GMTfig[h]{GMT_-B_geo_2}{Geographic map border with both primary (P) and secondary (S) components.}


\subsubsection{Cartesian linear axes}
\index{linear axes}
\index{Axes!linear}

For non-geographic axes, the \textbf{MAP\_FRAME\_TYPE} setting is implicitly set to plain.  Other than that,
cartesian linear axes are very similar to geographic axes.  The annotation format may be controlled with
the \textbf{FORMAT\_FLOAT\_OUT} parameter.  By default, it is set to ``\%g'', which is a C language format statement
for floating point numbers\footnote{Please consult the man page for \emph{printf} or any book on C .},
and with this setting the various axis routines will automatically determine
how many decimal points should be used by inspecting the \emph{stride} settings.  If \textbf{FORMAT\_FLOAT\_OUT} is set
to another format it will be used directly (.e.g, ``\%.2f'' for a fixed, two decimals format).
Note that for these axes you may use the \emph{unit} setting to
add a unit string to each annotation (see Figure~\ref{fig:GMT_-B_linear}).

\GMTfig[h]{GMT_-B_linear}{Linear Cartesian projection axis.  Long tickmarks accompany
annotations, shorter ticks indicate frame interval.  The axis label is
optional.  We used \Opt{R}0/12/0/1 \Opt{JX}3/0.4 \Opt{Ba}4\textbf{f}2\textbf{g}1:Frequency::,\protect\%:.}

\subsubsection{Cartesian log$_{10}$ axes}
\index{log$_{10}$ axes}
\index{Logarithmic axes}
\index{Axes!log$_{10}$}
\index{Axes!Logarithmic}

Due to the logarithmic nature of annotation spacings, the \emph{stride} parameter takes on specific
meanings.  The following concerns are specific to log axes:

\begin{enumerate}
\item \emph{stride} must be 1, 2, 3, or a negative integer $-n$.  Annotations/ticks will
then occur at 1, 1--2--5, or 1,2,3,4,...,9, respectively, for each magnitude range.
For $-n$ the annotations will take place every \emph{n}'th magnitude.

\item Append \textbf{l} to \emph{stride}. Then, log$_{10}$ of the annotation
is plotted at every integer log$_{10}$ value (e.g., $x = 100$ will be annotated as ``2'')
[Default annotates $x$ as is].

\item Append \textbf{p} to \emph{stride}.  Then, annotations appear as 10
raised to log$_{10}$ of the value (e.g., $10^{-5}$).

\end{enumerate}

\GMTfig[h]{GMT_-B_log}{Logarithmic projection axis using separate values for annotation,
frame, and grid intervals.  (top) Here, we have chosen to annotate the actual
values.  Interval = 1 means every whole power of 10, 2 means 1, 2, 5 times
powers of 10, and 3 means every 0.1 times powers of 10.  We used
\Opt{R}1/1000/0/1 \Opt{JX}3l/0.4 \Opt{Ba}1\textbf{f}2\textbf{g}3.
(middle) Here, we have chosen to
annotate log$_{10}$ of the actual values, with \Opt{Ba}1\textbf{f}2\textbf{g}3\textbf{l}. 
(bottom) We annotate every power of 10 using log$_{10}$ of the actual values
as exponents, with \Opt{Ba}1\textbf{f}2\textbf{g}3\textbf{p}.}

\index{Exponential axis}
\index{Axes!exponential}
\index{Axes!power}
\subsubsection{Cartesian exponential axes}
Normally, \emph{stride} will be used to create equidistant (in the user's unit) annotations
or ticks, but because of the exponential nature of the axis, such annotations may converge
on each other at one end of the axis.  To avoid this problem, you can
append \textbf{p }to \emph{stride}, and the annotation
interval is expected to be in transformed units, yet the annotation itself will be plotted
as un-transformed units.  E.g., if \emph{stride} = 1 and power = 0.5 (i.e., sqrt),
then equidistant annotations labeled 1, 4, 9, ... will appear.

\GMTfig[h]{GMT_-B_pow}{Exponential or power projection axis.  (top) Using an exponent of 0.5
yields a $\sqrt{x}$ axis.  Here, intervals refer to actual data values, in
\Opt{R}0/100/0/1 \Opt{JX}3\textbf{p}0.5/0.4\ \Opt{Ba}20\textbf{f}10\textbf{g}5.
(bottom) Here, intervals refer to projected values, although the annotation
uses the corresponding unprojected values, as in \Opt{Ba}3\textbf{f}2\textbf{g}1\textbf{p}.}

\index{Time axis}
\index{Axes!time}
\subsubsection{Cartesian time axes}

What sets time axis apart from the other kinds of plot axes is the numerous ways in which we
may want to tick and annotate the axis.  Not only do we have both primary and secondary annotation
items but we also have interval annotations versus tickmark annotations, numerous time units,
and several ways in which to modify the plot.  We will demonstrate this flexibility with a
series of examples.  While all our examples will only show a single $x$-axis, time-axis is supported for all axes.

Our first example shows a time period of almost two months in Spring 2000.  We want to annotate the month
intervals as well as the date at the start of each week:

\script{GMT_-B_time1}

These commands result in Figure~\ref{fig:GMT_-B_time1}.  Note the leading hyphen in the \textbf{FORMAT\_DATE\_MAP}
removes leading zeros from calendar items (e.g., 02 becomes 2).
\GMTfig[h]{GMT_-B_time1}{Cartesian time axis, example 1.}

The next example shows two different ways to annotate an axis portraying 2 days in July 1969:

\script{GMT_-B_time2}

The lower example (Figure~\ref{fig:GMT_-B_time2}) chooses to annotate the weekdays (by specifying
\textbf{a}1\textbf{K}) while the upper
example choses dates (by specifying \textbf{a}1\textbf{D}).  Note how the clock format only selects hours and minutes (no seconds) and
the date format selects a month name, followed by one space and a two-digit day-of-month number.

\GMTfig[h!]{GMT_-B_time2}{Cartesian time axis, example 2.}

The third example presents two years, annotating both the years and every 3rd month.

\script{GMT_-B_time3}

Note that while the year annotation is centered on the 1-year interval, the month annotations must be centered
on the corresponding month and \emph{not} the 3-month interval.  The \textbf{FORMAT\_DATE\_MAP} selects month
name only and \textbf{FORMAT\_TIME\_PRIMARY\_MAP} selects the 1-character, upper case abbreviation of month names using
the current language (selected by \textbf{TIME\_LANGUAGE}).
\GMTfig[h!]{GMT_-B_time3}{Cartesian time axis, example 3.}

The fourth example (Figure~\ref{fig:GMT_-B_time4}) only shows a few hours of a day, using relative time by
specifying \textbf{t} in the \Opt{R} option while the \textbf{TIME\_UNIT} is \textbf{d} (for days).
We select both primary and
secondary annotations, ask for a 12-hour clock, and let time go from right to left:

\script{GMT_-B_time4}
\GMTfig[h!]{GMT_-B_time4}{Cartesian time axis, example 4.}

The fifth example shows a few weeks of time (Figure~\ref{fig:GMT_-B_time5}).  The lower axis shows ISO weeks with
week numbers and abbreviated names of the weekdays.   The upper uses Gregorian weeks (which start at the day chosen
by \textbf{TIME\_WEEK\_START}); they do not have numbers.

\script{GMT_-B_time5}
\GMTfig[h!]{GMT_-B_time5}{Cartesian time axis, example 5.}

Our sixth example shows the first five months of 1996, and we have annotated each month with an abbreviated, upper case
name and 2-digit year.  Only the primary axes information is specified.

\script{GMT_-B_time6}

\GMTfig[h!]{GMT_-B_time6}{Cartesian time axis, example 6.}

Our seventh and final example illustrates annotation of year-days.  Unless we specify the formatting with a leading hyphen
in  \textbf{FORMAT\_DATE\_MAP} we get 3-digit integer days.  Note that in order to have the two years
annotated we need to allow for the annotation of small fractional intervals; normally such truncated interval must
be at least half of a full interval.

\script{GMT_-B_time7}

\GMTfig[h!]{GMT_-B_time7}{Cartesian time axis, example 7.}

\index{Custom axis}
\index{Axes!custom}
\subsubsection{Custom axes}
\label{sec:custaxes}
Irregularly spaced annotations or annotations based on look-up tables can be implemented using
the \emph{custom} annotation mechanism.  Here, we given the \textbf{c} (custom) type to the \Opt{B}
option followed by a filename that contains the annotations (and or tick/grid-lines) for one axis.
The file can contain any number of comments (lines starting with \#) and any number of records of
the format
\\
\emph{coord}\quad	\emph{type}\quad	[\emph{label}]
\\
The \emph{coord} is the location of the desired annotation, tick, or grid-line, whereas
\emph{type} is a string composed of letters from \textbf{a} (annotation), \textbf{i} interval
annotation, \textbf{f} frame tick, and \textbf{g} gridline.  You must use either \textbf{a} or \textbf{i}
within one file; no mixing is allowed.  The coordinates should be arranged in increasing order.
If \emph{label} is given it replaces the normal
annotation based on the \emph{coord} value.  Our last example shows such a custom basemap
with an interval annotations on the \emph{x}-axis and irregular annotations on the \emph{y}-axis.

\script{GMT_-B_custom}

\GMTfig[h!]{GMT_-B_custom}{Custom and irregular annotations, tick-marks, and gridlines.}

\subsection{Portrait plot orientation: The \Opt{P} option} 
\index{Orientation!of plot}
\index{Plot!orientation}
\index{Landscape orientation}
\index{Orientation!landscape}
\index{Portrait orientation \Opt{P}}
\index{Orientation!portrait \Opt{P}}
\index{\Opt{P} (portrait orientation)}

\GMTfig[h]{GMT_-P}{Users can specify Landscape [Default] or Portrait (\Opt{P}) orientation.}

\Opt{P} selects Portrait plotting mode\footnote{For historical reasons, the \GMT\
Default is Landscape, see \GMTprog{gmt.conf} to change this.}.  In general,
a plot has an \emph{x}-axis increasing from left to
right and a \emph{y}-axis increasing from bottom to top.  If the
paper is turned so that the long dimension of the paper is
parallel to the \emph{x}-axis then the plot is said to have
\emph{Landscape} orientation.  If the long dimension of
the paper parallels the \emph{y}-axis the orientation is called
\emph{Portrait} (think of taking pictures with a camera
and these words make sense).  The
default Landscape orientation is obtained by translating the origin in the
\emph{x}-direction (by the width of the chosen paper \textbf{PAPER\_MEDIA)} and then rotating the
coordinate system counterclockwise by 90\DS.  By default the \textbf{PS\_MEDIA} is
set to Letter (or A4 if SI is chosen); this value must be changed
when using different media, such as 11" x 17" or large format plotters
(Figure~\ref{fig:GMT_-P}).


\subsection{Plot overlays: The \Opt{K} \Opt{O} options}
\index{Overlay plot \Opt{O} \Opt{K}}
\index{Plot!overlay \Opt{O} \Opt{K}}
\index{Plot!continue \Opt{O} \Opt{K}}
\index{\Opt{K} (continue plot)}
\index{\Opt{O} (overlay plot)}

\GMTfig[h]{GMT_-OK}{A final \PS\ file consists of any number of individual pieces.}

The \Opt{K} and \Opt{O} options control the generation of \PS\ code for multiple
overlay plots.  All \PS\ files must have a header (for initializations),
a body (drawing the figure), and a trailer (printing it out) (see
Figure~\ref{fig:GMT_-OK}).  Thus,
when overlaying several \GMT\ plots we must make sure that the first plot
call omits the trailer, that all intermediate calls omit both header and
trailer, and that the final overlay omits the header.
\Opt{K} omits the trailer which implies that more \PS\ code will be appended
later [Default terminates the plot system].  \Opt{O} selects Overlay plot
mode and omits the header information [Default initializes a new plot system].
Most unexpected results for multiple overlay plots can be traced to the
incorrect use of these options.  If you run only one plot
program, ignore both the \Opt{O} and \Opt{K} options; they are
only used when stacking plots. 

\subsection{Timestamps on plots: The \Opt{U} option} 
\index{\Opt{U} (plot timestamp)}
\index{Timestamp}
\index{UNIX@\UNIX!timestamp}

\Opt{U} draws \UNIX\ System time stamp.  Optionally, append an arbitrary
text string (surrounded by double quotes), or the code \textbf{c}, which will
plot the current command string (Figure~\ref{fig:GMT_-U}).

\GMTfig[h]{GMT_-U}{The \Opt{U} option makes it easy to ``date'' a plot.}

\subsection{Verbose feedback: The \Opt{V} option} 
\index{\Opt{V} (verbose mode)}
\index{Verbose (\Opt{V})}
\label{sec:verbose}
\Opt{V} selects verbose mode, which will send progress reports to
\emph{stderr} [Default runs ``silently''].  The interpretation of
this option can be toggled by changing the default \textbf{GMT\_VERBOSE}.

\subsection{Plot positioning and layout: The \Opt{X} \Opt{Y} options}
\index{\Opt{X} (shift plot in $x$)}
\index{\Opt{Y} (shift plot in $y$)}
\index{Plot!offset}
\index{Offset, plot}
\GMTfig[h]{GMT_-XY}{Plot origin can be translated freely with \Opt{X} \Opt{Y}.}

\Opt{X} and \Opt{Y} shift origin of plot by (\emph{xoff},\emph{yoff})
inches (Default is (\textbf{MAP\_ORIGIN\_X}, \textbf{MAP\_ORIGIN\_Y}) for new plots\footnote{Ensures that
boundary annotations do not fall off the page.} and (0,0) for overlays (\Opt{O})).
By default, all translations are relative to the previous origin
(see Figure~\ref{fig:GMT_-XY}).  Supply offset as \textbf{c} to center the
plot in that direction relative to the page margin.
Absolute translations (i.e., relative to a fixed point (0,0) at the
lower left corner of the paper) can be achieve by prepending ``a''
to the offsets.  Subsequent overlays will be co-registered with the
previous plot unless the origin is shifted using these options.
The offsets are measured in the current coordinates system (which can
be rotated using the initial \Opt{P} option; subsequent \Opt{P} options
for overlays are ignored).

\subsection{OGR/GMT GIS i/o: The \Opt{a} option}
\index{Table!GIS}
\index{Table!OGR}
\index{GIS tables}
\index{OGR/GIS tables}
\index{Input!aspatial \Opt{a}}
\index{\Opt{a} (process aspatial data)}

\GMT\ relies on external tools to translate geospatial files such as shapefiles
into a format we can read.  The tool \progname{ogr2ogr} in the GDAL package can do such
translations and preserve the aspatial metadata via a new OGR/GMT format
specification (See Appendix Q).  For this to be useful we need a mechanism
to associate certain metadata values with required input and output columns expected
by \GMT\ programs.  The \Opt{a} option allows you to supply one or more
comma-separated associations \emph{col=name}, where
\emph{name} is the name of an aspatial attribute field in a OGR/GMT file
and whose value we wish to as data input for column \emph{col}.  The
given aspatial field thus replaces any other value already set.  Note
that \emph{col = 0} is the first data columns.  Note that if no aspatial attributes
are needed then the \Opt{a} option is not needed -- \GMT\ will still process
and read such data files.

\subsubsection{OGR/GMT input with \Opt{a} option}
If you need to populate GMT data columns with (constant) values specified by aspatial
attributes, use \Opt{a} and append any number of comma-separated \emph{col=name}
associations.  E.g., \emph{2=depth} will read the spatial \emph{x,y} columns from the
file and add a third (\emph{z}) column based on the value of the aspatial field called
\emph{depth}.  You can also associate aspatial fields with other settings such as
labels, fill colors, pens, and values used to look-up colors.  Do so by letting
the \emph{col} value be one of \textbf{D}, \textbf{G}, \textbf{L}, \textbf{T}, \textbf{W}, or \textbf{Z}.  This
works analogously to how standard multi-segment files can pass such options via
its segment headers (See Appendix B).

\subsubsection{OGR/GMT output with \Opt{a} option}
You can also make \GMT\ table-writing tools output the OGR/GMT format directly.
Again, specify if certain \GMT\ data columns with constant values should be stored
as aspatial metadata using the \emph{col=name}[:\emph{type}], where you can optionally
specify what data type it should be (double, integer, string, logical, byte, or
datetime) [double is default].  As for input, you can also use the special \emph{col}
entries of \textbf{D}, \textbf{G}, \textbf{L}, \textbf{T}, \textbf{W}, or \textbf{Z} to have values stored
as options in segment headers be used as the source for the name aspatial field.
Finally, for output you must append +\textbf{g}\emph{geometry}, where \emph{geometry}
can be any of [\textbf{M}]\textbf{POINT}$|$\textbf{LINE}$|$\textbf{POLY}; the \textbf{M} represent
the multi-versions of these three geometries.  Use upper-case +\textbf{G} to signal that
you want to split any line or polygon features that straddle the Dateline.

\subsection{Binary table i/o: The \Opt{b} option}
\index{Table!binary}
\index{Table!netCDF}
\index{Binary tables}
\index{NetCDF tables}
\index{Input!binary \Opt{bi}}
\index{Output!binary \Opt{bo}}
\index{\Opt{bi} (select binary input)}
\index{\Opt{bo} (select binary output)}

All \GMT\ programs that accept table data input may read ASCII, native binary, or netCDF data.
When using native binary data the user must be aware
of the fact that \GMT\ has no way of determining the actual
number of columns in the file.  You must therefore pass that
information to \GMT\ via the binary \Opt{bi}[\emph{n}]\textbf{t} option,
where \emph{n} is the actual number of data columns and \textbf{t}
must be one of \textbf{c} (signed 1-byte character), \textbf{u}
(unsigned 1-byte character), \textbf{h} (signed 2-byte int), \textbf{H}
(unsigned 2-byte int), \textbf{i} (signed 4-byte int), \textbf{I}
(unsigned 4-byte int), \textbf{l} (signed 8-byte int), \textbf{K}
(unsigned 8-byte int), \textbf{f} (4-byte single-precision float), and
\textbf{d} (8-byte double-precision float).  For a mixed-type data record
you can concatenate several [\emph{n}]\textbf{t} combinations,
separated by commas.  You may append \textbf{w}
to any of the items to force byte-swapping.  Alternatively, append
\textbf{+L}$|$\textbf{B} to indicate that the entire data file
should be read or written as little- or big-endian, respectively.
Here, \emph{n} is the number of each items in your binary file.
Note that \emph{n} may be larger than \emph{m}, the number of
columns that the \GMT\ program requires to do its task.
If \emph{n} is not given then it defaults to \emph{m} and all columns
are assumed to be of the single specified type \textbf{t} [\textbf{d} (double), if not set].
If \emph{n} $<$ \emph{m} an error is generated.

For binary output, use the \Opt{bo}[\emph{n}]\textbf{t} option; see \Opt{bi} for further
details.

Because of its meta data, reading netCDF tables (i.e., netCDF files containing 1-dimensional arrays)
is quite a bit less complex than reading native binary files. When feeding netCDF tables to programs
like \GMTprog{psxy}, the program will automatically recognize the format and read whatever amount of
columns are needed for that program. To steer which columns are to be read, the user can
append the suffix \textbf{?}\emph{var1}\textbf{/}\emph{var2}\textbf{/}\emph{...} to the netCDF file name,
where \emph{var1}, \emph{var2}, etc.\
are the names of the variables to be processed. No \Opt{bi} option is needed in this case.

Currently, netCDF tables can only be input, not output.
For more information, see Appendix~\ref{app:B}.

\subsection{Number of Copies: The \Opt{c} option}
\index{\Opt{c} (set \# of copies)}
\index{Number of copies}

The \Opt{c} option specifies the number of plot copies [Default is 1].  This
value is embedded in the \PS\ file and will make a printer issue the chosen
number of copies without respooling.

\subsection{Data type selection: The \Opt{f} option}
\label{sec:fg_option}
\index{Table!format}
\index{Input!format \Opt{fi}}
\index{Output!format \Opt{fo}}
\index{\Opt{fi} (set input format)}
\index{\Opt{fo} (set output format)}

When map projections are not required we must explicitly state
what kind of data each input or output column contains.  This is accomplished with
the \Opt{f} option.  Following an optional \textbf{i} (for input only) or \textbf{o} (for output
only), we append a text string with information about each column (or range of columns) separated by commas.
Each string starts with the column number (0 is first column) followed by either
\textbf{x} (longitude), \textbf{y} (latitude), \textbf{T} (absolute calendar time) or \textbf{t} (relative time).  If
several consecutive columns have the same format you may specify a range of columns
rather than a single column, i.e., 0--4 for the first 5 columns.  For example, if our
input file has geographic coordinates (latitude, longitude) with absolute calendar
coordinates in the columns 3 and 4, we would specify \textbf{fi}0\textbf{y},1\textbf{x},3-4\textbf{T}.  All other columns
are assumed to have the default, floating point format and need not be set individually.
The shorthand \Opt{f}[\textbf{i}$|$\textbf{o}]\textbf{g} means \Opt{f}[\textbf{i}$|$\textbf{o}]0x,1y (i.e., geographic coordinates).
For more information, see Sections~\ref{sec:input data} and~\ref{sec:output data}.

\subsection{Data gap detection: The \Opt{g} option}
\index{\Opt{g} (activate data gap detection)}
\label{sec:gap}
\GMT\ has several mechanisms that can determine line segmentation.  Typically, data segments are
separated by multiple segment header records (see Appendix B).  However, if key data
columns contain a NaN we may also use that information to break lines into multiple
segments.  This behavior is modified by the parameter \textbf{IO\_NAN\_RECORDS} which
by default is set to \emph{skip}, meaning such records are considered bad and simply
skipped.  If you wish such records to indicate a segment boundary then set this parameter
to \emph{pass}.  Finally, you may wish to indicate gaps based on the data values themselves.
The \Opt{g} option is used to detect gaps based on one or more criteria (use \Opt{g+} if
\emph{all} the criteria must be met; otherwise only one of the specified criteria needs
to be met to signify a data gap).  Gaps can be based on excessive jumps in the \emph{x}- or
\emph{y}-coordinates (\Opt{gx} or \Opt{gy}), or on the distance between points (\Opt{gd}).
Append the \emph{gap} distance and optionally a unit for actual distances.
For geographic data the optional unit may be arc \textbf{d}egree, \textbf{m}inute, and \textbf{s}econd,
or m\textbf{e}ter [Default], \textbf{f}eet, \textbf{k}ilometer, \textbf{M}iles, or \textbf{n}autical miles.
For programs that map data to map coordinates you can optionally specify these criteria to apply to
the projected coordinates (by using upper-case \Opt{gX}, \Opt{gY} or \Opt{gD}).
In that case, choose from \textbf{c}entimeter, \textbf{i}nch or \textbf{p}oint
[Default unit is controlled by \textbf{PROJ\_LENGTH\_UNIT}].
Note: For \Opt{gx} or \Opt{gy} with time data the unit is instead controlled by \textbf{TIME\_UNIT}.

\subsection{Header data records: The \Opt{h} option}
\index{\Opt{h} (header records)}
\index{Header records \Opt{h}}
\label{sec:header}
The \Opt{h}[\textbf{i}$|$\textbf{o}][\emph{n\_recs}] option lets \GMT\ know that input file(s) have
one [Default] or more header records.  If there are more than one header
record you must specify the number after the \Opt{h} option, e.g., \Opt{h}4.  The
default number of header records if \Opt{h} is used is one of the many parameters
in the \filename{gmt.conf} file (\textbf{IO\_N\_HEADER\_RECS}, by default 1),
but can be overridden by \Opt{h}\emph{n\_header\_recs}.
Note that blank lines and records that be start with the character \# are
automatically skipped.  Normally, programs that both read and write tables will
output the header records that are found on input.  Use \Opt{hi} to suppress the
writing of header records.  Use \Opt{ho} to tell programs to output a header record
identifying each data column.

When \Opt{b} is used to indicate binary data the \Opt{h} takes on a slightly different meaning.
Now, the \emph{n\_recs} argument is taken to mean how many \emph{bytes} should be
skipped (on input) or padded with the space character (on output).

\subsection{Input columns selection: The \Opt{i} option}
\index{\Opt{i} (Input columns selection)}
\index{Columns selection \Opt{i}}
\label{sec:incols}
The \Opt{i}\emph{columns} option allows you to specify which
input file data columns to use and it what order.  By default, \GMT\ will
read all the data columns in the file, starting with the first column (0).
Using \Opt{i} modifies that process.  For instance, to use the 4th, 7th, and
3rd data column as the required \emph{x,y,z} to \GMTprog{blockmean} you would
specify \Opt{i}3,6,2 (since 0 is the first column). The chosen data columns
will be used as is.  Optionally, you can specify that input columns should be
transformed according to a linear or logarithmic conversion.  Do so by appending
[\textbf{l}][\textbf{s}\emph{scale}][\textbf{o}\emph{offset}] to each column (or range
of columns).  All items are optional: The \textbf{l} implies we should first take
$\log_{10}$ of the data [leave as is]. Next, we may scale the result by the given \emph{scale} [1].
Finally, we add in the specified \emph{offset} [0].

\subsection{Grid interpolation parameters: The \Opt{n} option}
\index{\Opt{n} (Grid resampling parameters)}
\index{Resampling \Opt{n}}
\label{sec:resample}
The \Opt{n}\emph{type} option controls parameters 
used for 2-D grids resampling.  You can select the type
of spline used (\Opt{nb} for B-spline smoothing, \Opt{nc} for bicubic [Default],
\Opt{nl} for bilinear, or \Opt{nn} for nearest-node value).
For programs that support it, antialiasing is by default on;
optionally, append \textbf{+a} to switch off antialiasing.
By default, boundary conditions are set according to the grid type and extent.
Change boundary conditions by appending \textbf{+b}\emph{BC},
where \emph{BC} is either \textbf{g} for geographic boundary conditions
or one (or both) of  \textbf{n} and \textbf{p} for natural or periodic boundary
conditions, respectively.  Append \textbf{x} or \textbf{y} to only apply the condition
in one dimension.  E.g., \Opt{nb+nxpy} would imply natural boundary conditions in
the \emph{x} direction and periodic conditions in the \emph{y} direction.
Finally, append \textbf{+t}\emph{threshold} to control
how close to nodes with NaN the interpolation should go.  A
\emph{threshold} of 1.0 requires all (4 or 16) nodes involved in the
interpolation to be non-NaN. 0.5 will interpolate about half way from a
non-NaN value; 0.1 will go about 90\% of the way, etc.

\subsection{Output columns selection: The \Opt{o} option}
\index{\Opt{o} (Output columns selection)}
\index{Columns selection \Opt{o}}
\label{sec:outcols}
The \Opt{o}\emph{columns} option allows you to specify which
columns to write on output and it what order.  By default, \GMT\ will
write all the data columns produced by the program.
Using \Opt{o} modifies that process.  For instance, to write just the 4th and
2nd data column to the output you would use \Opt{o}3,1 (since 0 is the first column).

\subsection{Perspective view: The \Opt{p} option}
\index{\Opt{p} Perspective view}
\index{Perspective view}

All plotting programs that normally produce a flat, two-dimensional illustration can
be told to view this flat illustration from a particular vantage point, resulting in
a perspective view.  You can select perspective view with the \Opt{p} option by setting
the azimuth and elevation of the viewpoint [Default is 180/90].
When \Opt{p} is used in consort with \Opt{Jz} or \Opt{JZ}, a third value can be appended which indicates
at which \emph{z}-level all 2D material, like the plot frame, is plotted (in perspective) 
[Default is at the bottom of the z-axis].
For frames used for animation, you may want to append \textbf{+} to fix the center
of your data domain (or specify a particular world coordinate point with \textbf{+w}\emph{lon0/lat}[\emph{z}])
which will project to the center of your page size (or you may specify the coordinates
of the \emph{projected} view point with \textbf{+v}\emph{x0/y0}.
When \Opt{p} is used without any further arguments, the values from the last use of \Opt{p} in a previous \GMT\
command will be used.

\subsection{Grid registration: The \Opt{r} option}
\index{\Opt{r} Grid registration}
\index{grid file!registration|(}
\label{sec:grid_registration}
All 2-D grids in \GMT\ have their nodes organized in one of two ways, known
as \emph{gridline}- and \emph{pixel} registration.  The \GMT\ default is
gridline registration; programs that allow for the creation of grids can
use the \Opt{r} option to select pixel registration instead.

\subsubsection{Gridline registration}
\index{grid file!registration!grid line|(}

In this registration, the nodes are centered on the grid line
intersections and the data points represent the average value
in a cell of dimensions ($x_{inc} \cdot y_{inc}$) centered on each
node (Figure~\ref{fig:GMT_registration}).
In the case of grid line registration the number of nodes are
related to region and grid spacing by \\

\[ \begin{array}{ccl} 
nx & =  &       (x_{max} - x_{min}) / x_{inc} + 1       \\ 
ny & =  &       (y_{max} - y_{min}) / y_{inc} + 1
\end{array} \]
which for the example in Figure~\ref{fig:GMT_registration} yields $nx = ny = 4$.

\GMTfig[h]{GMT_registration}{Gridline- and pixel-registration of data nodes.}

\index{grid file!registration!grid line|)}

\subsubsection{Pixel registration}
\index{grid file!registration!pixel|(}

Here, the nodes are centered in the grid cells, i.e., the areas
between grid lines, and the data points represent the average
values within each cell (Figure~\ref{fig:GMT_registration}.
In the case of pixel registration the number of nodes are related
to region and grid spacing by \\

\[ \begin{array}{ccl} 
nx & =  &       (x_{max} - x_{min}) / x_{inc}   \\ 
ny & =  &       (y_{max} - y_{min}) / y_{inc}
\end{array} \]
Thus, given the same region (\Opt{R}) and grid spacing, the pixel-registered grids have one less
column and one less row than the gridline-registered grids; here we
find $nx = ny = 3$.

\index{grid file!registration!pixel|)}
\index{grid file!registration|)}

\subsection{NaN-record treatment: The \Opt{s} option}
\index{\Opt{s} NaN-record treatment}
\index{NaN}

We can use this option to suppress output for records whose \emph{z}-value equals NaN
(by default we output all records).  Alternatively, append \textbf{r} to reverse the suppression,
i.e., only output the records whose \emph{z}-value equals NaN.  Use \Opt{sa} to suppress
output records where one or more fields (and not necessarily \emph{z}) equal NaN.  Finally, you can supply
a comma-separated list of all columns or column ranges to consider for this NaN test.

\subsection{Layer PDF transparency: The \Opt{t} option}
\index{\Opt{t} Layer PDF transparency}
\index{Transparency}
\label{sec:ltransp}

While the \PS\ language does not support transparency, PDF does, and via \PS\ extensions
one can manipulate the transparency levels of objects.  The \Opt{t} option allows you to change
the transparency level for the current overlay by appending a percentage in the 0--100 range; the
default is 0, or opaque.  Transparency may also be controlled
on a feature by feature basis when setting color or fill (see section~\ref{sec:fill}).

\subsection{Latitude/Longitude or Longitude/Latitude?: The \Opt{:} option}
\index{\Opt{:} (input and/or output is $y,x$, not $x,y$)}
\index{lat/lon input}

For geographical data, the first column is expected to contain longitudes
and the second to contain latitudes.  To reverse this expectation you must
apply the \Opt{:} option.  Optionally, append \textbf{i} or \textbf{o} to restrict
the effect to input or output only.  Note that command line arguments that may take
geographic coordinates (e.g., \Opt{R}) \emph{always} expect longitude before
latitude. Also, geographical grids are expected to have the longitude as
first (minor) dimension.

\index{Command line!standardized options|)}%

\section{Command line history}
\label{sec:gmtcommands}
\index{Command line!history}
\index{History, command line}

\GMT\ programs ``remember'' the standardized command line options
(See Section~\ref{sec:stopt}) given during their previous invocations
and this provides a shorthand notation for complex options.
For example, if a basemap was created with an oblique Mercator
projection, specified as

\vspace{\baselineskip} 

\texttt{-Joc170W/25:30S/33W/56:20N/1:500000} \\ 

\vspace{\baselineskip} 
\noindent
then a subsequent \GMTprog{psxy} command to plot symbols only needs
to state \Opt{J}o in order to activate the same projection.  In
contrast, note that \Opt{J} by itself will pick the most recently used projection.
Previous commands are maintained in the file \filename{.gmtcomands},
of which there will be one in each directory you run the programs
from.  This is handy if you create separate directories for
separate projects since chances are that data manipulations
and plotting for each project will share many of the same options.
Note that an option spelled out on the command line will always
override the last entry in the \filename{.gmtcomands} file and,
if execution is successful, will replace this entry as the
previous option argument in the \filename{.gmtcomands} file.
If you call several \GMT\ modules piped together then \GMT\ cannot
guarantee that the \filename{.gmtcomands} file is processed
in the intended order from left to right.  The only guarantee
is that the file will not be clobbered since \GMT\ uses advisory
file locking.  The uncertainty in processing order makes the use
of shorthands in pipes unreliable.  We therefore recommend that you
only use shorthands in single process command lines, and spell out
the full command option when using chains of commands connected with
pipes.

\section{Usage messages, syntax- and general error messages}
\index{Usage messages}
\index{Messages!usage}
\index{Syntax messages}
\index{Messages!syntax}
\index{Error messages}
\index{Messages!error}

Each program carries a usage message.  If you enter the program
name without any arguments, the program will write the complete
usage message to standard error (your screen, unless you
redirect it).  This message explains in detail what all the
valid arguments are.  If you enter the program name followed
by a \emph{hyphen} (--) only you will get a shorter version
which only shows the command line syntax and no detailed
explanations.  If you incorrectly specify an option or omit
a required option, the program will produce syntax errors and
explain what the correct syntax for these options should be.
If an error occurs during the running of a program, the
program will in some cases recognize this and give you an
error message.  Usually this will also terminate the run.
The error messages generally begin with the name of the
program in which the error occurred; if you have several
programs piped together this tells you where the trouble is. 

\section{Standard input or file, header records}
\index{Standard input}
\index{Input!standard}
\index{Header records \Opt{h}}
\index{Record, header \Opt{h}}
\index{\Opt{h} (header records)}
\index{GMT\_DATADIR}

Most of the programs which expect table data input can read
either standard input or input in one or several files.
These programs will try to read \emph{stdin} unless you type
the filename(s) on the command line without the above hyphens.
(If the program sees a hyphen, it reads the next character
as an instruction; if an argument begins without a hyphen,
it tries to open this argument as a filename).
This feature allows you to connect programs with pipes if
you like.  If your input is ASCII and has one or more header
records that do not begin with \#, you must use the \Opt{h}
option (see Section~\ref{sec:header}).
ASCII files may in many cases also contain segment-headers
separating data segments.  These are called ``multi-segment files''.
For binary table data the \Opt{h} option may specify how many bytes should
be skipped before the data section is reached.
Binary files may also contain segment-headers
separating data segments.  These segment-headers are simply data records
whose fields are all set to NaN;
see Appendix~\ref{app:B} for complete documentation. 

If filenames are given for reading, \GMT\ programs will first look for them in the
current directory.  If the file is not found, the programs
will look in two other directories pointed to by environmental
parameters (if set).  These are \textbf{GMT\_USERDIR} and
\textbf{GMT\_DATADIR}, and they may be set by the user to point to directories
that contain data sets of general use.  Normally, the \textbf{GMT\_DATADIR} directory (or directories:
add multiple paths by separating them with colons (semi-colons under Windows)) will
hold data sets of a general nature (tables, grids), although a particular use
is to make available large grids accessible via the supplemental programs \GMTprog{grdraster}
or \GMTprog{img2grd}; see Appendix~\ref{app:A} for information
about these supplemental programs.  The \textbf{GMT\_USERDIR} directory may hold miscellaneous
data sets more specific to the user; this directory also stores \GMT\ defaults and other
configuration files.  Any directory that ends in a trailing slash (/) will be searched
recursively.  Data sets that the user finds are often needed
may be placed in these directories, thus eliminating the need to specify
a full path to the file.  Program output is always written to the
current directory unless a full path has been specified.

\section{Verbose operation}
\index{Verbose (\Opt{V})}
\index{\Opt{V} (verbose mode)}

Most of the programs take an optional \Opt{V} argument
which will run the program in the ``verbose'' mode (see Section~\ref{sec:verbose}).
Verbose will write to standard error information about the progress
of the operation you are running.  Verbose reports things
such as counts of points read, names of data files
processed, convergence of iterative solutions, and the like.
Since these messages are written to \emph{stderr},  the
verbose talk remains separate from your data output. 
You may optionally choose among five levels of \emph{verbosity}; each level
adds more messages with an increasing level of details.  The levels are
\begin{description}
	\item [0]: Complete silence, not even fatal error messages.
	\item [1]: Fatal error messages.
	\item [2]: Warnings and progress messages [Default].
	\item [3]: Detailed progress messages.
	\item [4]: Debugging messages.
\end{description}
The verbosity is cumulative, i.e., level 3 means all messages of levels 0--3
will be reported.

\section{Program output}
\index{Output!standard}
\index{Output!error}
\index{grid file!formats!netCDF}

Most programs write their results, including \PS\
plots, to standard output.  The exceptions are those which may
create binary netCDF grid files such as \GMTprog{surface} (due to
the design of netCDF a filename must be provided; however,
alternative binary output formats allowing piping are available; see Section~\ref{sec:grdformats}).
With \UNIX\, you can redirect standard output to a file or pipe it
into another process.  Error messages, usage messages, and
verbose comments are written to standard error in all cases.
You can use \UNIX\ to redirect standard error as well,
if you want to create a log file of what you are doing.
The syntax for redirection differ among the main shells (Bash and C-shell).

\section{Input data formats}
\label{sec:input data}
\index{Input!format}

Most of the time, \GMT\ will know what kind of $x$ and $y$ coordinates it is reading because you have selected
a particular coordinate transformation or map projection.  However,
there may be times when you must explicitly specify what you are
providing as input using the \Opt{f} switch. When binary input data are expected (\Opt{bi}) you must
specify exactly the format of the records.  However, for ASCII input there are numerous
ways to encode data coordinates (which may be separated by white-space or commas).  Valid input data are generally
of the same form as the arguments to the \Opt{R} option (see Section~\ref{sec:R}), with additional
flexibility for calendar data.  Geographical coordinates, for example, can be given in decimal degrees
(e.g., -123.45417) or in the
[\PM]\emph{ddd}[:\emph{mm}[:\emph{ss}[\emph{.xxx}]]][\textbf{W}$|$\textbf{E}$|$\textbf{S}$|$\textbf{N}]
format (e.g., 123:27:15W).

Because of the widespread use of incompatible and ambiguous formats, the processing of input
date components is guided by the template \textbf{FORMAT\_DATE\_IN} in your
\filename{gmt.conf} file; it is by default set to \emph{yyyy-mm-dd}.  Y2K-challenged input data such as
29/05/89 can be processed by setting \textbf{FORMAT\_DATE\_IN}
to dd/mm/yy.  A complete description of possible formats is given in the \GMTprog{gmt.conf}
man page.  The \emph{clock} string is more standardized but issues like 12- or 24-hour clocks complicate matters
as well as the presence or absence of delimiters between fields.  Thus, the processing of input
clock coordinates is guided by the template \textbf{FORMAT\_CLOCK\_IN} which defaults to \emph{hh:mm:ss.xxx}.

\GMT\ programs that require a map projection argument will implicitly know what kind of data to expect, and the
input processing is done accordingly.  However, some programs that simply report on minimum and maximum
values or just do a reformatting of the data will in general not know what to expect, and furthermore there is
no way for the programs to know what kind of data other columns (beyond the leading $x$ and $y$ columns) contain.
In such instances we must
explicitly tell \GMT\ that we are feeding it data in the specific geographic or calendar formats (floating point
data are assumed by default).  We specify the data type via the \Opt{f} option (which sets both input and output
formats; use \Opt{fi} and \Opt{fo} to set input and output separately).  For instance, to specify that the
the first two columns are longitude and latitude, and that the third column (e.g., $z$) is absolute calendar time, we add
\Opt{fi}0x,1y,2T to the command line.  For more details, see the man page for the program you need to use.

\section{Output data formats}
\label{sec:output data}
\index{Output!format}

The numerical output from \GMT\ programs can be binary (when \Opt{bo} is used) or ASCII [Default].
In the latter case the issue of formatting becomes important.  \GMT\ provides extensive
machinery for allowing just about any imaginable format to be used on output.  Analogous to
the processing of input data, several templates guide the formatting process.  These are
\textbf{FORMAT\_DATE\_OUT} and \textbf{FORMAT\_CLOCK\_OUT} for calendar-time coordinates,
\textbf{FORMAT\_GEO\_OUT} for geographical coordinates, and \textbf{FORMAT\_FLOAT\_OUT} for generic
floating point data.  In addition, the user have control over how columns are separated via
the \textbf{FIELD\_SEPARATOR} parameter.  Thus, as an example, it is possible to create limited
FORTRAN-style card records by setting \textbf{FORMAT\_FLOAT\_OUT} to \%7.3lf and \textbf{FIELD\_SEPARATOR} to
none [Default is tab].

\section{\PS\ features}
\index{PostScript@\PS!features}
\PS\ is a command language for driving graphics
devices such as laser printers.  It is ASCII text which you
can read and edit as you wish (assuming you have some knowledge
of the syntax).  We prefer this to binary metafile plot
systems since such files cannot easily be modified after they
have been created.  \GMT\ programs also write many comments to
the plot file which make it easier for users to orient
themselves should they need to edit the file (e.g., \% Start
of x-axis)\footnote{To keep \PS\ files small, such comments are by default
turned off; see \textbf{PS\_COMMENTS} to enable them.}.  All \GMT\ programs create \PS\ code by
calling the \textbf{PSL} plot library (The user may call these
functions from his/her own C or FORTRAN plot programs. See the
manual pages for \textbf{PSL} syntax).  Although \GMT\ programs
can create very individualized plot code, there will always be
cases not covered by these programs.  Some knowledge of
\PS\ will enable the user to add such features
directly into the plot file.  By default, \GMT\ will produce
freeform \PS\ output with embedded printer directives.  To
produce Encapsulated \PS\ (EPS) that can be imported into graphics programs such as
\progname{CorelDraw}, \progname{Illustrator} or \progname{InkScape} for further
embellishment, change the \textbf{PS\_MEDIA} setting in the \filename{gmt.conf}
file.  See Appendix~\ref{app:C} and the \GMTprog{gmt.conf} man page for more details.

\section{Specifying pen attributes}

\index{Attributes!pen}%
\index{Pen!setting attributes}%
\label{sec:pen}
A pen in \GMT\ has three attributes: \emph{width}, \emph{color},
and \emph{style}.  Most programs will accept pen attributes in
the form of an option argument, with commas separating the
given attributes, e.g.,

\vspace{\baselineskip} 

\par \Opt{W}[\emph{width}[\textbf{c$|$i$|$p}]],[\emph{color}],[\emph{style}[\textbf{c$|$i$|$p$|$}]]\par 

\begin{description}
\index{Pen!width}
\index{Width, pen}
\index{Attributes!pen!width}%
\item[$\rightarrow$]\emph{Width} is by default measured in points (1/72 of an inch).
Append \textbf{c}, \textbf{i}, or \textbf{p} to specify pen width in cm, inch, or points,
respectively.
Minimum-thickness pens can be achieved by
giving zero width, but the result is device-dependent.  Finally, a few
predefined pen names can be used: default, faint, and \{thin, thick, fat\}[er$|$est],
and obese.  Table~\ref{tbl:pennames} shows this list and the corresponding pen widths.
\begin{table}[h]
\centering
\begin{tabular}{|l|c||l|c|} \hline
\multicolumn{1}{|c|}{\emph{Pen name}}	&	\multicolumn{1}{c|}{\emph{Width}}	&	\multicolumn{1}{|c|}{\emph{Pen name}}	&	\multicolumn{1}{c|}{\emph{Width}} \\ \hline
faint		&	0	&	thicker		&	1.5p \\ \hline 
default		&	0.25p	&	thickest	&	2p \\ \hline
thinnest	&	0.25p	&	fat		&	3p \\ \hline
thinner		&	0.50p	&	fatter		&	6p \\ \hline 
thin		&	0.75p	&	fattest		&	12p \\ \hline  
thick		&	1.0p	&	obese		&	18p \\	\hline 
\end{tabular}
\caption{\gmt\ predefined pen widths.}
\label{tbl:pennames}
\end{table}

\index{Pen!color}
\index{Color!pen}
\index{Color!RGB system}
\index{Color!HSV system}
\index{Color!CMYK system}
\index{Attributes!pen!color}%
\item[$\rightarrow$]The \emph{color} can be specified in five different ways:
\begin{enumerate}
\item Gray. Specify a \emph{gray} shade in the range 0--255 (linearly going from black [0] to white [255]).
\item RGB. Specify \emph{r}/\emph{g}/\emph{b}, each ranging from 0--255.  Here 0/0/0 is black, 255/255/255 is white,
255/0/0 is red, etc.
\item HSV. Specify \emph{hue}-\emph{saturation}-\emph{value}, with the former in the 0--360 degree range while the latter
two take on the range 0--1\footnote{For an overview of color systems such as HSV, see Appendix~\ref{app:I}.}.
\item CMYK. Specify \emph{cyan}/\emph{magenta}/\emph{yellow}/\emph{black}, each ranging from 0--100\%.
\item Name.  Specify one of 663 valid color names.  Use \progname{man gmtcolors} to list all valid names.
A very small yet versatile subset consists of the 29 choices \emph{white}, \emph{black}, and
[light$|$dark]\{\emph{red,
orange, yellow, green, cyan, blue, magenta, gray$|$grey, brown}\}.
The color names are case-insensitive, so mixed upper and lower case can be used (like
\emph{DarkGreen}).
\end{enumerate}

\index{Pen!style}
\index{Style, pen}
\index{Attributes!pen!style}%
\item[$\rightarrow$]The \emph{style} attribute controls the appearance
of the line.  A ``.'' yields a dotted line, whereas a dashed pen is requested with ``-''.
Also combinations of dots and dashes, like ``.-'' for a dot-dashed line, are allowed.
The lengths of dots and dashes are scaled relative to the pen width (dots has
a length that equals the pen width while dashes are 8 times as long; gaps between
segments are 4 times the pen width).
For more detailed attributes including exact dimensions you may specify \emph{string}:\emph{offset},
where \emph{string} is a series of numbers separated by underscores.
These numbers represent a pattern by indicating the length of line
segments and the gap between segments.  The \emph{offset} phase-shifts the
pattern from the beginning the line.  For example, if you want a yellow line of width
0.1~cm that alternates between long dashes (4 points), an 8 point gap, then
a 5 point dash, then another 8 point gap, with pattern offset by 2 points
from the origin, specify \Opt{W}0.1c,yellow,4\_8\_5\_8:2p.
Just as with pen width, the default style units are points, but can also be explicitly specified in cm, inch, or points
(see \emph{width} discussion above). 
\end{description} 
Table~\ref{tbl:penex} contains additional examples of pen specifications suitable for, say, \GMTprog{psxy}.
\begin{table}[h]
\centering
\begin{tabular}{|l|l|} \hline
\multicolumn{1}{|c|}{\emph{Pen example}}	&	\multicolumn{1}{c|}{\emph{Comment}} \\ \hline
\Opt{W}0.5p		&	0.5 point wide line of default color and style \\ \hline 
\Opt{W}green		&	Green line with default width and style \\ \hline
\Opt{W}thin,red,-	&	Dashed, thin red line \\ \hline 
\Opt{W}fat,.		&	Fat dotted line with default color \\ \hline 
\Opt{W}0.1c,120-1-1	&	Green (in h-s-v) pen, 1~mm thick \\ \hline 
\Opt{W}faint,100/0/0/0,..-	&	Very thin, cyan (in c/m/y/k), dot-dot-dashed line \\ \hline 
\end{tabular}
\caption{A few examples of pen specifications. The default width, color and style are determined by the \textbf{MAP\_DEFAULT\_PEN} parameter.}
\label{tbl:penex}
\end{table}
In addition to these pen settings there are several \PS\ settings that can affect the appearance of lines.
These are controlled via the \GMT\ defaults settings \textbf{PS\_LINE\_CAP}, \textbf{PS\_LINE\_JOIN},
and \textbf{PS\_MITER\_LIMIT}. They determine how a line segment ending is rendered, be it at the
termination of a solid line or at the end of all dashed line segments making up a line, and how a
straight lines of finite thickness should behave when joined at a common point.  By default, line segments
have rectangular ends, but this can change to give rounded ends.  When \textbf{PS\_LINE\_CAP} is set to
round the a segment length of zero will appear as a circle.  This can be used to created circular dotted
lines, and by manipulating the phase shift in the \emph{style} attribute and plotting the same line twice
one can even alternate the color of adjacent items.  Figure~\ref{fig:GMT_linecap} shows various lines made
in this fashion.  See the \GMTprog{gmt.conf} man page for more information.
\GMTfig[h]{GMT_linecap}{Line appearance can be varied by using \textbf{PS\_LINE\_CAP} and the \emph{style} attribute.}

\section{Specifying area fill attributes}

\index{Attributes!fill!color}%
\index{Attributes!fill!pattern}%
\index{Fill!attributes!color}%
\index{Fill!attributes!pattern}%
\index{Color!fill}%
\index{Pattern!fill}%
\index{Pattern!color}%
\index{\Opt{GP} \Opt{Gp}}
\label{sec:fill}

Many plotting programs will allow the user to draw filled polygons or
symbols.  The fill specification may take two forms: 

\vspace{\baselineskip} 

\par \Opt{G}\emph{fill}\par 

\par \Opt{Gp}\emph{dpi/pattern}[:\textbf{B}\emph{color}[\textbf{F}\emph{color}]]\par 

\vspace{\baselineskip} 
\noindent
\begin{description}
\item [fill:]
In the first case we may specify a \emph{gray} shade (0--255), RGB color
(\emph{r}/\emph{g}/\emph{b} all in the 0--255 range or in hexadecimal \emph{\#rrggbb}), HSV color (\emph{hue}-\emph{saturation}-\emph{value}
in the 0--360, 0--1, 0--1 range), CMYK color (\emph{cyan}/\emph{magenta}/\emph{yellow}/\emph{black},
each ranging from 0--100\%), or a valid color \emph{name}; in that respect it is similar
to specifying the pen color settings (see pen color discussion under Section~\ref{sec:pen}).
\item [pattern:]
The second form allows us to use a predefined bit-image pattern.
\emph{pattern} can either be a number in the range 1--90 or the name of a 1-,
8-, or 24-bit Sun raster file.  The former will result in one of the 90
predefined 64 x 64 bit-patterns provided with \GMT\ and reproduced in Appendix~\ref{app:E}.
The latter allows the user to create customized, repeating images using
standard Sun raster files\footnote{Convert other graphics formats to Sun ras format using
ImageMagick's \progname{convert} program.}.  The \emph{dpi} parameter sets the resolution of
this image on the page;  the area fill is thus made up of a series of these
``tiles''.  Specifying \emph{dpi} as 0 will result in highest resolution
obtainable given the present dpi setting in \filename{gmt.conf}.
By specifying upper case \Opt{GP} instead of \Opt{Gp} the image will be
bit-reversed, i.e., white and black areas will be interchanged (only applies
to 1-bit images or predefined bit-image patterns).  For these patterns and
other 1-bit images one may specify alternative background and foreground
colors (by appending :\textbf{B}\emph{color}[\textbf{F}\emph{color}]) that will
replace the default white and black pixels, respectively.  Setting one of the
fore- or background colors to -- yields a \emph{transparent} image where only the
back- \emph{or} foreground pixels will be painted.
\end{description}

Due to \PS\ implementation limitations the raster images used with
\Opt{G} must be less than 146 x 146 pixels in size; for larger images see
\GMTprog{psimage}.  The format of Sun raster files is outlined in Appendix~\ref{app:B}.
Note that under \PS\ Level 1 the patterns are filled by using
the polygon as a \emph{clip path}.  Complex clip paths may require
more memory than the \PS\ interpreter has been assigned.
There is therefore the possibility that some \PS\ interpreters
(especially those supplied with older laserwriters) will run out of memory
and abort.  Should that occur we recommend that you use a regular gray-shade
fill instead of the patterns.  Installing more memory in your printer
\emph{may or may not} solve the problem! 

Table~\ref{tbl:fillex} contains a few examples of fill specifications.

\begin{table}[h]
\centering
\begin{tabular}{|l|l|} \hline
\multicolumn{1}{|c|}{\emph{Fill example}}	&	\multicolumn{1}{c|}{\emph{Comment}} \\ \hline
\Opt{G}128		&	Solid gray \\ \hline
\Opt{G}127/255/0	&	Chartreuse, R/G/B-style \\ \hline
\Opt{G}\#00ff00 & Green, hexadecimal RGB code \\ \hline  
\Opt{G}25-0.86-0.82	&	Chocolate, h-s-v -- style \\ \hline 
\Opt{G}DarkOliveGreen1	&	One of the named colors \\ \hline 
\Opt{Gp}300/7		&	Simple diagonal hachure pattern in b/w at 300 dpi\\ \hline 
\Opt{Gp}300/7:Bred	&	Same, but with red lines on white \\ \hline 
\Opt{Gp}300/7:BredF-	&	Now the gaps between red lines are transparent \\ \hline 
\Opt{Gp}100/marble.ras	&	Using user image of marble as the fill at 100 dpi \\ \hline 
\end{tabular}
\caption{A few examples of fill specifications.}
\label{tbl:fillex}
\end{table}

\index{Specifying fonts}
\index{Fonts, specifying}
\index{Font!attributes!size}%
\index{Font!attributes!type}%
\index{Font!attributes!fill}%
\index{Font!attributes!transparency}%
\index{Font!attributes!outline}%
\label{sec:fonts}

\section{Specifying Fonts}

The fonts used by \GMT\ are typically set indirectly via the \GMT\ defaults parameters.
However, some programs, like \GMTprog{pstext} may wish to have this information passed directly.
A font is specified by a comma-delimited attribute list of \emph{size}, \emph{fonttype} and \emph{fill}, each of which is optional.
The \emph{size} is the font size (usually in points) but \textbf{c}, \textbf{i} or \textbf{p} can be added to indicate a specific unit.
The \emph{fonttype} is the name (case sensitive!) of the font or its equivalent numerical ID (e.g., Helvetica-Bold or 1).
\emph{fill} specifies the gray shade, color or pattern of the text (see section~\ref{sec:fill} above).
Optionally, you may append \textbf{=}\emph{pen} to the \emph{fill} value in order to draw the text outline with the specified
\emph{pen};
if used you may optionally skip the filling of the text by setting \emph{fill} to \textbf{-}.
If any of the attributes is omitted their default or previous setting will be retained.
See Appendix G for a list of all fonts recognized by \GMT.

\section{Stroke, Fill and Font Transparency}

\index{Attributes!transparency}%
\index{Fill!attributes!transparency}%
\index{Font!attributes!transparency}%
\index{Color!transparency}%
\index{Pattern!transparency}%
\index{Pattern!transparency}%
\label{sec:transp}

The \PS\ language has no built-in mechanism for transparency.  However, \PS\
extensions make it possible to request transparency, and tools that can render such extensions
will produce transparency effects.  We specify transparency in percent: 0 is opaque [Default]
while 100 is fully transparent (i.e., nothing will show).  As noted in section~\ref{sec:ltransp}, we can control
transparency on a layer-by-layer basis using the \Opt{t} option.  However, we may also set
transparency as an attribute of stroke or fill (including for fonts) settings.  Here, transparency is requested by
appending @\emph{transparency} to colors or pattern fills.  The transparency \emph{mode} can be changed
by using the \GMT\ default parameter \textbf{PS\_TRANSPARENCY}; the default is Normal but you can choose among
Color, ColorBurn, ColorDodge, Darken, Difference, Exclusion, HardLight, Hue,
Lighten, Luminosity, Multiply, Normal, Overlay, Saturation, SoftLight, and Screen.
For more information, see for instance (search online for) the Adobe pdfmark Reference Manual.
Most printers and many \PS\ viewers can neither print nor show transparency. They will simply ignore your attempt to
create transparency and will plot any material as opaque. \progname{GhostScript} and its derivatives
such as \GMT's \GMTprog{ps2raster} support transparency (if compiled with the correct build option).
Note: If you use \progname{Acrobat Distiller} to create a PDF file you must first change some settings to make
transparency effective: change the parameter /AllowTransparency to true in your *.joboptions file.

\section{Color palette tables}

\index{Color!palette tables|(}
\index{CPT!file|(}

Several programs, such as those which read 2-D gridded data sets and create
colored images or shaded reliefs, need to be told what colors to use and
over what \emph{z}-range each color applies.  This is the purpose of the
color palette table (CPT file).  These files may also be used by \GMTprog{psxy}
and \GMTprog{psxyz} to plot color-filled symbols.  For most applications, you
will simply create a CPT file using the tool \GMTprog{makecpt} which will
take an existing color table and resample it to fit your chosen data
range, or use \GMTprog{grd2cpt} to build a CPT file based on the data distribution
in one or more given grid files.  However, in some situations you will need to make a CPT file by
hand or using text tools like \progname{awk} or \progname{perl}.

Color palette tables (CPT) comes in two flavors: (1) Those designed to work with
categorical data (e.g., data where interpolation of values is undefined) and
(2) those designed for regular, continuously-varying data.  In both cases
the \emph{fill} information follows the format given in Section \ref{sec:fill}.

\subsection{Categorical CPT files}

Categorical data are information on which normal numerical operations are not
defined. As an example, consider various land classifications (desert, forest, glacier, etc.)
and it is clear that even if we assigned a numerical value to these categories (e.g.,
desert = 1, forest = 2, etc) it would be meaningless to compute average values (what would
1.5 mean?).  For such data a special format of the CPT files are provided.  Here,
each category is assigned a unique key, a color or pattern, and an optional
label (usually the category name) marked by a leading semi-colon.  Keys must be
monotonically increasing but do not need to be consecutive.  The format is

\begin{center}
\begin{tabular}{lll}
key$_1$ &  \emph{fill} &  [;\emph{label}] \\ 
\ldots & & \\ 
key$_n$ &  \emph{fill} &  [;\emph{label}] \\ 
\end{tabular} 
\end{center}

The \emph{fill} information follows the format given in Section \ref{sec:fill}.
While not always applicable to categorical data, the background color (for
\emph{key}-values $<$ \emph{key$_1$}), foreground color
(for \emph{key}-values $>$ \emph{key$_{n}$}), and not-a-number (NaN) color (for
\emph{key}-values = NaN) are all defined in the \filename{gmt.conf}
file, but can be overridden by the statements

\begin{center}
\begin{tabular}{llll}
B &  R$_{back}$ &  G$_{back}$ &  B$_{back}$ \\ 
F &  R$_{fore}$ &  G$_{fore}$ &  B$_{fore}$ \\ 
N &  R$_{nan}$ &  G$_{nan}$ &  B$_{nan}$ \\
\end{tabular}
\end{center}

\subsection{Regular CPT files}

Suitable for continuous data types and allowing for color interpolations, the
format of the regular CPT files is: 

\begin{center}
\begin{tabular}{llllll}
z$_0$ &  Color$_{min}$  &  z$_1$ & Color$_{max}$ &  [\bf{A}] & [;\emph{label}] \\ 
\ldots & & & & & \\ 
z$_{n-2}$ &  Color$_{min}$ &  z$_{n-1}$ &  Color$_{max}$ &  [\textbf{A}] & [;\emph{label}] \\
\end{tabular} 
\end{center}

Thus, for each ``\emph{z}-slice'', defined as the interval between two boundaries
(e.g., \emph{z$_0$} to \emph{z$_1$}), the color can be constant (by letting Color$_{max}$
= Color$_{min}$ or -) or a continuous, linear function of \emph{z}.  If patterns are
used then the second (max) pattern must be set to -.  The optional flag \textbf{A} is used to indicate annotation
of the color scale when plotted using \GMTprog{psscale}.  The optional flag \textbf{A} may be \textbf{L},
\textbf{U}, or \textbf{B} to select annotation of the lower, upper, or both limits
of the particular $z$-slice, respectively.  However, the standard \Opt{B} option can be used
by \GMTprog{psscale} to affect annotation and ticking of color scales. Just as other \GMT{} programs, the \emph{stride} can be omitted to determine the annotation and tick interval automatically (e.g., \Opt{Baf}).  The optional
semicolon followed by a text label will make \GMTprog{psscale}, when used with the \Opt{L} option,
place the supplied label instead of formatted \emph{z}-values.

As for categorical tables, the background color (for \emph{z}-values $<$ \emph{z$_0$}), foreground color
(for \emph{z}-values $>$ \emph{z$_{n-1}$}), and not-a-number (NaN) color (for
\emph{z}-values = NaN) are all defined in the \filename{gmt.conf}
file, but can be overridden by the statements

\begin{center}
\begin{tabular}{ll}
B &  Color$_{back}$ \\ 
F &  Color$_{fore}$ \\ 
N &  Color$_{nan}$ \\
\end{tabular}
\end{center}

\noindent
which can be inserted into the beginning or end of the CPT file.  If
you prefer the HSV system, set the
\filename{gmt.conf} parameter accordingly and replace red, green,
blue with hue, saturation, value.  Color palette tables that contain
gray-shades only may replace the \emph{r/g/b} triplets with a single gray-shade
in the 0--255 range.  For CMYK, give four values in the 0--100 range.
Both the min and max color specifications in one \emph{z}-slice must use
the same color system, i.e., you cannot mix ``red'' and 0/255/100 on the
same line.

A few programs (i.e., those that plot polygons such as \GMTprog{grdview},
\GMTprog{psscale}, \GMTprog{psxy} and \GMTprog{psxyz}) can accept pattern fills instead
of gray-shades.  You must specify the pattern as in Section~\ref{sec:fill} (no
leading \Opt{G} of course), and only the first pattern (for low $z$) is used (we cannot
interpolate between patterns).  
Finally, some programs let you skip features whose $z$-slice in the CPT file has
gray-shades set to --.  As an example, consider

\begin{center}
\begin{tabular}{lllll}
30 &  p200/16 &  80 & -- \\ 
80 &  -- &  100 &  -- \\
100 &  200/0/0  &  200/255/255  &  0 \\
200 &  yellow &  300 & green  \\ 
\end{tabular} 
\end{center}
\noindent
where slice $30 < z < 80$ is painted with pattern \# 16 at 200 dpi,
slice $80 < z < 100$ is skipped, slice $100 < z < 200$ is
painted in a range of dark red to yellow, whereas the slice $200 < z < 300$
will linearly yield colors from yellow to green, depending on the actual value
of $z$.

\index{Color!palette tables|)}
\index{CPT!file|)}

\index{Artificial illumination}
\index{Illumination, artificial}
\index{Shaded relief}
\index{Relief, shaded}

Some programs like \GMTprog{grdimage} and \GMTprog{grdview} apply artificial
illumination to achieve shaded relief maps.  This is typically done
by finding the directional gradient in the direction of the artificial
light source and scaling the gradients to have approximately a normal
distribution on the interval [-1,+1].  These intensities are used
to add ``white'' or ``black'' to the color as defined by the \emph{z}-values
and the CPT file.  An intensity of zero leaves the color unchanged.
Higher values will brighten the color, lower values will darken it,
all without changing the original hue of the color (see Appendix~\ref{app:I}
for more details).  The illumination is decoupled from the data
grid file in that a separate grid file holding intensities in the
[-1,+1] range must be provided.  Such intensity files can be
derived from the data grid using \GMTprog{grdgradient} and modified
with \GMTprog{grdhisteq}, but could equally well be a separate data set.
E.g., some side-scan sonar systems collect both bathymetry and
backscatter intensities, and one may want to use the latter information
to specify the illumination of the colors defined by the former.
Similarly, one could portray magnetic anomalies superimposed on
topography by using the former for colors and the latter for shading. 

\section{The Drawing of Vectors}
\label{sec:vectors}
\index{Vectors!Cartesian}
\index{Vectors!Circular}
\index{Vectors!Great-circle}

\GMT\ supports plotting vectors in various forms.  A vector is one of many symbols
that may be plotted by \GMTprog{psxy} and \GMTprog{psxyz}, is the main feature
in \GMTprog{grdvector}, and is indirectly used by other programs.  All vectors plotted by \GMT\ consist of two separate
parts: The vector line (controlled by the chosen pen attributes) and the optional
vector head(s) (controlled by the chosen fill).  We distinguish between three types of vectors:
\begin{enumerate}
	\item Cartesian vectors are plotted as straight lines.  They can be specified by a
	start point and the direction and length (in map units) of the vector, or by its beginning and end point.
	They may also be specified giving the azimuth and length (in km) instead.
	
	\item Circular vectors are (as the name implies) drawn as circular arcs and can be used to indicate opening
	angles.  It accepts an origin, a radius, and the beginning and end angles.
	
	\item Geo-vectors are drawn using great circle arcs.  They are specified by a beginning point
	and the azimuth and length (in km) of the vector, or by its beginning and end point.
\end{enumerate}
\GMTfig[h]{GMT_arrows}{Examples of Cartesian (left), circular (middle), and geo-vectors (right) for different
attribute specifications.  Note that both full and half arrow-heads can be specified, as well as no head at all.}
There are numerous attributes you can modify, including how the vector should be justified relative
to the given point (beginning, center, or end), where heads (if any) should be placed, if the head should just be the
left or right half, if the vector attributes should shrink for vectors whose length are less than a given cutoff
length, and the size and shape of the head.  These attributes are detailed further in the relevant manual pages.

\section{Character escape sequences}
\label{sec:escape}
\index{Characters!European}
\index{Characters!escape sequences}
\index{Characters!escape sequences!subscript}
\index{Characters!escape sequences!superscript}
\index{Characters!escape sequences!switch fonts}
\index{Characters!escape sequences!composite character}
\index{Characters!escape sequences!small caps}
\index{Characters!escape sequences!octal character}
\index{Escape sequences!characters}
\index{European characters}
\index{Text!European}
\index{Text!escape sequences}
\index{Text!subscript}
\index{Subscripts}
\index{Text!superscript}
\index{Superscripts}
\index{Characters!composite}
\index{Composite characters}
\index{Font!switching}
\index{Font!symbol}
\index{Symbol font}
\index{"@, printing}
\index{Small caps}
\index{Characters!octal}
\index{Octal characters}

For annotation labels or text strings plotted with \GMTprog{pstext},
\GMT\ provides several escape sequences that allow the user to
temporarily switch to the symbol font, turn on sub- or superscript,
etc., within words.  These conditions are toggled on/off by the
escape sequence @\textbf{x}, where \textbf{x} can be one of several types.
The escape sequences recognized in \GMT\ are listed in Table~\ref{tbl:escape}. 
Only one level of sub- or superscript is supported.
Note that under Windows the percent symbol indicates a batch variable,
hence you must use two percent-signs for each one required in the escape sequence for font switching.

\begin{table}[H]
\centering
\begin{tabular}{|l|l|} \hline
\multicolumn{1}{|c|}{\emph{Code}}	&	\multicolumn{1}{c|}{\emph{Effect}} \\ \hline
@\~	&	Turns symbol font on or off \\ \hline 
@+	&	Turns superscript on or off \\ \hline 
@-	&	Turns subscript on or off \\ \hline 
@\#	&	Turns small caps on or off \\ \hline 
@\_	&	Turns underline on or off \\ \hline 
@\%\emph{fontno}\%	&	Switches to another font; @\%\% resets to previous font \\ \hline 
@:\emph{size}:	&	Switches to another font size; @:: resets to previous size \\ \hline 
@;\emph{color};	&	Switches to another font color; @;; resets to previous color \\ \hline 
@!	&	Creates one composite character of the next two characters \\ \hline 
@@	&	Prints the @ sign itself \\ \hline 
\end{tabular}
\caption{\gmt\ text escape sequences.}
\label{tbl:escape}
\end{table}

Shorthand notation for a few special European characters has also been
added (Table~\ref{tbl:scand}):
\index{Characters!escape sequences!European}
\index{European characters}

\begin{table}[H]
\centering
\begin{tabular}{|l|l||l|l|} \hline
\emph{Code} & \emph{Effect}  & \emph{Code} & \emph{Effect} \\ \hline
@E &  \AE   & @e &  \ae   \\ \hline
@O &  \O    & @o &  \o    \\ \hline
@A &  \AA   & @a &  \aa   \\ \hline
@C &  \c{C} & @c &  \c{c} \\ \hline
@N &  \~{N} & @n &  \~{n} \\ \hline
@U &  \"{U} & @u &  \"{u} \\ \hline
@s &  \ss   &    &        \\ \hline
\end{tabular}
\caption{Shortcuts for some European characters.}
\label{tbl:scand}
\end{table}

\PS\ fonts used in \GMT\ may be re-encoded to include
several accented characters used in many European languages.  To
access these, you must specify the full octal code $\backslash$xxx
allowed for your choice of character encodings
determined by the \textbf{PS\_CHAR\_ENCODING} setting described
in the \GMTprog{gmt.conf} man page.  Only the special characters
belonging to a particular encoding will
be available.  Many characters not directly available by
using single octal codes may be constructed with the composite
character mechanism @!.
 
Some examples of escape sequences and embedded octal codes in \GMT\ strings using the
Standard+ encoding: 

\begin{tabbing}
XXX\=\verb|2@~p@~r@+2@+h@-0@- E\363tv\363s|XXXX \== XXXX\=text \kill 
\>\verb|2@~p@~r@+2@+h@-0@- E\363tv\363s| \> = \> 2$\pi r^2h_0$ E\"{o}tv\"{o}s \\ 
\>\verb|10@+-3 @Angstr@om|		 \> =	\> 10$^{-3}$ \AA ngstr\o m \\ 
\>\verb|Se@nor Gar@con|	 \> = \> Se\~{n}or Gar\c{c}on \\ 
\>\verb|M@!\305anoa stra@se|	 \> = \> M\={a}noa stra\ss e \\ 
\>\verb|A@\#cceleration@\# (ms@+-2@+)|	 \> = \> \sc{Acceleration (ms$^{-2}$)}
\end{tabbing} 

The option in \GMTprog{pstext} to draw a rectangle surrounding the text
will not work for strings with escape sequences.  A chart of characters
and their octal codes is given in Appendix~\ref{app:F}. 

\section{Grid file format specifications}
\index{grid file!formats|(}
\index{grid file!formats!netCDF}
\index{grid file!formats!COARDS}
\index{grid file!formats!floats}
\index{grid file!formats!shorts}
\index{grid file!formats!unsigned char}
\index{grid file!formats!bits}
\index{grid file!formats!raster file}
\index{grid file!formats!custom format}
\label{sec:grdformats}
\GMT\ has the ability to read and write grids using more than one grid file
format (see Table \ref{tbl:grdformats} for supported format and their IDs).  For
reading, \GMT\ will automatically determine the format of grid files, while for writing
you will normally have to append \emph{=ID} to the filename if you want \GMT\ to use a different format
than the default.

By default, \GMT\ will create new grid files using the \textbf{nf} format; however,
this behavior can be overridden by setting the \textbf{GRID\_FORMAT} defaults parameter
to any of the other recognized values (or by appending \emph{=ID}).

\GMT\ can also read netCDF grid files produced by other software packages, provided the grid files
satisfy the COARDS and Hadley Centre conventions for netCDF grids. Thus, products created under
those conventions (provided the grid is 2-, 3-, 4-, or 5-dimensional) can be read directly by \GMT\ and the netCDF grids
written by \GMT\ can be read by other programs that conform to those conventions. Three such programs are
\htmladdnormallink{\progname{ncview}}{http://meteora.ucsd.edu/~pierce/ncview_home_page.html},
\htmladdnormallink{\progname{Panoply}}{http://www.giss.nasa.gov/tools/panoply/}
and
\htmladdnormallink{\progname{ncBrowse}}{http://www.epic.noaa.gov/java/ncBrowse/}; others can be found
on the \htmladdnormallink{netCDF website}{http://www.unidata.ucar.edu/software/netcdf/software.html}.

In addition, users with some C-programming experience may add
their own read/write functions and link them with the \GMT\ library
to extend the number of predefined formats.  Technical information
on this topic can be found in the source file \filename{gmt\_customio.c}. 
Users who are considering this approach should contact the \GMT\ team.

\begin{table}[H]
\centering
\begin{tabular}{|l|l|} \hline
\multicolumn{1}{|c|}{\emph{ID}}	&	\multicolumn{1}{c|}{\emph{GMT 4 netCDF standard formats}} \\ \hline \hline
nb & GMT netCDF format (byte)   (COARDS-compliant)		\\ \hline
ns & GMT netCDF format (short)  (COARDS-compliant)		\\ \hline
ni & GMT netCDF format (int)    (COARDS-compliant)		\\ \hline
nf & GMT netCDF format (float)  (COARDS-compliant)		\\ \hline
nd & GMT netCDF format (double) (COARDS-compliant) 		\\ \hline \hline
\multicolumn{1}{|c|}{\emph{ID}}	&	\multicolumn{1}{c|}{\emph{GMT 3 netCDF legacy formats}} \\ \hline \hline
cb & GMT netCDF format (byte)	(depreciated) \\ \hline
cs & GMT netCDF format (short)	(depreciated) \\ \hline
ci & GMT netCDF format (int)	(depreciated) \\ \hline
cf & GMT netCDF format (float)	(depreciated) \\ \hline
cd & GMT netCDF format (double)	(depreciated) \\ \hline \hline
\multicolumn{1}{|c|}{\emph{ID}}	&	\multicolumn{1}{c|}{\emph{GMT native binary formats}} \\ \hline \hline
bm & GMT native, C-binary format (bit-mask)	\\ \hline
bb & GMT native, C-binary format (byte)		\\ \hline
bs & GMT native, C-binary format (short)	\\ \hline
bi & GMT native, C-binary format (int)		\\ \hline
bf & GMT native, C-binary format (float)	\\ \hline
bd & GMT native, C-binary format (double)	\\ \hline \hline
\multicolumn{1}{|c|}{\emph{ID}}	&	\multicolumn{1}{c|}{\emph{Miscellaneous grid formats}} \\ \hline \hline
rb & SUN raster file format (8-bit standard)	\\ \hline
rf & GEODAS grid format GRD98 (NGDC)		\\ \hline
sf & Golden Software Surfer format 6 (float)	\\ \hline
sd & Golden Software Surfer format 7 (double)	\\ \hline
af & Atlantic Geoscience Center AGC (float)	\\ \hline
ei & ESRI Arc/Info ASCII Grid Interchange format (integer)	\\ \hline
ef & ESRI Arc/Info ASCII Grid Interchange format (float)	\\ \hline
gd & Read-only via GDAL\footnote{Requires building \GMT\ with GDAL.} (float)	\\ \hline
\end{tabular}
\caption{\gmt\ grid file formats.}
\label{tbl:grdformats}
\end{table}

Because some formats have limitations on the range of values they can store
it is sometimes necessary to provide more
than simply the name of the file and its ID on the command line.  For instance,
a native short integer file may use a unique value to signify an empty node
or NaN, and the data may need translation and scaling prior to use.
Therefore, all \GMT\ programs that read or write grid files will decode
the given filename as follows:

\vspace{\baselineskip} 

\par 	name[=\emph{ID}[/\emph{scale}/\emph{offset}[/\emph{nan}]]]\par 

\vspace{\baselineskip} 

\noindent
where everything in brackets is optional.  If you are reading a grid
then no options are needed: just continue to pass
the name of the grid file.  However, if you write another format you must
append the =\emph{ID} string, where \emph{ID} is the format code
listed above.  In addition, should you want to (1) multiply the data by
a scale factor, and (2) add a constant offset you must append the
/\emph{scale}/\emph{offset} modifier.  Finally, if you need to indicate
that a certain data value should be interpreted as a NaN (not-a-number)
you must append the /\emph{nan} suffix to the scaling string (it cannot
go by itself; note the nesting of the brackets!). 

Some of the grid formats allow writing to standard output and reading
from standard input which means you can connect \GMT\ programs that
operate on grid files with pipes, thereby speeding up execution and
eliminating the need for large, intermediate grid files.  You specify
standard input/output by leaving out the filename entirely.
That means the ``filename'' will begin with
``=\emph{ID}\,'' since no \GMT\  netCDF format 
allow piping (due to the design of netCDF). 

Everything looks clearer after a few examples: 

\begin{enumerate}
\item To write a native binary float grid file, specify the name as \filename{my\_file.f4=bf}.

\item To read a native short integer grid file, multiply the data by 10 and then
add 32000, but first let values that equal 32767 be set to NaN,
use the filename \filename{my\_file.i2=bs/10/32000/32767}. 

\item To read a Golden Software ``surfer'' format 6 grid file, just pass the file name,
e.g., \filename{my\_surferfile.grd}. 

\item To read a 8-bit standard Sun raster file (with values in the 0--255 range)
and convert it to a \PM 1 range, give the name as
\filename{rasterfile=rb/7.84313725e-3/-1} (i.e., 1/127.5).

\item To write a native binary short integer grid file to standard output after subtracting
32000 and dividing its values by 10, give filename as \filename{=bs/0.1/-3200}. 

\end{enumerate} 

Programs that both read and/or write more than one grid file may
specify different formats and/or scaling for the files involved.
The only restriction with the embedded grid specification mechanism
is that no grid files may actually use the ``=''
character as part of their name (presumably, a small sacrifice). 

\index{grid file!suffix}

One can also define special file suffixes to imply a specific file
format; this approach represents a more intuitive and user-friendly
way to specify the various file formats.  The user may create a file
called \filename{.gmt\_io} in the current directory, home directory
or in the directory \filename{$\sim$/.gmt} and define any
number of custom formats.  The following is an example of a
\filename{.gmt\_io} file:
\index{.gmt\_io}

\vspace{\baselineskip} 

\noindent
\begin{tabbing}
MMM\=\# suffix \=format\_id \=scale \=offset \=NaNxxx\=Comments \kill 
\>\# GMT i/o shorthand file \\ 
\>\# It can have any number of comment lines like this one anywhere \\
\>\# suffix \> format\_id	\> scale \> offset \>NaN\>Comments \\ 
\>grd \> nf \> - \> - \> - \>Default format \\ 
\>b \> bf \> - \> - \> - \> Native binary floats \\ 
\>i2 \> bs \> - \> - \> 32767 \> 2-byte integers with NaN value \\ 
\>ras \> rb \> - \> - \> - \> Sun raster files \\ 
\>byte \> bb \> - \> - \> 255 \> Native binary 1-byte grids \\ 
\>bit \> bm \> - \> - \> - \> Native binary 0 or 1 grids \\ 
\>mask \> bm \> - \> - \> 0 \> Native binary 1 or NaN masks \\ 
\>faa \> bs \> 0.1 \> - \> 32767 \> Native binary gravity in 0.1 mGal
\end{tabbing} 

These suffices can be anything that makes sense to the user.  To
activate this mechanism, set parameter \textbf{IO\_GRIDFILE\_SHORTHAND} to TRUE in
your \filename{gmt.conf} file.  Then, using the filename
\filename{stuff.i2} is equivalent to saying \filename{stuff.i2=bs/1/0/32767},
and the filename \filename{wet.mask} means wet.mask=bm/1/0/0.  For a
file intended for masking, i.e., the nodes are either 1 or NaN,
the bit or mask format file may be as small as 1/32 the size of the
corresponding grid float format file. 
\index{grid file!formats|)}

\section{Options for COARDS-compliant netCDF files}
\label{sec:netcdf}
\index{grid file!formats!netCDF|(}
\index{grid file!formats!COARDS|(}

When the netCDF file contains more than one 2-dimensional variable, \GMT\ programs
will load the first such variable in the file and ignore all others. Alternatively,
the user can select the
required variable by adding the suffix ``?\emph{varname}'' to the file name. For example,
to get information on the variable ``slp'' in file \filename{file.nc}, use:
\begin{verbatim}
	grdinfo "file.nc?slp"
\end{verbatim}
Since COARDS-compliant netCDF files are the default, the additional suffix ``=nf'' can be omitted.

In case the named variable is 3-dimensional, \GMT\ will load the first (bottom) layer. If another
layer is required, either add ``[\emph{index}]'' or ``(\emph{level})'', where \emph{index} is
the index of the third (depth) variable (starting at 0 for the first layer) and \emph{level}
is the numerical value of the third (depth) variable associated with the requested layer.
To indicate the second layer of the 3-D variable ``slp'' use as file name: \filename{file.nc?slp[1]}.

When you supply the numerical value for the third variable using ``(\emph{level})'',
\GMT\ will pick the layer closest to that value. No interpolation is performed.

Note that the question mark, brackets and parentheses have special meanings on Unix-based platforms. Therefore,
you will need to either \emph{escape} these characters, by placing a backslash in front of them, or place the
whole file name plus modifiers between single quotes or double quotes.

A similar approach is followed for loading 4-dimensional grids. Consider a 4-dimensional grid with the following
variables:
\begin{verbatim}
	lat(lat): 0, 1, 2, 3, 4, 5, 6, 7, 8, 9
	lon(lon): 0, 1, 2, 3, 4, 5, 6, 7, 8, 9
	depth(depth): 0, 10, 20, 30, 40, 50, 60, 70, 80, 90
	time(time): 0, 12, 24, 36, 48
	pressure(time,depth,lat,lon): (5000 values)
\end{verbatim}
To get information on the 10$\times$10 grid of pressure at depth 10 and at time 24, one would use:
\begin{verbatim}
	grdinfo "file.nc?pressure[2,1]"
\end{verbatim}
or (only in case the coordinates increase linearly):
\begin{verbatim}
	grdinfo "file.nc?pressure(24,10)"
\end{verbatim}

The COARDS conventions set restrictions on the names that can be used for the units of the variables
and coordinates. For example, the units of longitude and latitude are ``degrees\_east'' and ``degrees\_north'',
respectively. Here is an example of the header of a COARDS compliant netCDF file (to be obtained using
\progname{ncdump}):
\begin{verbatim}
netcdf M2_fes2004 {
dimensions:
        lon = 2881 ;
        lat = 1441 ;
variables:
        float lon(lon) ;
                lon:long_name = "longitude" ;
                lon:units = "degrees_east" ;
                lon:actual_range = 0., 360. ;
        float lat(lat) ;
                lat:long_name = "latitude" ;
                lat:units = "degrees_north" ;
                lat:actual_range = -90., 90. ;
        short amp(lat, lon) ;
                amp:long_name = "amplitude" ;
                amp:unit = "m" ;
                amp:scale_factor = 0.0001 ;
                amp:add_offset = 3. ;
                amp:_FillValue = -32768s ;
        short pha(lat, lon) ;
                pha:long_name = "phase" ;
                pha:unit = "degrees" ;
                pha:scale_factor = 0.01 ;
                pha:_FillValue = -32768s ;
\end{verbatim}
This file contains two grids, which can be plotted separately using the names \filename{M2\_fes2004.nc?amp}
and \filename{M2\_fes2004.nc?pha}. The attributes \verb|long_name| and \verb|unit| for each variable are
combined in \GMT\ to a single unit string. For example, after reading the grid \verb|y_unit| equals
\verb|latitude [degrees_north]|. The same method can be used in reverse to set the proper variable names
and units when writing a grid. However, when the coordinates are set properly as geographical or time axes,
\GMT\ will take care of this. The user is, however, still responsible for setting the variable name and
unit of the z-coordinate. The default is simply ``z''.

\index{grid file!formats!netCDF|)}
\index{grid file!formats!COARDS|)}

\section{Options for grids and images read via GDAL}
\label{sec:GDAL}
\index{grid file!formats!GDAL|(}

If the support has been configured during installation, then \GMT\ can read a variety
of grid and image formats via GDAL. This extends the capability of \GMT\ to handle
data sets from a variety of sources.

\subsection{Reading multi-band images}

\GMTprog{grdimage} and \GMTprog{psimage} both lets the user select individual bands
in a multi-band image file and treats the result as an image (that is the values,
in the 0--255 range, are treated as colors, not data). To select individual bands
you use the \textbf{+b}{\it band-number} mechanism that must be appended to the image
filename. Here, {\it band-number} can be the number of one individual band (the
counting starts at zero), or it could be a comma-separated list of bands. For example
\begin{verbatim}
psimage jpeg_image_with_three_bands.jpg+b0
\end{verbatim}
will plot only the first band (i.e., the red band) of the jpeg image as a gray-scale image, and
\begin{verbatim}
psimage jpeg_image_with_three_bands.jpg+b2,1,0
\end{verbatim}
will plot the same image in color but where the RGB band order has been reversed.

Instead of treating them as images, all other \GMT\ programs that process grids can
read individual bands from an image but will consider the values to be regular data.
For example, let \filename{multiband} be the name of a multi-band file with a near
infrared component in band 4 and red in band 3. We will compute the NDVI (Normalized
Difference Vegetation Index), which is defined as NDVI = (NIR - R) / (NIR + R), as
\begin{verbatim}
grdmath multiband=gd+b3 multiband=gd+b2 SUB multiband=gd+b3 \
	multiband=gd+b2 ADD DIV = ndvi.nc
\end{verbatim}
The resulting grid \filename{ndvi.nc} can then be plotted as usual.

\subsection{Reading more complex multi-band IMAGES or GRIDS}

It is also possible to access to sub-datasets in a multi-band grid. The next example
shows how we can extract the SST from the MODIS file \filename{A20030012003365.L3m\_YR\_NSST\_9}
that is stored in the HDF ``format''. We need to run the GDAL program \GMTprog{gdalinfo} on the
file because we first must extract the necessary metadata from the file:

\begin{verbatim}
gdalinfo A20030012003365.L3m_YR_NSST_9
Driver: HDF4/Hierarchical Data Format Release 4
Files: A20030012003365.L3m_YR_NSST_9
Size is 512, 512
Coordinate System is `'
Metadata:
 Product Name=A20030012003365.L3m_YR_NSST_9
 Sensor Name=MODISA
 Sensor=
 Title=MODISA Level-3 Standard Mapped Image
...
 Scaling=linear
 Scaling Equation=(Slope*l3m_data) + Intercept = Parameter value
 Slope=0.000717185
 Intercept=-2
 Scaled Data Minimum=-2
 Scaled Data Maximum=45
 Data Minimum=-1.999999
 Data Maximum=34.76
Subdatasets:
 SUBDATASET_1_NAME=HDF4_SDS:UNKNOWN:"A20030012003365.L3m_YR_NSST_9":0
 SUBDATASET_1_DESC=[2160x4320] l3m_data (16-bit unsigned integer)
 SUBDATASET_2_NAME=HDF4_SDS:UNKNOWN:"A20030012003365.L3m_YR_NSST_9":1
 SUBDATASET_2_DESC=[2160x4320] l3m_qual (8-bit unsigned integer)
\end{verbatim}

Now, to access this file with \GMT\ we need to use the =gd mechanism and append the name
of the sub-dataset that we want to extract. Here, a simple example using \GMTprog{grdinfo}
would be

\scriptsize
\begin{verbatim}
grdinfo A20030012003365.L3m_YR_NSST_9=gd?HDF4_SDS:UNKNOWN:"A20030012003365.L3m_YR_NSST_9:0"

HDF4_SDS:UNKNOWN:A20030012003365.L3m_YR_NSST_9:0: Title: Grid imported via GDAL
HDF4_SDS:UNKNOWN:A20030012003365.L3m_YR_NSST_9:0: Command:
HDF4_SDS:UNKNOWN:A20030012003365.L3m_YR_NSST_9:0: Remark:
HDF4_SDS:UNKNOWN:A20030012003365.L3m_YR_NSST_9:0: Gridline node registration used
HDF4_SDS:UNKNOWN:A20030012003365.L3m_YR_NSST_9:0: Grid file format: gd = Import through GDAL (convert to float)
HDF4_SDS:UNKNOWN:A20030012003365.L3m_YR_NSST_9:0: x_min: 0.5 x_max: 4319.5 x_inc: 1 name: x nx: 4320
HDF4_SDS:UNKNOWN:A20030012003365.L3m_YR_NSST_9:0: y_min: 0.5 y_max: 2159.5 y_inc: 1 name: y ny: 2160
HDF4_SDS:UNKNOWN:A20030012003365.L3m_YR_NSST_9:0: z_min: 0 z_max: 65535 name: z
HDF4_SDS:UNKNOWN:A20030012003365.L3m_YR_NSST_9:0: scale_factor: 1 add_offset: 0
\end{verbatim}
\normalsize

Be warned, however, that things are not yet completed because while the data are scaled
according to the equation printed above (``Scaling Equation=(Slope*l3m\_data) + Intercept
= Parameter value''), this scaling is not applied by GDAL on reading so it cannot be done
automatically by \GMT. One solution is to do the reading and scaling via \GMTprog{grdmath}
first, i.e.,

\scriptsize
\begin{verbatim}
grdmath A20030012003365.L3m_YR_NSST_9=gd?HDF4_SDS:UNKNOWN:"A20030012003365.L3m_YR_NSST_9:0" \
	0.000717185 MUL -2 ADD = sst.nc
\end{verbatim}
\normalsize
\noindent
then plot the \filename{sst.nc} directly.

\index{grid file!formats!GDAL|)}

\section{The NaN data value}
\index{NaN}
\index{Not-a-Number}

For a variety of data processing and plotting tasks there is a need to acknowledge that
a data point is missing or unassigned.  In the ``old days'' such information was passed
by letting a value like -9999.99 take on the special meaning of ``this is not really a
value, it is missing''.  The problem with this scheme is that -9999.99 (or any other
floating point value) may be a perfectly reasonable data value and in such a scenario
would be skipped.  The solution adopted in \GMT\ is to use the IEEE concept Not-a-Number
(NaN) for this purpose.  Mathematically, a NaN is what you get if you do an undefined
mathematical operation like $0/0$; in ASCII data files they appear as the textstring NaN.
This value is internally stored with a particular bit pattern
defined by IEEE so that special action can be taken when it is encountered by programs.
In particular, a standard library function called \texttt{isnan} is used to test if a floating point
is a NaN.  \GMT\ uses these tests extensively to determine if a value is suitable for plotting
or processing (if a NaN is used in a calculation the result would become NaN as well).  Data points
whose values equal NaN are not normally plotted (or plotted with the special NaN color given in
\filename{gmt.conf}).  Several tools such as \GMTprog{xyz2grd}, \GMTprog{gmtmath}, and
\GMTprog{grdmath} can convert user data to NaN and vice versa, thus facilitating arbitrary
masking and clipping of data sets.  Note that a few computers do not have native IEEE hardware
support.  At this point, this applies to some of the older Cray super-computers.  Users on such
machines may have to adopt the old `-9999.99'' scheme to achieve the desired results.

Data records that contain NaN values for the \emph{x} or \emph{y} columns (or the \emph{z} column
for cases when 3-D Cartesian data are expected) are usually skipped during reading.  However,
the presence of these bad records can be interpreted in two different ways, and this behavior
is controlled by the \textbf{IO\_NAN\_RECORDS} defaults parameter.  The default setting (\emph{gap})
considers such records to indicate a gap in an otherwise continuous series of points (e.g., a line),
and programs can act upon this information, e.g., not to draw a line across the gap or to break the line
into separate segments.  The alternative setting (\emph{bad}) makes no such interpretation and
simply reports back how many bad records were skipped during reading; see Section~\ref{sec:gap} for details.

\section{\gmt\ environment parameters}
\index{Environment parameters}

\GMT\ relies on several environment parameters, in particular to find data files and program settings.
\begin{description}
\item [\$GMT\_SHAREDIR] points to the \GMT\ share directory where all run-time support files
	such as coastlines, custom symbols, \PS\ macros, color tables, and much more reside.  If this
	parameter is not set it defaults to the share sub-directory selected during the \GMT\
	install process (i.e., your answer to question C.9 on the web install form, or specified by the
	option \filename{--datarootdir} in the \filename{configure} step of the installation), which normally
	is the share directory under the \GMT\ installation directory.
\item [\$GMT\_DATADIR] points to one or more directories where large and/or widely used data files can
	be placed.  All \GMT\ programs look in these directories when a file is specified on the
	command line and it is not present in the current directory.  This allows maintainers to
	consolidate large data files and to simplify scripting that use these files since the absolute path need not be specified.
	Separate multiple directories with colons (:); under Windows you use semi-colons (;).  Any directory
	name that ends in a trailing / will be search recursively (not under Windows).
\item [\$GMT\_USERDIR] points to a directory where the user may place custom configuration files
	(e.g., an alternate \filename{coastline.conf} file, preferred default settings in \filename{gmt.conf},
	custom symbols and color palettes, and
	shorthands for gridfile extensions via \filename{.gmt\_io}).  Users may also place their own
	data files in this directory as \GMT\ programs will search for files given on the command
	line in both \textbf{\$GMT\_DATADIR} and \textbf{\$GMT\_USERDIR}.
\item [\$GMT\_TMPDIR] is where \GMT\ will write its state parameters via the two files
	\filename{.gmtcommands} and \filename{gmt.conf}.  If \textbf{\$GMT\_TMPDIR} is not set,
	these files are written to the current directory. See Appendix~\ref{app:P} for more on
	the use of \textbf{\$GMT\_TMPDIR}.
\end{description}
Note that files whose full path is given will never be searched for in any of these directories.
