%------------------------------------------
%	$Id: GMT_Chapter_3.tex,v 1.46 2011-01-02 20:09:05 guru Exp $
%
%	The GMT Documentation Project
%	Copyright 2000-2011.
%	Paul Wessel and Walter H. F. Smith
%------------------------------------------
%

\chapter{\gmt\ overview and quick reference}
\label{ch:3}
\index{GMT@\GMT!overview|(}
\thispagestyle{headings}

\section{\gmt\ summary}
The following is a summary of all the programs supplied with \GMT\
and a very short description of their purpose.  For more details,
see the individual \UNIX\ manual pages or the online web documentation. 
For a listing sorted by program purpose, see Section~\ref{sec:purpose}.
\begin{tabbing}
nearneighborxxxxx \=					\kill
\GMTprog{blockmean}	\>	L$_2$ (\emph{x},\emph{y},\emph{z}) table data filter/decimator \\ 
\GMTprog{blockmedian}	\>	L$_1$ (\emph{x},\emph{y},\emph{z}) table data filter/decimator \\ 
\GMTprog{blockmode}	\>	Mode estimate (\emph{x},\emph{y},\emph{z}) table data filter/decimator \\ 
\GMTprog{filter1d}	\>	Filter 1-D table data sets (time series) \\ 
\GMTprog{fitcircle}	\>	Finds the best-fitting great or small circle for a set of points \\ 
\GMTprog{gmt2rgb}	\>	Convert Sun raster or grid file to red, green, blue component grids \\ 
\GMTprog{gmtconvert}	\>	Convert data tables from one format to another \\ 
\GMTprog{gmtdefaults}	\>	List the current default settings \\ 
\GMTprog{gmtmath}	\>	Mathematical operations on table data \\ 
\GMTprog{gmtselect}	\>	Select subsets of table data based on multiple spatial criteria \\ 
\GMTprog{gmtset}	\>	Change selected parameters in current \filename{.gmtdefaults4} file \\ 
\GMTprog{grd2cpt}	\>	Make color palette table from a grid files \\ 
\GMTprog{grd2xyz}	\>	Conversion from 2-D grid file to table data \\ 
\GMTprog{grdblend}	\>	Blend several partially over-lapping grid files onto one grid \\ 
\GMTprog{grdclip}	\>	Limit the \emph{z}-range in gridded data sets \\ 
\GMTprog{grdcontour}	\>	Contouring of 2-D gridded data sets \\ 
\GMTprog{grdcut}	\>	Cut a sub-region from a grid file \\ 
\GMTprog{grdedit}	\>	Modify header information in a 2-D grid file \\ 
\GMTprog{grdfft}	\>	Perform operations on grid files in the frequency domain \\ 
\GMTprog{grdfilter}	\>	Filter 2-D gridded data sets in the space domain \\ 
\GMTprog{grdgradient}	\>	Compute directional gradient from grid files \\ 
\GMTprog{grdhisteq}	\>	Histogram equalization for grid files \\ 
\GMTprog{grdimage}	\>	Produce images from 2-D gridded data sets \\ 
\GMTprog{grdinfo}	\>	Get information about grid files \\ 
\GMTprog{grdlandmask}	\>	Create masking grid files from shoreline data base \\ 
\GMTprog{grdmask}	\>	Reset grid nodes in/outside a clip path to constants \\ 
\GMTprog{grdmath}	\>	Mathematical operations on grid files \\ 
\GMTprog{grdpaste}	\>	Paste together grid files along a common edge \\ 
\GMTprog{grdproject}	\>	Project gridded data sets onto a new coordinate system \\ 
\GMTprog{grdreformat}	\>	Converts grid files into other grid formats \\ 
\GMTprog{grdsample}	\>	Resample a 2-D gridded data set onto a new grid \\ 
\GMTprog{grdtrack}	\>	Sampling of 2-D gridded data set along 1-D track \\ 
\GMTprog{grdtrend}	\>	Fits polynomial trends to grid files \\ 
\GMTprog{grdvector}	\>	Plotting of 2-D gridded vector fields \\ 
\GMTprog{grdview}	\>	3-D perspective imaging of 2-D gridded data sets \\ 
\GMTprog{grdvolume}	\>	Calculate volumes under a surface within specified contour \\ 
\GMTprog{greenspline}	\>	Interpolation using Green's functions for splines in 1--3 dimensions \\ 
\GMTprog{makecpt}	\>	Make color palette tables \\ 
\GMTprog{mapproject}	\>	Transformation of coordinate systems for table data \\ 
\GMTprog{minmax}	\>	Report extreme values in table data files \\ 
\GMTprog{nearneighbor}	\>	Nearest-neighbor gridding scheme \\ 
\GMTprog{project}	\>	Project table data onto lines or great circles \\ 
\GMTprog{ps2raster}	\>	Crop and convert PostScript files to raster images, EPS, and PDF \\
\GMTprog{psbasemap}	\>	Create a basemap plot \\ 
\GMTprog{psclip}	\>	Use polygon files to define clipping paths \\ 
\GMTprog{pscoast}	\>	Plot (and fill) coastlines, borders, and rivers on maps \\ 
\GMTprog{pscontour}	\>	Contour or image raw table data by triangulation \\ 
\GMTprog{pshistogram}	\>	Plot a histogram \\ 
\GMTprog{psimage}	\>	Plot Sun raster files on a map \\ 
\GMTprog{pslegend}	\>	Plot a legend on a map \\ 
\GMTprog{psmask}	\>	Create overlay to mask out regions on maps \\ 
\GMTprog{psrose}	\>	Plot sector or rose diagrams \\ 
\GMTprog{psscale}	\>	Plot gray scale or color scale on maps \\ 
\GMTprog{pstext}	\>	Plot text strings on maps \\ 
\GMTprog{pswiggle}	\>	Draw table data time-series along track on maps \\ 
\GMTprog{psxy}		\>	Plot symbols, polygons, and lines on maps \\ 
\GMTprog{psxyz}		\>	Plot symbols, polygons, and lines in 3-D \\ 
\GMTprog{sample1d}	\>	Resampling of 1-D table data sets \\ 
\GMTprog{spectrum1d}	\>	Compute various spectral estimates from time-series \\ 
\GMTprog{splitxyz}	\>	Split \emph{xyz} files into several segments \\ 
\GMTprog{surface}	\>	A continuous curvature gridding algorithm \\ 
\GMTprog{trend1d}	\>	Fits polynomial or Fourier trends to $y = f(x)$ series \\ 
\GMTprog{trend2d}	\>	Fits polynomial trends to $z = f(x,y)$ series \\ 
\GMTprog{triangulate}	\>	Perform optimal Delauney triangulation and gridding \\ 
\GMTprog{xyz2grd}	\>	Convert an equidistant table \emph{xyz} file to a 2-D grid file
\end{tabbing}
\index{GMT@\GMT!overview|)}

\section{\gmt\ quick reference}
\index{GMT@\GMT!quick reference|(}
\label{sec:purpose}
Instead of an alphabetical listing, this section contains a summary
sorted by program purpose.  Also included is a quick summary of the
standard command line options and a breakdown of the \Opt{J} option
for each of the over 30 projections available in \GMT. \\

\begin{center}

\begin{tabular}{|ll|}
\multicolumn{2}{c}{\bf FILTERING OF 1-D AND 2-D DATA} \\\hline
\GMTprog{blockmean}	&	L$_2$ estimate ($x, y, z$) data filters/decimators \\ \hline
\GMTprog{blockmedian}	&	L$_1$ estimate ($x, y, z$) data filters/decimators \\ \hline
\GMTprog{blockmode}	&	Mode estimate ($x, y, z$) data filters/decimators \\ \hline
\GMTprog{filter1d}	&	Filter 1-D data (time series) \\ \hline
\GMTprog{grdfilter}	&	Filter 2-D data in space domain \\ \hline
\multicolumn{2}{c}{\bf PLOTTING OF 1-D and 2-D DATA} \\ \hline
\GMTprog{grdcontour}	&	Contouring of 2-D gridded data\\ \hline
\GMTprog{grdimage}	&	Produce images from 2-D gridded data \\ \hline
\GMTprog{grdvector}	&	Plot vector fields from 2-D gridded data \\ \hline
\GMTprog{grdview}	&	3-D perspective imaging of 2-D gridded data \\ \hline
\GMTprog{psbasemap}	&	Create a basemap frame \\ \hline
\GMTprog{psclip}	&	Use polygon files as clipping paths \\ \hline
\GMTprog{pscoast}	&	Plot coastlines, filled continents, rivers, and political borders \\ \hline
\GMTprog{pscontour}	&	Direct contouring or imaging of \emph{xyz} data by triangulation \\ \hline
\GMTprog{pshistogram}	&	Plot a histogram \\ \hline
\GMTprog{psimage}	&	Plot Sun raster files on a map \\ \hline
\GMTprog{pslegend}	&	Plot a legend on a map \\ \hline
\GMTprog{psmask}	&	Create overlay to mask specified regions of a map \\ \hline
\GMTprog{psrose}	&	Plot sector or rose diagrams \\ \hline
\GMTprog{psscale}	&	Plot gray scale or color scale \\ \hline
\GMTprog{pstext}	&	Plot text strings \\ \hline
\GMTprog{pswiggle}	&	Draw anomalies along track \\ \hline
\GMTprog{psxy}		&	Plot symbols, polygons, and lines in 2-D \\ \hline
\GMTprog{psxyz}		&	Plot symbols, polygons, and lines in 3-D \\ \hline
\end{tabular}

\begin{tabular}{|ll|}
\multicolumn{2}{c}{\bf GRIDDING OF (X,Y,Z) TABLE DATA} \\ \hline
\GMTprog{greenspline}	&	Interpolation using Green's functions for splines in 1--3 dimensions \\ \hline
\GMTprog{nearneighbor}	&	Nearest-neighbor gridding scheme \\ \hline
\GMTprog{surface}	&	Continuous curvature gridding algorithm \\ \hline
\GMTprog{triangulate}	&	Perform optimal Delauney triangulation on \emph{xyz} data \\ \hline
\multicolumn{2}{c}{\bf SAMPLING OF 1-D AND 2-D DATA} \\ \hline
\GMTprog{grdsample}	&	Resample a 2-D gridded data onto new grid \\ \hline
\GMTprog{grdtrack}	&	Sampling of 2-D data along 1-D track \\ \hline
\GMTprog{sample1d}	&	Resampling of 1-D data \\ \hline
\multicolumn{2}{c}{\bf PROJECTION AND MAP-TRANSFORMATION} \\ \hline
\GMTprog{grdproject}	&	Transform gridded data to a new coordinate system \\ \hline
\GMTprog{mapproject}	&	Transform table data to a new coordinate system \\ \hline
\GMTprog{project}	&	Project data onto lines or great circles \\ \hline
\multicolumn{2}{c}{\bf INFORMATION} \\ \hline
\GMTprog{gmtdefaults}	&	List the current default settings \\ \hline
\GMTprog{gmtset}	&	Command-line editing of parameters in the \filename{.gmtdefaults4} file \\ \hline
\GMTprog{grdinfo}	&	Get information about the content of grid files \\ \hline
\GMTprog{minmax}	&	Report extreme values in table data files \\ \hline
\multicolumn{2}{c}{\bf MISCELLANEOUS} \\ \hline
\GMTprog{gmtmath}	&	Reverse Polish Notation (RPN) calculator for table data \\ \hline
\GMTprog{makecpt}	&	Create GMT color palette tables \\ \hline
\GMTprog{spectrum1d}	&	Compute spectral estimates from time-series \\ \hline
\GMTprog{triangulate}	&	Perform optimal Delauney triangulation on xyz data \\ \hline
\multicolumn{2}{c}{\bf CONVERT OR EXTRACT SUBSETS OF DATA} \\ \hline
\GMTprog{gmt2rgb}	&	Convert Sun raster or grid file to red, green, blue component grids \\ \hline 
\GMTprog{gmtconvert}	&	Convert table data from one format to another \\ \hline
\GMTprog{gmtselect}	&	Select table data subsets based on multiple spatial criteria \\ \hline
\GMTprog{grd2xyz}	&	Convert 2-D gridded data to table data \\ \hline
\GMTprog{grdcut}	&	Cut a sub-region from a grid file \\ \hline
\GMTprog{grdblend}	&	Blend several partially over-lapping grid files onto one grid \\ \hline
\GMTprog{grdpaste}	&	Paste together grid files along common edge \\ \hline
\GMTprog{grdreformat}	&	Convert from one grid format to another \\ \hline
\GMTprog{splitxyz}	&	Split ($x, y, z$) table data into several segments \\ \hline
\GMTprog{xyz2grd}	&	Convert table data to 2-D grid file \\ \hline
\multicolumn{2}{c}{\bf DETERMINE TRENDS IN 1-D AND 2-D DATA} \\ \hline
\GMTprog{fitcircle}	&	Finds best-fitting great or small circles \\ \hline
\GMTprog{grdtrend}	&	Fits polynomial trends to grid files ($z = f(x, y)$) \\ \hline
\GMTprog{trend1d}	&	Fits polynomial or Fourier trends to $y = f(x)$ series \\ \hline
\GMTprog{trend2d}	&	Fits polynomial trends to $z = f(x, y)$ series \\ \hline
\multicolumn{2}{c}{\bf OTHER OPERATIONS ON 2-D GRIDS} \\ \hline
\GMTprog{grd2cpt}	&	Make color palette table from grid file \\ \hline
\GMTprog{grdclip}	&	Limit the $z$--range in gridded data sets \\ \hline
\GMTprog{grdedit}	&	Modify grid header information \\ \hline
\GMTprog{grdfft}	&	Operate on grid files in frequency domain \\ \hline
\GMTprog{grdgradient}	&	Compute directional gradients from grid files \\ \hline
\GMTprog{grdhisteq}	&	Histogram equalization for grid files \\ \hline
\GMTprog{grdlandmask}	&	Creates mask grid file from coastline database \\ \hline
\GMTprog{grdmask}	&	Set grid nodes in/outside a clip path to constants \\ \hline
\GMTprog{grdmath}	&	Reverse Polish Notation (RPN) calculator for grid files \\ \hline
\GMTprog{grdvolume}	&	Calculate volume under a surface within a contour \\ \hline
\multicolumn{2}{c}{\bf MANIPULATING \GMT\ POSTSCRIPT FILES} \\ \hline
\GMTprog{ps2raster}	&	Crop and convert PostScript files to raster images, EPS and PDF \\ \hline
\end{tabular}

\clearpage

\index{Standardized command line options|(}
\index{\Opt{B} (set annotations and ticks)|(}
\index{\Opt{H} (header records)|(}
\index{\Opt{J} (set map projection)|(}
\index{Projection!azimuthal!Lambert \Opt{Ja} \Opt{JA}|(}
\index{Lambert azimuthal projection \Opt{Ja} \Opt{JA}|(}
\index{\Opt{Ja} \Opt{JA} (Lambert azimuthal)|(}
\index{Projection!conic!Albers \Opt{Jb} \Opt{JB}|(}
\index{Albers conic projection \Opt{Jb} \Opt{JB}|(}
\index{\Opt{Jb} \Opt{JB} (Albers)|(}
\index{Projection!cylindrical!Cassini \Opt{Jc} \Opt{JC}|(}
\index{Cassini projection \Opt{Jc} \Opt{JC}|(}
\index{\Opt{Jc} \Opt{JC} (Cassini)|(}
\index{Projection!cylindrical!stereographic \Opt{Jcyl\_stere} \Opt{JCyl\_stere}|(}
\index{Cylindrical stereographic projection \Opt{Jcyl\_stere} \Opt{JCyl\_stere}|(}
\index{\Opt{Jcyl\_stere} \Opt{JCyl\_stere} (Cylindrical stereographic)|(}
\index{Projection!conic!equidistant \Opt{Jd} \Opt{JD}|(}
\index{Equidistant conic projection \Opt{Jd} \Opt{JD}|(}
\index{\Opt{Jd} \Opt{JD} (Equidistant conic)|(}
\index{Projection!azimuthal!equidistant \Opt{Je} \Opt{JE}|(}
\index{Azimuthal equidistant projection \Opt{Je} \Opt{JE}|(}
\index{\Opt{Je} \Opt{JE} (Azimuthal equidistant)|(}
\index{Projection!azimuthal!gnomonic \Opt{Jf} \Opt{JF}|(}
\index{Gnomonic projection \Opt{Jf} \Opt{JF}|(}
\index{\Opt{Jf} \Opt{JF} (Gnomonic)|(}
\index{Projection!azimuthal!orthographic \Opt{Jg} \Opt{JG}|(}
\index{Orthographic projection \Opt{Jg} \Opt{JG}|(}
\index{\Opt{Jg} \Opt{JG} (Orthographic)|(}
\index{Hammer projection \Opt{Jh} \Opt{JH}|(}
\index{Projection!miscellaneous!Hammer \Opt{Jh} \Opt{JH}|(}
\index{\Opt{Jh} \Opt{JH} (Hammer)|(}
\index{Sinusoidal projection \Opt{Ji} \Opt{JI}|(}
\index{Projection!miscellaneous!Sinusoidal \Opt{Ji} \Opt{JI}|(}
\index{\Opt{Ji} \Opt{JI} (Sinusoidal)|(}
\index{Projection!cylindrical!Miller \Opt{Jj} \Opt{JJ}|(}
\index{Miller cylindrical projection \Opt{Jj} \Opt{JJ}|(}
\index{\Opt{Jj} \Opt{JJ} (Miller)|(}
\index{Eckert IV and VI projection \Opt{Jk} \Opt{JK}|(}
\index{Projection!miscellaneous!Eckert IV and VI \Opt{Jk} \Opt{JK}|(}
\index{\Opt{Jk} \Opt{JK} (Eckert IV and VI)|(}
\index{Projection!conic!Lambert \Opt{Jl} \Opt{JL}|(}
\index{Lambert conic projection \Opt{Jl} \Opt{JL}|(}
\index{\Opt{Jl} \Opt{JL} (Lambert conic)|(}
\index{Projection!cylindrical!Mercator \Opt{Jm} \Opt{JM}|(}
\index{Mercator projection \Opt{Jm} \Opt{JM}|(}
\index{\Opt{Jm} \Opt{JM} (Mercator)|(}
\index{Robinson projection \Opt{Jn} \Opt{JN}|(}
\index{Projection!miscellaneous!Robinson \Opt{Jn} \Opt{JN}|(}
\index{\Opt{Jn} \Opt{JN} (Robinson)|(}
\index{Projection!cylindrical!oblique Mercator \Opt{Jo} \Opt{JO}|(}
\index{Oblique Mercator projection \Opt{Jo} \Opt{JO}|(}
\index{\Opt{Jo} \Opt{JO} (Oblique Mercator)|(}
\index{Projection!polar ($\theta, r$) \Opt{Jp} \Opt{JP}|(}
\index{Polar ($\theta, r$) projection \Opt{Jp} \Opt{JP}|(}
\index{\Opt{Jp} \Opt{JP} (Polar ($\theta, r$) projections)|(}
\index{Projection!conic!polyconic \Opt{Jpoly} \Opt{JPoly}|(}
\index{Polyconic projection \Opt{Jpoly} \Opt{JPoly}|(}
\index{\Opt{Jpoly} \Opt{JPoly} (Polyconic)|(}
\index{Projection!cylindrical!equidistant \Opt{Jq} \Opt{JQ}|(}
\index{Equidistant cylindrical projection \Opt{Jq} \Opt{JQ}|(}
\index{\Opt{Jq} \Opt{JQ} (Cylindrical equidistant)|(}
\index{Winkel Tripel projection \Opt{Jr} \Opt{JR}|(}
\index{Projection!miscellaneous!Winkel Tripel \Opt{Jr} \Opt{JR}|(}
\index{\Opt{Jr} \Opt{JR} (Winkel Tripel)|(}
\index{Projection!azimuthal!stereographic \Opt{Js} \Opt{JS}|(}
\index{Stereographic projection \Opt{Js} \Opt{JS}|(}
\index{\Opt{Js} \Opt{JS} (Stereographic)|(}
\index{Projection!cylindrical!transverse Mercator \Opt{Jt} \Opt{JT}|(}
\index{Projection!cylindrical!UTM \Opt{Ju} \Opt{JU}|(}
\index{Transverse Mercator projection \Opt{Jt} \Opt{JT}|(}
\index{UTM projection \Opt{Ju} \Opt{JU}|(}
\index{\Opt{Jt} \Opt{JT} (Transverse Mercator)|(}
\index{\Opt{Ju} \Opt{JU} (UTM)|(}
\index{Van der Grinten projection \Opt{Jv} \Opt{JV}|(}
\index{Projection!miscellaneous!Van der Grinten \Opt{Jv} \Opt{JV}|(}
\index{\Opt{Jv} \Opt{JV} (Van der Grinten)|(}
\index{Mollweide projection \Opt{Jw} \Opt{JW}|(}
\index{Projection!miscellaneous!Mollweide \Opt{Jw} \Opt{JW}|(}
\index{\Opt{Jw} \Opt{JW} (Mollweide)|(}
\index{Projection!linear|(}
\index{Projection!linear!Cartesian|(}
\index{Cartesian linear projection|(}
\index{Linear projection|(}
\index{\Opt{Jx} \Opt{JX} (Non-map projections)|(}
\index{Projection!logarithmic|(}
\index{Logarithmic projection|(}
\index{Projection!power (exponential)|(}
\index{Power (exponential) projection|(}
\index{Projection!linear!geographic|(}
\index{Geographic Linear projection|(}
\index{Projection!cylindrical!general \Opt{Jy} \Opt{JY}|(}
\index{General cylindrical projection \Opt{Jy} \Opt{JY}|(}
\index{\Opt{Jy} \Opt{JY} (General cylindrical)|(}
\index{\Opt{K} (continue plot)|(}
\index{\Opt{O} (overlay plot)|(}
\index{\Opt{P} (portrait orientation)|(}
\index{\Opt{R} (set region)|(}
\index{\Opt{U} (plot timestamp)|(}
\index{\Opt{V} (verbose mode)|(}
\index{\Opt{X} (shift plot in $x$)|(}
\index{\Opt{Y} (shift plot in $y$)|(}
\index{\Opt{b} (binary i/o)|(}
\index{\Opt{c} (set \# of copies)|(}
\index{\Opt{f} (formatting of i/o)|(}
\index{\Opt{g} (detect gaps in i/o)|(}
\index{\Opt{m} (use multi-segment i/o)|(}
\index{\Opt{:} (input and/or output is $y,x$, not $x,y$)|(}



\newenvironment{cmdlineopts}[1]{
\newcommand{\lon}[1]{$lon_{##1}$}
\newcommand{\lat}[1]{$lat_{##1}$}
\newcommand{\Wi}{\emph{width}}
\newcommand{\wi}{\emph{scale}}
\newcommand{\ho}{[/\emph{horizon}]}
\newcommand{~}{\hspace{0.2in}}

\begin{tabular}{|ll|}
\multicolumn{2}{c}{\bf STANDARDIZED COMMAND LINE OPTIONS #1} \\ \hline
\multicolumn{2}{|l|}{\Opt{B}[\textbf{p}$|$\textbf{s}]\emph{xinfo}[\emph{/yinfo}[\emph{/zinfo}]][\emph{WESNZwesnz+}][\emph{:.title:}] Tickmarks. Each \emph{info} is} \\
\multicolumn{2}{|l|}{~[\textbf{t}]\emph{stride}[\PM \emph{phase}][\textbf{u}][\textbf{l}$|$\textbf{p}][:"label":][:="prefix":][:,"unit":], where \textbf{l} and \textbf{p} apply to $\log_{10}$ axes only,} \\
\multicolumn{2}{|l|}{~and type \textbf{t} = \{\textbf{a, A, f, g, i, I}\}; unit \textbf{u} = \{\textbf{c, C, d, D, h, H, K, k, m, M, o, O, r, R, u, U, y, Y}\}} \\
\multicolumn{2}{|l|}{~The leading \textbf{p}$|$\textbf{s} selects primary [Default] or secondary axis items} \\ \hline
\Opt{H}[\textbf{i}][\emph{n\_headers}]		&	ASCII [input] tables have header record[s] \\ \hline
\Opt{J}	(upper case for \Wi, lower case for \wi) &	Map projection \\ \hline
}{
\Opt{K}	&	Append more PS later \\ \hline
\Opt{O}	&	This is an overlay plot \\ \hline
\Opt{P}	&	Select Portrait orientation \\ \hline
\Opt{R}\emph{west/east/south/north}[\emph{/zmin/zmax}][\textbf{r}] & Specify Region of interest \\ \hline
\Opt{U}[[\emph{just}]/\emph{dx}/\emph{dy}/][\emph{label}]	&	Plot time-stamp on plot \\ \hline
\Opt{V}	&	Run in verbose mode \\ \hline
\Opt{X}[\textbf{a}$|$\textbf{c}$|$\textbf{r}]\emph{off}[\textbf{u}]	&	Shift plot origin in $x$-direction \\ \hline
\Opt{Y}[\textbf{a}$|$\textbf{c}$|$\textbf{r}]\emph{off}[\textbf{u}]	&	Shift plot origin in $y$-direction \\ \hline
\Opt{b}[\textbf{i}$|$\textbf{o}][\textbf{c}$|$\textbf{s}$|$\textbf{S}$|$\textbf{d}$|$\textbf{D}][\emph{ncol}]	&	Select binary input or output \\ \hline
\Opt{c}\emph{copies}	&	Set number of plot copies [1] \\ \hline
\Opt{f}[\textbf{i}$|$\textbf{o}]\emph{colinfo}	&	Set formatting of ASCII input or output \\ \hline
\Opt{g}[\textbf{+}]\textbf{x}$|$\textbf{X}$|$\textbf{y}$|$\textbf{Y}$|$\textbf{d}$|$\textbf{D}\emph{gap}[\textbf{u}]	&	Segment data by detecting gaps \\ \hline
\Opt{m}[\textbf{i}$|$\textbf{o}]\emph{flag}	&	Set multi-segment data mode \\ \hline
\Opt{:}[\textbf{i}$|$\textbf{o}]	&	Expect \emph{y}/\emph{x} input rather than \emph{x}/\emph{y} \\ \hline
\end{tabular}
}

\begin{cmdlineopts}{WITH OLD PROJECTION CODES}
~\Opt{JA}\lon0/\lat0\ho/\Wi	&	Lambert azimuthal equal area \\ \hline
~\Opt{JB}\lon0/\lat0/\lat1/\lat2/\Wi	&	Albers conic equal area \\ \hline
~\Opt{JC}\lon0/\lat0/\Wi	&	Cassini cylindrical \\ \hline
~\Opt{JCyl\_stere/}[\lon0/[\lat0/]]\Wi & Cylindrical stereographic \\ \hline
~\Opt{JD}\lon0/\lat0/\lat1/\lat2/\Wi	&	Equidistant conic \\ \hline
~\Opt{JE}\lon0/\lat0\ho/\Wi	&	Azimuthal equidistant \\ \hline
~\Opt{JF}\lon0/\lat0\ho/\Wi	&	Azimuthal gnomonic \\ \hline
~\Opt{JG}\lon0/\lat0\ho/\Wi	&	Azimuthal orthographic \\ \hline
~\Opt{JG}\lon0/\lat0/\emph{alt}/\emph{azim}/\emph{tilt}/\emph{twist}/\emph{W}/\emph{H}/\Wi & General perspective\\\hline
~\Opt{JH}[\lon0/]\Wi	&	Hammer equal area \\ \hline
~\Opt{JI}[\lon0/]\Wi	&	Sinusoidal equal area \\ \hline
~\Opt{JJ}[\lon0/]\Wi	&	Miller cylindrical \\ \hline
~\Opt{JKf}[\lon0/]\Wi	&	Eckert IV equal area \\ \hline
~\Opt{JKs}[\lon0/]\Wi	&	Eckert VI equal area \\ \hline
~\Opt{JL}\lon0/\lat0/\lat1/\lat2/\Wi	&	Lambert conic conformal \\ \hline
~\Opt{JM}[\lon0/[\lat0/]]\Wi	&	Mercator cylindrical \\ \hline
~\Opt{JN}[\lon0/]\Wi	&	Robinson \\ \hline
~\Opt{JOa}\lon0/\lat0/\emph{azim}/\Wi	&	Oblique Mercator, 1:	origin and azimuth \\ \hline
~\Opt{JOb}\lon0/\lat0/\lon1/\lat1/\Wi	&	Oblique Mercator, 2:	two points \\ \hline
~\Opt{JOc}\lon0/\lat0/\lon{p}/\lat{p}/\Wi	&	Oblique Mercator, 3:	origin and pole \\ \hline
~\Opt{JP}[\textbf{a}]\Wi[/\emph{origin}]	&	Polar [azimuthal] ($\theta, r$) (or cylindrical) \\ \hline
~\Opt{JPoly}[\lon0/[\lat0/]]\Wi	&	(American) polyconic \\ \hline
~\Opt{JQ}[\lon0/[\lat0/]]\Wi	&	Equidistant cylindrical \\ \hline
~\Opt{JR}[\lon0/]\Wi	&	Winkel Tripel \\ \hline
~\Opt{JS}\lon0/\lat0/\ho/\Wi	&	General stereographic \\ \hline
~\Opt{JT}\lon0/[\lat0/]\Wi	&	Transverse Mercator \\ \hline
~\Opt{JU}\emph{zone}/\Wi	&	Universal Transverse Mercator (UTM) \\ \hline
~\Opt{JV}[\lon0/]\Wi	&	Van der Grinten \\ \hline
~\Opt{JW}[\lon0/]\Wi	&	Mollweide \\ \hline
~\Opt{JX}\Wi[\textbf{l}$|$\textbf{p}\emph{exp}$|$\textbf{T}$|$\textbf{t}][/\emph{height}[\textbf{l}$|$\textbf{p}\emph{exp}$|$\textbf{T}$|$\textbf{t}]][\textbf{d}]	&	Linear, log$_{10}$, $x^a$--$y^b$, and time \\ \hline
~\Opt{JY}\lon0/\lat0/\Wi	&	Cylindrical equal area \\ \hline
\end{cmdlineopts}

\begin{cmdlineopts}{(WITH \progname{Proj4} PROJECTION CODES)}
~\Opt{Jaea/}\lon0/\lat0/\lat1/\lat2/\wi	&	Albers conic equal area \\ \hline
~\Opt{Jaeqd/}\lon0/\lat0\ho/\wi	&	Azimuthal equidistant \\ \hline
~\Opt{Jcass/}\lon0/\lat0/\wi	&	Cassini cylindrical \\ \hline
~\Opt{Jcea/}\lon0/\lat0/\wi	&	Cylindrical equal area \\ \hline
~\Opt{Jcyl\_stere/}[\lon0/[\lat0/]]\wi & Cylindrical stereographic \\ \hline
~\Opt{Jeqc/}[\lon0/[\lat0/]]\wi	&	Equidistant cylindrical \\ \hline
~\Opt{Jeqdc/}\lon0/\lat0/\lat1/\lat2/\wi	&	Equidistant conic \\ \hline
~\Opt{Jgnom/}\lon0/\lat0\ho/\wi	&	Azimuthal gnomonic \\ \hline
~\Opt{Jhammer/}[\lon0/]\wi	&	Hammer equal area \\ \hline
~\Opt{Jeck4/}[\lon0/]\wi	&	Eckert IV equal area \\ \hline
~\Opt{Jeck6/}[\lon0/]\wi	&	Eckert VI equal area \\ \hline
~\Opt{Jlaea/}\lon0/\lat0\ho/\wi	&	Lambert azimuthal equal area \\ \hline
~\Opt{Jlcc/}\lon0/\lat0/\lat1/\lat2/\wi	&	Lambert conic conformal \\ \hline
~\Opt{Jmerc/}[\lon0/[\lat0/]]\wi	&	Mercator cylindrical \\ \hline
~\Opt{Jmill/}[\lon0/]\wi	&	Miller cylindrical \\ \hline
~\Opt{Jmoll/}[\lon0/]\wi	&	Mollweide \\ \hline
~\Opt{Jnsper/}\lon0/\lat0/\emph{alt}/\emph{azim}/\emph{tilt}/\emph{twist}/\emph{W}/\emph{H}/\wi & General perspective\\\hline
~\Opt{Jomerc/}\lon0/\lat0/\emph{azim}/\wi	&	Oblique Mercator, 1:	origin and azimuth \\ \hline
~\Opt{Jomerc/}\lon0/\lat0/\lon1/\lat1/\wi	&	Oblique Mercator, 2:	two points \\ \hline
~\Opt{Jomercp/}\lon0/\lat0/\lon{p}/\lat{p}/\wi	&	Oblique Mercator, 3:	origin and pole \\ \hline
~\Opt{Jortho/}\lon0/\lat0\ho/\wi	&	Azimuthal orthographic \\ \hline
~\Opt{Jpolar/}[\textbf{a}]\wi[/\emph{origin}]	&	Polar [azimuthal] ($\theta, r$) (or cylindrical) \\ \hline
~\Opt{Jpoly}[\lon0/[\lat0/]]\Wi	&	(American) polyconic \\ \hline
~\Opt{Jrobin/}[\lon0/]\wi	&	Robinson \\ \hline
~\Opt{Jsinu/}[\lon0/]\wi	&	Sinusoidal equal area \\ \hline
~\Opt{Jstere/}\lon0/\lat0/\ho/\wi	&	General stereographic \\ \hline
~\Opt{Jtmerc/}\lon0/[\lat0/]\wi	&	Transverse Mercator \\ \hline
~\Opt{Jutm/}\emph{zone}/\wi	&	Universal Transverse Mercator (UTM) \\ \hline
~\Opt{Jvandg/}[\lon0/]\wi	&	Van der Grinten \\ \hline
~\Opt{Jwintri/}[\lon0/]\wi	&	Winkel Tripel \\ \hline
~\Opt{Jxy/}\emph{xscale}[\textbf{l}$|$\textbf{p}\emph{exp}$|$\textbf{T}$|$\textbf{t}][/\emph{yscale}[\textbf{l}$|$\textbf{p}\emph{exp}$|$\textbf{T}$|$\textbf{t}]][\textbf{d}]	&	Linear, log$_{10}$, $x^a$--$y^b$, and time \\ \hline
\end{cmdlineopts}

\index{Standardized command line options|)}
\index{\Opt{B} (set annotations and ticks)|)}
\index{\Opt{H} (header records)|)}
\index{\Opt{J} (set map projection)|)}
\index{Projection!azimuthal!Lambert \Opt{Ja} \Opt{JA}|)}
\index{Lambert azimuthal projection \Opt{Ja} \Opt{JA}|)}
\index{\Opt{Ja} \Opt{JA} (Lambert azimuthal)|)}
\index{Projection!conic!Albers \Opt{Jb} \Opt{JB}|)}
\index{Albers conic projection \Opt{Jb} \Opt{JB}|)}
\index{\Opt{Jb} \Opt{JB} (Albers)|)}
\index{Projection!cylindrical!Cassini \Opt{Jc} \Opt{JC}|)}
\index{Cassini projection \Opt{Jc} \Opt{JC}|)}
\index{\Opt{Jc} \Opt{JC} (Cassini)|)}
\index{Projection!cylindrical!stereographic \Opt{Jcyl\_stere} \Opt{JCyl\_stere}|)}
\index{Cylindrical stereographic projection \Opt{Jcyl\_stere} \Opt{JCyl\_stere}|)}
\index{\Opt{Jcyl\_stere} \Opt{JCyl\_stere} (Cylindrical stereographic)|)}
\index{Projection!conic!equidistant \Opt{Jd} \Opt{JD}|)}
\index{Equidistant conic projection \Opt{Jd} \Opt{JD}|)}
\index{\Opt{Jd} \Opt{JD} (Equidistant conic)|)}
\index{Projection!azimuthal!equidistant \Opt{Je} \Opt{JE}|)}
\index{Azimuthal equidistant projection \Opt{Je} \Opt{JE}|)}
\index{\Opt{Je} \Opt{JE} (Azimuthal equidistant)|)}
\index{Projection!azimuthal!gnomonic \Opt{Jf} \Opt{JF}|)}
\index{Gnomonic projection \Opt{Jf} \Opt{JF}|)}
\index{\Opt{Jf} \Opt{JF} (Gnomonic)|)}
\index{Projection!azimuthal!orthographic \Opt{Jg} \Opt{JG}|)}
\index{Orthographic projection \Opt{Jg} \Opt{JG}|)}
\index{\Opt{Jg} \Opt{JG} (Orthographic)|)}
\index{Hammer projection \Opt{Jh} \Opt{JH}|)}
\index{Projection!miscellaneous!Hammer \Opt{Jh} \Opt{JH}|)}
\index{\Opt{Jh} \Opt{JH} (Hammer)|)}
\index{Sinusoidal projection \Opt{Ji} \Opt{JI}|)}
\index{Projection!miscellaneous!Sinusoidal \Opt{Ji} \Opt{JI}|)}
\index{\Opt{Ji} \Opt{JI} (Sinusoidal)|)}
\index{Projection!cylindrical!Miller \Opt{Jj} \Opt{JJ}|)}
\index{Miller cylindrical projection \Opt{Jj} \Opt{JJ}|)}
\index{\Opt{Jj} \Opt{JJ} (Miller)|)}
\index{Eckert IV and VI projection \Opt{Jk} \Opt{JK}|)}
\index{Projection!miscellaneous!Eckert IV and VI \Opt{Jk} \Opt{JK}|)}
\index{\Opt{Jk} \Opt{JK} (Eckert IV and VI)|)}
\index{Projection!conic!Lambert \Opt{Jl} \Opt{JL}|)}
\index{Lambert conic projection \Opt{Jl} \Opt{JL}|)}
\index{\Opt{Jl} \Opt{JL} (Lambert conic)|)}
\index{Projection!cylindrical!Mercator \Opt{Jm} \Opt{JM}|)}
\index{Mercator projection \Opt{Jm} \Opt{JM}|)}
\index{\Opt{Jm} \Opt{JM} (Mercator)|)}
\index{Robinson projection \Opt{Jn} \Opt{JN}|)}
\index{Projection!miscellaneous!Robinson \Opt{Jn} \Opt{JN}|)}
\index{\Opt{Jn} \Opt{JN} (Robinson)|)}
\index{Projection!cylindrical!oblique Mercator \Opt{Jo} \Opt{JO}|)}
\index{Oblique Mercator projection \Opt{Jo} \Opt{JO}|)}
\index{\Opt{Jo} \Opt{JO} (Oblique Mercator)|)}
\index{Projection!polar ($\theta, r$) \Opt{Jp} \Opt{JP}|)}
\index{Polar ($\theta, r$) projection \Opt{Jp} \Opt{JP}|)}
\index{\Opt{Jp} \Opt{JP} (Polar ($\theta, r$) projections)|)}
\index{Projection!conic!polyconic \Opt{Jpoly} \Opt{JPoly}|)}
\index{Polyconic projection \Opt{Jpoly} \Opt{JPoly}|)}
\index{\Opt{Jpoly} \Opt{JPoly} (Polyconic)|)}
\index{Projection!cylindrical!equidistant \Opt{Jq} \Opt{JQ}|)}
\index{Equidistant cylindrical projection \Opt{Jq} \Opt{JQ}|)}
\index{\Opt{Jq} \Opt{JQ} (Cylindrical equidistant)|)}
\index{Winkel Tripel projection \Opt{Jr} \Opt{JR}|)}
\index{Projection!miscellaneous!Winkel Tripel \Opt{Jr} \Opt{JR}|)}
\index{\Opt{Jr} \Opt{JR} (Winkel Tripel)|)}
\index{Projection!azimuthal!stereographic \Opt{Js} \Opt{JS}|)}
\index{Stereographic projection \Opt{Js} \Opt{JS}|)}
\index{\Opt{Js} \Opt{JS} (Stereographic)|)}
\index{Projection!cylindrical!transverse Mercator \Opt{Jt} \Opt{JT}|)}
\index{Projection!cylindrical!UTM \Opt{Ju} \Opt{JU}|)}
\index{Transverse Mercator projection \Opt{Jt} \Opt{JT}|)}
\index{UTM projection \Opt{Ju} \Opt{JU}|)}
\index{\Opt{Jt} \Opt{JT} (Transverse Mercator)|)}
\index{\Opt{Ju} \Opt{JU} (UTM)|)}
\index{Van der Grinten projection \Opt{Jv} \Opt{JV}|)}
\index{Projection!miscellaneous!Van der Grinten \Opt{Jv} \Opt{JV}|)}
\index{\Opt{Jv} \Opt{JV} (Van der Grinten)|)}
\index{Mollweide projection \Opt{Jw} \Opt{JW}|)}
\index{Projection!miscellaneous!Mollweide \Opt{Jw} \Opt{JW}|)}
\index{\Opt{Jw} \Opt{JW} (Mollweide)|)}
\index{Projection!linear|)}
\index{Projection!linear!Cartesian|)}
\index{Cartesian linear projection|)}
\index{Linear projection|)}
\index{\Opt{Jx} \Opt{JX} (Non-map projections)|)}
\index{Projection!logarithmic|)}
\index{Logarithmic projection|)}
\index{Projection!power (exponential)|)}
\index{Power (exponential) projection|)}
\index{Projection!linear!geographic|)}
\index{Geographic Linear projection|)}
\index{Projection!cylindrical!general \Opt{Jy} \Opt{JY}|)}
\index{General cylindrical projection \Opt{Jy} \Opt{JY}|)}
\index{\Opt{Jy} \Opt{JY} (General cylindrical)|)}
\index{\Opt{K} (continue plot)|)}
\index{\Opt{O} (overlay plot)|)}
\index{\Opt{P} (portrait orientation)|)}
\index{\Opt{R} (set region)|)}
\index{\Opt{U} (plot timestamp)|)}
\index{\Opt{V} (verbose mode)|)}
\index{\Opt{X} (shift plot in $x$)|)}
\index{\Opt{Y} (shift plot in $y$)|)}
\index{\Opt{b} (binary i/o)|)}
\index{\Opt{c} (set \# of copies)|)}
\index{\Opt{f} (formatting of i/o)|)}
\index{\Opt{g} (detect gaps in i/o)|)}
\index{\Opt{m} (use multi-segment i/o)|)}
\index{\Opt{:} (input and/or output is $y,x$, not $x,y$)|)}

\index{GMT@\GMT!quick reference|)}

\end{center}
