%------------------------------------------
%	$Id$
%
%	The GMT Documentation Project
%	Copyright (c) 2000-2011.
%	P. Wessel, W. H. F. Smith, R. Scharroo, and J. Luis
%------------------------------------------
%
\chapter{\gmt\ Coordinate Transformations}
\label{ch:5}
\thispagestyle{headings}

\GMT\ programs read real-world coordinates and convert them to positions on a plot.
This is achieved by selecting one of several coordinate transformations or projections.
We distinguish between three sets of such conversions:

\begin{itemize}
\item Cartesian coordinate transformations
\item Polar coordinate transformations
\item Map coordinate transformations
\end{itemize}

The next chapter will be dedicated to \GMT\ map projections in its entirety.  Meanwhile, the present chapter
will summarize the properties of the Cartesian and Polar coordinate transformations available in \GMT, list
which parameters define them, and demonstrate how they are used to create simple plot axes.  We will mostly
be using \GMTprog{psbasemap} (and occasionally \GMTprog{psxy}) to demonstrate the various transformations.
Our illustrations may differ from those you reproduce with the same commands because of different settings
in our \filename{gmt.conf} file.)  Finally, note that while we will specify dimensions in inches (by
appending \textbf{i}), you may want to use cm (\textbf{c}), or points (\textbf{p}) as unit instead
(see the \GMTprog{gmt.conf} man page). 

\section{Cartesian transformations}
\index{Projection!cartesian|(}
\index{Transformation!cartesian|(}

\GMT\ Cartesian coordinate transformations come in three flavors:

\begin{itemize}
\item Linear coordinate transformation
\item Log$_{10}$ coordinate transformation
\item Power (exponential) coordinate transformation
\end{itemize}

These transformations convert input coordinates $(x,y)$ to locations $(x', y')$ on a plot.
There is no coupling between $x$ and $y$ (i.e., $x' = f(x)$ and $y' = f(y)$);
it is a \textbf{one-dimensional} projection. Hence, we may use separate transformations
for the $x$- and $y$-axes (and $z$-axes for 3-D plots).   Below, we will use the expression
$u' = f(u)$, where $u$ is either $x$ or $y$ (or $z$ for 3-D plots).
The coefficients in $f(u)$ depend on the desired plot
size (or scale), the chosen $(x,y)$ domain, and the nature of $f$ itself.


Two subsets of linear will be discussed
separately; these are a polar (cylindrical) projection and a linear projection applied to
geographic coordinates (with a 360\DS\ periodicity in the $x$-coordinate).  We will show examples
of all of these projections using dummy data sets created with
\GMTprog{gmtmath}, a ``Reverse Polish Notation'' (RPN) calculator that
operates on or creates table data: 

\index{Reverse Polish Notation (RPN)}
\index{RPN (Reverse Polish Notation)}
\script{../gmtmath}

%---------------------------------------LINEAR TRANSFORMATION----------------------------------------------

\subsection{Cartesian linear transformation (\Opt{Jx} \Opt{JX})}
\index{Projection!cartesian!linear|(}
\index{Transformation!cartesian!linear|(}
\index{Projection!linear|(}
\index{Cartesian!linear projection|(}
\index{Linear projection|(}
\index{\Opt{Jx} \Opt{JX} (Non-map projections)|(}

There are in fact three different uses of the Cartesian linear transformation, each
associated with specific command line options.  The different manifestations result
from specific properties of three kinds of data:

\begin{enumerate}
\item Regular floating point coordinates
\item Geographic coordinates
\item Calendar time coordinates
\end{enumerate}

\subsubsection{Regular floating point coordinates}

Selection of the Cartesian linear transformation with regular floating point coordinates
will result in a simple linear scaling $u' = au + b$ of the input coordinates.  The projection
is defined by stating

\begin{itemize}
\item scale in inches/unit (\Opt{Jx}) or axis length in inches (\Opt{JX})
\end{itemize}

If the \emph{y}-scale or \emph{y}-axis length is different from that of
the \emph{x}-axis (which is most often the case), separate the two
scales (or lengths) by a slash, e.g., \Opt{Jx}0.1i/0.5i or \Opt{JX}8i/5i. 
Thus, our $y = \sqrt{x}$ data sets will plot as shown in Figure~\ref{fig:GMT_linear}.

\GMTfig{GMT_linear}{Linear transformation of Cartesian coordinates.}

The complete commands given to produce this plot were 

\script{GMT_linear}
\index{Axes!direction}
\index{Axes!reverse}

\noindent
Normally, the user's \emph{x}-values will increase to the right
and the \emph{y}-values will increase upwards.  It should be noted
that in many situations it is desirable to have the direction of
positive coordinates be reversed.  For example, when plotting
depth on the \emph{y}-axis it makes more sense to have the positive
direction downwards.  All that is required to reverse the sense of
positive direction is to supply a negative scale (or axis length).
Finally, sometimes it is convenient to specify the width (or height)
of a map and let the other dimension be computed based on the implied
scale and the range of the other axis.  To do this, simply specify
the length to be recomputed as 0.

\subsubsection{Geographic coordinates}
\label{sec:linear}
\index{Projection!linear!geographic|(}
\index{Transformation!linear!geographic|(}
\index{Geographic Linear projection|(}

\GMTfig{GMT_linear_d}{Linear transformation of map coordinates.}

While the Cartesian linear projection is primarily designed for regular floating point
\emph{x},\emph{y} data, it is sometimes necessary to plot geographical
data in a linear projection.  This poses a problem since longitudes
have a 360\DS\ periodicity.  \GMT\ therefore needs to be informed
that it has been given geographical coordinates even though a linear transformation
has been chosen.  We do so by adding a \textbf{g} (for geographical) or \textbf{d} (for degrees)
directly after \Opt{R} or by appending a \textbf{g} or \textbf{d} to
the end of the \Opt{Jx} (or \Opt{JX}) option.  As an example, we
want to plot a crude world map centered on 125\DS E.  Our command will be 

\script{GMT_linear_d} 

\noindent
with the result reproduced in Figure~\ref{fig:GMT_linear_d}.
\index{Projection!linear!geographic|)}
\index{Transformation!linear!geographic|)}
\index{Geographic Linear projection|)}

\subsubsection{Calendar time coordinates}
\label{sec:time}
\index{Projection!linear!calendar|(}
\index{Transformation!linear!calendar|(}
\index{Calendar Linear projection|(}

\GMTfig{GMT_linear_cal}{Linear transformation of calendar coordinates.}

Several particular issues arise when we seek to make linear plots using calendar date/time as
the input coordinates.  As far as setting up the coordinate transformation we must indicate whether
our input data have absolute time coordinates or relative time coordinates.  For the
former we append \textbf{T} after the axis scale (or width), while for the latter we append \textbf{t} at the
end of the \Opt{Jx} (or \Opt{JX}) option.
However, other command line arguments (like the \Opt{R} option) may already specify whether the time
coordinate is absolute or relative. An
absolute time entry must be given as [\emph{date}]\textbf{T}[\emph{clock}]
(with \emph{date} given as \emph{yyyy}[-\emph{mm}[-\emph{dd}]], \emph{yyyy}[-\emph{jjj}], or \emph{yyyy}[-\textbf{W}\emph{ww}[-\emph{d}]], and \emph{clock} using
the \emph{hh}[:\emph{mm}[:\emph{ss}[\emph{.xxx}]]] 24-hour clock format) whereas the relative time is simply
given as the units of time since the epoch followed by \textbf{t} (see \textbf{TIME\_UNIT} and \textbf{TIME\_EPOCH} for information
on specifying the time unit and the epoch).  As a simple example, we will make a plot of a school week
calendar (Figure~\ref{fig:GMT_linear_cal}).

\script{GMT_linear_cal} 

When the coordinate ranges provided by the \Opt{R} option and the projection type given by \Opt{JX}
(including the optional \textbf{d}, \textbf{g}, \textbf{t} or \textbf{T}) conflict, \GMT\ will warn the
users about it. In general, the options provided with \Opt{JX} will prevail.

\index{Projection!linear!calendar|)}
\index{Transformation!linear!calendar|)}
\index{Calendar Linear projection|)}

\index{Projection!cartesian!linear|)}
\index{Transformation!cartesian!linear|)}
\index{Projection!linear|)}
\index{Cartesian!linear projection|)}
\index{Linear projection|)}

%---------------------------------------LOG10 TRANSFORMATION----------------------------------------------

\subsection{Cartesian logarithmic projection}
\index{Projection!cartesian!logarithmic|(}
\index{Transformation!cartesian!logarithmic|(}
\index{Logarithmic projection|(}

\GMTfig[h]{GMT_log}{Logarithmic transformation of $x$-coordinates.}

The log$_{10}$ transformation is simply $u' = a \log_{10}(u) + b$ and is selected by appending an \textbf{l}
(lower case L) immediately following the scale (or axis length)
value.  Hence, to produce a plot in which the \emph{x}-axis is
logarithmic (the \emph{y}-axis remains linear, i.e., a semi-log plot), try 

\script{GMT_log} 

\par Note that if \emph{x}- and \emph{y}-scaling are different and
a log$_{10}$-log$_{10}$ plot is desired, the \textbf{l} must be
appended twice: Once after the \emph{x}-scale (before the /) and
once after the \emph{y}-scale. 
\index{Projection!cartesian!logarithmic|)}
\index{Transformation!cartesian!logarithmic|)}
\index{Logarithmic projection|)}

%---------------------------------------POWER TRANSFORMATION----------------------------------------------

\subsection{Cartesian power projection}
\index{Projection!cartesian!power (exponential)|(}
\index{Transformation!cartesian!power (exponential)|(}
\index{Power (exponential) projection|(}

\GMTfig[h]{GMT_pow}{Exponential or power transformation of $x$-coordinates.}

This projection uses $u' = a u^b + c$ and allows us to explore exponential relationships like \emph{x$^p$} versus \emph{y$^q$}.
While $p$ and $q$ can be any values, we will select $p
= 0.5$ and $q = 1$ which means we will plot $x$ versus $\sqrt{x}$.
We indicate this scaling by appending a \textbf{p} (lower case P) followed
by the desired exponent, in our case 0.5.  Since $q = 1$ we do not
need to specify \textbf{p}1 since it is identical to the linear transformation.
Thus our command becomes

\script{GMT_pow} 
\index{Projection!cartesian!power (exponential)|)}
\index{Transformation!cartesian!power (exponential)|)}
\index{Power (exponential) projection|)}
\index{\Opt{Jx} \Opt{JX} (Non-map projections)|)}

%---------------------------------------POLAR TRANSFORMATION----------------------------------------------

\section{Linear projection with polar ($\theta, r$) coordinates (\Opt{Jp } \Opt{JP})}
\index{Projection!polar ($\theta, r$) \Opt{Jp} \Opt{JP}|(}
\index{Polar ($\theta, r$) projection \Opt{Jp} \Opt{JP}|(}
\index{\Opt{Jp} \Opt{JP} (Polar ($\theta, r$) projections)|(}

\GMTfig{GMT_polar}{Polar (Cylindrical) transformation of
($\theta, r$) coordinates.}

This transformation converts polar coordinates (angle $\theta$ and radius $r$)
to positions on a plot.  Now $x' = f(\theta,r)$ and $y' = g(\theta,r)$, hence it is similar
to a regular map projection because $x$ and $y$ are coupled and $x$ (i.e., $\theta$) has a 360\DS\ periodicity.
With input and output points both in the plane it is a \textbf{two-dimensional} projection.
The transformation comes in two flavors:

\begin{enumerate}
\item Normally, $\theta$ is understood to be directions counter-clockwise from the horizontal axis, but we may choose
to specify an angular offset [whose default value is zero].  We will call this offset $\theta_0$.
Then, $x' = f(\theta, r) = ar \cos (\theta-\theta_0) + b$ and $y' = g(\theta, r) = ar \sin (\theta-\theta_0) + c$.
\item Alternatively, $\theta$ can be interpreted to be azimuths clockwise from the vertical axis, yet we may again
choose to specify the angular offset [whose default value is zero].
Then, $x' = f(\theta, r) = ar \cos (90 - (\theta-\theta_0)) + b$ and $y' = g(\theta, r) = ar \sin (90 - (\theta-\theta_0)) + c$.
\end{enumerate}

Consequently, the polar transformation is defined by providing

\begin{itemize}
\item scale in inches/unit (\Opt{Jp}) or full width of plot in inches (\Opt{JP})
\item Optionally, insert \textbf{a} after \textbf{p$|$P} to indicate CW azimuths rather than CCW directions
\item Optionally, append /$origin$ in degrees to indicate an angular offset [0]
\item Optionally, append \textbf{r} to reverse the radial direction (here, \emph{south} and \emph{north} must be elevations in 0--90\DS\ range).
\item Optionally, append \textbf{z} to annotate depths rather than radius.
\end{itemize}

As an example of this projection we will create a gridded data set
in polar coordinates $z(\theta, r) = r^2 \cdot \cos{4\theta}$
using \GMTprog{grdmath}, a RPN calculator that operates on or
creates grid files.

\script{GMT_polar} 

We used \GMTprog{grdcontour} to make a contour map of this data.  Because
the data file only contains values with $2 \leq r \leq 4$, a donut
shaped plot appears in Figure~\ref{fig:GMT_polar}.\
\index{Projection!polar ($\theta, r$) \Opt{Jp} \Opt{JP}|)}
\index{Polar ($\theta, r$) projection \Opt{Jp} \Opt{JP}|)}
\index{\Opt{Jp} \Opt{JP} (Polar ($\theta, r$) projections)|)}
