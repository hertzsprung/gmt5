%------------------------------------------
%	$Id: GMT_Chapter_5.tex,v 1.3 2001-09-21 21:39:22 pwessel Exp $
%
%	The GMT Documentation Project
%	Copyright 2000-2001.
%	Paul Wessel and Walter H. F. Smith
%------------------------------------------
%
\chapter{\gmt\ Transformations and Projections}
\thispagestyle{headings}

\GMT\ programs that read coordinate data will need to know how
to convert the input coordinates to positions on a map or plot.
This is achieved by selecting one of several transformations or projections.
We distinguish between two sets of scalings:

\begin{description}
\item [Linear Transformations].  This label refers to a set of procedures
that transform input coordinates $(x,y)$ to locations $(x', y')$ on a plot.
There is no coupling between $x$ and $y$, i.e., $x' = f(x)$ and $y' = f(y)$.
For that reason, we may use separate transformations for the $x$-, $y$-, and
$z$-axes.
Below, we will use the expression $u' = f(u)$, where $u$ is either $x$ or $y$.
The constant scales and offsets in $f(t)$ are determined from the desired plot
size (or scale), the chosen $(x,y)$ domain, and the nature of $f$.
This group contains

\begin{itemize}
\item Linear transformation.  Here, $u' = au + b$
\item Log$_{10}$ transformation.  Here, $x' = a \log_{10}(x) + b$.
\item power (exponential) transformation.  Here, $x' = a x^b + c$.
\end{itemize}

Two special versions of the linear projection deserve special mention.

\begin{enumerate}
\item It is sometimes useful to plot geographic coordinates (longitude, latitude) using
a linear projection.  However, longitude is periodic in 360\DS\ and hence special handling
must be used to ensure that this is considered.
\item When the input coordinate is absolute time (calendar dates and clocks) there are special
issues that are related to unit conversion and the non-equidistant anotation separations for
some units.  Other than that, the final internal representation of time is linearly transformed
like any other coordinate.
\end{enumerate}

\item [Polar Transformation]. This transforms polar coordinates (angle $\theta$ and radius $r$)
to positions on a plot.  Now, $x' = f(\theta,r)$ and $y' = g(\theta,r)$ hence it is similar
to a regular map projection in the sense that $x$ and $y$ are coupled and x ($\theta$) has a 360\DS\ periodicity.
The transformation comes in two flavors:

\begin{enumerate}
\item Normally, $\theta$ is understood to be directions counter-clockwise from the horizontal axis, but we may choose
to specify an angular offset [whose default value is zero].  We will call this offset $\theta_0$.
Then, $x' = f(\theta, r) = ar \cos (\theta-\theta_0) + b$ and $y' = g(\theta, r) = ar \sin (\theta-\theta_0) + c$.
\item Alternatively, $\theta$ can be interpreted to be azimuths clockwise from the vertical axis, yet we may again
choose to specify the angular offset [whose default value is zero].
Then, $x' = f(\theta, r) = ar \cos (90 - (\theta-\theta_0)) + b$ and $y' = g(\theta, r) = ar \sin (90 - (\theta-\theta_0)) + c$.
\end{enumerate}

\item [Map Projections].  This group consists of 23 different map projections.  They all project the input coordinates
longitude and latitude to positions on a map.  In general, $x' = f(x,y)$ and $y' = g(x,y)$, where $f$ and $g$ can be
quite nasty and we will refrain from presenting details in this document.  The interested read is referred to
{\it Snyder} [1987]\footnote{Snyder, J. P., 1987, Map Projections \- A Working Manual, U.S. Geological Survey Prof. Paper 1395.}.
\end{description}

The purpose of the following two chapters is to summarize the properties
of transformations and map projections available in \GMT, what parameters define
them, and demonstrate how they are used to create simple
basemaps.  We will mostly be using the \GMTprog{pscoast}
command for geographic maps and \GMTprog{psbasemap} (and occasionally \GMTprog{psxy}) for other transformations.
Our illustrations may differ from yours because of different settings in our
\filename{.gmtdefaults} file.)  Note that while we will
specify dimensions in inches (by appending {\bf i}), you may
want to use cm ({\bf c}), meters ({\bf m}), or points ({\bf p})
as unit instead (see the \GMTprog{gmtdefaults} man page). 
