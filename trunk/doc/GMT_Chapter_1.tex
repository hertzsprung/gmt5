%------------------------------------------
%	$Id: GMT_Chapter_1.tex,v 1.19 2004-09-14 20:24:56 pwessel Exp $
%
%	The GMT Documentation Project
%	Copyright 2000-2004.
%	Paul Wessel and Walter H. F. Smith
%------------------------------------------
%
\chapter{Preface} 
\thispagestyle{headings}

While \GMT\ has served the map-making and data processing needs of scientists since 1988\footnote{when
version 1.0 was released at Lamont-Doherty Earth Observatory.}, the current global use was
heralded by the first official release in {\it EOS Trans. AGU} in the fall of 1991.  Since then,
\GMT\ has grown to become a standard tool for many Earth and Ocean scientists.  Development
has at times been rapid, and numerous releases have seen the light of day since the early versions.
For a history of the changes from release to release, see the online
\htmladdnormallink{Release Announcements}{http://gmt.soest.hawaii.edu/gmt/gmt_releases.html}
and the file \filename{CHANGES} in the main \GMT\ directory.

The success of \GMT\ is to a large degree due to the input of the user community. In fact, most of the
capabilities and options in \GMT\ programs originated as user requests.
We would like to hear from you should you have any suggestions for future enhancements and modification.
Please send your comments to
\htmladdnormallink{the GMT help list}{mailto:gmt-help@hawaii.edu}.

\section{What is new in \gmt\ 4?}

\GMT\ 4 represents a major overhaul of the package, hence the major version number increment.  There are four
categories of changes that have been implemented:
\begin{description}
\item [Time-series support.]  \GMT\ can now read and write time-series data where
the time coordinates are of the form {\it date}{\bf T}{\it clock}\footnote{Use standard
\UNIX\ tools such as \progname{awk} or \progname{perl} to reformat files should
your {\it date} and {\it clock} components reside in separate columns.}.  The formats
used for {\it date} and {\it clock} are under the user's control.  Both Gregorian
and ISO calendars are supported.  Frame annotation for time-series are now supported
via the \Opt{B} option; there are many new modifiers reflecting the vast number of
ways one may want to annotate time axes, including support for primary and secondary
annotation levels and the day- and month-names in numerous languages (send us the info
we need if your language is not supported).  The capability to handle time (in \Opt{R},
\Opt{J}, \Opt{B}, i/o, and plotting) required considerable changes ``under the hood'',
including the introduction of numerous new \GMTprog{gmtdefaults} parameters to make
time series support as ``generic'' as we need it to be.
\item [New Tools.]  Three new tools have been added:
\begin{enumerate}
\item \GMTprog{gmt2rgb}: Make red, green, and blue component gridfiles from an image (to be
used with new options for false color imaging or image draping by \GMTprog{grdimage} or \GMTprog{grdview}).
\item \GMTprog{grdblend}: Blend several partially over-lapping grdfiles onto one combined grid.  Output
grid is written one row at the time so enormous grids can be created.
\item \GMTprog{pslegend}: Design and plot elaborate legends on maps.
\end{enumerate}
\item [New Program Options.]  Many programs have received additional options or
features that enhances their usefulness:
\begin{itemize}
\item \GMTprog{blockmean}:	New option \Opt{Sw} will return weight sum while \Opt{Sz} returns
the data sums (i.e., the same as \Opt{S}).
\item \GMTprog{filter1d}:	New filters \Opt{Fl$|$L$|$u$|$U} that return extreme values.
\item \GMTprog{gmtconvert}:	Added new options \Opt{F}, \Opt{A},  and \Opt{I} that simulate
\UNIX\ \progname{cut}, \progname{paste}, and \progname{tail} \Opt{r} (or \progname{tac}) capabilities.
Option \Opt{E} reports first and last point per segment only, \Opt{L} lists the segment headers only,
while \Opt{S} lists records from segments whose header matches a given text pattern.
\item \GMTprog{gmtmath}:	Added new operators for solving least squares problems ({\bf COL, LSQFIT}),
finding function roots ({\bf ROOTS}), and evaluating critical values ({\bf CHICRIT, FCRIT, TCRIT, ZCRIT}).
We also added some general functions ({\bf SINC, LOG2, LRAND}) and miscellaneous operations ({\bf FLIPUD, NEQ}).
The \Opt{S} option may now take a modifier to select first or last record only.
\item \GMTprog{gmtselect}:	New option  \Opt{Z} to pass or skip based of $z$-range.
\item \GMTprog{grd2cpt}:	New options  \Opt{Q} for logarithmic scales, \Opt{E} for equidistant color
intervals, \Opt{R} for selecting a grid sub-region, and \Opt{N} to suppress output of B, N, F colors.
\item \GMTprog{grd2xyz}:	New option \Opt{W} to write constant weight factor as 4th output column,
and ability to process several grid files at the same time.
\item \GMTprog{grdcontour}:	Expanded the \Opt{G} option to handle 5 algorithms (4 new) for the placement
of contour labels. 
\item \GMTprog{grdedit}:	New option \Opt{N} to replace selected node values given {\it x, y, z} data
in table form (options \Opt{H}, \Opt{b}, \Opt{f}, and \Opt{:} added for file support).
\item \GMTprog{grdfilter}:	New geospatial filters \Opt{Fl$|$L$|$u$|$U} that return extreme values. 
\item \GMTprog{grdimage}:	New option for colormasking (\Opt{Q}; \PS\ Level 3 only), \PS\ image 
interpolation (\Opt{E}{\it -dpi}), and false RGB color image (if given three grids).
\item \GMTprog{grdinfo}:	New option to create argument for \GMTprog{makecpt} (\Opt{T}) and to round-off
region (\Opt{I}). 
\item \GMTprog{grdmath}:	Added new operators for critical values ({\bf CHICRIT, FCRIT, TCRIT, ZCRIT}),
geospatial analysis ({\bf LDIST,  PDIST, INSIDE}) and for calculating azimuths ({\bf CAS, SAZ}).  We have
also added some general functions ({\bf SINC, LOG2, LRAND}) and a few grid operations ({\bf FLIPLR, FLIPUD, ROTX, ROTY, NEQ,
INRANGE}).  We may now create multiple output grids from a single command.
\item \GMTprog{grdreformat}: Option to suppress header in raw output (\Opt{N}).
\item \GMTprog{grdsample}:	Option to push the bilinear interpolation closer to nodes that are NaN (\Opt{Q}).
\item \GMTprog{grdtrack}:	Options to retrieve nearest node value (\Opt{N}, no interpolation) and to push
the bilinear interpolation closer to nodes that are NaN (\Opt{Q}).
\item \GMTprog{grdview}:	Colormasking (\Opt{Qc}, PS Level 3 only), draping of images via red, green,
and blue component grids (\Opt{G}).  Also,  drapegrids can have higher resolution than the relief grid.
\item \GMTprog{makecpt}:	New options \Opt{Q} for logarithmic scales and \Opt{N} to suppress output
of B, N, F colors.
\item \GMTprog{mapproject}:	New options for datum conversions (\Opt{T}, \Opt{E}, and \Opt{Q}), azimuth and
back-azimuth (\Opt{A}) and point distance (\Opt{G}) calculations, and to calculate distances to a line (\Opt{L}).
\item \GMTprog{minmax}:		Added \Opt{T}{\it dz} option to produce \Opt{T} string for \GMTprog{makecpt},
\Opt{E} for returning extreme records, and the \Opt{I} option was extended to handle any number of columns when \Opt{C} is used.
\item \GMTprog{psbasemap}:	Extended \Opt{L} to allow alternate label and justification and added \Opt{T}
for directional rose ornament or magnetic compass directions.
\item \GMTprog{pscoast}:	Extended \Opt{L} to allow alternate label and justification and added \Opt{T}
for directional rose ornament or magnetic compass directions.
\item \GMTprog{pscontour}:	Expanded the \Opt{G} option to handle 5 algorithms (4 new) for the placement
of contour labels. 
\item \GMTprog{psimage}:	\PS\ image interpolation (\Opt{W}{\it -xlength}), and justification option
in \Opt{C}.
\item \GMTprog{psscale}:	Options to annotate on opposite side (\Opt{A}) and to set back or foreground
triangle only (\Opt{E}[{\bf b$|$f}] ).  Also, draw discrete color-key table with centered annotations by appending an optional
{\it gap} to the \Opt{L} option. 
\item \GMTprog{pstext}: 	New option \Opt{A} should azimuths rather than angles be given,
\item \GMTprog{psxy}: 	Line color control (via \Opt{C}), symbol position offset (with \Opt{D}), custom symbols access 
(with \Opt{Sk}; use any of the 35 (Appendix N) that come with \GMT\ or design your own), many new symbols (horizontal and vertical dashes,
pentagon, octagon, rectangle, double-header and centered vectors), and annotated (``quoted'') lines with \Opt{Sq}.
\item \GMTprog{psxyz}: 	Same, plus vertical dash symbol.
\item \GMTprog{xyz2grd}: 	Added \Opt{Au$|$l} for upper/lower value at each node.
\end{itemize}
\item [General enhancements.]  These affect most of the programs:
\begin{itemize}
\item The coastline data have been updated to GSHHS version 1.3.  About 50 or so polygons had lingering
crossovers and some had duplicate points or failed to close; these have now been fixed. Major
errors in the Puget Sound coastline have also been corrected.
\item New shorthand to repeat the most recently used projection (\Opt{J}).
\item Options for phase-shifting the stride and supply a prefix for frame annotations (\Opt{B}).
\item Override \GMT\ defaults directly on the command line with any number of {--}{--}\emph{PAR=value} options.
\item Now choose from 63 ellipsoids and 223 datums, plus add additional custom values.
\item Numerous new \GMT\ defaults parameters, mostly in support of time-series functionality.
\item Shorthand for global regions (\Opt{Rg} for \Opt{R}0/360/-90/90 and \Opt{Rd} for \Opt{R}-180/180/-90/90).
\item Full support for either RGB, HSV, or CMYK in pen/fill command-line options or in cpt files.
\item Support for English color names (e.g., red, lightbrown).
\item Choice of unit when specifying pen thickness (cm, inch, point).
\item Easier pen specification mechanism, with predefined names for certain pen thicknesses.
\item Centering of plots on current page with \Opt{Xc}, \Opt{Yc}.
\item More control over input/output table formats (\Opt{f}, \Opt{:}[{\bf i$|$o}]).
\item Ability to read and write NOAA/NGDC GRD98 grid format.
\item Ability to add additional fonts.
\item Custom paper media size (useful for posters and large maps).
\item All text are now justified by the \PS\ interpreter.
\item Better support for various international character encodings.
\item New Appendices M (color tables), N (custom symbols), O (contours and ``quoted lines''), and P
(using both \GMT\ 3 and 4).
\item New hidden files \filename{.gmtdefaults4} and \filename{.gmtcommands4} to ensure peaceful coexistence with \GMT\ 3-series.
\item Data files in directories pointed to by the three environmental parameters {\bf \$GMT\_DATADIR}, {\bf \$GMT\_GRIDDIR},
and {\bf \$GMT\_IMGDIR} can be specified without their full path names when used as input files.
\item We have added five new examples for a total of 25.
\item Bourne shell utility \progname{gmtswitch} simplifies switching between installed \GMT\ versions.
\end{itemize}
\end{description}

